% Options for packages loaded elsewhere
\PassOptionsToPackage{unicode}{hyperref}
\PassOptionsToPackage{hyphens}{url}
\documentclass[
]{book}
\usepackage{xcolor}
\usepackage{amsmath,amssymb}
\setcounter{secnumdepth}{5}
\usepackage{iftex}
\ifPDFTeX
  \usepackage[T1]{fontenc}
  \usepackage[utf8]{inputenc}
  \usepackage{textcomp} % provide euro and other symbols
\else % if luatex or xetex
  \usepackage{unicode-math} % this also loads fontspec
  \defaultfontfeatures{Scale=MatchLowercase}
  \defaultfontfeatures[\rmfamily]{Ligatures=TeX,Scale=1}
\fi
\usepackage{lmodern}
\ifPDFTeX\else
  % xetex/luatex font selection
\fi
% Use upquote if available, for straight quotes in verbatim environments
\IfFileExists{upquote.sty}{\usepackage{upquote}}{}
\IfFileExists{microtype.sty}{% use microtype if available
  \usepackage[]{microtype}
  \UseMicrotypeSet[protrusion]{basicmath} % disable protrusion for tt fonts
}{}
\makeatletter
\@ifundefined{KOMAClassName}{% if non-KOMA class
  \IfFileExists{parskip.sty}{%
    \usepackage{parskip}
  }{% else
    \setlength{\parindent}{0pt}
    \setlength{\parskip}{6pt plus 2pt minus 1pt}}
}{% if KOMA class
  \KOMAoptions{parskip=half}}
\makeatother
\usepackage{color}
\usepackage{fancyvrb}
\newcommand{\VerbBar}{|}
\newcommand{\VERB}{\Verb[commandchars=\\\{\}]}
\DefineVerbatimEnvironment{Highlighting}{Verbatim}{commandchars=\\\{\}}
% Add ',fontsize=\small' for more characters per line
\usepackage{framed}
\definecolor{shadecolor}{RGB}{248,248,248}
\newenvironment{Shaded}{\begin{snugshade}}{\end{snugshade}}
\newcommand{\AlertTok}[1]{\textcolor[rgb]{0.94,0.16,0.16}{#1}}
\newcommand{\AnnotationTok}[1]{\textcolor[rgb]{0.56,0.35,0.01}{\textbf{\textit{#1}}}}
\newcommand{\AttributeTok}[1]{\textcolor[rgb]{0.13,0.29,0.53}{#1}}
\newcommand{\BaseNTok}[1]{\textcolor[rgb]{0.00,0.00,0.81}{#1}}
\newcommand{\BuiltInTok}[1]{#1}
\newcommand{\CharTok}[1]{\textcolor[rgb]{0.31,0.60,0.02}{#1}}
\newcommand{\CommentTok}[1]{\textcolor[rgb]{0.56,0.35,0.01}{\textit{#1}}}
\newcommand{\CommentVarTok}[1]{\textcolor[rgb]{0.56,0.35,0.01}{\textbf{\textit{#1}}}}
\newcommand{\ConstantTok}[1]{\textcolor[rgb]{0.56,0.35,0.01}{#1}}
\newcommand{\ControlFlowTok}[1]{\textcolor[rgb]{0.13,0.29,0.53}{\textbf{#1}}}
\newcommand{\DataTypeTok}[1]{\textcolor[rgb]{0.13,0.29,0.53}{#1}}
\newcommand{\DecValTok}[1]{\textcolor[rgb]{0.00,0.00,0.81}{#1}}
\newcommand{\DocumentationTok}[1]{\textcolor[rgb]{0.56,0.35,0.01}{\textbf{\textit{#1}}}}
\newcommand{\ErrorTok}[1]{\textcolor[rgb]{0.64,0.00,0.00}{\textbf{#1}}}
\newcommand{\ExtensionTok}[1]{#1}
\newcommand{\FloatTok}[1]{\textcolor[rgb]{0.00,0.00,0.81}{#1}}
\newcommand{\FunctionTok}[1]{\textcolor[rgb]{0.13,0.29,0.53}{\textbf{#1}}}
\newcommand{\ImportTok}[1]{#1}
\newcommand{\InformationTok}[1]{\textcolor[rgb]{0.56,0.35,0.01}{\textbf{\textit{#1}}}}
\newcommand{\KeywordTok}[1]{\textcolor[rgb]{0.13,0.29,0.53}{\textbf{#1}}}
\newcommand{\NormalTok}[1]{#1}
\newcommand{\OperatorTok}[1]{\textcolor[rgb]{0.81,0.36,0.00}{\textbf{#1}}}
\newcommand{\OtherTok}[1]{\textcolor[rgb]{0.56,0.35,0.01}{#1}}
\newcommand{\PreprocessorTok}[1]{\textcolor[rgb]{0.56,0.35,0.01}{\textit{#1}}}
\newcommand{\RegionMarkerTok}[1]{#1}
\newcommand{\SpecialCharTok}[1]{\textcolor[rgb]{0.81,0.36,0.00}{\textbf{#1}}}
\newcommand{\SpecialStringTok}[1]{\textcolor[rgb]{0.31,0.60,0.02}{#1}}
\newcommand{\StringTok}[1]{\textcolor[rgb]{0.31,0.60,0.02}{#1}}
\newcommand{\VariableTok}[1]{\textcolor[rgb]{0.00,0.00,0.00}{#1}}
\newcommand{\VerbatimStringTok}[1]{\textcolor[rgb]{0.31,0.60,0.02}{#1}}
\newcommand{\WarningTok}[1]{\textcolor[rgb]{0.56,0.35,0.01}{\textbf{\textit{#1}}}}
\usepackage{longtable,booktabs,array}
\usepackage{calc} % for calculating minipage widths
% Correct order of tables after \paragraph or \subparagraph
\usepackage{etoolbox}
\makeatletter
\patchcmd\longtable{\par}{\if@noskipsec\mbox{}\fi\par}{}{}
\makeatother
% Allow footnotes in longtable head/foot
\IfFileExists{footnotehyper.sty}{\usepackage{footnotehyper}}{\usepackage{footnote}}
\makesavenoteenv{longtable}
\usepackage{graphicx}
\makeatletter
\newsavebox\pandoc@box
\newcommand*\pandocbounded[1]{% scales image to fit in text height/width
  \sbox\pandoc@box{#1}%
  \Gscale@div\@tempa{\textheight}{\dimexpr\ht\pandoc@box+\dp\pandoc@box\relax}%
  \Gscale@div\@tempb{\linewidth}{\wd\pandoc@box}%
  \ifdim\@tempb\p@<\@tempa\p@\let\@tempa\@tempb\fi% select the smaller of both
  \ifdim\@tempa\p@<\p@\scalebox{\@tempa}{\usebox\pandoc@box}%
  \else\usebox{\pandoc@box}%
  \fi%
}
% Set default figure placement to htbp
\def\fps@figure{htbp}
\makeatother
\setlength{\emergencystretch}{3em} % prevent overfull lines
\providecommand{\tightlist}{%
  \setlength{\itemsep}{0pt}\setlength{\parskip}{0pt}}
\usepackage[]{natbib}
\bibliographystyle{plainnat}
\usepackage{booktabs}
\usepackage{booktabs}
\usepackage{longtable}
\usepackage{array}
\usepackage{multirow}
\usepackage{wrapfig}
\usepackage{float}
\usepackage{colortbl}
\usepackage{pdflscape}
\usepackage{tabu}
\usepackage{threeparttable}
\usepackage{threeparttablex}
\usepackage[normalem]{ulem}
\usepackage{makecell}
\usepackage{xcolor}
\usepackage{fontspec}
\usepackage{multicol}
\usepackage{hhline}
\newlength\Oldarrayrulewidth
\newlength\Oldtabcolsep
\usepackage{hyperref}
\usepackage{bookmark}
\IfFileExists{xurl.sty}{\usepackage{xurl}}{} % add URL line breaks if available
\urlstyle{same}
\hypersetup{
  pdftitle={R Tutorial for MKT 4320},
  pdfauthor={Jeffrey Meyer, Ph.D., Bowling Green State University},
  hidelinks,
  pdfcreator={LaTeX via pandoc}}

\title{R Tutorial for MKT 4320}
\usepackage{etoolbox}
\makeatletter
\providecommand{\subtitle}[1]{% add subtitle to \maketitle
  \apptocmd{\@title}{\par {\large #1 \par}}{}{}
}
\makeatother
\subtitle{Marketing Analytics}
\author{Jeffrey Meyer, Ph.D., Bowling Green State University}
\date{2025}

\begin{document}
\maketitle

{
\setcounter{tocdepth}{1}
\tableofcontents
}
\chapter*{Preface}\label{preface}
\addcontentsline{toc}{chapter}{Preface}

\section*{About this book}\label{about-this-book}

This book is a companion resource for \textbf{MKT 4320: Marketing Analytics} at Bowling Green State University.

Its purpose is to help you \textbf{use R effectively for applied marketing analytics}, not to provide a comprehensive or theoretical treatment of the R language. The focus is on learning how to run analyses, interpret results, and apply methods to real marketing problems.

This book is intentionally practical and course-driven. Chapters are designed to support lab assignments, projects, and examples used throughout the semester.

\begin{center}\rule{0.5\linewidth}{0.5pt}\end{center}

\section*{What this book is (and is not)}\label{what-this-book-is-and-is-not}

This book \textbf{is}:

\begin{itemize}
\tightlist
\item
  A guided introduction to R and RStudio for marketing analytics
\item
  A reference for the types of analyses used in MKT 4320
\item
  A practical companion to lab assignments and course projects
\end{itemize}

This book \textbf{is not}:

\begin{itemize}
\tightlist
\item
  A complete R programming textbook
\item
  A reference for every possible R function or package
\item
  A substitute for reading lab instructions carefully
\end{itemize}

You are not expected to memorize R syntax. You \emph{are} expected to run code, modify examples, interpret output, and explain results.

\begin{center}\rule{0.5\linewidth}{0.5pt}\end{center}

\section*{The MKT4320BGSU package}\label{the-mkt4320bgsu-package}

Many examples in this book rely on the \textbf{MKT4320BGSU} R package.

This package contains:

\begin{itemize}
\tightlist
\item
  Custom functions used in the course
\item
  Datasets for labs and demonstrations
\item
  Tools designed specifically for teaching marketing analytics
\end{itemize}

Because the package is actively maintained, this book may be updated over time to reflect changes in functions, workflows, or best practices.

\begin{center}\rule{0.5\linewidth}{0.5pt}\end{center}

\section*{How to use this book}\label{how-to-use-this-book}

Most weeks, you will use this book alongside a lab assignment. A typical workflow looks like this:

\begin{enumerate}
\def\labelenumi{\arabic{enumi}.}
\tightlist
\item
  Read the relevant sections of the book
\item
  Open the corresponding lab file
\item
  Run the example code
\item
  Modify the code to answer lab questions
\item
  Interpret results and write conclusions
\end{enumerate}

Code examples are meant to be run. Errors are normal and are part of the learning process.

\begin{center}\rule{0.5\linewidth}{0.5pt}\end{center}

\section*{A note on updates}\label{a-note-on-updates}

This book is a \textbf{living document}. Content may be revised, expanded, or clarified as the course progresses or as the supporting R package evolves.

When updates occur, they are intended to improve clarity, consistency, or alignment with course objectives.

\begin{center}\rule{0.5\linewidth}{0.5pt}\end{center}

\section*{Acknowledgments}\label{acknowledgments}

Portions of this material are adapted from a variety of sources, including R documentation, textbooks, online tutorials, and teaching materials developed over time. Where appropriate, sources are noted within individual chapters.

\begin{center}\rule{0.5\linewidth}{0.5pt}\end{center}

\section*{What's next}\label{whats-next}

The first chapter covers \textbf{Setup}, including how to install and configure R, RStudio, and the required packages for the course.

From there, the book moves into core R fundamentals and then into the specific analytical methods used in marketing analytics.

\chapter{Setup}\label{setup}

\section{What you need}\label{what-you-need}

To work through this book, you will need:

\begin{itemize}
\tightlist
\item
  Access to R/RStudio
\item
  An internet connection (for installing packages)
\item
  A basic familiarity with saving and opening files on your computer
\end{itemize}

\section{R and RStudio (quick overview)}\label{r-and-rstudio-quick-overview}

R is the language that runs your analysis. RStudio is the interface you use to write code, run code, view plots, and manage files. You have three options for using R/RStudio:

\begin{enumerate}
\def\labelenumi{\arabic{enumi}.}
\tightlist
\item
  Use OSC Classroom On Demand\\
  Using the OSC Classroom on Demand is by far the easiest way to use R/RStudio for this course. All packages (see below for more info on packages) should already be loaded and tested to ensure compatability.
\item
  Install on your own machine\\
  This option involves two steps. First, install ``base'' R from the Comprehensive R Archive Network (CRAN): \url{https://cran.r-project.org/}. Second, install RStudio: \url{https://posit.co/download/rstudio-desktop/\#download}.
\item
  Use a computer in an on-campus computer lab.\\
  This option should be considered a last resort. I cannot guarantee all packages will be able to be installed on university machines, and you would need to install the packages every time you use the machine.
\end{enumerate}

\section{Navigating RStudio}\label{navigating-rstudio}

RStudio's interface is typically organized into four main panes in a default layout: the Source Editor, the R Console, the Environment/History, and the Files/Plots/Packages/Help/Viewer pane.

\begin{itemize}
\tightlist
\item
  \textbf{Source} (\emph{usually} top-left)\\
  This pane is where you write, edit, and save your R scripts or R Markdown files. It functions like a text editor with features like syntax highlighting and code completion. Code is not executed immediately upon typing. You must explicitly send lines or sections of code to the Console (commonly using the ``Run'' button or keyboard shortcuts like Ctrl+Enter on Windows/Linux or Cmd+Enter on Mac) to execute them. This pane only appears when you have a file open.
\item
  \textbf{Console} (\emph{usually} bottom-left)\\
  This pane is where commands are actually run and text-based output, warnings, or error messages are displayed. You can type R commands directly into the console for immediate execution. The console shows a \texttt{\textgreater{}} prompt when it is ready to accept a new command. Commands typed directly here are not saved automatically, which is why writing in the Source editor is recommended for complete analyses.
\item
  \textbf{Environments} (\emph{usually} top-right)\\
  This pane helps manage your current R session and the objects within it. It may contain several tabs, including:

  \begin{itemize}
  \tightlist
  \item
    Environment Tab: Displays all the active objects (e.g., data frames, variables, functions, vectors) you've created or loaded during your current R session. You can inspect brief summaries of these objects here.
  \item
    History Tab: Provides a log of all the commands that have been successfully executed in the console, which can be useful for reviewing past work.
  \item
    Other Tabs: May also include tabs for Connections, Build, or Tutorials, depending on your RStudio configuration.
  \end{itemize}
\item
  \textbf{Output} (\emph{usually} top-right)\\
  This multi-purpose pane provides access to various tools for project management and output viewing
  Files: A file browser for navigating your computer's directory structure and managing files within your current working directory.
  Plots: Where all the visualizations and graphs you create with R code will be displayed. It also may contain several tab, including:

  \begin{itemize}
  \tightlist
  \item
    Packages: Lists all installed R packages and allows you to install new ones or load them into your current session.
  \item
    Help: The built-in documentation browser for R functions and packages. You can search for help directly here or by using a command like \texttt{?function\_name} in the console.
  \item
    Viewer: Used for displaying local web content, such as interactive HTML outputs, Shiny apps, or interactive plots generated by certain packages.
  \item
    Plots: Use for displaying static plots generated by code.
  \end{itemize}
\end{itemize}

The layout of these panes can be customized via the \emph{Tools} -\textgreater{} \emph{Global Options} -\textgreater{} \emph{Pane Layout} menu. For more information on the panes in RStudio, please visit \url{https://docs.posit.co/ide/user/ide/guide/ui/ui-panes.html}.

\section{Installing and loading packages}\label{installing-and-loading-packages}

Packages extend R. You typically:

\begin{enumerate}
\def\labelenumi{\arabic{enumi}.}
\tightlist
\item
  Install a package once per computer
\item
  Load the package each time you start a new R session
\end{enumerate}

\subsection{Install packages (one-time)}\label{install-packages-one-time}

Packages are installed using the \texttt{install.packages("package\_name")} function. For example, run the code below in the Console. If you see messages scrolling by, that is normal.

\begin{Shaded}
\begin{Highlighting}[]
\FunctionTok{install.packages}\NormalTok{(}\StringTok{"tidyverse"}\NormalTok{)}
\FunctionTok{install.packages}\NormalTok{(}\StringTok{"devtools"}\NormalTok{)}
\end{Highlighting}
\end{Shaded}

\subsection{Load packages (each session)}\label{load-packages-each-session}

Packages are loaded using the \texttt{library(package\_name)} function. You may see messages and/or warnings scroll by when loading package, which is normal behavior.

\begin{Shaded}
\begin{Highlighting}[]
\FunctionTok{library}\NormalTok{(tidyverse)}
\FunctionTok{library}\NormalTok{(devtools)}
\end{Highlighting}
\end{Shaded}

\section{Installing the MKT4320BGSU package}\label{installing-the-mkt4320bgsu-package}

The MKT4320BGSU package contains the functions and datasets used throughout this course.

\subsection{Install from GitHub}\label{install-from-github}

\begin{Shaded}
\begin{Highlighting}[]
\NormalTok{devtools}\SpecialCharTok{::}\FunctionTok{install\_github}\NormalTok{(}\StringTok{"jdmeyer73/MKT4320BGSU"}\NormalTok{)}
\end{Highlighting}
\end{Shaded}

\subsection{Load the package}\label{load-the-package}

\begin{Shaded}
\begin{Highlighting}[]
\FunctionTok{library}\NormalTok{(MKT4320BGSU)}
\end{Highlighting}
\end{Shaded}

\subsection{Verify everything works}\label{verify-everything-works}

If the chunks below run without errors, you are set.

\begin{Shaded}
\begin{Highlighting}[]
\FunctionTok{data}\NormalTok{(}\StringTok{"directmktg"}\NormalTok{)}
\FunctionTok{head}\NormalTok{(directmktg)}
\end{Highlighting}
\end{Shaded}

\begin{verbatim}
# A tibble: 6 x 5
    userid   age buy   gender salary
     <dbl> <dbl> <fct> <fct>   <dbl>
1 15624510    19 No    Male       19
2 15810944    35 No    Male       20
3 15668575    26 No    Female     43
4 15603246    27 No    Female     57
5 15804002    19 No    Male       76
6 15728773    27 No    Male       58
\end{verbatim}

\section{Getting help when you're stuck}\label{getting-help-when-youre-stuck}

These are the most useful help tools in R:

\begin{itemize}
\tightlist
\item
  \texttt{?function\_name} opens help for a function
\item
  \texttt{help.search("keyword")} searches help files
\item
  If an error happens, read the last line carefully (it often tells you what to fix)
\end{itemize}

Examples:

\begin{Shaded}
\begin{Highlighting}[]
\NormalTok{?mean}
\FunctionTok{help.search}\NormalTok{(}\StringTok{"logistic regression"}\NormalTok{)}
\end{Highlighting}
\end{Shaded}

\section{How you'll use this book}\label{how-youll-use-this-book}

Most weeks, your workflow will look like this:

\begin{enumerate}
\def\labelenumi{\arabic{enumi}.}
\tightlist
\item
  Read the relevant sections of the book
\item
  Open the lab file for the week
\item
  Run the example code
\item
  Modify the code to answer lab questions
\item
  Interpret results and write conclusions
\end{enumerate}

Errors are normal---debugging is part of learning analytics.

\section{What's next}\label{whats-next-1}

In the next chapter, we'll cover basic R fundamentals you'll use constantly, including:

\begin{itemize}
\tightlist
\item
  objects and assignment
\item
  vectors and data frames
\item
  common functions
\item
  importing data and basic data manipulation
\end{itemize}

\chapter{R Basics}\label{r-basics}

This chapter introduces the core R concepts you will use throughout the course. The goal is not to be exhaustive, but to give you enough familiarity with R's basic building blocks so that later analytical methods make sense.

By the end of this chapter, you should be comfortable creating objects, working with vectors and data frames, indexing data, and performing simple data manipulation tasks.

\begin{center}\rule{0.5\linewidth}{0.5pt}\end{center}

\section{Basics of R Commands}\label{basics-of-r-commands}

\begin{itemize}
\tightlist
\item
  R is case sensitive
\item
  When using the console, use the keyboard {~↑~} and {~↓~} arrow keys to easily cycle through previous commands typed.
\item
  When using the text editor (i.e., a script file) in the source pane, use the \texttt{Ctrl}+\texttt{Enter} (Windows/Linux) or \texttt{Cmd} + \texttt{Enter} (Mac) keyboard shortcut to submit a line of code directly to the console.

  \begin{itemize}
  \tightlist
  \item
    The entire line does \textbf{not} need to be highlighted; the cursor needs to be anywhere on the line to be submitted.
  \end{itemize}
\item
  When using the text editor/script file, the ``\#'' symbol signifies a comment

  \begin{itemize}
  \tightlist
  \item
    Everything after is ignored
  \item
    It can be on the same line:
  \end{itemize}
\end{itemize}

\begin{Shaded}
\begin{Highlighting}[]
\NormalTok{x }\OtherTok{\textless{}{-}} \DecValTok{100}   \CommentTok{\# Assign 100 to x}
\end{Highlighting}
\end{Shaded}

\begin{itemize}
\tightlist
\item
  It can be on separate lines:
\end{itemize}

\begin{Shaded}
\begin{Highlighting}[]
\CommentTok{\# Assign 100 to x}
\NormalTok{x }\OtherTok{\textless{}{-}} \DecValTok{100}
\end{Highlighting}
\end{Shaded}

\begin{center}\rule{0.5\linewidth}{0.5pt}\end{center}

\section{Operators}\label{operators}

Mathematical and logical operators are used frequently.

\begin{table}

\caption{\label{tab:roperators}R Operators}
\centering
\begin{tabular}[t]{l|l}
\hline
Description & Operator\\
\hline
\multicolumn{2}{l}{\textbf{Mathematical}}\\
\hline
\hspace{1em}addition & \$+\$\\
\hline
\hspace{1em}subtraction & \$-\$\\
\hline
\hspace{1em}multiplication & \$*\$\\
\hline
\hspace{1em}division & \$/\$\\
\hline
\hspace{1em}exponentiation & \textasciicircum{} or \$**\$\\
\hline
\multicolumn{2}{l}{\textbf{Logical}}\\
\hline
\hspace{1em}less than & <\\
\hline
\hspace{1em}less than or equal to & <=\\
\hline
\hspace{1em}greater than & >\\
\hline
\hspace{1em}greater than or equal to & >=\\
\hline
\hspace{1em}exactly equal to & ==\\
\hline
\hspace{1em}not equal to & !=\\
\hline
\hspace{1em}Not x & !x\\
\hline
\hspace{1em}x OR y & x|y\\
\hline
\hspace{1em}x AND y & x\&y\\
\hline
\hspace{1em}test if X is TRUE & isTRUE(x)\\
\hline
\end{tabular}
\end{table}

\begin{center}\rule{0.5\linewidth}{0.5pt}\end{center}

\section{Objects and assignment}\label{objects-and-assignment}

In R, almost everything you work with is an \textbf{object}. Objects store values, data, or results from functions.

You create objects using the assignment operator \texttt{\textless{}-}. In an RStudio script file or in the console, you can use a keyboard shortcut to produce the assignment operator. For Windows/Linux, \texttt{Alt} + \texttt{-}. For Mac, \texttt{Option} + \texttt{-}.

\begin{Shaded}
\begin{Highlighting}[]
\NormalTok{x }\OtherTok{\textless{}{-}} \DecValTok{10}
\NormalTok{x}
\end{Highlighting}
\end{Shaded}

\begin{verbatim}
[1] 10
\end{verbatim}

You can overwrite objects by assigning a new value:

\begin{Shaded}
\begin{Highlighting}[]
\NormalTok{x }\OtherTok{\textless{}{-}} \DecValTok{25}
\NormalTok{x}
\end{Highlighting}
\end{Shaded}

\begin{verbatim}
[1] 25
\end{verbatim}

As stated before, object names are case sensitive

\begin{Shaded}
\begin{Highlighting}[]
\NormalTok{x }\OtherTok{\textless{}{-}} \DecValTok{100}
\NormalTok{X}
\end{Highlighting}
\end{Shaded}

\begin{verbatim}
Error:
! object 'X' not found
\end{verbatim}

\begin{center}\rule{0.5\linewidth}{0.5pt}\end{center}

\section{Vectors}\label{vectors}

A \textbf{vector} is a collection of values of the same type.

\subsection{Creating vectors}\label{creating-vectors}

Vectors are aften created using the concatenate function, \texttt{c(item1,\ item2,\ ...)}

\begin{Shaded}
\begin{Highlighting}[]
\NormalTok{ages }\OtherTok{\textless{}{-}} \FunctionTok{c}\NormalTok{(}\DecValTok{18}\NormalTok{, }\DecValTok{21}\NormalTok{, }\DecValTok{25}\NormalTok{, }\DecValTok{30}\NormalTok{)}
\NormalTok{ages}
\end{Highlighting}
\end{Shaded}

\begin{verbatim}
[1] 18 21 25 30
\end{verbatim}

Common vector types include:

\begin{itemize}
\tightlist
\item
  Numeric
\item
  Character
\item
  Logical
\end{itemize}

\begin{Shaded}
\begin{Highlighting}[]
\NormalTok{names }\OtherTok{\textless{}{-}} \FunctionTok{c}\NormalTok{(}\StringTok{"Alex"}\NormalTok{, }\StringTok{"Jamie"}\NormalTok{, }\StringTok{"Taylor"}\NormalTok{, }\StringTok{"Pat"}\NormalTok{)}
\NormalTok{passed }\OtherTok{\textless{}{-}} \FunctionTok{c}\NormalTok{(}\ConstantTok{TRUE}\NormalTok{, }\ConstantTok{FALSE}\NormalTok{, }\ConstantTok{TRUE}\NormalTok{, }\ConstantTok{TRUE}\NormalTok{)}
\end{Highlighting}
\end{Shaded}

The class of a vector can be checked with the \texttt{class(ojbect\_name)} or \texttt{str(object\_name)} function.

\begin{Shaded}
\begin{Highlighting}[]
\FunctionTok{class}\NormalTok{(ages)}
\end{Highlighting}
\end{Shaded}

\begin{verbatim}
[1] "numeric"
\end{verbatim}

\begin{Shaded}
\begin{Highlighting}[]
\FunctionTok{str}\NormalTok{(names)}
\end{Highlighting}
\end{Shaded}

\begin{verbatim}
 chr [1:4] "Alex" "Jamie" "Taylor" "Pat"
\end{verbatim}

\begin{Shaded}
\begin{Highlighting}[]
\FunctionTok{class}\NormalTok{(passed)}
\end{Highlighting}
\end{Shaded}

\begin{verbatim}
[1] "logical"
\end{verbatim}

Vectors can only hold a single class/type of value. When multiple classes are included, the values are coerced to the most general type.

\begin{Shaded}
\begin{Highlighting}[]
\NormalTok{mixed }\OtherTok{\textless{}{-}} \FunctionTok{c}\NormalTok{(}\DecValTok{1}\NormalTok{, }\ConstantTok{FALSE}\NormalTok{, }\FloatTok{3.5}\NormalTok{, }\StringTok{"Hello!"}\NormalTok{)}
\NormalTok{mixed}
\end{Highlighting}
\end{Shaded}

\begin{verbatim}
[1] "1"      "FALSE"  "3.5"    "Hello!"
\end{verbatim}

\begin{Shaded}
\begin{Highlighting}[]
\FunctionTok{class}\NormalTok{(mixed)}
\end{Highlighting}
\end{Shaded}

\begin{verbatim}
[1] "character"
\end{verbatim}

The \texttt{c()} function can be used to add to existing vectors, or combine vectors. Type coercion will be applied as needed.

\begin{Shaded}
\begin{Highlighting}[]
\NormalTok{ages2 }\OtherTok{\textless{}{-}} \FunctionTok{c}\NormalTok{(ages, }\DecValTok{29}\NormalTok{, }\DecValTok{24}\NormalTok{)}
\NormalTok{ages}
\end{Highlighting}
\end{Shaded}

\begin{verbatim}
[1] 18 21 25 30
\end{verbatim}

\begin{Shaded}
\begin{Highlighting}[]
\NormalTok{ages2}
\end{Highlighting}
\end{Shaded}

\begin{verbatim}
[1] 18 21 25 30 29 24
\end{verbatim}

\begin{Shaded}
\begin{Highlighting}[]
\NormalTok{ages\_names }\OtherTok{\textless{}{-}} \FunctionTok{c}\NormalTok{(ages, names)}
\NormalTok{ages\_names}
\end{Highlighting}
\end{Shaded}

\begin{verbatim}
[1] "18"     "21"     "25"     "30"     "Alex"   "Jamie"  "Taylor" "Pat"   
\end{verbatim}

\begin{Shaded}
\begin{Highlighting}[]
\FunctionTok{class}\NormalTok{(ages\_names)}
\end{Highlighting}
\end{Shaded}

\begin{verbatim}
[1] "character"
\end{verbatim}

\subsection{Vectorized operations}\label{vectorized-operations}

R is vectorized, meaning operations apply to all elements at once.

\begin{Shaded}
\begin{Highlighting}[]
\NormalTok{ages}
\end{Highlighting}
\end{Shaded}

\begin{verbatim}
[1] 18 21 25 30
\end{verbatim}

\begin{Shaded}
\begin{Highlighting}[]
\NormalTok{ages }\SpecialCharTok{+} \DecValTok{1}
\end{Highlighting}
\end{Shaded}

\begin{verbatim}
[1] 19 22 26 31
\end{verbatim}

\begin{Shaded}
\begin{Highlighting}[]
\NormalTok{ages }\SpecialCharTok{*} \DecValTok{2}
\end{Highlighting}
\end{Shaded}

\begin{verbatim}
[1] 36 42 50 60
\end{verbatim}

\subsection{Vector Length}\label{vector-length}

The number of items in a vector can be checked with the \texttt{length(object\_name)} function, but can also be seen using the \texttt{str(object\_name)} function from earlier.

\begin{Shaded}
\begin{Highlighting}[]
\FunctionTok{length}\NormalTok{(ages2)}
\end{Highlighting}
\end{Shaded}

\begin{verbatim}
[1] 6
\end{verbatim}

\begin{Shaded}
\begin{Highlighting}[]
\FunctionTok{str}\NormalTok{(ages2)}
\end{Highlighting}
\end{Shaded}

\begin{verbatim}
 num [1:6] 18 21 25 30 29 24
\end{verbatim}

\begin{center}\rule{0.5\linewidth}{0.5pt}\end{center}

\section{Data frames}\label{data-frames}

A \textbf{data frame} is a table where:

\begin{itemize}
\tightlist
\item
  Each column is a variable
\item
  Each row is an observation
\end{itemize}

Data frames are the most common way to handle data sets and to provide data to statistical functions.

Data frames can be created using the \texttt{data.frame(objects)} function:

\begin{Shaded}
\begin{Highlighting}[]
\CommentTok{\# Creating a data frame all in one step}
\NormalTok{students }\OtherTok{\textless{}{-}} \FunctionTok{data.frame}\NormalTok{(}
  \AttributeTok{id =} \DecValTok{1}\SpecialCharTok{:}\DecValTok{4}\NormalTok{,}
  \AttributeTok{age =} \FunctionTok{c}\NormalTok{(}\DecValTok{18}\NormalTok{, }\DecValTok{21}\NormalTok{, }\DecValTok{25}\NormalTok{, }\DecValTok{30}\NormalTok{),}
  \AttributeTok{major =} \FunctionTok{c}\NormalTok{(}\StringTok{"MKT"}\NormalTok{, }\StringTok{"FIN"}\NormalTok{, }\StringTok{"MKT"}\NormalTok{, }\StringTok{"MKT"}\NormalTok{))}
\NormalTok{students}
\end{Highlighting}
\end{Shaded}

\begin{verbatim}
  id age major
1  1  18   MKT
2  2  21   FIN
3  3  25   MKT
4  4  30   MKT
\end{verbatim}

\begin{Shaded}
\begin{Highlighting}[]
\CommentTok{\# Creating a data frame by combining vectors}
\NormalTok{new\_students }\OtherTok{\textless{}{-}} \FunctionTok{data.frame}\NormalTok{(}\FunctionTok{c}\NormalTok{(names,ages,passed))}
\NormalTok{new\_students}
\end{Highlighting}
\end{Shaded}

\begin{verbatim}
   c.names..ages..passed.
1                    Alex
2                   Jamie
3                  Taylor
4                     Pat
5                      18
6                      21
7                      25
8                      30
9                    TRUE
10                  FALSE
11                   TRUE
12                   TRUE
\end{verbatim}

You will work with data frames constantly in this course.

\begin{center}\rule{0.5\linewidth}{0.5pt}\end{center}

\section{Indexing and sequencing}\label{indexing-and-sequencing}

Indexing is used to obtain particular elements of a data structure (vectors, matrices, data frames). Sequences are useful for indexing and loops.

\subsection{Indexing vectors}\label{indexing-vectors}

Use square brackets \texttt{{[}{]}} to select elements.

\begin{Shaded}
\begin{Highlighting}[]
\NormalTok{ages[}\DecValTok{1}\NormalTok{]}
\end{Highlighting}
\end{Shaded}

\begin{verbatim}
[1] 18
\end{verbatim}

\begin{Shaded}
\begin{Highlighting}[]
\NormalTok{ages[}\DecValTok{2}\SpecialCharTok{:}\DecValTok{4}\NormalTok{]}
\end{Highlighting}
\end{Shaded}

\begin{verbatim}
[1] 21 25 30
\end{verbatim}

Logical indexing is also common:

\begin{Shaded}
\begin{Highlighting}[]
\NormalTok{ages[ages }\SpecialCharTok{\textgreater{}} \DecValTok{21}\NormalTok{]}
\end{Highlighting}
\end{Shaded}

\begin{verbatim}
[1] 25 30
\end{verbatim}

\subsection{Sequencing}\label{sequencing}

Use the \texttt{\#:\#} coding or the \texttt{seq(from\ =\ ,\ to\ =\ ,\ by\ =\ )} function to create a sequence.

\begin{Shaded}
\begin{Highlighting}[]
\DecValTok{1}\SpecialCharTok{:}\DecValTok{10}
\end{Highlighting}
\end{Shaded}

\begin{verbatim}
 [1]  1  2  3  4  5  6  7  8  9 10
\end{verbatim}

\begin{Shaded}
\begin{Highlighting}[]
\FunctionTok{seq}\NormalTok{(}\AttributeTok{from =} \DecValTok{0}\NormalTok{, }\AttributeTok{to =} \DecValTok{1}\NormalTok{, }\AttributeTok{by =} \FloatTok{0.2}\NormalTok{)}
\end{Highlighting}
\end{Shaded}

\begin{verbatim}
[1] 0.0 0.2 0.4 0.6 0.8 1.0
\end{verbatim}

\begin{Shaded}
\begin{Highlighting}[]
\FunctionTok{seq}\NormalTok{(}\DecValTok{0}\NormalTok{,}\DecValTok{100}\NormalTok{,}\DecValTok{10}\NormalTok{)}
\end{Highlighting}
\end{Shaded}

\begin{verbatim}
 [1]   0  10  20  30  40  50  60  70  80  90 100
\end{verbatim}

\begin{center}\rule{0.5\linewidth}{0.5pt}\end{center}

\section{Common functions}\label{common-functions}

Functions take inputs (arguments) and return outputs.

Examples of commonly used functions:

\begin{Shaded}
\begin{Highlighting}[]
\FunctionTok{mean}\NormalTok{(ages)}
\end{Highlighting}
\end{Shaded}

\begin{verbatim}
[1] 23.5
\end{verbatim}

\begin{Shaded}
\begin{Highlighting}[]
\FunctionTok{min}\NormalTok{(ages)}
\end{Highlighting}
\end{Shaded}

\begin{verbatim}
[1] 18
\end{verbatim}

\begin{Shaded}
\begin{Highlighting}[]
\FunctionTok{max}\NormalTok{(ages)}
\end{Highlighting}
\end{Shaded}

\begin{verbatim}
[1] 30
\end{verbatim}

\begin{Shaded}
\begin{Highlighting}[]
\FunctionTok{summary}\NormalTok{(ages)}
\end{Highlighting}
\end{Shaded}

\begin{verbatim}
   Min. 1st Qu.  Median    Mean 3rd Qu.    Max. 
  18.00   20.25   23.00   23.50   26.25   30.00 
\end{verbatim}

To learn about a function:

\begin{Shaded}
\begin{Highlighting}[]
\NormalTok{?mean}
\end{Highlighting}
\end{Shaded}

\begin{center}\rule{0.5\linewidth}{0.5pt}\end{center}

\section{Missing (and Other Interesting) Values}\label{missing-and-other-interesting-values}

In R, missing values are assigned a special constant, {NA}.

{NA} is not a character value, but a type of its own. Any math performed on a value of {NA} becomes {NA}.

\begin{Shaded}
\begin{Highlighting}[]
\NormalTok{ages\_missing }\OtherTok{\textless{}{-}} \FunctionTok{c}\NormalTok{(ages, }\ConstantTok{NA}\NormalTok{, }\ConstantTok{NA}\NormalTok{)}
\NormalTok{ages\_missing}
\end{Highlighting}
\end{Shaded}

\begin{verbatim}
[1] 18 21 25 30 NA NA
\end{verbatim}

\begin{Shaded}
\begin{Highlighting}[]
\FunctionTok{mean}\NormalTok{(ages\_missing)}
\end{Highlighting}
\end{Shaded}

\begin{verbatim}
[1] NA
\end{verbatim}

However, many commands contain a option, \texttt{na.rm=TRUE}, to ignore {NA} data when performing the function.

\begin{Shaded}
\begin{Highlighting}[]
\FunctionTok{mean}\NormalTok{(ages\_missing, }\AttributeTok{na.rm=}\ConstantTok{TRUE}\NormalTok{)}
\end{Highlighting}
\end{Shaded}

\begin{verbatim}
[1] 23.5
\end{verbatim}

R also has special types for infinity, {Inf}, and undefined numbers (i.e., ``not a number''), {NaN}. To see this in action, take the natural log, \texttt{log()}, of certain numbers. Notice that R provides a warning when the {NaN} is found.

\begin{Shaded}
\begin{Highlighting}[]
\FunctionTok{log}\NormalTok{(}\SpecialCharTok{{-}}\DecValTok{1}\NormalTok{)   }\CommentTok{\# Not a number}
\end{Highlighting}
\end{Shaded}

\begin{verbatim}
Warning in log(-1): NaNs produced
\end{verbatim}

\begin{verbatim}
[1] NaN
\end{verbatim}

\begin{Shaded}
\begin{Highlighting}[]
\FunctionTok{log}\NormalTok{(}\DecValTok{0}\NormalTok{)    }\CommentTok{\# Infinity}
\end{Highlighting}
\end{Shaded}

\begin{verbatim}
[1] -Inf
\end{verbatim}

\begin{center}\rule{0.5\linewidth}{0.5pt}\end{center}

\section{Factors}\label{factors}

Character data can be converted into nominal \textbf{factors} using the \texttt{as.factor(object\_name)} function. Each unique character value will be a level of the factor. Behind the scenes, R stores the values as integers, with a separate list of labels. When the data type is set as a factor, R knows how to handle it appropriately in the model. The levels can be accessed with the \texttt{levels(object\_name)} function.

\begin{Shaded}
\begin{Highlighting}[]
\NormalTok{school\_year }\OtherTok{\textless{}{-}} \FunctionTok{c}\NormalTok{(}\StringTok{"JR"}\NormalTok{, }\StringTok{"SR"}\NormalTok{, }\StringTok{"SR"}\NormalTok{, }\StringTok{"SO"}\NormalTok{, }\StringTok{"FR"}\NormalTok{, }\StringTok{"JR"}\NormalTok{)}
\FunctionTok{class}\NormalTok{(school\_year)}
\end{Highlighting}
\end{Shaded}

\begin{verbatim}
[1] "character"
\end{verbatim}

\begin{Shaded}
\begin{Highlighting}[]
\NormalTok{school\_year }\OtherTok{\textless{}{-}} \FunctionTok{as.factor}\NormalTok{(school\_year)}
\FunctionTok{str}\NormalTok{(school\_year)}
\end{Highlighting}
\end{Shaded}

\begin{verbatim}
 Factor w/ 4 levels "FR","JR","SO",..: 2 4 4 3 1 2
\end{verbatim}

\begin{Shaded}
\begin{Highlighting}[]
\FunctionTok{levels}\NormalTok{(school\_year)}
\end{Highlighting}
\end{Shaded}

\begin{verbatim}
[1] "FR" "JR" "SO" "SR"
\end{verbatim}

\begin{center}\rule{0.5\linewidth}{0.5pt}\end{center}

\section{What's next}\label{whats-next-2}

In the next chapter, we focus on \textbf{using functions effectively in R}.

You will learn how to:

\begin{itemize}
\tightlist
\item
  pass arguments to functions,
\item
  work with positional versus named arguments,
\item
  understand default values, and
\item
  read function documentation more efficiently.
\end{itemize}

These skills are essential for working with both built-in R functions and the custom functions provided in the \textbf{MKT4320BGSU} package.

\chapter{Functions in R}\label{functions-in-r}

Most of what you do in R involves \textbf{using functions}. A function takes inputs (called \textbf{arguments}), performs an operation, and returns an output.

You have already used several functions, such as \texttt{mean()}, \texttt{summary()}, and \texttt{seq()}.

\begin{Shaded}
\begin{Highlighting}[]
\FunctionTok{mean}\NormalTok{(ages)}
\end{Highlighting}
\end{Shaded}

\begin{verbatim}
[1] 23.5
\end{verbatim}

\begin{Shaded}
\begin{Highlighting}[]
\FunctionTok{summary}\NormalTok{(ages)}
\end{Highlighting}
\end{Shaded}

\begin{verbatim}
   Min. 1st Qu.  Median    Mean 3rd Qu.    Max. 
  18.00   20.25   23.00   23.50   26.25   30.00 
\end{verbatim}

\section{Function arguments}\label{function-arguments}

Functions often require one or more arguments. For example:

\begin{Shaded}
\begin{Highlighting}[]
\FunctionTok{mean}\NormalTok{(ages)}
\end{Highlighting}
\end{Shaded}

\begin{verbatim}
[1] 23.5
\end{verbatim}

Here, \texttt{ages} is passed to the function \texttt{mean()} as its first argument.

Some functions require multiple arguments:

\begin{Shaded}
\begin{Highlighting}[]
\FunctionTok{seq}\NormalTok{(}\AttributeTok{from =} \DecValTok{0}\NormalTok{, }\AttributeTok{to =} \DecValTok{1}\NormalTok{, }\AttributeTok{by =} \FloatTok{0.2}\NormalTok{)}
\end{Highlighting}
\end{Shaded}

\begin{verbatim}
[1] 0.0 0.2 0.4 0.6 0.8 1.0
\end{verbatim}

\begin{center}\rule{0.5\linewidth}{0.5pt}\end{center}

\section{Positional vs named arguments}\label{positional-vs-named-arguments}

Arguments can be passed to a function in \textbf{two ways}.

\subsection{Positional arguments}\label{positional-arguments}

If arguments are supplied \textbf{in the correct order}, you do not need to name them.

\begin{Shaded}
\begin{Highlighting}[]
\FunctionTok{seq}\NormalTok{(}\DecValTok{0}\NormalTok{, }\DecValTok{1}\NormalTok{, }\FloatTok{0.2}\NormalTok{)}
\end{Highlighting}
\end{Shaded}

\begin{verbatim}
[1] 0.0 0.2 0.4 0.6 0.8 1.0
\end{verbatim}

This works because R knows the expected order of arguments for \texttt{seq()}:

\begin{enumerate}
\def\labelenumi{\arabic{enumi}.}
\tightlist
\item
  \texttt{from}
\item
  \texttt{to}
\item
  \texttt{by}
\end{enumerate}

Positional arguments are concise, but they can make code harder to read and easier to misuse.

\begin{center}\rule{0.5\linewidth}{0.5pt}\end{center}

\subsection{Named arguments (recommended)}\label{named-arguments-recommended}

You can explicitly name arguments using \texttt{argument\ =\ value}.

\begin{Shaded}
\begin{Highlighting}[]
\FunctionTok{seq}\NormalTok{(}\AttributeTok{from =} \DecValTok{0}\NormalTok{, }\AttributeTok{to =} \DecValTok{1}\NormalTok{, }\AttributeTok{by =} \FloatTok{0.2}\NormalTok{)}
\end{Highlighting}
\end{Shaded}

\begin{verbatim}
[1] 0.0 0.2 0.4 0.6 0.8 1.0
\end{verbatim}

Advantages of named arguments:

\begin{itemize}
\tightlist
\item
  Code is easier to read
\item
  Order does not matter
\item
  Fewer mistakes when functions have many arguments
\end{itemize}

For example, this works even though the order is different:

\begin{Shaded}
\begin{Highlighting}[]
\FunctionTok{seq}\NormalTok{(}\AttributeTok{by =} \FloatTok{0.2}\NormalTok{, }\AttributeTok{to =} \DecValTok{1}\NormalTok{, }\AttributeTok{from =} \DecValTok{0}\NormalTok{)}
\end{Highlighting}
\end{Shaded}

\begin{verbatim}
[1] 0.0 0.2 0.4 0.6 0.8 1.0
\end{verbatim}

In this course, \textbf{naming arguments is recommended}, especially for complex functions.

\begin{center}\rule{0.5\linewidth}{0.5pt}\end{center}

\section{Default argument values}\label{default-argument-values}

Many function arguments have \textbf{default values}. If you do not specify them, R uses the default.

Example:

\begin{Shaded}
\begin{Highlighting}[]
\FunctionTok{mean}\NormalTok{(ages)}
\end{Highlighting}
\end{Shaded}

\begin{verbatim}
[1] 23.5
\end{verbatim}

The function \texttt{mean()} has optional arguments such as \texttt{na.rm}, which defaults to \texttt{FALSE}.

\begin{Shaded}
\begin{Highlighting}[]
\FunctionTok{mean}\NormalTok{(ages, }\AttributeTok{na.rm =} \ConstantTok{TRUE}\NormalTok{)}
\end{Highlighting}
\end{Shaded}

\begin{verbatim}
[1] 23.5
\end{verbatim}

You only need to specify arguments when:

\begin{itemize}
\tightlist
\item
  You want a non-default behavior
\item
  The function requires the argument
\end{itemize}

\begin{center}\rule{0.5\linewidth}{0.5pt}\end{center}

\section{Mixing positional and named arguments}\label{mixing-positional-and-named-arguments}

You can mix positional and named arguments, but \textbf{positional arguments must come first}.

This is valid

\begin{Shaded}
\begin{Highlighting}[]
\FunctionTok{mean}\NormalTok{(ages, }\AttributeTok{na.rm =} \ConstantTok{TRUE}\NormalTok{)}
\end{Highlighting}
\end{Shaded}

\begin{verbatim}
[1] 23.5
\end{verbatim}

This will result in an error.

\begin{Shaded}
\begin{Highlighting}[]
\FunctionTok{mean}\NormalTok{(}\AttributeTok{na.rm =} \ConstantTok{TRUE}\NormalTok{, ages)}
\end{Highlighting}
\end{Shaded}

\begin{verbatim}
[1] 23.5
\end{verbatim}

A good rule of thumb:

\begin{itemize}
\tightlist
\item
  Use positional arguments for the \textbf{main input}
\item
  Use named arguments for \textbf{options and settings}
\end{itemize}

\begin{center}\rule{0.5\linewidth}{0.5pt}\end{center}

\section{Viewing function documentation}\label{viewing-function-documentation}

To understand how a function works and what arguments it accepts, use the help system.

\begin{Shaded}
\begin{Highlighting}[]
\NormalTok{?mean}
\end{Highlighting}
\end{Shaded}

The help page shows:

\begin{itemize}
\tightlist
\item
  What the function does
\item
  Required and optional arguments
\item
  Default values
\item
  Examples
\end{itemize}

When in doubt, \textbf{read the argument list first}.

\begin{center}\rule{0.5\linewidth}{0.5pt}\end{center}

\section{Common mistakes to avoid}\label{common-mistakes-to-avoid}

\begin{itemize}
\tightlist
\item
  Forgetting parentheses. This refers to the function itself, not the result.:
\end{itemize}

\begin{Shaded}
\begin{Highlighting}[]
\NormalTok{mean}
\end{Highlighting}
\end{Shaded}

\begin{verbatim}
function (x, ...) 
UseMethod("mean")
<bytecode: 0x0000029c291e1a90>
<environment: namespace:base>
\end{verbatim}

\begin{itemize}
\tightlist
\item
  Misspelling argument names:
\end{itemize}

\begin{Shaded}
\begin{Highlighting}[]
\FunctionTok{mean}\NormalTok{(ages, }\AttributeTok{na\_remove =} \ConstantTok{TRUE}\NormalTok{)}
\end{Highlighting}
\end{Shaded}

\begin{verbatim}
[1] 23.5
\end{verbatim}

\begin{itemize}
\tightlist
\item
  Assuming argument order without checking documentation
\end{itemize}

\begin{center}\rule{0.5\linewidth}{0.5pt}\end{center}

\section{Key takeaway}\label{key-takeaway}

You do \textbf{not} need to memorize every function or argument. Instead, focus on:

\begin{itemize}
\tightlist
\item
  Understanding what a function expects as input
\item
  Knowing when to use named arguments
\item
  Reading help files when unsure
\end{itemize}

These habits will make your R code more readable, more reliable, and easier to debug.

\begin{center}\rule{0.5\linewidth}{0.5pt}\end{center}

\section{What's next}\label{whats-next-3}

In the next chapter, we turn to \textbf{working with course data}.

You will learn how to:

\begin{itemize}
\tightlist
\item
  inspect and understand real datasets used in the course,
\item
  distinguish between numeric and factor variables,
\item
  identify missing or problematic values, and
\item
  perform basic data manipulation using the \texttt{dplyr} package.
\end{itemize}

These skills will allow you to move from isolated examples to analyzing complete datasets and will serve as the foundation for descriptive analysis, visualization, and modeling later in the semester.

\chapter{Working with Data}\label{working-with-data}

This chapter focuses on working with real datasets used throughout the course. You will learn how course data are made available, how to inspect and understand variables, and how to perform basic data transformations using the \texttt{dplyr} package.

The goal is to help you move from small, isolated examples to working confidently with full datasets.

\begin{center}\rule{0.5\linewidth}{0.5pt}\end{center}

\section{Course data vs importing data}\label{course-data-vs-importing-data}

In this course, many datasets are provided directly through the \textbf{MKT4320BGSU} package. These datasets are accessed using the \texttt{data()} function. For the next two chapters, we'll be using the \texttt{directmktg} dataset.

\begin{Shaded}
\begin{Highlighting}[]
\FunctionTok{data}\NormalTok{(}\StringTok{"directmktg"}\NormalTok{)}
\end{Highlighting}
\end{Shaded}

Using \texttt{data()} has several advantages:

\begin{itemize}
\tightlist
\item
  Everyone is working with the same dataset
\item
  No file paths are required
\item
  Fewer import-related errors
\end{itemize}

When possible, labs will rely on datasets loaded with \texttt{data()}.

\subsection{Importing external data}\label{importing-external-data}

In some cases, you may work with your own data files (e.g., CSV files).

\begin{Shaded}
\begin{Highlighting}[]
\NormalTok{mydata }\OtherTok{\textless{}{-}} \FunctionTok{read.csv}\NormalTok{(}\StringTok{"mydata.csv"}\NormalTok{)}
\end{Highlighting}
\end{Shaded}

Imported data behave the same way as course datasets once they are loaded into R. The key difference is how they enter your workspace.

As a general rule:

\begin{itemize}
\tightlist
\item
  Use \texttt{data()} for course-provided datasets
\item
  Use \texttt{read.csv()} (or similar functions) for your own files
\end{itemize}

\begin{center}\rule{0.5\linewidth}{0.5pt}\end{center}

\section{Inspecting datasets and variables}\label{inspecting-datasets-and-variables}

Before analyzing data, it is critical to understand what is in the dataset.

\subsection{Viewing the data}\label{viewing-the-data}

The \texttt{head(data\_object)} first few rows (the default is \texttt{n=6}) and helps you understand the structure of the data. There is a similar function, \texttt{tail()}, that looks at the last \texttt{n} rows.

\begin{Shaded}
\begin{Highlighting}[]
\FunctionTok{head}\NormalTok{(directmktg)}
\end{Highlighting}
\end{Shaded}

\begin{verbatim}
    userid age buy gender salary
1 15624510  19  No   Male     19
2 15810944  35  No   Male     20
3 15668575  26  No Female     43
4 15603246  27  No Female     57
5 15804002  19  No   Male     76
6 15728773  27  No   Male     58
\end{verbatim}

\begin{Shaded}
\begin{Highlighting}[]
\FunctionTok{head}\NormalTok{(directmktg, }\AttributeTok{n=}\DecValTok{5}\NormalTok{)}
\end{Highlighting}
\end{Shaded}

\begin{verbatim}
    userid age buy gender salary
1 15624510  19  No   Male     19
2 15810944  35  No   Male     20
3 15668575  26  No Female     43
4 15603246  27  No Female     57
5 15804002  19  No   Male     76
\end{verbatim}

\begin{Shaded}
\begin{Highlighting}[]
\FunctionTok{tail}\NormalTok{(directmktg, }\AttributeTok{n=}\DecValTok{5}\NormalTok{)}
\end{Highlighting}
\end{Shaded}

\begin{verbatim}
      userid age buy gender salary
396 15691863  46 Yes Female     41
397 15706071  51 Yes   Male     23
398 15654296  50 Yes Female     20
399 15755018  36  No   Male     33
400 15594041  49 Yes Female     36
\end{verbatim}

To see the entire data object, you can use the \texttt{View(data\_object)} function, which will open the data in a spreadsheet-like format in the Source pane.

\begin{Shaded}
\begin{Highlighting}[]
\FunctionTok{View}\NormalTok{(directmktg)}
\end{Highlighting}
\end{Shaded}

\subsection{Dataset structure}\label{dataset-structure}

The \texttt{str(object)} function is one of the most important commands in R. When used with a dataset object, it shows:

\begin{itemize}
\tightlist
\item
  Variable names
\item
  Variable types (numeric, factor, character)
\item
  A preview of values
\end{itemize}

\begin{Shaded}
\begin{Highlighting}[]
\FunctionTok{str}\NormalTok{(directmktg)}
\end{Highlighting}
\end{Shaded}

\begin{verbatim}
'data.frame':   400 obs. of  5 variables:
 $ userid: num  15624510 15810944 15668575 15603246 15804002 ...
 $ age   : num  19 35 26 27 19 27 27 32 25 35 ...
 $ buy   : Factor w/ 2 levels "No","Yes": 1 1 1 1 1 1 1 2 1 1 ...
 $ gender: Factor w/ 2 levels "Male","Female": 1 1 2 2 1 1 2 2 1 2 ...
 $ salary: num  19 20 43 57 76 58 84 150 33 65 ...
\end{verbatim}

\subsection{Variable summaries}\label{variable-summaries}

When used on the entire dataset, the \texttt{summary(object\_name)} function provides a summary of all variables. The output depends on the variable type:

\begin{itemize}
\tightlist
\item
  Numeric variables: min, max, mean, quartiles
\item
  Factor variables: counts by level
\end{itemize}

\begin{Shaded}
\begin{Highlighting}[]
\FunctionTok{summary}\NormalTok{(directmktg)}
\end{Highlighting}
\end{Shaded}

\begin{verbatim}
     userid              age         buy         gender        salary      
 Min.   :15566689   Min.   :18.00   No :257   Male  :196   Min.   : 15.00  
 1st Qu.:15626764   1st Qu.:29.75   Yes:143   Female:204   1st Qu.: 43.00  
 Median :15694342   Median :37.00                          Median : 70.00  
 Mean   :15691540   Mean   :37.66                          Mean   : 69.74  
 3rd Qu.:15750363   3rd Qu.:46.00                          3rd Qu.: 88.00  
 Max.   :15815236   Max.   :60.00                          Max.   :150.00  
\end{verbatim}

\begin{center}\rule{0.5\linewidth}{0.5pt}\end{center}

\section{Data frames, variables, and indexing}\label{data-frames-variables-and-indexing}

A \textbf{data frame} is a table of rows and columns. In R, you will often need to reference a specific variable (column) or select a subset of rows/columns.

\subsection{Accessing a variable (a column)}\label{accessing-a-variable-a-column}

Use the \texttt{\$} operator to access a column by name.

\begin{Shaded}
\begin{Highlighting}[]
\NormalTok{directmktg}\SpecialCharTok{$}\NormalTok{age}
\end{Highlighting}
\end{Shaded}

\begin{verbatim}
  [1] 19 35 26 27 19 27 27 32 25 35 26 26 20 32 18 29 47 45 46 48 45 47 48 45 46 47 49 47 29 31 31 27 21 28 27 35 33 30
 [39] 26 27 27 33 35 30 28 23 25 27 30 31 24 18 29 35 27 24 23 28 22 32 27 25 23 32 59 24 24 23 22 31 25 24 20 33 32 34
 [77] 18 22 28 26 30 39 20 35 30 31 24 28 26 35 22 30 26 29 29 35 35 28 35 28 27 28 32 33 19 21 26 27 26 38 39 37 38 37
[115] 42 40 35 36 40 41 36 37 40 35 41 39 42 26 30 26 31 33 30 21 28 23 20 30 28 19 19 18 35 30 34 24 27 41 29 20 26 41
[153] 31 36 40 31 46 29 26 32 32 25 37 35 33 18 22 35 29 29 21 34 26 34 34 23 35 25 24 31 26 31 32 33 33 31 20 33 35 28
[191] 24 19 29 19 28 34 30 20 26 35 35 49 39 41 58 47 55 52 40 46 48 52 59 35 47 60 49 40 46 59 41 35 37 60 35 37 36 56
[229] 40 42 35 39 40 49 38 46 40 37 46 53 42 38 50 56 41 51 35 57 41 35 44 37 48 37 50 52 41 40 58 45 35 36 55 35 48 42
[267] 40 37 47 40 43 59 60 39 57 57 38 49 52 50 59 35 37 52 48 37 37 48 41 37 39 49 55 37 35 36 42 43 45 46 58 48 37 37
[305] 40 42 51 47 36 38 42 39 38 49 39 39 54 35 45 36 52 53 41 48 48 41 41 42 36 47 38 48 42 40 57 36 58 35 38 39 53 35
[343] 38 47 47 41 53 54 39 38 38 37 42 37 36 60 54 41 40 42 43 53 47 42 42 59 58 46 38 54 60 60 39 59 37 46 46 42 41 58
[381] 42 48 44 49 57 56 49 39 47 48 48 47 45 60 39 46 51 50 36 49
\end{verbatim}

A column is often a vector, which means you can use vector functions on it.

\begin{Shaded}
\begin{Highlighting}[]
\FunctionTok{mean}\NormalTok{(directmktg}\SpecialCharTok{$}\NormalTok{age)}
\end{Highlighting}
\end{Shaded}

\begin{verbatim}
[1] 37.655
\end{verbatim}

\begin{Shaded}
\begin{Highlighting}[]
\FunctionTok{summary}\NormalTok{(directmktg}\SpecialCharTok{$}\NormalTok{age)}
\end{Highlighting}
\end{Shaded}

\begin{verbatim}
   Min. 1st Qu.  Median    Mean 3rd Qu.    Max. 
  18.00   29.75   37.00   37.66   46.00   60.00 
\end{verbatim}

\subsection{Indexing a data frame}\label{indexing-a-data-frame}

Data frames can be indexed (or subsetted) using:

\begin{itemize}
\tightlist
\item
  \texttt{data{[}rows,\ cols{]}}
\item
  leave rows blank to keep all rows: \texttt{data{[},\ cols{]}}
\item
  leave cols blank to keep all columns: \texttt{data{[}rows,\ {]}}
\end{itemize}

In addition, sequencing can be used to get multiple rows and/or columns. By default, subsetting a single column returns a vector. To retain it as a data frame, use the \texttt{drop\ =\ FALSE} argument.

\begin{Shaded}
\begin{Highlighting}[]
\NormalTok{directmktg[}\DecValTok{1}\NormalTok{, ]               }\CommentTok{\# first row, all columns}
\NormalTok{directmktg[}\DecValTok{1}\SpecialCharTok{:}\DecValTok{5}\NormalTok{, ]             }\CommentTok{\# first 5 rows, all columns}
\NormalTok{directmktg[, }\DecValTok{2}\NormalTok{]               }\CommentTok{\# all rows, column 2 only (age), as a vector}
\NormalTok{directmktg[, }\DecValTok{2}\NormalTok{, drop}\OtherTok{=}\ConstantTok{FALSE}\NormalTok{]   }\CommentTok{\# all rows, column 2 only (age), as a data frame}
\end{Highlighting}
\end{Shaded}

You can also access index columns using the column name columns. in two other ways. First, if you know the column number, you can using \texttt{{[}{[}\ {]}{]}} (useful when column names are stored in an object).

\begin{Shaded}
\begin{Highlighting}[]
\NormalTok{directmktg[}\DecValTok{1}\NormalTok{, }\StringTok{"age"}\NormalTok{]                 }\CommentTok{\# first row, age column}
\NormalTok{directmktg[}\DecValTok{1}\SpecialCharTok{:}\DecValTok{5}\NormalTok{, }\FunctionTok{c}\NormalTok{(}\StringTok{"age"}\NormalTok{, }\StringTok{"gender"}\NormalTok{)]  }\CommentTok{\# first 5 rows, age and gender columns}
\end{Highlighting}
\end{Shaded}

\subsection{Filtering rows using a condition (base R)}\label{filtering-rows-using-a-condition-base-r}

You can filter rows by using a logical condition inside the row index.

\begin{Shaded}
\begin{Highlighting}[]
\NormalTok{directmktg[directmktg}\SpecialCharTok{$}\NormalTok{age }\SpecialCharTok{\textgreater{}=} \DecValTok{55}\NormalTok{, ]}
\end{Highlighting}
\end{Shaded}

\begin{verbatim}
      userid age buy gender salary
65  15605000  59  No Female     83
205 15660866  58 Yes Female    101
207 15654230  55 Yes Female    130
213 15707596  59  No Female     42
216 15779529  60 Yes Female    108
220 15732987  59 Yes   Male    143
224 15593715  60 Yes   Male    102
228 15685346  56 Yes   Male    133
244 15769596  56 Yes Female    104
248 15775590  57 Yes Female    122
259 15569641  58 Yes Female     95
263 15672821  55 Yes Female    125
272 15688172  59 Yes Female     76
273 15791373  60 Yes   Male     42
275 15692819  57 Yes Female     26
276 15727467  57 Yes   Male     74
281 15609669  59 Yes Female     88
293 15625395  55 Yes   Male     39
301 15736397  58 Yes Female     38
335 15814553  57 Yes   Male     60
337 15664907  58 Yes   Male    144
356 15606472  60 Yes   Male     34
366 15807525  59 Yes Female     29
367 15574372  58 Yes Female     47
371 15611430  60 Yes Female     46
372 15774744  60 Yes   Male     83
374 15708791  59 Yes   Male    130
380 15749381  58 Yes Female     23
385 15806901  57 Yes Female     33
386 15775335  56 Yes   Male     60
394 15635893  60 Yes   Male     42
\end{verbatim}

This keeps only rows where \texttt{age\ \textgreater{}=\ 55}.

\begin{center}\rule{0.5\linewidth}{0.5pt}\end{center}

\subsection{Specific variable summaries (base R)}\label{specific-variable-summaries-base-r}

You can use variable naming or indexing to get specific variable summaries using the \texttt{summary()} function.

\begin{Shaded}
\begin{Highlighting}[]
\FunctionTok{summary}\NormalTok{(directmktg}\SpecialCharTok{$}\NormalTok{age)}
\end{Highlighting}
\end{Shaded}

\begin{verbatim}
   Min. 1st Qu.  Median    Mean 3rd Qu.    Max. 
  18.00   29.75   37.00   37.66   46.00   60.00 
\end{verbatim}

\begin{Shaded}
\begin{Highlighting}[]
\FunctionTok{summary}\NormalTok{(directmktg[,}\DecValTok{2}\SpecialCharTok{:}\DecValTok{3}\NormalTok{])}
\end{Highlighting}
\end{Shaded}

\begin{verbatim}
      age         buy     
 Min.   :18.00   No :257  
 1st Qu.:29.75   Yes:143  
 Median :37.00            
 Mean   :37.66            
 3rd Qu.:46.00            
 Max.   :60.00            
\end{verbatim}

\begin{center}\rule{0.5\linewidth}{0.5pt}\end{center}

\section{Data transformations (base R)}\label{data-transformations-base-r}

Data transformation refers to modifying or creating variables so they are more useful for analysis. This includes:

\begin{itemize}
\tightlist
\item
  creating new variables
\item
  recoding variables
\end{itemize}

Base R can do all of these. Even if you prefer \texttt{dplyr} (covered later), it is helpful to understand what is happening ``under the hood.''

\subsection{Creating a new variable}\label{creating-a-new-variable}

This example creates a \textbf{new variable} in the data frame using base R. The expression on the right-hand side,
\texttt{directmktg\$age\ /\ 10}, is evaluated first. Because \texttt{directmktg\$age} is a vector, the division is applied to \textbf{each observation}, and the result is a new vector of the same length.

Using the assignment operator \texttt{\textless{}-}, this new vector is stored as a column named \texttt{age10} in the \texttt{directmktg} data frame.

\begin{Shaded}
\begin{Highlighting}[]
\NormalTok{directmktg}\SpecialCharTok{$}\NormalTok{age10 }\OtherTok{\textless{}{-}}\NormalTok{ directmktg}\SpecialCharTok{$}\NormalTok{age }\SpecialCharTok{/} \DecValTok{10}
\FunctionTok{head}\NormalTok{(directmktg, }\AttributeTok{n=}\DecValTok{5}\NormalTok{)}
\end{Highlighting}
\end{Shaded}

\begin{verbatim}
    userid age buy gender salary age10
1 15624510  19  No   Male     19   1.9
2 15810944  35  No   Male     20   3.5
3 15668575  26  No Female     43   2.6
4 15603246  27  No Female     57   2.7
5 15804002  19  No   Male     76   1.9
\end{verbatim}

\subsection{\texorpdfstring{Recoding with \texttt{ifelse()}}{Recoding with ifelse()}}\label{recoding-with-ifelse}

A common transformation is converting a categorical outcome into a numeric indicator.

\begin{Shaded}
\begin{Highlighting}[]
\NormalTok{directmktg}\SpecialCharTok{$}\NormalTok{buy\_binary }\OtherTok{\textless{}{-}} \FunctionTok{ifelse}\NormalTok{(directmktg}\SpecialCharTok{$}\NormalTok{buy }\SpecialCharTok{==} \StringTok{"Yes"}\NormalTok{, }\DecValTok{1}\NormalTok{, }\DecValTok{0}\NormalTok{)}
\FunctionTok{table}\NormalTok{(directmktg}\SpecialCharTok{$}\NormalTok{buy, directmktg}\SpecialCharTok{$}\NormalTok{buy\_binary)}
\end{Highlighting}
\end{Shaded}

\begin{verbatim}
     
        0   1
  No  257   0
  Yes   0 143
\end{verbatim}

Another common recoding is converting a numeric value into categories, which can be done with nested \texttt{ifelse()} statements.

\begin{Shaded}
\begin{Highlighting}[]
\NormalTok{directmktg}\SpecialCharTok{$}\NormalTok{salary\_cat }\OtherTok{\textless{}{-}} \FunctionTok{ifelse}\NormalTok{(directmktg}\SpecialCharTok{$}\NormalTok{salary }\SpecialCharTok{\textless{}=}\DecValTok{50}\NormalTok{, }\StringTok{"Low"}\NormalTok{,}
                               \FunctionTok{ifelse}\NormalTok{(directmktg}\SpecialCharTok{$}\NormalTok{salary }\SpecialCharTok{\textless{}=}\DecValTok{80}\NormalTok{, }\StringTok{"Med"}\NormalTok{, }\StringTok{"High"}\NormalTok{))}
\FunctionTok{head}\NormalTok{(directmktg[,}\FunctionTok{c}\NormalTok{(}\StringTok{"salary"}\NormalTok{,}\StringTok{"salary\_cat"}\NormalTok{)], }\DecValTok{10}\NormalTok{)}
\end{Highlighting}
\end{Shaded}

\begin{verbatim}
   salary salary_cat
1      19        Low
2      20        Low
3      43        Low
4      57        Med
5      76        Med
6      58        Med
7      84       High
8     150       High
9      33        Low
10     65        Med
\end{verbatim}

\begin{center}\rule{0.5\linewidth}{0.5pt}\end{center}

\section{What's next}\label{whats-next-4}

In the next chapter, we introduce the \textbf{\texttt{dplyr} package} for data manipulation. While the tasks performed with \texttt{dplyr} are similar to those you have already seen using base R, \texttt{dplyr} provides a more consistent and expressive way to work with data. These tools will be used extensively in later chapters for descriptive analysis, visualization, and modeling.

\chapter{\texorpdfstring{\emph{dplyr} Package}{dplyr Package}}\label{dplyr-package}

In this course, data transformations are primarily performed using the \texttt{dplyr} package (pronounced DEE ply er). This package makes data manipulation easier and more intuitive (for most). \texttt{dplyr} is built around the five main ``verbs'' shown below that make up a majority of data manipulation. However, there are other functions that dplyr uses to also help with data manipulation.

\begin{itemize}
\tightlist
\item
  \texttt{select} is used to subset columns
\item
  \texttt{filter} is used to subset rows
\item
  \texttt{mutate} is used to add new columns based on calculations (usually with other columns)
\item
  \texttt{summarise} is use to perform summary calculations (e.g., mean, max, etc.) on data set
\item
  \texttt{group\_by} is used to group rows of a data frame with the same value in specified columns
\end{itemize}

In addition, \texttt{dplyr} uses the pipe, \texttt{\%\textgreater{}\%}, to string together a series of functions. Think of functions strung together as upstream and downstream functions. The function to the left of \texttt{\%\textgreater{}\%} is the upstream function, while the function to the right is the downstream function.

By default, the downstream function assumes the value coming from the upstream function is the first argument in its function. Therefore, the first argument can be omitted. If the downstream function needs to use the value from the upstream function assigned to a different argument, a \texttt{.} is simply put in the position of that argument

\begin{center}\rule{0.5\linewidth}{0.5pt}\end{center}

\section{\texorpdfstring{The \emph{dplyr} workflow}{The dplyr workflow}}\label{the-dplyr-workflow}

The \texttt{dplyr} package is designed to make data manipulation clear and readable.

\begin{Shaded}
\begin{Highlighting}[]
\FunctionTok{library}\NormalTok{(dplyr)}
\end{Highlighting}
\end{Shaded}

A typical \texttt{dplyr} workflow:

\begin{enumerate}
\def\labelenumi{\arabic{enumi}.}
\tightlist
\item
  Start with a data frame
\item
  Apply a sequence of transformation verbs
\item
  Save or display the result
\end{enumerate}

\begin{center}\rule{0.5\linewidth}{0.5pt}\end{center}

\section{Selecting variables}\label{selecting-variables}

Use \texttt{select()} to keep only the variables you need. This does not modify the original dataset unless you save the result.

\begin{Shaded}
\begin{Highlighting}[]
\NormalTok{directmktg }\SpecialCharTok{\%\textgreater{}\%}
  \FunctionTok{select}\NormalTok{(userid, age, gender) }\SpecialCharTok{\%\textgreater{}\%}
  \FunctionTok{head}\NormalTok{()}
\end{Highlighting}
\end{Shaded}

\begin{verbatim}
    userid age gender
1 15624510  19   Male
2 15810944  35   Male
3 15668575  26 Female
4 15603246  27 Female
5 15804002  19   Male
6 15728773  27   Male
\end{verbatim}

``Negative'' selection can also be done by using the \texttt{-} (minus sign) before a variable name or a vector of variable names.

\begin{Shaded}
\begin{Highlighting}[]
\NormalTok{directmktg }\SpecialCharTok{\%\textgreater{}\%}
  \FunctionTok{select}\NormalTok{(}\SpecialCharTok{{-}}\NormalTok{gender) }\SpecialCharTok{\%\textgreater{}\%}
  \FunctionTok{head}\NormalTok{()}
\end{Highlighting}
\end{Shaded}

\begin{verbatim}
    userid age buy salary age10 buy_binary salary_cat
1 15624510  19  No     19   1.9          0        Low
2 15810944  35  No     20   3.5          0        Low
3 15668575  26  No     43   2.6          0        Low
4 15603246  27  No     57   2.7          0        Med
5 15804002  19  No     76   1.9          0        Med
6 15728773  27  No     58   2.7          0        Med
\end{verbatim}

\begin{Shaded}
\begin{Highlighting}[]
\NormalTok{directmktg }\SpecialCharTok{\%\textgreater{}\%}
  \FunctionTok{select}\NormalTok{(}\SpecialCharTok{{-}}\FunctionTok{c}\NormalTok{(age,gender)) }\SpecialCharTok{\%\textgreater{}\%}
  \FunctionTok{head}\NormalTok{()}
\end{Highlighting}
\end{Shaded}

\begin{verbatim}
    userid buy salary age10 buy_binary salary_cat
1 15624510  No     19   1.9          0        Low
2 15810944  No     20   3.5          0        Low
3 15668575  No     43   2.6          0        Low
4 15603246  No     57   2.7          0        Med
5 15804002  No     76   1.9          0        Med
6 15728773  No     58   2.7          0        Med
\end{verbatim}

\begin{center}\rule{0.5\linewidth}{0.5pt}\end{center}

\section{Filtering observations}\label{filtering-observations}

Use \texttt{filter()} to keep rows that meet certain conditions. (Note: the \texttt{nrow(object\_name)} from base R provides the number of rows in the data frame).

\begin{Shaded}
\begin{Highlighting}[]
\FunctionTok{nrow}\NormalTok{(directmktg)}
\end{Highlighting}
\end{Shaded}

\begin{verbatim}
[1] 400
\end{verbatim}

\begin{Shaded}
\begin{Highlighting}[]
\NormalTok{directmktg }\SpecialCharTok{\%\textgreater{}\%}
  \FunctionTok{filter}\NormalTok{(age }\SpecialCharTok{\textgreater{}=} \DecValTok{35}\NormalTok{) }\SpecialCharTok{\%\textgreater{}\%}
  \FunctionTok{nrow}\NormalTok{()}
\end{Highlighting}
\end{Shaded}

\begin{verbatim}
[1] 254
\end{verbatim}

Multiple conditions can be combined:

\begin{Shaded}
\begin{Highlighting}[]
\NormalTok{directmktg }\SpecialCharTok{\%\textgreater{}\%}
  \FunctionTok{filter}\NormalTok{(age }\SpecialCharTok{\textgreater{}=} \DecValTok{35}\NormalTok{, }
\NormalTok{         gender }\SpecialCharTok{==} \StringTok{"Male"}\NormalTok{) }\SpecialCharTok{\%\textgreater{}\%}
  \FunctionTok{nrow}\NormalTok{()}
\end{Highlighting}
\end{Shaded}

\begin{verbatim}
[1] 124
\end{verbatim}

\begin{center}\rule{0.5\linewidth}{0.5pt}\end{center}

\section{Creating new variables}\label{creating-new-variables}

Use \texttt{mutate()} to create or transform variables. New variables are added to the data frame.

\begin{Shaded}
\begin{Highlighting}[]
\NormalTok{directmktg }\SpecialCharTok{\%\textgreater{}\%}
  \FunctionTok{select}\NormalTok{(userid, age) }\SpecialCharTok{\%\textgreater{}\%}
  \FunctionTok{mutate}\NormalTok{(}\AttributeTok{age10 =}\NormalTok{ age }\SpecialCharTok{/} \DecValTok{10}\NormalTok{) }\SpecialCharTok{\%\textgreater{}\%}
  \FunctionTok{head}\NormalTok{()}
\end{Highlighting}
\end{Shaded}

\begin{verbatim}
    userid age age10
1 15624510  19   1.9
2 15810944  35   3.5
3 15668575  26   2.6
4 15603246  27   2.7
5 15804002  19   1.9
6 15728773  27   2.7
\end{verbatim}

\begin{center}\rule{0.5\linewidth}{0.5pt}\end{center}

\section{\texorpdfstring{Summaries with \texttt{summarise()}}{Summaries with summarise()}}\label{summaries-with-summarise}

The \texttt{summarise()} function is used to compute \textbf{summary statistics} from a data frame.
It can be used \textbf{with or without grouping}.

When \texttt{summarise()} is used \textbf{without} \texttt{group\_by()}, it computes summaries over the
entire dataset.

\begin{Shaded}
\begin{Highlighting}[]
\NormalTok{directmktg }\SpecialCharTok{\%\textgreater{}\%}
  \FunctionTok{summarise}\NormalTok{(}
    \AttributeTok{n =} \FunctionTok{n}\NormalTok{(),}
    \AttributeTok{mean\_age =} \FunctionTok{mean}\NormalTok{(age),}
    \AttributeTok{buy\_rate =} \FunctionTok{mean}\NormalTok{(buy }\SpecialCharTok{==} \StringTok{"Yes"}\NormalTok{)}
\NormalTok{  )}
\end{Highlighting}
\end{Shaded}

\begin{verbatim}
    n mean_age buy_rate
1 400   37.655   0.3575
\end{verbatim}

Here's what this code is doing:

\begin{itemize}
\tightlist
\item
  \texttt{n()} counts the total number of observations in the dataset
\item
  \texttt{mean(age)} computes the overall average age
\item
  \texttt{mean(buy\ ==\ "Yes")} computes the overall purchase rate
\end{itemize}

The result is a data frame with \textbf{one row}, where each column represents a summary
statistic for the full dataset.

\begin{center}\rule{0.5\linewidth}{0.5pt}\end{center}

\section{\texorpdfstring{Grouped summaries with \texttt{group\_by()} and \texttt{summarise()}}{Grouped summaries with group\_by() and summarise()}}\label{grouped-summaries-with-group_by-and-summarise}

Often, you want to compute summaries \textbf{separately for different groups}, such as
customer segments or demographic categories.

In \texttt{dplyr}, this is done by combining \texttt{group\_by()} with \texttt{summarize()}.

\begin{Shaded}
\begin{Highlighting}[]
\NormalTok{directmktg }\SpecialCharTok{\%\textgreater{}\%}
  \FunctionTok{group\_by}\NormalTok{(gender)}
\end{Highlighting}
\end{Shaded}

\begin{verbatim}
# A tibble: 400 x 8
# Groups:   gender [2]
     userid   age buy   gender salary age10 buy_binary salary_cat
      <dbl> <dbl> <fct> <fct>   <dbl> <dbl>      <dbl> <chr>     
 1 15624510    19 No    Male       19   1.9          0 Low       
 2 15810944    35 No    Male       20   3.5          0 Low       
 3 15668575    26 No    Female     43   2.6          0 Low       
 4 15603246    27 No    Female     57   2.7          0 Med       
 5 15804002    19 No    Male       76   1.9          0 Med       
 6 15728773    27 No    Male       58   2.7          0 Med       
 7 15598044    27 No    Female     84   2.7          0 High      
 8 15694829    32 Yes   Female    150   3.2          1 High      
 9 15600575    25 No    Male       33   2.5          0 Low       
10 15727311    35 No    Female     65   3.5          0 Med       
# i 390 more rows
\end{verbatim}

The \texttt{group\_by()} function does not change the data values. Instead, it tells R how
the data should be \textbf{temporarily divided into groups} for the next operation.

At this point, no calculations have been performed. Once the data are grouped, \texttt{summarise()} computes statistics \textbf{within each group}.

\begin{Shaded}
\begin{Highlighting}[]
\NormalTok{directmktg }\SpecialCharTok{\%\textgreater{}\%}
  \FunctionTok{group\_by}\NormalTok{(gender) }\SpecialCharTok{\%\textgreater{}\%}
  \FunctionTok{summarise}\NormalTok{(}
    \AttributeTok{n =} \FunctionTok{n}\NormalTok{(),}
    \AttributeTok{mean\_age =} \FunctionTok{mean}\NormalTok{(age),}
    \AttributeTok{buy\_rate =} \FunctionTok{mean}\NormalTok{(buy }\SpecialCharTok{==} \StringTok{"Yes"}\NormalTok{)}
\NormalTok{  )}
\end{Highlighting}
\end{Shaded}

\begin{verbatim}
# A tibble: 2 x 4
  gender     n mean_age buy_rate
  <fct>  <int>    <dbl>    <dbl>
1 Male     196     36.9    0.337
2 Female   204     38.4    0.377
\end{verbatim}

Step by step:

\begin{itemize}
\tightlist
\item
  \texttt{group\_by(gender)} splits the data into separate groups based on gender
\item
  \texttt{n()} counts observations within each group
\item
  \texttt{mean(age)} computes the average age within each group
\item
  \texttt{mean(buy\ ==\ "Yes")} computes the purchase rate within each group
\end{itemize}

The result is a data frame with \textbf{one row per group} and \textbf{one column per summary statistic}.

After \texttt{summarise()} runs, the grouping structure is automatically dropped, so the
result behaves like a regular data frame.

\begin{center}\rule{0.5\linewidth}{0.5pt}\end{center}

\section{\texorpdfstring{Connecting \texttt{summarize()} to base R}{Connecting summarize() to base R}}\label{connecting-summarize-to-base-r}

Conceptually, grouped summaries in \texttt{dplyr} perform the same task as a multi-step
process in base R:

\begin{enumerate}
\def\labelenumi{\arabic{enumi}.}
\tightlist
\item
  Split the data into groups
\item
  Compute summary statistics for each group
\item
  Combine the results into a table
\end{enumerate}

For example, in base R you might compute group means using functions such as
\texttt{aggregate()} or by manually subsetting the data.

The advantage of \texttt{group\_by()} and \texttt{summarise()} is that these steps are expressed
\textbf{explicitly and readably}, making your data transformations easier to follow,
debug, and modify.

\begin{center}\rule{0.5\linewidth}{0.5pt}\end{center}

\section{Combining transformations}\label{combining-transformations}

As you've seen, one of the main advantages of \texttt{dplyr} is that multiple steps can be chained together.

\begin{Shaded}
\begin{Highlighting}[]
\NormalTok{directmktg\_clean }\OtherTok{\textless{}{-}}\NormalTok{ directmktg }\SpecialCharTok{\%\textgreater{}\%}
  \FunctionTok{filter}\NormalTok{(age }\SpecialCharTok{\textgreater{}=} \DecValTok{35}\NormalTok{) }\SpecialCharTok{\%\textgreater{}\%}
  \FunctionTok{mutate}\NormalTok{(}\AttributeTok{age10 =}\NormalTok{ age }\SpecialCharTok{/} \DecValTok{10}\NormalTok{,}
         \AttributeTok{buy\_binary =} \FunctionTok{ifelse}\NormalTok{(buy }\SpecialCharTok{==} \StringTok{"Yes"}\NormalTok{, }\DecValTok{1}\NormalTok{, }\DecValTok{0}\NormalTok{)) }\SpecialCharTok{\%\textgreater{}\%}
  \FunctionTok{select}\NormalTok{(userid, age10, gender, buy\_binary)}

\FunctionTok{head}\NormalTok{(directmktg\_clean)}
\end{Highlighting}
\end{Shaded}

\begin{verbatim}
    userid age10 gender buy_binary
1 15810944   3.5   Male          0
2 15727311   3.5 Female          0
3 15733883   4.7   Male          1
4 15617482   4.5   Male          1
5 15704583   4.6   Male          1
6 15621083   4.8 Female          1
\end{verbatim}

This approach keeps data preparation transparent and reproducible.

\begin{center}\rule{0.5\linewidth}{0.5pt}\end{center}

\section{Key takeaway}\label{key-takeaway-1}

Before any modeling or visualization, you should be able to:
- load data reliably,
- inspect variable types and values,
- identify missing or problematic data, and
- transform data into a usable analytical form.

These steps are essential for sound marketing analytics.

\begin{center}\rule{0.5\linewidth}{0.5pt}\end{center}

\section{What's next}\label{whats-next-5}

In the next chapter, we will use cleaned and well-understood data to perform \textbf{descriptive analysis}, including frequency tables, crosstabs, measures of central tendency and dispersion, and correlation.

\chapter{Descriptive Analysis}\label{descriptive-analysis}

Descriptive analysis summarizes and helps you understand your data \textbf{numerically}. In marketing analytics, it is often your first ``sanity check'' before modeling or visualization.

In this chapter you will learn how to:

\begin{itemize}
\tightlist
\item
  inspect a dataset and its variables,
\item
  create frequency tables and crosstabs,
\item
  compute measures of central tendency and dispersion,
\item
  and compute and interpret correlations.
\end{itemize}

For this chapter, we'll be using the \texttt{airlinesat\_small} dataset from the \texttt{MKT4320BGSU}.

\begin{Shaded}
\begin{Highlighting}[]
\FunctionTok{data}\NormalTok{(airlinesat\_small)}
\end{Highlighting}
\end{Shaded}

\begin{center}\rule{0.5\linewidth}{0.5pt}\end{center}

\section{Descriptive Analysis}\label{descriptive-analysis-1}

A quick descriptive workflow often looks like:

\begin{enumerate}
\def\labelenumi{\arabic{enumi})}
\tightlist
\item
  \textbf{Confirm the data structure} (rows/columns, variable types)
\item
  \textbf{Look for missing values and odd ranges}
\item
  \textbf{Summarize key variables} (categorical → counts; numeric → mean/SD, etc.)
\item
  \textbf{Compare segments} (e.g., by treatment, gender, region, etc.)
\end{enumerate}

\subsection{Inspecting the dataset}\label{inspecting-the-dataset}

The following functions are useful for inspecting the dataset.

\begin{itemize}
\tightlist
\item
  \texttt{str()} shows each variables' type, displays factor levels if present, and gives a compact preview of values for each variable
\item
  \texttt{dim()} returns the number of rows and columsn to quickly tell you sample size and number of variables
\item
  \texttt{names()} provides all column names, which helps with selecting variables for writing code or formulas
\item
  \texttt{head()} gives the first six rows of the dataset to allow for visual inspection of raw values
\end{itemize}

\begin{Shaded}
\begin{Highlighting}[]
\CommentTok{\# Take a look at the data frame}
\FunctionTok{str}\NormalTok{(airlinesat\_small)}
\end{Highlighting}
\end{Shaded}

\begin{verbatim}
'data.frame':   1065 obs. of  13 variables:
 $ age           : num  30 55 56 43 44 40 39 41 33 51 ...
 $ country       : Factor w/ 5 levels "at","ch","de",..: 2 2 2 4 2 2 2 2 2 3 ...
 $ flight_class  : Factor w/ 3 levels "Business","Economy",..: 2 1 2 2 1 3 2 1 2 1 ...
 $ flight_latest : Factor w/ 6 levels "within the last 12 months",..: 4 3 5 3 6 5 6 3 3 4 ...
 $ flight_purpose: Factor w/ 2 levels "Business","Leisure": 2 1 1 2 1 2 1 1 2 1 ...
 $ flight_type   : Factor w/ 2 levels "Domestic","International": 1 2 1 1 2 2 1 2 1 2 ...
 $ gender        : Factor w/ 2 levels "female","male": 2 2 1 1 1 2 2 2 2 2 ...
 $ language      : Factor w/ 3 levels "English","French",..: 2 1 1 2 1 3 2 2 2 3 ...
 $ nflights      : num  2 6 8 7 25 16 35 9 3 4 ...
 $ status        : Factor w/ 3 levels "Blue","Gold",..: 1 2 1 1 2 2 1 2 1 2 ...
 $ nps           : num  6 10 8 8 6 7 8 7 8 8 ...
 $ overall_sat   : num  2 6 2 4 2 4 4 4 4 3 ...
 $ reputation    : num  3 6 4 6 5 3 3 4 2 4 ...
\end{verbatim}

\begin{Shaded}
\begin{Highlighting}[]
\CommentTok{\# Dimensions (rows, columns)}
\FunctionTok{dim}\NormalTok{(airlinesat\_small)}
\end{Highlighting}
\end{Shaded}

\begin{verbatim}
[1] 1065   13
\end{verbatim}

\begin{Shaded}
\begin{Highlighting}[]
\CommentTok{\# Variable names}
\FunctionTok{names}\NormalTok{(airlinesat\_small)}
\end{Highlighting}
\end{Shaded}

\begin{verbatim}
 [1] "age"            "country"        "flight_class"   "flight_latest"  "flight_purpose" "flight_type"   
 [7] "gender"         "language"       "nflights"       "status"         "nps"            "overall_sat"   
[13] "reputation"    
\end{verbatim}

\begin{Shaded}
\begin{Highlighting}[]
\CommentTok{\# First few rows}
\FunctionTok{head}\NormalTok{(airlinesat\_small)}
\end{Highlighting}
\end{Shaded}

\begin{verbatim}
  age country flight_class            flight_latest flight_purpose   flight_type gender language nflights status nps
1  30      ch      Economy within the last 6 months        Leisure      Domestic   male   French        2   Blue   6
2  55      ch     Business within the last 3 months       Business International   male  English        6   Gold  10
3  56      ch      Economy    within the last month       Business      Domestic female  English        8   Blue   8
4  43      fr      Economy within the last 3 months        Leisure      Domestic female   French        7   Blue   8
5  44      ch     Business     within the last week       Business International female  English       25   Gold   6
6  40      ch        First    within the last month        Leisure International   male   German       16   Gold   7
  overall_sat reputation
1           2          3
2           6          6
3           2          4
4           4          6
5           2          5
6           4          3
\end{verbatim}

\subsection{Missing values (quick checks)}\label{missing-values-quick-checks}

It is also important to check for missing values, because they can have an adverse effect on some analyses. The \texttt{is.na()} function is often used for this, which returns a \texttt{TRUE} if a value is \texttt{NA}. Ultimately, this will tell which values in the data are missing to help decide how to handle them.

\begin{Shaded}
\begin{Highlighting}[]
\CommentTok{\# Missing values by variable}
\FunctionTok{colSums}\NormalTok{(}\FunctionTok{is.na}\NormalTok{(airlinesat\_small))}
\end{Highlighting}
\end{Shaded}

\begin{verbatim}
           age        country   flight_class  flight_latest flight_purpose    flight_type         gender       language 
             0              0              0              0              0              0              0              0 
      nflights         status            nps    overall_sat     reputation 
             0              0              0              0              0 
\end{verbatim}

\begin{Shaded}
\begin{Highlighting}[]
\CommentTok{\# Total missing values in the dataset}
\FunctionTok{sum}\NormalTok{(}\FunctionTok{is.na}\NormalTok{(airlinesat\_small))}
\end{Highlighting}
\end{Shaded}

\begin{verbatim}
[1] 0
\end{verbatim}

\begin{center}\rule{0.5\linewidth}{0.5pt}\end{center}

\section{Frequency Tables}\label{frequency-tables}

Frequency tables summarize \textbf{categorical} variables (and sometimes binned numeric variables).

\subsection{One-way frequency table}\label{one-way-frequency-table}

To create a frequency table, use the \texttt{table()} function in R. Counts are often converted into proportions or rates to make results easier to interpret and compare.

In R, the \texttt{proportions()} function is used to convert frequency tables into proportions. Multiply the table by 100 to get percentages.

\begin{Shaded}
\begin{Highlighting}[]
\CommentTok{\# Frequency table for a categorical variable}
\FunctionTok{table}\NormalTok{(airlinesat\_small}\SpecialCharTok{$}\NormalTok{gender)}
\end{Highlighting}
\end{Shaded}

\begin{verbatim}

female   male 
   280    785 
\end{verbatim}

\begin{Shaded}
\begin{Highlighting}[]
\CommentTok{\# Add proportions (shares)}
\FunctionTok{proportions}\NormalTok{(}\FunctionTok{table}\NormalTok{(airlinesat\_small}\SpecialCharTok{$}\NormalTok{gender))}
\end{Highlighting}
\end{Shaded}

\begin{verbatim}

   female      male 
0.2629108 0.7370892 
\end{verbatim}

\begin{Shaded}
\begin{Highlighting}[]
\CommentTok{\# Add percentages}
\DecValTok{100} \SpecialCharTok{*} \FunctionTok{proportions}\NormalTok{(}\FunctionTok{table}\NormalTok{(airlinesat\_small}\SpecialCharTok{$}\NormalTok{gender))}
\end{Highlighting}
\end{Shaded}

\begin{verbatim}

  female     male 
26.29108 73.70892 
\end{verbatim}

\begin{center}\rule{0.5\linewidth}{0.5pt}\end{center}

\section{Crosstabs}\label{crosstabs}

A crosstab is a frequency table for \textbf{two categorical variables}. Crosstabs are used frequently in marketing to compare segments (e.g., purchase by gender).

\subsection{Base R}\label{base-r}

Base R does not do a great job of easily creating crosstabs and testing for independence of the two variables. Instead, a multistep process is required:

\begin{itemize}
\item
  Create the two-way frequency table using the \texttt{table(rowvar,\ colvar)} function and assign it to a separate object
\item
  Display the two-way freq table by just using the table name

\begin{Shaded}
\begin{Highlighting}[]
\CommentTok{\# Counts}
\NormalTok{ct }\OtherTok{\textless{}{-}} \FunctionTok{table}\NormalTok{(airlinesat\_small}\SpecialCharTok{$}\NormalTok{flight\_class, airlinesat\_small}\SpecialCharTok{$}\NormalTok{gender)}
\NormalTok{ct}
\end{Highlighting}
\end{Shaded}

\begin{verbatim}

           female male
  Business     39  146
  Economy     239  626
  First         2   13
\end{verbatim}
\item
  Use the function \texttt{proportions(tablename,\ margin)} on the newly created object to get column, row, or total percentages

  \begin{itemize}
  \item
    \texttt{proportions(tablename)} gives total percentages
  \item
    \texttt{proportions(tablename,\ 1)} gives row percentages
  \item
    \texttt{proportions(tablename,\ 2)} gives column percentages

\begin{Shaded}
\begin{Highlighting}[]
\FunctionTok{proportions}\NormalTok{(ct)             }\CommentTok{\# Total proportions: what pprportion of all respondents}
\end{Highlighting}
\end{Shaded}

\begin{verbatim}

                female        male
  Business 0.036619718 0.137089202
  Economy  0.224413146 0.587793427
  First    0.001877934 0.012206573
\end{verbatim}

\begin{Shaded}
\begin{Highlighting}[]
\FunctionTok{proportions}\NormalTok{(ct, }\AttributeTok{margin=}\DecValTok{1}\NormalTok{)   }\CommentTok{\# Row proportions: how are genders distributed within a row}
\end{Highlighting}
\end{Shaded}

\begin{verbatim}

              female      male
  Business 0.2108108 0.7891892
  Economy  0.2763006 0.7236994
  First    0.1333333 0.8666667
\end{verbatim}

\begin{Shaded}
\begin{Highlighting}[]
\FunctionTok{proportions}\NormalTok{(ct, }\AttributeTok{margin=}\DecValTok{2}\NormalTok{)   }\CommentTok{\# Column proportions: how is th outcome distributed within each gender}
\end{Highlighting}
\end{Shaded}

\begin{verbatim}

                female        male
  Business 0.139285714 0.185987261
  Economy  0.853571429 0.797452229
  First    0.007142857 0.016560510
\end{verbatim}
  \end{itemize}
\item
  Use the function \texttt{chisq.test(tablename)} on the newly created object to run the test of independence

\begin{Shaded}
\begin{Highlighting}[]
\FunctionTok{chisq.test}\NormalTok{(ct)}
\end{Highlighting}
\end{Shaded}

\begin{verbatim}
Warning in chisq.test(ct): Chi-squared approximation may be incorrect
\end{verbatim}

\begin{verbatim}

    Pearson's Chi-squared test

data:  ct
X-squared = 4.6912, df = 2, p-value = 0.09579
\end{verbatim}
\end{itemize}

\subsection{\texorpdfstring{Using package \emph{sjPlot}}{Using package sjPlot}}\label{using-package-sjplot}

The \textbf{sjPlot} package can print crosstabs with nicer formatting. Use the function \texttt{tab\_xtab(var.row=,\ var.col=,\ show.col.prc=TRUE)} to get a standard crosstab with column percentages and the chi-square test of independence.

\begin{Shaded}
\begin{Highlighting}[]
\FunctionTok{tab\_xtab}\NormalTok{(airlinesat\_small}\SpecialCharTok{$}\NormalTok{flight\_class,}
\NormalTok{         airlinesat\_small}\SpecialCharTok{$}\NormalTok{gender,}
         \AttributeTok{show.col.prc =} \ConstantTok{TRUE}\NormalTok{)}
\end{Highlighting}
\end{Shaded}

flight\_class

gender

Total

female

male

Business

{39}{13.9~\%}

{146}{18.6~\%}

{185}{17.4~\%}

Economy

{239}{85.4~\%}

{626}{79.7~\%}

{865}{81.2~\%}

First

{2}{0.7~\%}

{13}{1.7~\%}

{15}{1.4~\%}

Total

{280}{100~\%}

{785}{100~\%}

{1065}{100~\%}

χ2=4.691 · df=2 · Cramer's V=0.066 · Fisher's p=0.095

\begin{center}\rule{0.5\linewidth}{0.5pt}\end{center}

\section{Measures of Central Tendency and Dispersion}\label{measures-of-central-tendency-and-dispersion}

For \textbf{numeric} variables, the most common summaries are:

\begin{itemize}
\tightlist
\item
  Central tendency: mean, median, mode
\item
  Dispersion: variance, standard deviation, IQR, range
\end{itemize}

\subsection{Base R}\label{base-r-1}

Any individual summary statistic can be easily calculated using Base R with functions such as:

\begin{itemize}
\tightlist
\item
  \texttt{mean(var)} for mean
\item
  \texttt{sd(var)} for standard deviation
\item
  \texttt{quantile(var,\ .percentile)} for percentiles (e.g., `.50' would be median)
\end{itemize}

For summary statistics except for standard deviation, the \texttt{summary(object)} function can be used, where object can be a single variable or an entire data frame

\begin{Shaded}
\begin{Highlighting}[]
\CommentTok{\# Base R summary (min, quartiles, median, mean, max)}
\FunctionTok{summary}\NormalTok{(airlinesat\_small}\SpecialCharTok{$}\NormalTok{age)}
\end{Highlighting}
\end{Shaded}

\begin{verbatim}
   Min. 1st Qu.  Median    Mean 3rd Qu.    Max. 
  19.00   42.00   50.00   50.42   58.00  101.00 
\end{verbatim}

\begin{Shaded}
\begin{Highlighting}[]
\FunctionTok{summary}\NormalTok{(airlinesat\_small)}
\end{Highlighting}
\end{Shaded}

\begin{verbatim}
      age         country    flight_class                   flight_latest  flight_purpose        flight_type 
 Min.   : 19.00   at:108   Business:185   within the last 12 months:139   Business:525    Domestic     :558  
 1st Qu.: 42.00   ch: 66   Economy :865   within the last 2 days   : 57   Leisure :540    International:507  
 Median : 50.00   de:695   First   : 15   within the last 3 months :296                                      
 Mean   : 50.42   fr:  1                  within the last 6 months :187                                      
 3rd Qu.: 58.00   us:195                  within the last month    :253                                      
 Max.   :101.00                           within the last week     :133                                      
    gender       language      nflights         status         nps         overall_sat     reputation   
 female:280   English:233   Min.   :  1.00   Blue  :677   Min.   : 1.00   Min.   :1.00   Min.   :1.000  
 male  :785   French : 10   1st Qu.:  4.00   Gold  :143   1st Qu.: 6.00   1st Qu.:3.00   1st Qu.:4.000  
              German :822   Median :  8.00   Silver:245   Median : 8.00   Median :4.00   Median :5.000  
                            Mean   : 13.42                Mean   : 7.52   Mean   :3.74   Mean   :4.841  
                            3rd Qu.: 16.00                3rd Qu.: 9.00   3rd Qu.:5.00   3rd Qu.:6.000  
                            Max.   :457.00                Max.   :11.00   Max.   :7.00   Max.   :7.000  
\end{verbatim}

\begin{Shaded}
\begin{Highlighting}[]
\CommentTok{\# Mean and SD (remove missing values if present)}
\FunctionTok{mean}\NormalTok{(airlinesat\_small}\SpecialCharTok{$}\NormalTok{age, }\AttributeTok{na.rm =} \ConstantTok{TRUE}\NormalTok{)}
\end{Highlighting}
\end{Shaded}

\begin{verbatim}
[1] 50.41972
\end{verbatim}

\begin{Shaded}
\begin{Highlighting}[]
\FunctionTok{sd}\NormalTok{(airlinesat\_small}\SpecialCharTok{$}\NormalTok{age, }\AttributeTok{na.rm =} \ConstantTok{TRUE}\NormalTok{)}
\end{Highlighting}
\end{Shaded}

\begin{verbatim}
[1] 12.27464
\end{verbatim}

If you want a single, custom summary:

\begin{Shaded}
\begin{Highlighting}[]
\NormalTok{x }\OtherTok{\textless{}{-}}\NormalTok{ airlinesat\_small}\SpecialCharTok{$}\NormalTok{age}

\FunctionTok{c}\NormalTok{(}\AttributeTok{n =} \FunctionTok{sum}\NormalTok{(}\SpecialCharTok{!}\FunctionTok{is.na}\NormalTok{(x)),}
  \AttributeTok{mean =} \FunctionTok{mean}\NormalTok{(x, }\AttributeTok{na.rm =} \ConstantTok{TRUE}\NormalTok{),}
  \AttributeTok{sd =} \FunctionTok{sd}\NormalTok{(x, }\AttributeTok{na.rm =} \ConstantTok{TRUE}\NormalTok{),}
  \AttributeTok{median =} \FunctionTok{median}\NormalTok{(x, }\AttributeTok{na.rm =} \ConstantTok{TRUE}\NormalTok{),}
  \AttributeTok{iqr =} \FunctionTok{IQR}\NormalTok{(x, }\AttributeTok{na.rm =} \ConstantTok{TRUE}\NormalTok{),}
  \AttributeTok{min =} \FunctionTok{min}\NormalTok{(x, }\AttributeTok{na.rm =} \ConstantTok{TRUE}\NormalTok{),}
  \AttributeTok{max =} \FunctionTok{max}\NormalTok{(x, }\AttributeTok{na.rm =} \ConstantTok{TRUE}\NormalTok{))}
\end{Highlighting}
\end{Shaded}

\begin{verbatim}
         n       mean         sd     median        iqr        min        max 
1065.00000   50.41972   12.27464   50.00000   16.00000   19.00000  101.00000 
\end{verbatim}

\subsection{\texorpdfstring{Using package \emph{dplyr}}{Using package dplyr}}\label{using-package-dplyr}

With \textbf{dplyr}, you can summarize many variables and/or do summaries by groups.

Overall summaries:

\begin{Shaded}
\begin{Highlighting}[]
\NormalTok{airlinesat\_small }\SpecialCharTok{\%\textgreater{}\%}
  \FunctionTok{summarise}\NormalTok{(}\AttributeTok{n =} \FunctionTok{n}\NormalTok{(),}
            \AttributeTok{mean\_age =} \FunctionTok{mean}\NormalTok{(age, }\AttributeTok{na.rm =} \ConstantTok{TRUE}\NormalTok{),}
            \AttributeTok{sd\_age =} \FunctionTok{sd}\NormalTok{(age, }\AttributeTok{na.rm =} \ConstantTok{TRUE}\NormalTok{),}
            \AttributeTok{median\_age =} \FunctionTok{median}\NormalTok{(age, }\AttributeTok{na.rm =} \ConstantTok{TRUE}\NormalTok{),}
            \AttributeTok{iqr\_age =} \FunctionTok{IQR}\NormalTok{(age, }\AttributeTok{na.rm =} \ConstantTok{TRUE}\NormalTok{),}
            \AttributeTok{mean\_nflights =} \FunctionTok{mean}\NormalTok{(nflights, }\AttributeTok{na.rm =} \ConstantTok{TRUE}\NormalTok{),}
            \AttributeTok{sd\_nflights =} \FunctionTok{sd}\NormalTok{(nflights, }\AttributeTok{na.rm =} \ConstantTok{TRUE}\NormalTok{))}
\end{Highlighting}
\end{Shaded}

\begin{verbatim}
     n mean_age   sd_age median_age iqr_age mean_nflights sd_nflights
1 1065 50.41972 12.27464         50      16      13.41878    20.22647
\end{verbatim}

Group summaries (example: by \texttt{buy}):

\begin{Shaded}
\begin{Highlighting}[]
\NormalTok{airlinesat\_small }\SpecialCharTok{\%\textgreater{}\%}
  \FunctionTok{group\_by}\NormalTok{(status) }\SpecialCharTok{\%\textgreater{}\%}
  \FunctionTok{summarise}\NormalTok{(}\AttributeTok{n =} \FunctionTok{n}\NormalTok{(),}
            \AttributeTok{mean\_age =} \FunctionTok{mean}\NormalTok{(age, }\AttributeTok{na.rm =} \ConstantTok{TRUE}\NormalTok{),}
            \AttributeTok{sd\_age =} \FunctionTok{sd}\NormalTok{(age, }\AttributeTok{na.rm =} \ConstantTok{TRUE}\NormalTok{),}
            \AttributeTok{mean\_nflights =} \FunctionTok{mean}\NormalTok{(nflights, }\AttributeTok{na.rm =} \ConstantTok{TRUE}\NormalTok{),}
            \AttributeTok{sd\_nflights =} \FunctionTok{sd}\NormalTok{(nflights, }\AttributeTok{na.rm =} \ConstantTok{TRUE}\NormalTok{))}
\end{Highlighting}
\end{Shaded}

\begin{verbatim}
# A tibble: 3 x 6
  status     n mean_age sd_age mean_nflights sd_nflights
  <fct>  <int>    <dbl>  <dbl>         <dbl>       <dbl>
1 Blue     677     50.6   13.3          8.23        19.2
2 Gold     143     53     10.0         24.6         20.9
3 Silver   245     48.3   10.0         21.2         17.2
\end{verbatim}

\subsection{\texorpdfstring{Using package \emph{vtable}}{Using package vtable}}\label{using-package-vtable}

The \textbf{vtable} package can generate clean, compact descriptive tables with the \texttt{sumtable(data,\ vars=c(""))} function. For factor variables, it will provide the counts and percents of each level.

\begin{Shaded}
\begin{Highlighting}[]
\FunctionTok{sumtable}\NormalTok{(airlinesat\_small, }\FunctionTok{c}\NormalTok{(}\StringTok{"age"}\NormalTok{, }\StringTok{"nflights"}\NormalTok{, }\StringTok{"status"}\NormalTok{),}
         \AttributeTok{add.median=}\ConstantTok{TRUE}\NormalTok{,    }\CommentTok{\# \textquotesingle{}add.median=TRUE\textquotesingle{} includes a 50th percentile column}
         \AttributeTok{title=}\ConstantTok{NA}\NormalTok{)}
\end{Highlighting}
\end{Shaded}

\begin{tabular}{lllllllll}
\toprule
Variable & N & Mean & Std. Dev. & Min & Pctl. 25 & Pctl. 50 & Pctl. 75 & Max\\
\midrule
age & 1065 & 50 & 12 & 19 & 42 & 50 & 58 & 101\\
nflights & 1065 & 13 & 20 & 1 & 4 & 8 & 16 & 457\\
status & 1065 &  &  &  &  &  &  & \\
... Blue & 677 & 64\% &  &  &  &  &  & \\
... Gold & 143 & 13\% &  &  &  &  &  & \\
\addlinespace
... Silver & 245 & 23\% &  &  &  &  &  & \\
\bottomrule
\end{tabular}

The \texttt{sumtable()} function can also provide summary statistics by a grouping variable.

\begin{Shaded}
\begin{Highlighting}[]
\FunctionTok{sumtable}\NormalTok{(airlinesat\_small, }\FunctionTok{c}\NormalTok{(}\StringTok{"age"}\NormalTok{, }\StringTok{"nflights"}\NormalTok{, }\StringTok{"status"}\NormalTok{),}
         \AttributeTok{add.median=}\ConstantTok{TRUE}\NormalTok{, }
         \AttributeTok{group=}\StringTok{"gender"}\NormalTok{,}
         \AttributeTok{title=}\ConstantTok{NA}\NormalTok{)}
\end{Highlighting}
\end{Shaded}

\begin{tabular}{lllllllll}
\toprule
\multicolumn{1}{c}{gender} & \multicolumn{4}{c}{female} & \multicolumn{4}{c}{male} \\
\cmidrule(l{3pt}r{3pt}){1-1} \cmidrule(l{3pt}r{3pt}){2-5} \cmidrule(l{3pt}r{3pt}){6-9}
Variable & N & Mean & SD & Median & N & Mean & SD & Median\\
\midrule
age & 280 & 51 & 13 & 51 & 785 & 50 & 12 & 50\\
nflights & 280 & 9.3 & 12 & 5 & 785 & 15 & 22 & 9\\
status & 280 &  &  &  & 785 &  &  & \\
... Blue & 225 & 80\% &  &  & 452 & 58\% &  & \\
... Gold & 16 & 6\% &  &  & 127 & 16\% &  & \\
\addlinespace
... Silver & 39 & 14\% &  &  & 206 & 26\% &  & \\
\bottomrule
\end{tabular}

\begin{center}\rule{0.5\linewidth}{0.5pt}\end{center}

\section{Correlation}\label{correlation}

Correlation measures the strength of a \textbf{linear relationship} between two numeric variables.
The most common is Pearson correlation (the default).

\subsection{Base R}\label{base-r-2}

Base R can easily provide a correlation matrix of many variables, and it can provide a correlation test between two variables at a time, but it cannot produce a correlation matrix with p-values.

To get a correlation matrix, use the \texttt{cor()} function with a dataframe of the variables desired or using indexing on variable names. By default, it uses all observations, which can create \texttt{NA} values if any missing values exists. Therefore, the preference is to add the option \texttt{use\ =\ "pairwise.complete.obs"} to only calculate the correlation on non-missing values for each pair of variables.

\begin{Shaded}
\begin{Highlighting}[]
\NormalTok{mycorr\_df }\OtherTok{\textless{}{-}}\NormalTok{ airlinesat\_small }\SpecialCharTok{\%\textgreater{}\%}
  \FunctionTok{select}\NormalTok{(age, nflights, nps, overall\_sat, reputation)}
\FunctionTok{cor}\NormalTok{(mycorr\_df, }\AttributeTok{use =} \StringTok{"pairwise.complete.obs"}\NormalTok{)}
\end{Highlighting}
\end{Shaded}

\begin{verbatim}
                    age    nflights         nps overall_sat  reputation
age          1.00000000 -0.11576301  0.09867319  0.05903446  0.06082991
nflights    -0.11576301  1.00000000 -0.08949782 -0.05366975 -0.06364290
nps          0.09867319 -0.08949782  1.00000000  0.29961310  0.50712230
overall_sat  0.05903446 -0.05366975  0.29961310  1.00000000  0.17748688
reputation   0.06082991 -0.06364290  0.50712230  0.17748688  1.00000000
\end{verbatim}

\begin{Shaded}
\begin{Highlighting}[]
\FunctionTok{cor}\NormalTok{(airlinesat\_small[,}\FunctionTok{c}\NormalTok{(}\StringTok{"age"}\NormalTok{, }\StringTok{"nflights"}\NormalTok{, }\StringTok{"nps"}\NormalTok{, }\StringTok{"overall\_sat"}\NormalTok{, }\StringTok{"reputation"}\NormalTok{)],}
    \AttributeTok{use =} \StringTok{"pairwise.complete.obs"}\NormalTok{)}
\end{Highlighting}
\end{Shaded}

\begin{verbatim}
                    age    nflights         nps overall_sat  reputation
age          1.00000000 -0.11576301  0.09867319  0.05903446  0.06082991
nflights    -0.11576301  1.00000000 -0.08949782 -0.05366975 -0.06364290
nps          0.09867319 -0.08949782  1.00000000  0.29961310  0.50712230
overall_sat  0.05903446 -0.05366975  0.29961310  1.00000000  0.17748688
reputation   0.06082991 -0.06364290  0.50712230  0.17748688  1.00000000
\end{verbatim}

To get the correlation test for any one pair of variables, use the \texttt{cor.test(var1,\ var2)} function. By default, it includes only observations that are non-missing in both variables.

\begin{Shaded}
\begin{Highlighting}[]
\FunctionTok{cor.test}\NormalTok{(airlinesat\_small}\SpecialCharTok{$}\NormalTok{age, airlinesat\_small}\SpecialCharTok{$}\NormalTok{nflights)}
\end{Highlighting}
\end{Shaded}

\begin{verbatim}

    Pearson's product-moment correlation

data:  airlinesat_small$age and airlinesat_small$nflights
t = -3.7998, df = 1063, p-value = 0.000153
alternative hypothesis: true correlation is not equal to 0
95 percent confidence interval:
 -0.17461941 -0.05608231
sample estimates:
      cor 
-0.115763 
\end{verbatim}

\subsection{\texorpdfstring{Using package \emph{Hmisc}}{Using package Hmisc}}\label{using-package-hmisc}

The \textbf{Hmisc} package is useful for correlation matrices with p-values using the \texttt{rcorr()} function. This function expects a matrix (or data frame coerced to matrix).

\begin{Shaded}
\begin{Highlighting}[]
\FunctionTok{rcorr}\NormalTok{(}\FunctionTok{as.matrix}\NormalTok{(mycorr\_df))}
\end{Highlighting}
\end{Shaded}

\begin{verbatim}
              age nflights   nps overall_sat reputation
age          1.00    -0.12  0.10        0.06       0.06
nflights    -0.12     1.00 -0.09       -0.05      -0.06
nps          0.10    -0.09  1.00        0.30       0.51
overall_sat  0.06    -0.05  0.30        1.00       0.18
reputation   0.06    -0.06  0.51        0.18       1.00

n= 1065 


P
            age    nflights nps    overall_sat reputation
age                0.0002   0.0013 0.0541      0.0472    
nflights    0.0002          0.0035 0.0800      0.0378    
nps         0.0013 0.0035          0.0000      0.0000    
overall_sat 0.0541 0.0800   0.0000             0.0000    
reputation  0.0472 0.0378   0.0000 0.0000                
\end{verbatim}

\subsection{\texorpdfstring{Using package \emph{sjPlot}}{Using package sjPlot}}\label{using-package-sjplot-1}

The \textbf{sjPlot} package can produce formatted correlation tables for reports using the \texttt{tab\_corr()} function. By default, it uses listwise (or casewise) deletion, which removes an entire case (or row) if any value is \texttt{NA}, which may result in major data loss. Use the option \texttt{na.deletion\ =\ "pairwise")} to prevent this.

\begin{Shaded}
\begin{Highlighting}[]
\FunctionTok{tab\_corr}\NormalTok{(mycorr\_df,}
         \AttributeTok{triangle =} \StringTok{"lower"}\NormalTok{,}
         \AttributeTok{show.p =} \ConstantTok{TRUE}\NormalTok{,}
         \AttributeTok{na.deletion =} \StringTok{"pairwise"}\NormalTok{)}
\end{Highlighting}
\end{Shaded}

~

age

nflights

nps

overall\_sat

reputation

age

~

~

~

~

~

nflights

-0.116***

~

~

~

~

nps

0.099**

-0.089**

~

~

~

overall\_sat

0.059{}

-0.054{}

0.300***

~

~

reputation

0.061*

-0.064*

0.507***

0.177***

~

Computed correlation used pearson-method with pairwise-deletion.

\begin{center}\rule{0.5\linewidth}{0.5pt}\end{center}

\section{Why descriptive analysis matters}\label{why-descriptive-analysis-matters}

Descriptive statistics help you to:

\begin{itemize}
\tightlist
\item
  detect unusual values,
\item
  understand typical behavior,
\item
  compare groups,
\item
  and check whether results are plausible.
\end{itemize}

They also guide decisions about which models or visualizations are appropriate.

\begin{center}\rule{0.5\linewidth}{0.5pt}\end{center}

\section{What's next}\label{whats-next-6}

In the next chapter, we move from \textbf{numbers} to \textbf{visuals}. You will learn how to create effective data visualizations using a small amount of Base R, but primarily \texttt{ggplot2}. Visualizations allow you to see patterns, distributions, and relationships that are often difficult to detect from tables alone.

Together, descriptive statistics and visualization form the foundation of \textbf{exploratory data analysis}, which prepares you for modeling and inference later in the course.

\chapter{Data Visualization}\label{data-visualization}

Descriptive statistics summarize data numerically, but visualizations often reveal patterns, trends, and anomalies more quickly. This chapter introduces basic data visualization techniques in R. We begin with Base R graphics to understand core plotting concepts, then move to ggplot2, which provides a more flexible and powerful system for creating graphics.

\begin{center}\rule{0.5\linewidth}{0.5pt}\end{center}

\section{Base R Visualizations}\label{base-r-visualizations}

Base R graphics are built into R and require no additional packages. They are
useful for quick exploratory analysis and for understanding how plotting works
at a fundamental level.

\subsection{Histogram (Base R)}\label{histogram-base-r}

A histogram shows the distribution of a numeric variable. Histograms are useful for assessing the shape, spread, and potential outliers in a numeric variable.

\begin{Shaded}
\begin{Highlighting}[]
\FunctionTok{hist}\NormalTok{(airlinesat\_small}\SpecialCharTok{$}\NormalTok{age,}
     \AttributeTok{main =} \StringTok{"Histogram of Number of Age"}\NormalTok{,}
     \AttributeTok{xlab =} \StringTok{"Number of Age"}\NormalTok{,}
     \AttributeTok{col =} \StringTok{"lightgray"}\NormalTok{,}
     \AttributeTok{border =} \StringTok{"white"}\NormalTok{)}
\end{Highlighting}
\end{Shaded}

\pandocbounded{\includegraphics[keepaspectratio]{MKT4320_R_Tutorial_files/figure-latex/base-histogram-1.pdf}}

\subsection{Box-and-Whisker Plot (Base R)}\label{box-and-whisker-plot-base-r}

A box-and-whisker plot (often called a boxplot) summarizes the distribution of a numeric variable using five key values:

\begin{itemize}
\tightlist
\item
  Minimum
\item
  First quartile (25th percentile)
\item
  Median
\item
  Third quartile (75th percentile)
\item
  Maximum
\end{itemize}

Boxplots are especially useful for:

\begin{itemize}
\tightlist
\item
  Comparing distributions across groups
\item
  Identifying skewness
\item
  Detecting potential outliers
\end{itemize}

\begin{Shaded}
\begin{Highlighting}[]
\FunctionTok{boxplot}\NormalTok{(airlinesat\_small}\SpecialCharTok{$}\NormalTok{age,}
        \AttributeTok{main =} \StringTok{"Distribution of Age"}\NormalTok{,}
        \AttributeTok{ylab =} \StringTok{"Age"}\NormalTok{,}
        \AttributeTok{col  =} \StringTok{"lightgray"}\NormalTok{)}
\end{Highlighting}
\end{Shaded}

\pandocbounded{\includegraphics[keepaspectratio]{MKT4320_R_Tutorial_files/figure-latex/boxplot_base_r-1.pdf}}

\subsection{Scatterplot (Base R)}\label{scatterplot-base-r}

A scatterplot displays the relationship between two numeric variables. Scatterplots are commonly used to detect relationships, nonlinear patterns, and outliers.

\begin{Shaded}
\begin{Highlighting}[]
\FunctionTok{plot}\NormalTok{(airlinesat\_small}\SpecialCharTok{$}\NormalTok{age,}
\NormalTok{     airlinesat\_small}\SpecialCharTok{$}\NormalTok{nflights,}
     \AttributeTok{main =} \StringTok{"Age vs Number of Flights"}\NormalTok{,}
     \AttributeTok{xlab =} \StringTok{"Age"}\NormalTok{,}
     \AttributeTok{ylab =} \StringTok{"Number of Flights"}\NormalTok{,}
     \AttributeTok{pch =} \DecValTok{19}\NormalTok{,}
     \AttributeTok{col =} \StringTok{"darkgray"}\NormalTok{)}
\end{Highlighting}
\end{Shaded}

\pandocbounded{\includegraphics[keepaspectratio]{MKT4320_R_Tutorial_files/figure-latex/base-scatterplot-1.pdf}}

\subsection{Line Chart (Base R)}\label{line-chart-base-r}

Line charts are typically used for ordered data, such as time series or values summarized across an ordered variable. Line charts emphasize change across an ordered dimension rather than individual observations.

First, we'll simulate some time series data, and then we'll create the line chart.

\begin{Shaded}
\begin{Highlighting}[]
\CommentTok{\# Simulate monthly flight data}
\FunctionTok{set.seed}\NormalTok{(}\DecValTok{123}\NormalTok{)}
\NormalTok{months }\OtherTok{\textless{}{-}} \FunctionTok{seq}\NormalTok{(}\AttributeTok{from =} \FunctionTok{as.Date}\NormalTok{(}\StringTok{"2022{-}01{-}01"}\NormalTok{), }\AttributeTok{to =} \FunctionTok{as.Date}\NormalTok{(}\StringTok{"2023{-}12{-}01"}\NormalTok{), }\AttributeTok{by =} \StringTok{"month"}\NormalTok{)}
\NormalTok{n\_months }\OtherTok{\textless{}{-}} \FunctionTok{length}\NormalTok{(months)}
\NormalTok{flights }\OtherTok{\textless{}{-}} \FunctionTok{round}\NormalTok{(}\DecValTok{800} \SpecialCharTok{+}
    \FunctionTok{seq}\NormalTok{(}\DecValTok{0}\NormalTok{, }\DecValTok{100}\NormalTok{, }\AttributeTok{length.out =}\NormalTok{ n\_months) }\SpecialCharTok{+}          \CommentTok{\# upward trend}
    \DecValTok{80} \SpecialCharTok{*} \FunctionTok{sin}\NormalTok{(}\DecValTok{2} \SpecialCharTok{*}\NormalTok{ pi }\SpecialCharTok{*}\NormalTok{ (}\DecValTok{1}\SpecialCharTok{:}\NormalTok{n\_months) }\SpecialCharTok{/} \DecValTok{12}\NormalTok{) }\SpecialCharTok{+}         \CommentTok{\# seasonality}
    \FunctionTok{rnorm}\NormalTok{(n\_months, }\AttributeTok{mean =} \DecValTok{0}\NormalTok{, }\AttributeTok{sd =} \DecValTok{40}\NormalTok{))              }\CommentTok{\# random noise}

\NormalTok{flight\_ts }\OtherTok{\textless{}{-}} \FunctionTok{data.frame}\NormalTok{(}\AttributeTok{month =}\NormalTok{ months, }\AttributeTok{flights =}\NormalTok{ flights)}

\CommentTok{\# Line chart of flights by month}
\FunctionTok{plot}\NormalTok{(flight\_ts}\SpecialCharTok{$}\NormalTok{month,}
\NormalTok{     flight\_ts}\SpecialCharTok{$}\NormalTok{flights,}
     \AttributeTok{type =} \StringTok{"l"}\NormalTok{,}
     \AttributeTok{main =} \StringTok{"Number of Flights by Month"}\NormalTok{,}
     \AttributeTok{xlab =} \StringTok{"Month"}\NormalTok{,}
     \AttributeTok{ylab =} \StringTok{"Number of Flights"}\NormalTok{,}
     \AttributeTok{col  =} \StringTok{"steelblue"}\NormalTok{,}
     \AttributeTok{lwd  =} \DecValTok{2}\NormalTok{)}
\end{Highlighting}
\end{Shaded}

\pandocbounded{\includegraphics[keepaspectratio]{MKT4320_R_Tutorial_files/figure-latex/base-linechart-1.pdf}}

\subsection{Bar Chart (Base R)}\label{bar-chart-base-r}

Bar charts are used for categorical variables. They are useful for comparing counts or proportions across categories. They can also show results for different categories of another variable (i.e., a side-by-side bar chart).

\begin{Shaded}
\begin{Highlighting}[]
\NormalTok{status\_counts }\OtherTok{\textless{}{-}} \FunctionTok{table}\NormalTok{(airlinesat\_small}\SpecialCharTok{$}\NormalTok{status)}

\FunctionTok{barplot}\NormalTok{(}
\NormalTok{  status\_counts,}
  \AttributeTok{main =} \StringTok{"Loyalty Status"}\NormalTok{,}
  \AttributeTok{ylab =} \StringTok{"Frequency"}\NormalTok{,}
  \AttributeTok{col =} \StringTok{"lightblue"}
\NormalTok{)}
\end{Highlighting}
\end{Shaded}

\pandocbounded{\includegraphics[keepaspectratio]{MKT4320_R_Tutorial_files/figure-latex/base-barchart-1.pdf}}

\begin{Shaded}
\begin{Highlighting}[]
\NormalTok{status\_gender\_tab }\OtherTok{\textless{}{-}} \FunctionTok{table}\NormalTok{(airlinesat\_small}\SpecialCharTok{$}\NormalTok{status, airlinesat\_small}\SpecialCharTok{$}\NormalTok{gender)}
\FunctionTok{barplot}\NormalTok{(status\_gender\_tab,}
        \AttributeTok{beside =} \ConstantTok{TRUE}\NormalTok{,}
        \AttributeTok{col =} \FunctionTok{c}\NormalTok{(}\StringTok{"steelblue"}\NormalTok{, }\StringTok{"darkorange"}\NormalTok{),}
        \AttributeTok{main =} \StringTok{"Loyalty Status by Gender"}\NormalTok{,}
        \AttributeTok{xlab =} \StringTok{"Loyalty Status"}\NormalTok{,}
        \AttributeTok{ylab =} \StringTok{"Number of Customers"}\NormalTok{,}
        \AttributeTok{legend.text =} \ConstantTok{TRUE}\NormalTok{,}
        \AttributeTok{args.legend =} \FunctionTok{list}\NormalTok{(}\AttributeTok{title =} \StringTok{"Gender"}\NormalTok{, }\AttributeTok{x =} \StringTok{"topright"}\NormalTok{, }\AttributeTok{bty =} \StringTok{"n"}\NormalTok{))}
\end{Highlighting}
\end{Shaded}

\pandocbounded{\includegraphics[keepaspectratio]{MKT4320_R_Tutorial_files/figure-latex/base-barchart-2.pdf}}

\begin{center}\rule{0.5\linewidth}{0.5pt}\end{center}

\section{Moving Beyond Base R}\label{moving-beyond-base-r}

While Base R graphics are useful, they can become cumbersome when creating more complex plots or when consistent styling is needed. The \texttt{ggplot2} package provides a structured approach to visualization based on the \textbf{g}rammar of \textbf{g}raphics.

\begin{center}\rule{0.5\linewidth}{0.5pt}\end{center}

\section{\texorpdfstring{Introduction to \emph{ggplot2}}{Introduction to ggplot2}}\label{introduction-to-ggplot2}

In \texttt{ggplot2}, plots are built in layers from three main components: data, aesthetic mappings (i.e., a coordinate system identifying \texttt{x} and \texttt{y} variables), and geometric objects (i.e., how the data should be displayed). In addition, the plot can be enhanced by adding additional layers using the \texttt{+} operator. \texttt{ggplot2} also works very well with \texttt{dplyr} when data manipulation is needed prior to creating the plot.

\subsection{Histogram}\label{histogram}

Histograms in \texttt{ggplot} use the \texttt{geom\_histogram()} layer. As with many geoms in \texttt{ggplot2}, no options are \emph{required} in the geom. Here is a basic histogram.

\begin{Shaded}
\begin{Highlighting}[]
\FunctionTok{ggplot}\NormalTok{(airlinesat\_small, }\FunctionTok{aes}\NormalTok{(}\AttributeTok{x =}\NormalTok{ age)) }\SpecialCharTok{+}
  \FunctionTok{geom\_histogram}\NormalTok{()}
\end{Highlighting}
\end{Shaded}

\begin{verbatim}
`stat_bin()` using `bins = 30`. Pick better value `binwidth`.
\end{verbatim}

\pandocbounded{\includegraphics[keepaspectratio]{MKT4320_R_Tutorial_files/figure-latex/ggplot-histogram_basic-1.pdf}}

One of the benefits of \texttt{ggplot2} is the ease of making the plot look more visually appealing often more informative. Here is the histogram with additional options for \texttt{bins}, the \texttt{fill} color, and the outline \texttt{color}, along with labels using the \texttt{labs} layer.

\begin{Shaded}
\begin{Highlighting}[]
\FunctionTok{ggplot}\NormalTok{(airlinesat\_small, }\FunctionTok{aes}\NormalTok{(}\AttributeTok{x =}\NormalTok{ age)) }\SpecialCharTok{+}
  \FunctionTok{geom\_histogram}\NormalTok{(}\AttributeTok{bins =} \DecValTok{30}\NormalTok{,}
                 \AttributeTok{fill =} \StringTok{"orange"}\NormalTok{,}
                 \AttributeTok{color =} \StringTok{"white"}\NormalTok{) }\SpecialCharTok{+}
  \FunctionTok{labs}\NormalTok{(}\AttributeTok{title =} \StringTok{"Histogram of Age"}\NormalTok{,}
       \AttributeTok{x =} \StringTok{"Age"}\NormalTok{,}
       \AttributeTok{y=} \StringTok{"Count"}\NormalTok{)}
\end{Highlighting}
\end{Shaded}

\pandocbounded{\includegraphics[keepaspectratio]{MKT4320_R_Tutorial_files/figure-latex/ggplot-histogram_nice-1.pdf}}

\subsection{Box-and-Whiskers Plot}\label{box-and-whiskers-plot}

Box Plots are drawn with the \texttt{geom\_boxplot()} geom, which by default creates a box plot for a continuous \texttt{y} variable, but for each level of a discrete \texttt{x} variable. In addition, the standard box plot does not contain ``whiskers''. To get a box plot for only the continuous \texttt{y} variable, use \texttt{x\ =\ ""} as the discrete \texttt{x} variable. To add whiskers, include a \texttt{staplewidth\ =\ 1} within the \texttt{geom\_boxplot()}.

\begin{Shaded}
\begin{Highlighting}[]
\FunctionTok{ggplot}\NormalTok{(airlinesat\_small, }\FunctionTok{aes}\NormalTok{(}\AttributeTok{x =} \StringTok{""}\NormalTok{, }\AttributeTok{y=}\NormalTok{age)) }\SpecialCharTok{+}
  \FunctionTok{geom\_boxplot}\NormalTok{(}\AttributeTok{staplewidth =} \DecValTok{1}\NormalTok{) }\SpecialCharTok{+} 
  \FunctionTok{labs}\NormalTok{(}\AttributeTok{title=}\StringTok{"Boxplot for Age"}\NormalTok{,}
       \AttributeTok{x =} \StringTok{""}\NormalTok{,}
       \AttributeTok{y =} \StringTok{"Age"}\NormalTok{)}
\end{Highlighting}
\end{Shaded}

\pandocbounded{\includegraphics[keepaspectratio]{MKT4320_R_Tutorial_files/figure-latex/ggplot-boxplot_1-1.pdf}}

\subsubsection{Side-by-Side Boxplot}\label{side-by-side-boxplot}

To create side-by-side boxplots for a discrete variable, simply repace the \texttt{x\ =\ ""} with a variable name.

\begin{Shaded}
\begin{Highlighting}[]
\FunctionTok{ggplot}\NormalTok{(airlinesat\_small, }\FunctionTok{aes}\NormalTok{(}\AttributeTok{x =}\NormalTok{ gender, }\AttributeTok{y=}\NormalTok{age)) }\SpecialCharTok{+}
  \FunctionTok{geom\_boxplot}\NormalTok{(}\AttributeTok{staplewidth =} \DecValTok{1}\NormalTok{) }\SpecialCharTok{+} 
  \FunctionTok{labs}\NormalTok{(}\AttributeTok{title=}\StringTok{"Boxplot for Age by Gender"}\NormalTok{,}
       \AttributeTok{x =} \StringTok{"Gender"}\NormalTok{,}
       \AttributeTok{y =} \StringTok{"Age"}\NormalTok{)}
\end{Highlighting}
\end{Shaded}

\pandocbounded{\includegraphics[keepaspectratio]{MKT4320_R_Tutorial_files/figure-latex/ggplot-boxplot_2-1.pdf}}

\subsection{Scatterplot}\label{scatterplot}

Scatterplots are drawn with the \texttt{geom\_point()} geom and are used to show the relationship between two continuous variables.

\begin{Shaded}
\begin{Highlighting}[]
\FunctionTok{ggplot}\NormalTok{(airlinesat\_small, }\FunctionTok{aes}\NormalTok{(}\AttributeTok{x =}\NormalTok{ age, }\AttributeTok{y =}\NormalTok{ nflights)) }\SpecialCharTok{+}
  \FunctionTok{geom\_point}\NormalTok{() }\SpecialCharTok{+}
  \FunctionTok{labs}\NormalTok{(}\AttributeTok{title =} \StringTok{"Age vs Number of Flights"}\NormalTok{,}
       \AttributeTok{x =} \StringTok{"Age"}\NormalTok{,}
       \AttributeTok{y =} \StringTok{"Number of Flights"}\NormalTok{)}
\end{Highlighting}
\end{Shaded}

\pandocbounded{\includegraphics[keepaspectratio]{MKT4320_R_Tutorial_files/figure-latex/ggplot-scatterplot_1-1.pdf}}

\subsubsection{Scatterplot with a Categorical Variable}\label{scatterplot-with-a-categorical-variable}

More interesting scatterplots can be created by changing the color of the points by a third, discrete variable. This means adding a new aesthetic in the \texttt{aes()} part.

\begin{Shaded}
\begin{Highlighting}[]
\FunctionTok{ggplot}\NormalTok{(airlinesat\_small, }\FunctionTok{aes}\NormalTok{(}\AttributeTok{x =}\NormalTok{ age, }\AttributeTok{y =}\NormalTok{ nflights, }\AttributeTok{color=}\NormalTok{status)) }\SpecialCharTok{+}
  \FunctionTok{geom\_point}\NormalTok{() }\SpecialCharTok{+}
  \FunctionTok{labs}\NormalTok{(}\AttributeTok{title =} \StringTok{"Age vs Number of Flights by Loyalty Status"}\NormalTok{,}
       \AttributeTok{x =} \StringTok{"Age"}\NormalTok{,}
       \AttributeTok{y =} \StringTok{"Number of Flights"}\NormalTok{,}
       \AttributeTok{color =} \StringTok{"Loyalty Status"}\NormalTok{)}
\end{Highlighting}
\end{Shaded}

\pandocbounded{\includegraphics[keepaspectratio]{MKT4320_R_Tutorial_files/figure-latex/ggplot-scatterplot_2-1.pdf}}

\subsubsection{Scatterplot with a Trendline}\label{scatterplot-with-a-trendline}

Scatterplots become more helpful when we add a trend line. The most common trend line is a simple regression line, although others can be used. Use \texttt{geom\_smooth(method\ =\ "lm",\ se\ =\ FALSE)} to add a linear trend line.

\begin{Shaded}
\begin{Highlighting}[]
\FunctionTok{ggplot}\NormalTok{(airlinesat\_small, }\FunctionTok{aes}\NormalTok{(}\AttributeTok{x =}\NormalTok{ age, }\AttributeTok{y =}\NormalTok{ nps)) }\SpecialCharTok{+}
  \FunctionTok{geom\_point}\NormalTok{() }\SpecialCharTok{+}
  \FunctionTok{geom\_smooth}\NormalTok{(}\AttributeTok{method =} \StringTok{"lm"}\NormalTok{, }\AttributeTok{se =} \ConstantTok{FALSE}\NormalTok{) }\SpecialCharTok{+} 
  \FunctionTok{labs}\NormalTok{(}\AttributeTok{title =} \StringTok{"Age vs Number of Flights"}\NormalTok{,}
       \AttributeTok{x =} \StringTok{"Age"}\NormalTok{,}
       \AttributeTok{y =} \StringTok{"Number of Flights"}\NormalTok{)}
\end{Highlighting}
\end{Shaded}

\begin{verbatim}
`geom_smooth()` using formula = 'y ~ x'
\end{verbatim}

\pandocbounded{\includegraphics[keepaspectratio]{MKT4320_R_Tutorial_files/figure-latex/ggplot-scatterplot_3-1.pdf}}

\subsection{Line Chart}\label{line-chart}

Line charts are drawn with the \texttt{geom\_line()} geom and are used to emphasize change across an ordered dimension rather than individual observations. We'll use the simulated data from the line chart in base R created above.

\begin{Shaded}
\begin{Highlighting}[]
\FunctionTok{ggplot}\NormalTok{(flight\_ts, }\FunctionTok{aes}\NormalTok{(}\AttributeTok{x =}\NormalTok{ month, }\AttributeTok{y =}\NormalTok{ flights)) }\SpecialCharTok{+}
  \FunctionTok{geom\_line}\NormalTok{()}
\end{Highlighting}
\end{Shaded}

\pandocbounded{\includegraphics[keepaspectratio]{MKT4320_R_Tutorial_files/figure-latex/ggplot-linechart-1.pdf}}

\subsection{Bar Chart}\label{bar-chart}

In \texttt{ggplot2}, bar charts, \texttt{geom\_bar()}, are often used for plotting a single discrete variable, while column charts, \texttt{geom\_col()}, are used for plotting a discrete variable on the x axis and a continuous variable on the y axis. However, \texttt{geom\_col()} can be used for both if the data is ``preformatted'', which is easy with \texttt{dplyr}. Other than the first example below, all other examples use \texttt{geom\_col()} as it tends to be more flexible when combined with \texttt{dplyr}.

The second example below first organizes the data using \texttt{dplyr} and then passes that result to \texttt{ggplot}.

\begin{Shaded}
\begin{Highlighting}[]
\FunctionTok{ggplot}\NormalTok{(airlinesat\_small, }\FunctionTok{aes}\NormalTok{(}\AttributeTok{x =}\NormalTok{ gender)) }\SpecialCharTok{+}
  \FunctionTok{geom\_bar}\NormalTok{(}\AttributeTok{fill =} \StringTok{"lightblue"}\NormalTok{) }\SpecialCharTok{+}
  \FunctionTok{labs}\NormalTok{(}\AttributeTok{title =} \StringTok{"Gender"}\NormalTok{,}
       \AttributeTok{x =} \StringTok{"Gender"}\NormalTok{,}
       \AttributeTok{y =} \StringTok{"Count"}\NormalTok{)}
\end{Highlighting}
\end{Shaded}

\pandocbounded{\includegraphics[keepaspectratio]{MKT4320_R_Tutorial_files/figure-latex/ggplot-barchart-1.pdf}}

\begin{Shaded}
\begin{Highlighting}[]
\NormalTok{airlinesat\_small }\SpecialCharTok{\%\textgreater{}\%}
  \FunctionTok{group\_by}\NormalTok{(gender) }\SpecialCharTok{\%\textgreater{}\%}
  \FunctionTok{summarise}\NormalTok{(}\AttributeTok{n=}\FunctionTok{n}\NormalTok{()) }\SpecialCharTok{\%\textgreater{}\%}
  \FunctionTok{ggplot}\NormalTok{(}\FunctionTok{aes}\NormalTok{(}\AttributeTok{x =}\NormalTok{ gender, }\AttributeTok{y =}\NormalTok{ n)) }\SpecialCharTok{+} 
  \FunctionTok{geom\_col}\NormalTok{(}\AttributeTok{fill =} \StringTok{"lightblue"}\NormalTok{) }\SpecialCharTok{+}
  \FunctionTok{labs}\NormalTok{(}\AttributeTok{title =} \StringTok{"Gender"}\NormalTok{,}
       \AttributeTok{x =} \StringTok{"Gender"}\NormalTok{,}
       \AttributeTok{y =} \StringTok{"Count"}\NormalTok{)}
\end{Highlighting}
\end{Shaded}

\pandocbounded{\includegraphics[keepaspectratio]{MKT4320_R_Tutorial_files/figure-latex/ggplot-barchart-2.pdf}}

\subsubsection{Bar Charts for Continuous Variables}\label{bar-charts-for-continuous-variables}

Bar charts can be used to show a summary statistic (e.g., mean, median, etc.) of a continuous variable. This is easily done with \texttt{dplyr}. We can also add labels using the \texttt{geom\_text()} or \texttt{geom\_label()} geoms. \texttt{geom\_text()} tends to be more flexible. By default, \texttt{ggplot} places the label at the exact height of the value, but that will overlap with the bar itself. Use the \texttt{vjust\ =} option to change the position of the label vertically. Usually \texttt{vjust\ =\ 0.95} puts the label just inside the top of the bar, while \texttt{vjust\ =\ -.05} puts the label just outside the top of the bar. You can also change the color of the label with \texttt{color\ =}, the size of the label with \texttt{size\ =}, and you can round the label to a specific number of digits with \texttt{round()}.

\begin{Shaded}
\begin{Highlighting}[]
\NormalTok{airlinesat\_small }\SpecialCharTok{\%\textgreater{}\%}
  \FunctionTok{summarise}\NormalTok{(}\AttributeTok{mean\_age=}\FunctionTok{mean}\NormalTok{(age)) }\SpecialCharTok{\%\textgreater{}\%}
  \FunctionTok{ggplot}\NormalTok{(}\FunctionTok{aes}\NormalTok{(}\AttributeTok{x =} \StringTok{""}\NormalTok{, }\AttributeTok{y =}\NormalTok{ mean\_age)) }\SpecialCharTok{+} 
  \FunctionTok{geom\_col}\NormalTok{(}\AttributeTok{fill =} \StringTok{"navyblue"}\NormalTok{) }\SpecialCharTok{+}
  \FunctionTok{geom\_text}\NormalTok{(}\FunctionTok{aes}\NormalTok{(}\AttributeTok{label =} \FunctionTok{round}\NormalTok{(mean\_age,}\DecValTok{2}\NormalTok{)),}
            \AttributeTok{vjust =}\NormalTok{ .}\DecValTok{95}\NormalTok{,}
            \AttributeTok{color=}\StringTok{"white"}\NormalTok{,}
            \AttributeTok{size=}\DecValTok{8}\NormalTok{) }\SpecialCharTok{+}
  \FunctionTok{labs}\NormalTok{(}\AttributeTok{title =} \StringTok{"Mean Age"}\NormalTok{,}
       \AttributeTok{x =} \StringTok{"Age"}\NormalTok{,}
       \AttributeTok{y =} \StringTok{"Mean"}\NormalTok{)}
\end{Highlighting}
\end{Shaded}

\pandocbounded{\includegraphics[keepaspectratio]{MKT4320_R_Tutorial_files/figure-latex/ggplot-barchart_cont-1.pdf}}

\subsection{Grouped Bar Chart}\label{grouped-bar-chart}

When used with \texttt{dplyr}, it is easy to create side-by-side or stacked bar charts. In the code below, first a table is created and displayed using \texttt{dplyr}. Ultimately, we want to take those results pass them to a \texttt{ggplot} to create the chart.

\subsubsection{Stacked Bar Chart with Counts}\label{stacked-bar-chart-with-counts}

The default is to ``stack'' the counts.

\begin{Shaded}
\begin{Highlighting}[]
\NormalTok{airlinesat\_small }\SpecialCharTok{\%\textgreater{}\%}
  \FunctionTok{group\_by}\NormalTok{(gender, status) }\SpecialCharTok{\%\textgreater{}\%}
  \FunctionTok{summarise}\NormalTok{(}\AttributeTok{n=}\FunctionTok{n}\NormalTok{())}
\end{Highlighting}
\end{Shaded}

\begin{verbatim}
# A tibble: 6 x 3
# Groups:   gender [2]
  gender status     n
  <fct>  <fct>  <int>
1 female Blue     225
2 female Gold      16
3 female Silver    39
4 male   Blue     452
5 male   Gold     127
6 male   Silver   206
\end{verbatim}

\begin{Shaded}
\begin{Highlighting}[]
\NormalTok{airlinesat\_small }\SpecialCharTok{\%\textgreater{}\%}
  \FunctionTok{group\_by}\NormalTok{(gender, status) }\SpecialCharTok{\%\textgreater{}\%}
  \FunctionTok{summarise}\NormalTok{(}\AttributeTok{n=}\FunctionTok{n}\NormalTok{()) }\SpecialCharTok{\%\textgreater{}\%}
  \FunctionTok{ggplot}\NormalTok{(}\FunctionTok{aes}\NormalTok{(}\AttributeTok{x =}\NormalTok{ gender, }\AttributeTok{y =}\NormalTok{ n, }\AttributeTok{fill =}\NormalTok{ status)) }\SpecialCharTok{+}
  \FunctionTok{geom\_col}\NormalTok{() }\SpecialCharTok{+} 
  \FunctionTok{labs}\NormalTok{(}\AttributeTok{title =} \StringTok{"Loyalty Status by Gender (Stacked Count)"}\NormalTok{,}
       \AttributeTok{x =} \StringTok{"Gender"}\NormalTok{,}
       \AttributeTok{y =} \StringTok{"Count"}\NormalTok{,}
       \AttributeTok{fill =} \StringTok{"Status"}\NormalTok{)}
\end{Highlighting}
\end{Shaded}

\pandocbounded{\includegraphics[keepaspectratio]{MKT4320_R_Tutorial_files/figure-latex/ggplot-stacked-barchart_01-1.pdf}}

\subsubsection{Stacked Bar Chart with Percentages}\label{stacked-bar-chart-with-percentages}

We can create ``100\%'' stacked bar charts by calculating the percentages in the \texttt{dplyr} code. With the code below, the percentages will add up to 100\% for the first \texttt{group\_by} variable.

\begin{Shaded}
\begin{Highlighting}[]
\NormalTok{airlinesat\_small }\SpecialCharTok{\%\textgreater{}\%}
  \FunctionTok{group\_by}\NormalTok{(gender, status) }\SpecialCharTok{\%\textgreater{}\%}
  \FunctionTok{summarise}\NormalTok{(}\AttributeTok{n=}\FunctionTok{n}\NormalTok{()) }\SpecialCharTok{\%\textgreater{}\%}
  \FunctionTok{mutate}\NormalTok{(}\AttributeTok{perc =} \DecValTok{100} \SpecialCharTok{*}\NormalTok{ n}\SpecialCharTok{/}\FunctionTok{sum}\NormalTok{(n))}
\end{Highlighting}
\end{Shaded}

\begin{verbatim}
# A tibble: 6 x 4
# Groups:   gender [2]
  gender status     n  perc
  <fct>  <fct>  <int> <dbl>
1 female Blue     225 80.4 
2 female Gold      16  5.71
3 female Silver    39 13.9 
4 male   Blue     452 57.6 
5 male   Gold     127 16.2 
6 male   Silver   206 26.2 
\end{verbatim}

\begin{Shaded}
\begin{Highlighting}[]
\NormalTok{airlinesat\_small }\SpecialCharTok{\%\textgreater{}\%}
  \FunctionTok{group\_by}\NormalTok{(gender, status) }\SpecialCharTok{\%\textgreater{}\%}
  \FunctionTok{summarise}\NormalTok{(}\AttributeTok{n=}\FunctionTok{n}\NormalTok{()) }\SpecialCharTok{\%\textgreater{}\%}
  \FunctionTok{mutate}\NormalTok{(}\AttributeTok{perc =} \DecValTok{100} \SpecialCharTok{*}\NormalTok{ n}\SpecialCharTok{/}\FunctionTok{sum}\NormalTok{(n)) }\SpecialCharTok{\%\textgreater{}\%}
  \FunctionTok{ggplot}\NormalTok{(}\FunctionTok{aes}\NormalTok{(}\AttributeTok{x =}\NormalTok{ gender, }\AttributeTok{y =}\NormalTok{ perc, }\AttributeTok{fill =}\NormalTok{ status)) }\SpecialCharTok{+}
  \FunctionTok{geom\_col}\NormalTok{() }\SpecialCharTok{+} 
  \FunctionTok{labs}\NormalTok{(}\AttributeTok{title =} \StringTok{"Loyalty Status by Gender (Stacked Percent)"}\NormalTok{,}
       \AttributeTok{x =} \StringTok{"Gender"}\NormalTok{,}
       \AttributeTok{y =} \StringTok{"Percent"}\NormalTok{,}
       \AttributeTok{fill =} \StringTok{"Status"}\NormalTok{)}
\end{Highlighting}
\end{Shaded}

\pandocbounded{\includegraphics[keepaspectratio]{MKT4320_R_Tutorial_files/figure-latex/ggplot-stacked-barchart_02-1.pdf}}

\subsubsection{Side-by-Side Bar Chart with Counts/Percentages}\label{side-by-side-bar-chart-with-countspercentages}

To create side-by-side bar charts (vs.~stacked), use the \texttt{position\ =\ position\_dodge(width=.9)} option within the \texttt{geom\_col()}. If you add labels, you need to use \texttt{position\ =\ position\_dodge(width=.9)} in the \texttt{geom\_text}, otherwise the labels will be on top of each other.

\begin{Shaded}
\begin{Highlighting}[]
\NormalTok{airlinesat\_small }\SpecialCharTok{\%\textgreater{}\%}
  \FunctionTok{group\_by}\NormalTok{(gender, status) }\SpecialCharTok{\%\textgreater{}\%}
  \FunctionTok{summarise}\NormalTok{(}\AttributeTok{n=}\FunctionTok{n}\NormalTok{()) }\SpecialCharTok{\%\textgreater{}\%}
  \FunctionTok{ggplot}\NormalTok{(}\FunctionTok{aes}\NormalTok{(}\AttributeTok{x =}\NormalTok{ gender, }\AttributeTok{y =}\NormalTok{ n, }\AttributeTok{fill =}\NormalTok{ status)) }\SpecialCharTok{+}
  \FunctionTok{geom\_col}\NormalTok{(}\AttributeTok{position =} \FunctionTok{position\_dodge}\NormalTok{(}\AttributeTok{width=}\NormalTok{.}\DecValTok{9}\NormalTok{)) }\SpecialCharTok{+}
  \FunctionTok{geom\_text}\NormalTok{(}\FunctionTok{aes}\NormalTok{(}\AttributeTok{label =}\NormalTok{ n),}
            \AttributeTok{position =} \FunctionTok{position\_dodge}\NormalTok{(}\AttributeTok{width=}\NormalTok{.}\DecValTok{9}\NormalTok{),}
            \AttributeTok{vjust =} \SpecialCharTok{{-}}\NormalTok{.}\DecValTok{05}\NormalTok{) }\SpecialCharTok{+}
  \FunctionTok{labs}\NormalTok{(}\AttributeTok{title =} \StringTok{"Loyalty Status by Gender (Side{-}by{-}Side Count)"}\NormalTok{,}
       \AttributeTok{x =} \StringTok{"Gender"}\NormalTok{,}
       \AttributeTok{y =} \StringTok{"Count"}\NormalTok{,}
       \AttributeTok{fill =} \StringTok{"Status"}\NormalTok{)}
\end{Highlighting}
\end{Shaded}

\pandocbounded{\includegraphics[keepaspectratio]{MKT4320_R_Tutorial_files/figure-latex/ggplot-sidebyside-barchart_01-1.pdf}}

\begin{Shaded}
\begin{Highlighting}[]
\NormalTok{airlinesat\_small }\SpecialCharTok{\%\textgreater{}\%}
  \FunctionTok{group\_by}\NormalTok{(status, flight\_purpose) }\SpecialCharTok{\%\textgreater{}\%}
  \FunctionTok{summarise}\NormalTok{(}\AttributeTok{n =} \FunctionTok{n}\NormalTok{()) }\SpecialCharTok{\%\textgreater{}\%}
  \FunctionTok{mutate}\NormalTok{(}\AttributeTok{perc =} \DecValTok{100}\SpecialCharTok{*}\NormalTok{n}\SpecialCharTok{/}\FunctionTok{sum}\NormalTok{(n)) }\SpecialCharTok{\%\textgreater{}\%}
  \FunctionTok{ggplot}\NormalTok{(}\FunctionTok{aes}\NormalTok{(}\AttributeTok{x =}\NormalTok{ status, }\AttributeTok{y =}\NormalTok{ perc, }\AttributeTok{fill =}\NormalTok{ flight\_purpose)) }\SpecialCharTok{+}
  \FunctionTok{geom\_col}\NormalTok{(}\AttributeTok{position =} \FunctionTok{position\_dodge}\NormalTok{(}\AttributeTok{width=}\NormalTok{.}\DecValTok{9}\NormalTok{)) }\SpecialCharTok{+}
  \FunctionTok{geom\_text}\NormalTok{(}\FunctionTok{aes}\NormalTok{(}\AttributeTok{label =} \FunctionTok{round}\NormalTok{(perc, }\DecValTok{2}\NormalTok{)),}
            \AttributeTok{position =} \FunctionTok{position\_dodge}\NormalTok{(}\AttributeTok{width=}\NormalTok{.}\DecValTok{9}\NormalTok{),}
            \AttributeTok{vjust =} \SpecialCharTok{{-}}\NormalTok{.}\DecValTok{05}\NormalTok{) }\SpecialCharTok{+}
  \FunctionTok{labs}\NormalTok{(}\AttributeTok{title =} \StringTok{"Flight Purpose by Loyalty Status (Side{-}by{-}Side Percent)"}\NormalTok{,}
       \AttributeTok{x =} \StringTok{"Loyalty Status"}\NormalTok{,}
       \AttributeTok{y =} \StringTok{"Percent"}\NormalTok{,}
       \AttributeTok{fill =} \StringTok{"Flight Purpose"}\NormalTok{)}
\end{Highlighting}
\end{Shaded}

\pandocbounded{\includegraphics[keepaspectratio]{MKT4320_R_Tutorial_files/figure-latex/ggplot-sidebyside-barchart_01-2.pdf}}

\subsubsection{Side-by-Side Bar Chart with Continuous Variable}\label{side-by-side-bar-chart-with-continuous-variable}

We can also calculate a summary statistic (e.g., mean, median, etc.) for a continuous variable and use that to create a side-by-side bar chart of a discrete variable or two discrete variables.

\begin{Shaded}
\begin{Highlighting}[]
\NormalTok{airlinesat\_small }\SpecialCharTok{\%\textgreater{}\%}
  \FunctionTok{group\_by}\NormalTok{(status) }\SpecialCharTok{\%\textgreater{}\%}
  \FunctionTok{summarise}\NormalTok{(}\AttributeTok{mean=}\FunctionTok{mean}\NormalTok{(nflights)) }\SpecialCharTok{\%\textgreater{}\%}
  \FunctionTok{ggplot}\NormalTok{(}\FunctionTok{aes}\NormalTok{(}\AttributeTok{x =}\NormalTok{ status, }\AttributeTok{y =}\NormalTok{ mean)) }\SpecialCharTok{+}
  \FunctionTok{geom\_col}\NormalTok{(}\AttributeTok{position =} \FunctionTok{position\_dodge}\NormalTok{(}\AttributeTok{width=}\NormalTok{.}\DecValTok{9}\NormalTok{)) }\SpecialCharTok{+}
  \FunctionTok{geom\_text}\NormalTok{(}\FunctionTok{aes}\NormalTok{(}\AttributeTok{label =} \FunctionTok{round}\NormalTok{(mean, }\DecValTok{2}\NormalTok{)),}
            \AttributeTok{position =} \FunctionTok{position\_dodge}\NormalTok{(}\AttributeTok{width=}\NormalTok{.}\DecValTok{9}\NormalTok{),}
            \AttributeTok{vjust =}\NormalTok{ .}\DecValTok{95}\NormalTok{,}
            \AttributeTok{color=}\StringTok{"white"}\NormalTok{,}
            \AttributeTok{size=}\DecValTok{8}\NormalTok{) }\SpecialCharTok{+}
  \FunctionTok{labs}\NormalTok{(}\AttributeTok{title =} \StringTok{"Mean Number Flights by Loyalty Status"}\NormalTok{,}
       \AttributeTok{x =} \StringTok{"Loyalty Status"}\NormalTok{,}
       \AttributeTok{y =} \StringTok{"Mean Number of Flights"}\NormalTok{)}
\end{Highlighting}
\end{Shaded}

\pandocbounded{\includegraphics[keepaspectratio]{MKT4320_R_Tutorial_files/figure-latex/ggplot-sidebyside-barchart_02-1.pdf}}

\begin{Shaded}
\begin{Highlighting}[]
\NormalTok{airlinesat\_small }\SpecialCharTok{\%\textgreater{}\%}
  \FunctionTok{group\_by}\NormalTok{(status, flight\_purpose) }\SpecialCharTok{\%\textgreater{}\%}
  \FunctionTok{summarise}\NormalTok{(}\AttributeTok{mean =} \FunctionTok{mean}\NormalTok{(nflights)) }\SpecialCharTok{\%\textgreater{}\%}
  \FunctionTok{ggplot}\NormalTok{(}\FunctionTok{aes}\NormalTok{(}\AttributeTok{x =}\NormalTok{ status, }\AttributeTok{y =}\NormalTok{ mean, }\AttributeTok{fill =}\NormalTok{ flight\_purpose)) }\SpecialCharTok{+}
  \FunctionTok{geom\_col}\NormalTok{(}\AttributeTok{position =} \FunctionTok{position\_dodge}\NormalTok{(}\AttributeTok{width=}\NormalTok{.}\DecValTok{9}\NormalTok{)) }\SpecialCharTok{+}
  \FunctionTok{geom\_text}\NormalTok{(}\FunctionTok{aes}\NormalTok{(}\AttributeTok{label =} \FunctionTok{round}\NormalTok{(mean, }\DecValTok{2}\NormalTok{)),}
            \AttributeTok{position =} \FunctionTok{position\_dodge}\NormalTok{(}\AttributeTok{width=}\NormalTok{.}\DecValTok{9}\NormalTok{),}
            \AttributeTok{vjust =}\NormalTok{ .}\DecValTok{95}\NormalTok{,}
            \AttributeTok{color=}\StringTok{"white"}\NormalTok{,}
            \AttributeTok{size=}\DecValTok{6}\NormalTok{) }\SpecialCharTok{+}
  \FunctionTok{labs}\NormalTok{(}\AttributeTok{title =} \StringTok{"Flight Number of Flights by Loyalty Status and Flight Purpose"}\NormalTok{,}
       \AttributeTok{x =} \StringTok{"Loyalty Status"}\NormalTok{,}
       \AttributeTok{y =} \StringTok{"Mean Number of Flights"}\NormalTok{,}
       \AttributeTok{fill =} \StringTok{"Flight Purpose"}\NormalTok{)}
\end{Highlighting}
\end{Shaded}

\pandocbounded{\includegraphics[keepaspectratio]{MKT4320_R_Tutorial_files/figure-latex/ggplot-sidebyside-barchart_02-2.pdf}}

\begin{center}\rule{0.5\linewidth}{0.5pt}\end{center}

\section{Summary}\label{summary}

Base R graphics provide a simple way to create quick plots, while \texttt{ggplot2} offers
a more flexible and extensible framework for data visualization. In practice,
Base R is useful for fast checks, and \texttt{ggplot2} is preferred for exploratory
analysis and communication.

\begin{center}\rule{0.5\linewidth}{0.5pt}\end{center}

\section{What's Next}\label{whats-next-7}

In the next chapter, we move from visual summaries to formal modeling. You will learn how to use linear regression to quantify relationships between variables, estimate marginal effects, and make predictions while holding other factors constant.

\chapter{Linear Regression}\label{linear-regression}

Linear regression is one of the most widely used tools in marketing analytics.
It allows us to quantify relationships between an outcome variable and one or
more predictors, helping us explain variation, estimate marginal effects, and
generate predictions.

In earlier chapters, we summarized data numerically and visually. In this
chapter, we move beyond description to modeling relationships between variables.

Throughout this chapter, we use the \texttt{airlinesat\_small} dataset and focus on
interpretation rather than mathematical derivation.

\begin{center}\rule{0.5\linewidth}{0.5pt}\end{center}

\section{The Linear Regression Model}\label{the-linear-regression-model}

A linear regression model relates an outcome variable to one or more predictors.
In R, linear regression models are estimated using the \texttt{lm()} function.

The general structure is:

\begin{Shaded}
\begin{Highlighting}[]
\FunctionTok{lm}\NormalTok{(outcome }\SpecialCharTok{\textasciitilde{}}\NormalTok{ predictors, }\AttributeTok{data =}\NormalTok{ dataset)}
\end{Highlighting}
\end{Shaded}

\begin{center}\rule{0.5\linewidth}{0.5pt}\end{center}

\section{Simple Linear Regression}\label{simple-linear-regression}

We begin with a simple linear regression model containing a single predictor.

In this example, we model Net Promoter Score (\texttt{nps}) as a function of the number
of flights taken (\texttt{nflights}). If we don't ask for a summary we \textbf{only} get the coefficients. If we save the result as an object and then ask for a summary, we get the full, expected results.

\begin{Shaded}
\begin{Highlighting}[]
\FunctionTok{lm}\NormalTok{(nps }\SpecialCharTok{\textasciitilde{}}\NormalTok{ nflights, }\AttributeTok{data =}\NormalTok{ airlinesat\_small)}
\end{Highlighting}
\end{Shaded}

\begin{verbatim}

Call:
lm(formula = nps ~ nflights, data = airlinesat_small)

Coefficients:
(Intercept)     nflights  
    7.65848     -0.01031  
\end{verbatim}

\begin{Shaded}
\begin{Highlighting}[]
\NormalTok{model\_simple }\OtherTok{\textless{}{-}} \FunctionTok{lm}\NormalTok{(nps }\SpecialCharTok{\textasciitilde{}}\NormalTok{ nflights, }\AttributeTok{data =}\NormalTok{ airlinesat\_small)}
\FunctionTok{summary}\NormalTok{(model\_simple)}
\end{Highlighting}
\end{Shaded}

\begin{verbatim}

Call:
lm(formula = nps ~ nflights, data = airlinesat_small)

Residuals:
    Min      1Q  Median      3Q     Max 
-6.6482 -1.4318  0.4137  1.5476  8.0512 

Coefficients:
             Estimate Std. Error t value Pr(>|t|)    
(Intercept)  7.658478   0.085356   89.72  < 2e-16 ***
nflights    -0.010306   0.003518   -2.93  0.00347 ** 
---
Signif. codes:  0 '***' 0.001 '**' 0.01 '*' 0.05 '.' 0.1 ' ' 1

Residual standard error: 2.321 on 1063 degrees of freedom
Multiple R-squared:  0.00801,   Adjusted R-squared:  0.007077 
F-statistic: 8.583 on 1 and 1063 DF,  p-value: 0.003465
\end{verbatim}

Interpretation focuses on:

\begin{itemize}
\tightlist
\item
  The intercept: the predicted net promoter score (NPS) when the number of flights is zero
\item
  The slope on \texttt{nflights}: the expected change in NPS for one additional flight
\item
  R-squared: the proportion of variation in NPS explained by the model
\end{itemize}

\subsection{``Nice'' Output}\label{nice-output}

For nicer looking output, we can use either the \texttt{summ} function from the \texttt{jtools} package. It easily allows us to adjust the number of digits shown (\texttt{digits\ =\ \#} option). It can also show the standardized beta coefficients by using the \texttt{scale=TRUE} and \texttt{transform.response=TRUE} options together.

\begin{Shaded}
\begin{Highlighting}[]
\FunctionTok{summ}\NormalTok{(model\_simple, }\AttributeTok{digits =} \DecValTok{4}\NormalTok{)}
\end{Highlighting}
\end{Shaded}

\begin{table}[!h]
\centering
\begin{tabular}{lr}
\toprule
\cellcolor{gray!10}{Observations} & \cellcolor{gray!10}{1065}\\
Dependent variable & nps\\
\cellcolor{gray!10}{Type} & \cellcolor{gray!10}{OLS linear regression}\\
\bottomrule
\end{tabular}
\end{table} \begin{table}[!h]
\centering
\begin{tabular}{lr}
\toprule
\cellcolor{gray!10}{F(1,1063)} & \cellcolor{gray!10}{8.5832}\\
R² & 0.0080\\
\cellcolor{gray!10}{Adj. R²} & \cellcolor{gray!10}{0.0071}\\
\bottomrule
\end{tabular}
\end{table} \begin{table}[!h]
\centering
\begin{threeparttable}
\begin{tabular}{lrrrr}
\toprule
  & Est. & S.E. & t val. & p\\
\midrule
\cellcolor{gray!10}{(Intercept)} & \cellcolor{gray!10}{7.6585} & \cellcolor{gray!10}{0.0854} & \cellcolor{gray!10}{89.7243} & \cellcolor{gray!10}{0.0000}\\
nflights & -0.0103 & 0.0035 & -2.9297 & 0.0035\\
\bottomrule
\end{tabular}
\begin{tablenotes}
\item Standard errors: OLS
\end{tablenotes}
\end{threeparttable}
\end{table}

\begin{Shaded}
\begin{Highlighting}[]
\FunctionTok{summ}\NormalTok{(model\_simple, }\AttributeTok{digits =} \DecValTok{4}\NormalTok{, }\AttributeTok{scale =} \ConstantTok{TRUE}\NormalTok{, }\AttributeTok{transform.response =} \ConstantTok{TRUE}\NormalTok{)}
\end{Highlighting}
\end{Shaded}

\begin{table}[!h]
\centering
\begin{tabular}{lr}
\toprule
\cellcolor{gray!10}{Observations} & \cellcolor{gray!10}{1065}\\
Dependent variable & nps\\
\cellcolor{gray!10}{Type} & \cellcolor{gray!10}{OLS linear regression}\\
\bottomrule
\end{tabular}
\end{table} \begin{table}[!h]
\centering
\begin{tabular}{lr}
\toprule
\cellcolor{gray!10}{F(1,1063)} & \cellcolor{gray!10}{8.5832}\\
R² & 0.0080\\
\cellcolor{gray!10}{Adj. R²} & \cellcolor{gray!10}{0.0071}\\
\bottomrule
\end{tabular}
\end{table} \begin{table}[!h]
\centering
\begin{threeparttable}
\begin{tabular}{lrrrr}
\toprule
  & Est. & S.E. & t val. & p\\
\midrule
\cellcolor{gray!10}{(Intercept)} & \cellcolor{gray!10}{0.0000} & \cellcolor{gray!10}{0.0305} & \cellcolor{gray!10}{0.0000} & \cellcolor{gray!10}{1.0000}\\
nflights & -0.0895 & 0.0305 & -2.9297 & 0.0035\\
\bottomrule
\end{tabular}
\begin{tablenotes}
\item Standard errors: OLS; Continuous variables are mean-centered and scaled by 1 s.d.
\end{tablenotes}
\end{threeparttable}
\end{table}

\begin{center}\rule{0.5\linewidth}{0.5pt}\end{center}

\section{Multiple Linear Regression}\label{multiple-linear-regression}

In practice, outcomes are influenced by more than one factor. Multiple linear
regression allows us to include additional predictors to control for other
characteristics.

\subsection{Adding Additional Variables}\label{adding-additional-variables}

\begin{Shaded}
\begin{Highlighting}[]
\NormalTok{model\_multi }\OtherTok{\textless{}{-}} \FunctionTok{lm}\NormalTok{(nps }\SpecialCharTok{\textasciitilde{}}\NormalTok{ nflights }\SpecialCharTok{+}\NormalTok{ age, }\AttributeTok{data =}\NormalTok{ airlinesat\_small)}
\FunctionTok{summ}\NormalTok{(model\_multi, }\AttributeTok{digits =} \DecValTok{4}\NormalTok{)}
\end{Highlighting}
\end{Shaded}

\begin{table}[!h]
\centering
\begin{tabular}{lr}
\toprule
\cellcolor{gray!10}{Observations} & \cellcolor{gray!10}{1065}\\
Dependent variable & nps\\
\cellcolor{gray!10}{Type} & \cellcolor{gray!10}{OLS linear regression}\\
\bottomrule
\end{tabular}
\end{table} \begin{table}[!h]
\centering
\begin{tabular}{lr}
\toprule
\cellcolor{gray!10}{F(2,1062)} & \cellcolor{gray!10}{8.5875}\\
R² & 0.0159\\
\cellcolor{gray!10}{Adj. R²} & \cellcolor{gray!10}{0.0141}\\
\bottomrule
\end{tabular}
\end{table} \begin{table}[!h]
\centering
\begin{threeparttable}
\begin{tabular}{lrrrr}
\toprule
  & Est. & S.E. & t val. & p\\
\midrule
\cellcolor{gray!10}{(Intercept)} & \cellcolor{gray!10}{6.7861} & \cellcolor{gray!10}{0.3106} & \cellcolor{gray!10}{21.8516} & \cellcolor{gray!10}{0.0000}\\
nflights & -0.0091 & 0.0035 & -2.5822 & 0.0100\\
\cellcolor{gray!10}{age} & \cellcolor{gray!10}{0.0170} & \cellcolor{gray!10}{0.0058} & \cellcolor{gray!10}{2.9208} & \cellcolor{gray!10}{0.0036}\\
\bottomrule
\end{tabular}
\begin{tablenotes}
\item Standard errors: OLS
\end{tablenotes}
\end{threeparttable}
\end{table}

When additional predictors are included, coefficients are interpreted as
marginal effects holding other variables constant.

\begin{center}\rule{0.5\linewidth}{0.5pt}\end{center}

\section{Categorical Predictors and Reference Groups}\label{categorical-predictors-and-reference-groups}

Regression models can include categorical predictors. In R, factor variables
are automatically converted into indicator (dummy) variables.

\subsection{Example: Flight Class}\label{example-flight-class}

\begin{Shaded}
\begin{Highlighting}[]
\NormalTok{model\_cat }\OtherTok{\textless{}{-}} \FunctionTok{lm}\NormalTok{(nps }\SpecialCharTok{\textasciitilde{}}\NormalTok{ nflights }\SpecialCharTok{+}\NormalTok{ flight\_class,  }\AttributeTok{data =}\NormalTok{ airlinesat\_small)}
\FunctionTok{summ}\NormalTok{(model\_cat, }\AttributeTok{digits =} \DecValTok{4}\NormalTok{)}
\end{Highlighting}
\end{Shaded}

\begin{table}[!h]
\centering
\begin{tabular}{lr}
\toprule
\cellcolor{gray!10}{Observations} & \cellcolor{gray!10}{1065}\\
Dependent variable & nps\\
\cellcolor{gray!10}{Type} & \cellcolor{gray!10}{OLS linear regression}\\
\bottomrule
\end{tabular}
\end{table} \begin{table}[!h]
\centering
\begin{tabular}{lr}
\toprule
\cellcolor{gray!10}{F(3,1061)} & \cellcolor{gray!10}{4.1443}\\
R² & 0.0116\\
\cellcolor{gray!10}{Adj. R²} & \cellcolor{gray!10}{0.0088}\\
\bottomrule
\end{tabular}
\end{table} \begin{table}[!h]
\centering
\begin{threeparttable}
\begin{tabular}{lrrrr}
\toprule
  & Est. & S.E. & t val. & p\\
\midrule
\cellcolor{gray!10}{(Intercept)} & \cellcolor{gray!10}{7.7273} & \cellcolor{gray!10}{0.1793} & \cellcolor{gray!10}{43.0949} & \cellcolor{gray!10}{0.0000}\\
nflights & -0.0108 & 0.0035 & -3.0691 & 0.0022\\
\cellcolor{gray!10}{flight\_classEconomy} & \cellcolor{gray!10}{-0.0946} & \cellcolor{gray!10}{0.1881} & \cellcolor{gray!10}{-0.5029} & \cellcolor{gray!10}{0.6151}\\
flight\_classFirst & 1.0666 & 0.6232 & 1.7115 & 0.0873\\
\bottomrule
\end{tabular}
\begin{tablenotes}
\item Standard errors: OLS
\end{tablenotes}
\end{threeparttable}
\end{table}

One category is treated as the reference group. Coefficients for other categories
represent differences relative to that reference. In the example above, ``Business'' is treated as the reference group.

To change the reference group, use \texttt{relevel()}. It can be changed temporarily within the call to \texttt{lm()}, or it can be changed permanently in the dataframe.

\begin{Shaded}
\begin{Highlighting}[]
\CommentTok{\# Temporary level chance in formula}
\NormalTok{model\_cat\_relevel }\OtherTok{\textless{}{-}} \FunctionTok{lm}\NormalTok{(nps }\SpecialCharTok{\textasciitilde{}}\NormalTok{ nflights }\SpecialCharTok{+} \FunctionTok{relevel}\NormalTok{(flight\_class, }\AttributeTok{ref=}\StringTok{"Economy"}\NormalTok{),  }
                        \AttributeTok{data =}\NormalTok{ airlinesat\_small)}
\FunctionTok{summ}\NormalTok{(model\_cat\_relevel, }\AttributeTok{digits =} \DecValTok{4}\NormalTok{)}
\end{Highlighting}
\end{Shaded}

\begin{table}[!h]
\centering
\begin{tabular}{lr}
\toprule
\cellcolor{gray!10}{Observations} & \cellcolor{gray!10}{1065}\\
Dependent variable & nps\\
\cellcolor{gray!10}{Type} & \cellcolor{gray!10}{OLS linear regression}\\
\bottomrule
\end{tabular}
\end{table} \begin{table}[!h]
\centering
\begin{tabular}{lr}
\toprule
\cellcolor{gray!10}{F(3,1061)} & \cellcolor{gray!10}{4.1443}\\
R² & 0.0116\\
\cellcolor{gray!10}{Adj. R²} & \cellcolor{gray!10}{0.0088}\\
\bottomrule
\end{tabular}
\end{table} \begin{table}[!h]
\centering
\begin{threeparttable}
\begin{tabular}{lrrrr}
\toprule
  & Est. & S.E. & t val. & p\\
\midrule
\cellcolor{gray!10}{(Intercept)} & \cellcolor{gray!10}{7.6327} & \cellcolor{gray!10}{0.0907} & \cellcolor{gray!10}{84.1145} & \cellcolor{gray!10}{0.0000}\\
nflights & -0.0108 & 0.0035 & -3.0691 & 0.0022\\
\cellcolor{gray!10}{relevel(flight\_class, ref = "Economy")Business} & \cellcolor{gray!10}{0.0946} & \cellcolor{gray!10}{0.1881} & \cellcolor{gray!10}{0.5029} & \cellcolor{gray!10}{0.6151}\\
relevel(flight\_class, ref = "Economy")First & 1.1612 & 0.6052 & 1.9187 & 0.0553\\
\bottomrule
\end{tabular}
\begin{tablenotes}
\item Standard errors: OLS
\end{tablenotes}
\end{threeparttable}
\end{table}

\begin{Shaded}
\begin{Highlighting}[]
\NormalTok{airlinesat\_small}\SpecialCharTok{$}\NormalTok{flight\_class }\OtherTok{\textless{}{-}} \FunctionTok{relevel}\NormalTok{(airlinesat\_small}\SpecialCharTok{$}\NormalTok{flight\_class, }
                                         \AttributeTok{ref =} \StringTok{"Economy"}\NormalTok{)}
\end{Highlighting}
\end{Shaded}

\begin{center}\rule{0.5\linewidth}{0.5pt}\end{center}

\section{Interaction Effects}\label{interaction-effects}

An interaction allows the effect of one predictor to depend on the level of
another predictor. Interactions can be included in the formula using \texttt{*} or \texttt{:}

\begin{itemize}
\tightlist
\item
  \texttt{*} will include the interaction term \textbf{AND} each main effect
\item
  \texttt{:} will include \textbf{ONLY} the interaction term
\item
  Examples:

  \begin{itemize}
  \tightlist
  \item
    \texttt{y\ \textasciitilde{}\ x1\ +\ x2\ *\ x3} is the same as:\\
    \(y = x1 + x2 + x3 + (x2 \times x3)\)
  \item
    \texttt{y\ \textasciitilde{}\ x1\ +\ x2:x3\ is\ the\ same\ as:\ \ \ \$y\ =\ x1\ +\ (x2\ \textbackslash{}times\ x3)\$\ 0}y \textasciitilde{} x1 + x2 + x2:x3` is the same as:\\
    \(y = x1 + x2 + (x2 \times x3)\)
  \end{itemize}
\end{itemize}

\subsection{Example: Flights and Frequent Flier Status}\label{example-flights-and-frequent-flier-status}

\begin{Shaded}
\begin{Highlighting}[]
\NormalTok{model\_inter }\OtherTok{\textless{}{-}} \FunctionTok{lm}\NormalTok{(nps }\SpecialCharTok{\textasciitilde{}}\NormalTok{ age }\SpecialCharTok{+}\NormalTok{ nflights }\SpecialCharTok{*}\NormalTok{ status, }\AttributeTok{data =}\NormalTok{ airlinesat\_small)}
\FunctionTok{summ}\NormalTok{(model\_inter, }\AttributeTok{digits =} \DecValTok{4}\NormalTok{)}
\end{Highlighting}
\end{Shaded}

\begin{table}[!h]
\centering
\begin{tabular}{lr}
\toprule
\cellcolor{gray!10}{Observations} & \cellcolor{gray!10}{1065}\\
Dependent variable & nps\\
\cellcolor{gray!10}{Type} & \cellcolor{gray!10}{OLS linear regression}\\
\bottomrule
\end{tabular}
\end{table} \begin{table}[!h]
\centering
\begin{tabular}{lr}
\toprule
\cellcolor{gray!10}{F(6,1058)} & \cellcolor{gray!10}{5.0753}\\
R² & 0.0280\\
\cellcolor{gray!10}{Adj. R²} & \cellcolor{gray!10}{0.0225}\\
\bottomrule
\end{tabular}
\end{table} \begin{table}[!h]
\centering
\begin{threeparttable}
\begin{tabular}{lrrrr}
\toprule
  & Est. & S.E. & t val. & p\\
\midrule
\cellcolor{gray!10}{(Intercept)} & \cellcolor{gray!10}{6.8771} & \cellcolor{gray!10}{0.3136} & \cellcolor{gray!10}{21.9308} & \cellcolor{gray!10}{0.0000}\\
age & 0.0154 & 0.0058 & 2.6320 & 0.0086\\
\cellcolor{gray!10}{nflights} & \cellcolor{gray!10}{0.0003} & \cellcolor{gray!10}{0.0046} & \cellcolor{gray!10}{0.0565} & \cellcolor{gray!10}{0.9550}\\
statusGold & 0.1997 & 0.3155 & 0.6330 & 0.5269\\
\cellcolor{gray!10}{statusSilver} & \cellcolor{gray!10}{0.0803} & \cellcolor{gray!10}{0.2528} & \cellcolor{gray!10}{0.3178} & \cellcolor{gray!10}{0.7507}\\
\addlinespace
nflights:statusGold & -0.0158 & 0.0104 & -1.5241 & 0.1278\\
\cellcolor{gray!10}{nflights:statusSilver} & \cellcolor{gray!10}{-0.0266} & \cellcolor{gray!10}{0.0097} & \cellcolor{gray!10}{-2.7348} & \cellcolor{gray!10}{0.0063}\\
\bottomrule
\end{tabular}
\begin{tablenotes}
\item Standard errors: OLS
\end{tablenotes}
\end{threeparttable}
\end{table}

\begin{center}\rule{0.5\linewidth}{0.5pt}\end{center}

\section{\texorpdfstring{Margin Plots with \texttt{easy\_mp}}{Margin Plots with easy\_mp}}\label{sec-margin-plots}

Rather than using exploratory plots or manually creating margin plots, we use the \texttt{easy\_mp()} function from the
\texttt{MKT4320BGSU} package to visualize predicted values and marginal effects.

\subsection{\texorpdfstring{The \texttt{easy\_mp()} Function}{The easy\_mp() Function}}\label{the-easy_mp-function}

This function creates marginal effects plots for a focal predictor (with or without an interaction) from a linear regression (\texttt{lm}) or binary logistic regression (\texttt{glm} with \texttt{family\ =\ "binomial"}).

Usage: \texttt{easy\_mp(model,\ focal,\ int\ =\ NULL)}

Arguments:

\begin{itemize}
\tightlist
\item
  \texttt{model} is a fitted \texttt{lm} model or binary logistic \texttt{glm} model (\texttt{family\ =\ "binomial"}).
\item
  \texttt{focal} is the name of the focal predictor variable in quotations
\item
  \texttt{int} is the name of the interaction variable in quotations. Can be excluded if only the focal variable is wanted or no interaction exists.
\end{itemize}

Returns:
- \texttt{\$plot} is the margin plot
- \texttt{\$ptable} is the marginal effects table used to produce the plot

\subsection{WITHOUT Interactions}\label{without-interactions}

Below are examples of a margin plot for a continuous independent variable and for a categorical/factor independent variable. Note how the returned \texttt{\$plot} object is a \texttt{ggplot} that can be modified by adding layers.

\begin{Shaded}
\begin{Highlighting}[]
\NormalTok{model\_mp\_nointer }\OtherTok{\textless{}{-}} \FunctionTok{lm}\NormalTok{(nps }\SpecialCharTok{\textasciitilde{}}\NormalTok{ age }\SpecialCharTok{+}\NormalTok{ gender, }\AttributeTok{data =}\NormalTok{ airlinesat\_small)}

\CommentTok{\# Continuous Focal WITHOUT Interaction}
\NormalTok{mp\_age }\OtherTok{\textless{}{-}} \FunctionTok{easy\_mp}\NormalTok{(model\_mp\_nointer, }\AttributeTok{focal=}\StringTok{"age"}\NormalTok{)}
\NormalTok{mp\_age}\SpecialCharTok{$}\NormalTok{plot }\SpecialCharTok{+}
  \FunctionTok{labs}\NormalTok{(}\AttributeTok{title=}\StringTok{"No Interaction: Continuous Variable"}\NormalTok{)}
\end{Highlighting}
\end{Shaded}

\pandocbounded{\includegraphics[keepaspectratio]{MKT4320_R_Tutorial_files/figure-latex/lr-easymp-main-1.pdf}}

\begin{Shaded}
\begin{Highlighting}[]
\CommentTok{\# Factor Focal WITHOUT Interaction}
\NormalTok{mp\_gender }\OtherTok{\textless{}{-}} \FunctionTok{easy\_mp}\NormalTok{(model\_mp\_nointer, }\AttributeTok{focal=}\StringTok{"gender"}\NormalTok{)}
\NormalTok{mp\_gender}\SpecialCharTok{$}\NormalTok{plot }\SpecialCharTok{+}
  \FunctionTok{labs}\NormalTok{(}\AttributeTok{title=}\StringTok{"No Interaction: Factor Variable"}\NormalTok{)}
\end{Highlighting}
\end{Shaded}

\pandocbounded{\includegraphics[keepaspectratio]{MKT4320_R_Tutorial_files/figure-latex/lr-easymp-main-2.pdf}}

\subsection{WITH Interactions}\label{with-interactions}

Ultimately, there are four types of margin plots that can be created depending on the focal variable type and the interaction variable type:

\begin{itemize}
\tightlist
\item
  Continuous Focal IV \(\times\) Continuous Interaction IV
\item
  Continuous Focal IV \(\times\) Factor Interaction IV
\item
  Factor Focal IV \(\times\) Continuous Interaction IV
\item
  Factor Focal IV \(\times\) Factor Interaction IV
\end{itemize}

When the interaction term is continuous, the plot is created with representative values of the continuous variable (roughly the 1st percentile, the 99th percentile, and two evenly spaced values between those two).

\begin{Shaded}
\begin{Highlighting}[]
\NormalTok{model\_mp\_inter }\OtherTok{\textless{}{-}} \FunctionTok{lm}\NormalTok{(nps }\SpecialCharTok{\textasciitilde{}}\NormalTok{ age}\SpecialCharTok{*}\NormalTok{gender }\SpecialCharTok{+}\NormalTok{ age}\SpecialCharTok{*}\NormalTok{reputation }\SpecialCharTok{+}\NormalTok{ gender}\SpecialCharTok{*}\NormalTok{status, }\AttributeTok{data =}\NormalTok{ airlinesat\_small)}

\CommentTok{\# Continuous Focal WITH Continuous Interaction}
\NormalTok{mp\_age\_reputation }\OtherTok{\textless{}{-}} \FunctionTok{easy\_mp}\NormalTok{(model\_mp\_inter, }\AttributeTok{focal =} \StringTok{"age"}\NormalTok{, }\AttributeTok{int =} \StringTok{"reputation"}\NormalTok{)}
\NormalTok{mp\_age\_reputation}\SpecialCharTok{$}\NormalTok{plot }\SpecialCharTok{+}
  \FunctionTok{labs}\NormalTok{(}\AttributeTok{title=}\StringTok{"Interaction: Continuous Focal IV by Continuous Interaction IV"}\NormalTok{)}
\end{Highlighting}
\end{Shaded}

\pandocbounded{\includegraphics[keepaspectratio]{MKT4320_R_Tutorial_files/figure-latex/lr-easymp-interaction-1.pdf}}

\begin{Shaded}
\begin{Highlighting}[]
\CommentTok{\# Continuous Focal WITH Factor Interaction}
\NormalTok{mp\_age\_gender }\OtherTok{\textless{}{-}} \FunctionTok{easy\_mp}\NormalTok{(model\_mp\_inter, }\AttributeTok{focal =} \StringTok{"age"}\NormalTok{, }\AttributeTok{int =} \StringTok{"gender"}\NormalTok{)}
\NormalTok{mp\_age\_gender}\SpecialCharTok{$}\NormalTok{plot }\SpecialCharTok{+}
  \FunctionTok{labs}\NormalTok{(}\AttributeTok{title=}\StringTok{"Interaction: Continuous Focal IV by Factor Interaction IV"}\NormalTok{)}
\end{Highlighting}
\end{Shaded}

\pandocbounded{\includegraphics[keepaspectratio]{MKT4320_R_Tutorial_files/figure-latex/lr-easymp-interaction-2.pdf}}

\begin{Shaded}
\begin{Highlighting}[]
\CommentTok{\# Factor Focal WITH Continuous Interaction}
\NormalTok{mp\_gender\_age }\OtherTok{\textless{}{-}} \FunctionTok{easy\_mp}\NormalTok{(model\_mp\_inter, }\AttributeTok{focal =} \StringTok{"gender"}\NormalTok{, }\AttributeTok{int =} \StringTok{"age"}\NormalTok{)}
\NormalTok{mp\_gender\_age}\SpecialCharTok{$}\NormalTok{plot }\SpecialCharTok{+}
  \FunctionTok{labs}\NormalTok{(}\AttributeTok{title=}\StringTok{"Interaction: Factor Focal IV by Continuous Interaction IV"}\NormalTok{)}
\end{Highlighting}
\end{Shaded}

\pandocbounded{\includegraphics[keepaspectratio]{MKT4320_R_Tutorial_files/figure-latex/lr-easymp-interaction-3.pdf}}

\begin{Shaded}
\begin{Highlighting}[]
\CommentTok{\# Factor Focal WITH Factor Interaction}
\NormalTok{mp\_gender\_status }\OtherTok{\textless{}{-}} \FunctionTok{easy\_mp}\NormalTok{(model\_mp\_inter, }\AttributeTok{focal =} \StringTok{"gender"}\NormalTok{, }\AttributeTok{int =} \StringTok{"status"}\NormalTok{)}
\NormalTok{mp\_gender\_status}\SpecialCharTok{$}\NormalTok{plot }\SpecialCharTok{+}
  \FunctionTok{labs}\NormalTok{(}\AttributeTok{title=}\StringTok{"Interaction: Factor Focal IV by Factor Interaction IV"}\NormalTok{)}
\end{Highlighting}
\end{Shaded}

\pandocbounded{\includegraphics[keepaspectratio]{MKT4320_R_Tutorial_files/figure-latex/lr-easymp-interaction-4.pdf}}

\begin{center}\rule{0.5\linewidth}{0.5pt}\end{center}

\section{Prediction}\label{prediction}

Regression models can also be used for prediction. The function \texttt{predict()} can be used to predict the DV based on values of the IVs. We can either: (1) predict the expected value of the DV for each observation in the data, or (2) predict the expected value of the DV for new values of the IV(s). To use this function for (2), we must pass a data frame of values to the function, where the data frame contains ALL of the IVs and the value for each IV that we want.

Suppose we wanted to predict, with a confidence interval, the \texttt{nps} of someone that is 45 years old and had 25 flights on the airline, and also someone that is 25 years old and had 45 flights on the airline, based on our \texttt{model\_multi\ \textless{}-\ lm(nps\ \textasciitilde{}\ nflights\ +\ age,\ data\ =\ airlinesat\_small)} from above. First, we create the data frame of values

\begin{Shaded}
\begin{Highlighting}[]
\NormalTok{values }\OtherTok{\textless{}{-}} \FunctionTok{data.frame}\NormalTok{(}\AttributeTok{age=}\FunctionTok{c}\NormalTok{(}\DecValTok{45}\NormalTok{, }\DecValTok{25}\NormalTok{), }\AttributeTok{nflights=}\FunctionTok{c}\NormalTok{(}\DecValTok{25}\NormalTok{, }\DecValTok{45}\NormalTok{))}
\NormalTok{values}
\end{Highlighting}
\end{Shaded}

\begin{verbatim}
  age nflights
1  45       25
2  25       45
\end{verbatim}

Second, the data frame is passed to the \texttt{predict()} function with confidence intervals requested.

\begin{Shaded}
\begin{Highlighting}[]
\FunctionTok{predict}\NormalTok{(model\_multi, values, }\AttributeTok{interval=}\StringTok{"confidence"}\NormalTok{)}
\end{Highlighting}
\end{Shaded}

\begin{verbatim}
       fit      lwr      upr
1 7.322601 7.153951 7.491252
2 6.800657 6.431058 7.170255
\end{verbatim}

\begin{center}\rule{0.5\linewidth}{0.5pt}\end{center}

\section{What's Next}\label{whats-next-8}

In this chapter, you learned how to use linear regression to model relationships when the outcome variable is continuous, such as satisfaction or Net Promoter Score. Linear regression works well when predicted values can reasonably fall anywhere along a numeric scale.

Many marketing outcomes, however, are binary---for example, purchase vs.~no purchase, churn vs.~retention, or click vs.~no click. In these cases, linear regression is no longer appropriate.

In the next chapter, we introduce binary logistic regression, a modeling approach designed specifically for yes/no outcomes. You will learn how to estimate probabilities, interpret coefficients and marginal effects, and evaluate model performance in a way that is well-suited to common marketing decision problems.

\chapter{Binary Logistic Regression}\label{binary-logistic-regression}

\section{Why Logistic Regression in Marketing Analytics}\label{why-logistic-regression-in-marketing-analytics}

Many important marketing decisions involve \textbf{binary outcomes}:

\begin{itemize}
\tightlist
\item
  Did the customer buy or not?
\item
  Did the customer respond to a promotion?
\item
  Did the customer churn?
\end{itemize}

In these cases, the dependent variable takes on only two possible values. A standard linear regression model is not appropriate because it can produce predicted values below 0 or above 1 and does not model probabilities correctly.

\textbf{Binary logistic regression} is designed specifically for situations where the outcome is binary. Rather than predicting the outcome directly, logistic regression models the \textbf{probability} that the outcome occurs.

From a marketing perspective, this is powerful: instead of simply predicting ``buy'' or ``not buy,'' we can estimate how \emph{likely} a customer is to buy and then use those probabilities for targeting and decision-making.

\begin{center}\rule{0.5\linewidth}{0.5pt}\end{center}

\section{The Direct Marketing Data}\label{the-direct-marketing-data}

In this chapter, we use the \texttt{directmktg} dataset, which contains information on a direct marketing campaign.

The dataset consists of \textbf{400 prospects} and the following variables:

\begin{itemize}
\tightlist
\item
  \texttt{userid}: Unique identifier for each prospect\\
\item
  \texttt{age}: Prospect age in years\\
\item
  \texttt{gender}: Prospect gender (coded as provided)\\
\item
  \texttt{salary}: Prospect salary in thousands of dollars\\
\item
  \texttt{buy}: Purchase decision (``Yes'' or ``No'')
\end{itemize}

The marketing objective is:

Predict whether a prospect will purchase and estimate the probability of purchase.

\begin{Shaded}
\begin{Highlighting}[]
\FunctionTok{data}\NormalTok{(directmktg)}
\end{Highlighting}
\end{Shaded}

\begin{center}\rule{0.5\linewidth}{0.5pt}\end{center}

\section{Splitting Data into Training and Test Samples}\label{splitting-data-into-training-and-test-samples}

To evaluate predictive performance, we split the data into training and test samples using \texttt{splitsample()} from the \texttt{MKT4320BGSU} package.

Usage:

\begin{itemize}
\tightlist
\item
  \texttt{splitsample(data,\ outcome\ =\ NULL,\ group\ =\ NULL,\ choice\ =\ NULL,\ alt\ =\ NULL,}\strut \\
  \texttt{p\ =\ 0.75,\ seed\ =\ 4320)}
\item
  where:

  \begin{itemize}
  \tightlist
  \item
    \texttt{data} is the data frame to split.
  \item
    \texttt{outcome} is the outcome variable used for stratification. Required when group is \texttt{NULL}. Optional when group is provided. For binary logistic regression, it is required.
  \item
    \texttt{group} is NOT USED FOR BINARY LOGISTIC
  \item
    \texttt{choice} is NOT USED FOR BINARY LOGISTIC
  \item
    \texttt{alt} is NOT USED FOR BINARY LOGISTIC
  \item
    \texttt{p} is the proportion of observations to place in the training set. Must be strictly between 0 and 1. Default is 0.75.
  \item
    \texttt{seed} is the random seed for reproducibility. Default is 4320.
  \end{itemize}
\end{itemize}

\begin{Shaded}
\begin{Highlighting}[]
\NormalTok{sp }\OtherTok{\textless{}{-}} \FunctionTok{splitsample}\NormalTok{(directmktg, }\AttributeTok{outcome =} \StringTok{"buy"}\NormalTok{)}
\NormalTok{train }\OtherTok{\textless{}{-}}\NormalTok{ sp}\SpecialCharTok{$}\NormalTok{train}
\NormalTok{test  }\OtherTok{\textless{}{-}}\NormalTok{ sp}\SpecialCharTok{$}\NormalTok{test}
\end{Highlighting}
\end{Shaded}

The training data are used to estimate the model. The test data are reserved for out-of-sample evaluation.

\begin{center}\rule{0.5\linewidth}{0.5pt}\end{center}

\section{Estimating a Binary Logistic Regression Model}\label{estimating-a-binary-logistic-regression-model}

We estimate a logistic regression model using \texttt{glm()} with \texttt{family\ =\ "binomial"}. Remember, we want to estimate the model with the \texttt{train} data we just created.

\begin{Shaded}
\begin{Highlighting}[]
\NormalTok{mod }\OtherTok{\textless{}{-}} \FunctionTok{glm}\NormalTok{(buy }\SpecialCharTok{\textasciitilde{}}\NormalTok{ age }\SpecialCharTok{+}\NormalTok{ gender }\SpecialCharTok{+}\NormalTok{ salary, }\AttributeTok{data =}\NormalTok{ train, }\AttributeTok{family =} \StringTok{"binomial"}\NormalTok{)}
\FunctionTok{summary}\NormalTok{(mod)}
\end{Highlighting}
\end{Shaded}

\begin{verbatim}

Call:
glm(formula = buy ~ age + gender + salary, family = "binomial", 
    data = train)

Coefficients:
               Estimate Std. Error z value Pr(>|z|)    
(Intercept)  -13.166094   1.621695  -8.119 4.71e-16 ***
age            0.250215   0.032095   7.796 6.38e-15 ***
genderFemale  -0.406881   0.349811  -1.163    0.245    
salary         0.040632   0.006742   6.026 1.68e-09 ***
---
Signif. codes:  0 '***' 0.001 '**' 0.01 '*' 0.05 '.' 0.1 ' ' 1

(Dispersion parameter for binomial family taken to be 1)

    Null deviance: 392.94  on 300  degrees of freedom
Residual deviance: 210.68  on 297  degrees of freedom
AIC: 218.68

Number of Fisher Scoring iterations: 6
\end{verbatim}

The coefficients indicate how each variable affects the \textbf{log-odds} of purchase. The sign of each coefficient tells us whether the variable increases or decreases the likelihood of purchase.

For ``nicer'' looking results, the \texttt{summ()} function from the \texttt{jtools} package can be used.

\begin{Shaded}
\begin{Highlighting}[]
\FunctionTok{library}\NormalTok{(jtools)}
\FunctionTok{summ}\NormalTok{(mod, }\AttributeTok{digits =} \DecValTok{4}\NormalTok{)}
\end{Highlighting}
\end{Shaded}

\begin{table}[!h]
\centering
\begin{tabular}{lr}
\toprule
\cellcolor{gray!10}{Observations} & \cellcolor{gray!10}{301}\\
Dependent variable & buy\\
\cellcolor{gray!10}{Type} & \cellcolor{gray!10}{Generalized linear model}\\
Family & binomial\\
\cellcolor{gray!10}{Link} & \cellcolor{gray!10}{logit}\\
\bottomrule
\end{tabular}
\end{table} \begin{table}[!h]
\centering
\begin{tabular}{lr}
\toprule
\cellcolor{gray!10}{$\chi^2$(3)} & \cellcolor{gray!10}{182.2574}\\
p & 0.0000\\
\cellcolor{gray!10}{Pseudo-R² (Cragg-Uhler)} & \cellcolor{gray!10}{0.6231}\\
Pseudo-R² (McFadden) & 0.4638\\
\cellcolor{gray!10}{AIC} & \cellcolor{gray!10}{218.6842}\\
\addlinespace
BIC & 233.5126\\
\bottomrule
\end{tabular}
\end{table} \begin{table}[!h]
\centering
\begin{threeparttable}
\begin{tabular}{lrrrr}
\toprule
  & Est. & S.E. & z val. & p\\
\midrule
\cellcolor{gray!10}{(Intercept)} & \cellcolor{gray!10}{-13.1661} & \cellcolor{gray!10}{1.6217} & \cellcolor{gray!10}{-8.1187} & \cellcolor{gray!10}{0.0000}\\
age & 0.2502 & 0.0321 & 7.7961 & 0.0000\\
\cellcolor{gray!10}{genderFemale} & \cellcolor{gray!10}{-0.4069} & \cellcolor{gray!10}{0.3498} & \cellcolor{gray!10}{-1.1631} & \cellcolor{gray!10}{0.2448}\\
salary & 0.0406 & 0.0067 & 6.0265 & 0.0000\\
\bottomrule
\end{tabular}
\begin{tablenotes}
\item Standard errors: MLE
\end{tablenotes}
\end{threeparttable}
\end{table}

\begin{center}\rule{0.5\linewidth}{0.5pt}\end{center}

\section{Interpreting Odds Ratios}\label{interpreting-odds-ratios}

Because log-odds are difficult to interpret, we often convert coefficients to \textbf{odds ratios}. While we can do this by simply taking the natural exponents of the model coefficients (i.e., \(e^{coeff}\)), \texttt{exp(coef(mod))}, you can again use the \texttt{summ()} function from \texttt{jtools} and use the \texttt{exp\ =\ TRUE} option.

\begin{Shaded}
\begin{Highlighting}[]
\FunctionTok{summ}\NormalTok{(mod, }\AttributeTok{exp =} \ConstantTok{TRUE}\NormalTok{, }\AttributeTok{digits =} \DecValTok{4}\NormalTok{)}
\end{Highlighting}
\end{Shaded}

\begin{table}[!h]
\centering
\begin{tabular}{lr}
\toprule
\cellcolor{gray!10}{Observations} & \cellcolor{gray!10}{301}\\
Dependent variable & buy\\
\cellcolor{gray!10}{Type} & \cellcolor{gray!10}{Generalized linear model}\\
Family & binomial\\
\cellcolor{gray!10}{Link} & \cellcolor{gray!10}{logit}\\
\bottomrule
\end{tabular}
\end{table} \begin{table}[!h]
\centering
\begin{tabular}{lr}
\toprule
\cellcolor{gray!10}{$\chi^2$(3)} & \cellcolor{gray!10}{182.2574}\\
p & 0.0000\\
\cellcolor{gray!10}{Pseudo-R² (Cragg-Uhler)} & \cellcolor{gray!10}{0.6231}\\
Pseudo-R² (McFadden) & 0.4638\\
\cellcolor{gray!10}{AIC} & \cellcolor{gray!10}{218.6842}\\
\addlinespace
BIC & 233.5126\\
\bottomrule
\end{tabular}
\end{table} \begin{table}[!h]
\centering
\begin{threeparttable}
\begin{tabular}{lrrrrr}
\toprule
  & exp(Est.) & 2.5\% & 97.5\% & z val. & p\\
\midrule
\cellcolor{gray!10}{(Intercept)} & \cellcolor{gray!10}{0.0000} & \cellcolor{gray!10}{0.0000} & \cellcolor{gray!10}{0.0000} & \cellcolor{gray!10}{-8.1187} & \cellcolor{gray!10}{0.0000}\\
age & 1.2843 & 1.2060 & 1.3677 & 7.7961 & 0.0000\\
\cellcolor{gray!10}{genderFemale} & \cellcolor{gray!10}{0.6657} & \cellcolor{gray!10}{0.3354} & \cellcolor{gray!10}{1.3215} & \cellcolor{gray!10}{-1.1631} & \cellcolor{gray!10}{0.2448}\\
salary & 1.0415 & 1.0278 & 1.0553 & 6.0265 & 0.0000\\
\bottomrule
\end{tabular}
\begin{tablenotes}
\item Standard errors: MLE
\end{tablenotes}
\end{threeparttable}
\end{table}

An odds ratio greater than 1 indicates higher odds of purchase as the predictor increases. An odds ratio less than 1 indicates lower odds.

\begin{center}\rule{0.5\linewidth}{0.5pt}\end{center}

\section{Predicted Probabilities}\label{predicted-probabilities}

Logistic regression naturally produces predicted probabilities.

\begin{Shaded}
\begin{Highlighting}[]
\NormalTok{train}\SpecialCharTok{$}\NormalTok{phat }\OtherTok{\textless{}{-}} \FunctionTok{predict}\NormalTok{(mod, }\AttributeTok{type =} \StringTok{"response"}\NormalTok{)}
\FunctionTok{head}\NormalTok{(train}\SpecialCharTok{$}\NormalTok{phat)}
\end{Highlighting}
\end{Shaded}

\begin{verbatim}
           1            2            3            4            5            6 
0.0004805918 0.0267022049 0.0109742358 0.0048495633 0.0170625797 0.0321671256 
\end{verbatim}

These probabilities are often more useful than hard classifications because they allow ranking and targeting. However, these probabilities are also used in evaluating model fit through the classification matrix. NOTE: You do not have to create these probabilities manually like shown above.

\begin{center}\rule{0.5\linewidth}{0.5pt}\end{center}

\section{Classification and Cutoff Values}\label{classification-and-cutoff-values}

To classify prospects, we choose a probability cutoff (commonly 0.50). That is, a prospect/row is predicted to be ``positive'' if their predicted probability of the positive outcome is equal to 0.50 or greater. We use the \texttt{logistic\_classify()} function from the \texttt{MKT4320BGSU} package to create classification matrices.

Usage:

\begin{itemize}
\tightlist
\item
  \texttt{classify\_logistic(MOD,\ DATA,\ POSITIVE,\ CUTOFF\ =\ 0.5,\ DATA2\ =\ NULL,}\strut \\
  \texttt{LABEL1\ =\ "Sample\ 1",\ LABEL2\ =\ "Sample\ 2",\ digits\ =\ 3,\ ft\ =\ FALSE)}
\item
  where:

  \begin{itemize}
  \tightlist
  \item
    \texttt{MOD} is a fitted binary logistic regression glm object (i.e., family = ``binomial'').
  \item
    \texttt{DATA} a data frame for which the classification matrix should be produced. Usually \texttt{train}.
  \item
    \texttt{POSITIVE} is the level in the outcome variable representing the positive outcome.
  \item
    \texttt{CUTOFF} is the pobability cutoff for classification (default = 0.5).
  \item
    \texttt{DATA2} is an optional second data frame (e.g., test/holdout sample).
  \item
    \texttt{LABEL1} is the label for the first data set (default = ``Sample 1'')
  \item
    \texttt{LABEL2} is the label for the section data set, if provided (default = ``Sample 2'')
  \item
    \texttt{digits} is the number of decimal places (default = 3) to show.
  \item
    \texttt{ft} is a logical operator to return a nicer looking flextable (default = FALSE).
  \end{itemize}
\end{itemize}

\begin{Shaded}
\begin{Highlighting}[]
\FunctionTok{classify\_logistic}\NormalTok{(}\AttributeTok{MOD =}\NormalTok{ mod, }\AttributeTok{DATA =}\NormalTok{ train, }\AttributeTok{POSITIVE =} \StringTok{"Yes"}\NormalTok{)}
\end{Highlighting}
\end{Shaded}

\begin{verbatim}

Classification Matrix - Sample 1 (Cutoff = 0.50)
Accuracy = 0.844
PCC = 0.540

       No Yes Total
No    176  30   206
Yes    17  78    95
Total 193 108   301

Selected Statistics (Positive Class):
  Sensitivity (TPR): 0.722
  Specificity (TNR): 0.912
  Precision (PPV):   0.821
\end{verbatim}

We can also evaluate classification performance on the test sample and get the output in a ``nicer'' table.

\begin{Shaded}
\begin{Highlighting}[]
\FunctionTok{classify\_logistic}\NormalTok{(}\AttributeTok{MOD =}\NormalTok{ mod, }\AttributeTok{DATA =}\NormalTok{ train, }\AttributeTok{DATA2 =}\NormalTok{ test, }\AttributeTok{POSITIVE =} \StringTok{"Yes"}\NormalTok{,}
                  \AttributeTok{LABEL1 =} \StringTok{"Training"}\NormalTok{, }\AttributeTok{LABEL2 =} \StringTok{"Test"}\NormalTok{, }\AttributeTok{ft=}\ConstantTok{TRUE}\NormalTok{)}
\end{Highlighting}
\end{Shaded}

\section{Choosing a Cutoff}\label{choosing-a-cutoff}

The default cutoff of 0.50 may not be optimal. The \texttt{cutoff\_logistic()} function helps visualize tradeoffs between sensitivity, specificity, and accuracy.

Usage:

\begin{itemize}
\tightlist
\item
  \texttt{cutoff\_logistic(MOD,\ DATA,\ POSITIVE,\ LABEL1\ =\ "Sample\ 1",}\strut \\
  \texttt{DATA2\ =\ NULL,\ LABEL2\ =\ "Sample\ 2")}
\item
  where:

  \begin{itemize}
  \tightlist
  \item
    \texttt{MOD} is a fitted binary logistic regression model (glm with family = ``binomial'').
  \item
    \texttt{DATA} a data frame for which the classification matrix should be produced. Usually \texttt{train}.
  \item
    \texttt{POSITIVE} is the level in the outcome variable representing the positive outcome.
  \item
    \texttt{LABEL1} is the label for the first data set (default = ``Sample 1'').
  \item
    \texttt{DATA2} is an optional second data frame (e.g., test/holdout sample).
  \item
    \texttt{LABEL2} is the label for the section data set, if provided (default = ``Sample 2'').
  \end{itemize}
\end{itemize}

\begin{Shaded}
\begin{Highlighting}[]
\NormalTok{cut }\OtherTok{\textless{}{-}} \FunctionTok{cutoff\_logistic}\NormalTok{(}\AttributeTok{MOD =}\NormalTok{ mod, }\AttributeTok{DATA =}\NormalTok{ train, }\AttributeTok{DATA2 =}\NormalTok{ test, }\AttributeTok{POSITIVE =} \StringTok{"Yes"}\NormalTok{, }
                \AttributeTok{LABEL1 =} \StringTok{"Training"}\NormalTok{, }\AttributeTok{LABEL2 =} \StringTok{"Test"}\NormalTok{)}
\end{Highlighting}
\end{Shaded}

\pandocbounded{\includegraphics[keepaspectratio]{MKT4320_R_Tutorial_files/figure-latex/cutoff-diagnostics-1.pdf}} \pandocbounded{\includegraphics[keepaspectratio]{MKT4320_R_Tutorial_files/figure-latex/cutoff-diagnostics-2.pdf}}

\begin{center}\rule{0.5\linewidth}{0.5pt}\end{center}

\section{ROC Curve and AUC}\label{roc-curve-and-auc}

The ROC curve evaluates model discrimination across all possible cutoffs. The Area Under the Curve (AUC) summarizes overall classification performance. To get the ROC curve, use the \texttt{roc\_logistic()} function from the \texttt{MKT4320BGSU} package.

Usage:

\begin{itemize}
\tightlist
\item
  \texttt{roc\_logistic(MOD,\ DATA,\ LABEL1\ =\ "Sample\ 1",}\strut \\
  \texttt{DATA2\ =\ NULL,\ LABEL2\ =\ "Sample\ 2")}
\item
  where:

  \begin{itemize}
  \tightlist
  \item
    \texttt{MOD} is a fitted binary logistic regression model (glm with family = ``binomial'').
  \item
    \texttt{DATA} a data frame for ROC computation. Usually \texttt{train}.
  \item
    \texttt{LABEL1} is the label for the first data set (default = ``Sample 1'').
  \item
    \texttt{DATA2} is an optional second data frame (e.g., test/holdout sample) (default = NULL).
  \item
    \texttt{LABEL2} is the label for the section data set, if provided (default = ``Sample 2'').
  \end{itemize}
\end{itemize}

\begin{Shaded}
\begin{Highlighting}[]
\FunctionTok{roc\_logistic}\NormalTok{(}\AttributeTok{MOD =}\NormalTok{ mod, }\AttributeTok{DATA =}\NormalTok{ train, }\AttributeTok{DATA2 =}\NormalTok{ test, }
             \AttributeTok{LABEL1 =} \StringTok{"Training"}\NormalTok{,}\AttributeTok{LABEL2 =} \StringTok{"Test"}\NormalTok{)}
\end{Highlighting}
\end{Shaded}

\begin{verbatim}
Setting levels: control = No, case = Yes
\end{verbatim}

\begin{verbatim}
Setting direction: controls < cases
\end{verbatim}

\begin{verbatim}
Setting levels: control = No, case = Yes
\end{verbatim}

\begin{verbatim}
Setting direction: controls < cases
\end{verbatim}

\begin{verbatim}
$sample1
\end{verbatim}

\pandocbounded{\includegraphics[keepaspectratio]{MKT4320_R_Tutorial_files/figure-latex/roc-curve-1.pdf}}

\begin{verbatim}

$sample2
\end{verbatim}

\pandocbounded{\includegraphics[keepaspectratio]{MKT4320_R_Tutorial_files/figure-latex/roc-curve-2.pdf}}

\begin{center}\rule{0.5\linewidth}{0.5pt}\end{center}

\section{Gain and Lift Charts}\label{gain-and-lift-charts}

Gain and lift charts are especially useful for targeting decisions. These plots (and tables) show how much better the model performs relative to random targeting. To get the gain and lift charts, use the \texttt{gainlift\_logistic()} function from the \texttt{MKT4320BGSU} package.

Usage:

\begin{itemize}
\tightlist
\item
  \texttt{gainlift\_logistic(MOD,\ TRAIN,\ TEST,\ POSITIVE)}
\item
  where:

  \begin{itemize}
  \tightlist
  \item
    \texttt{MOD} is a fitted binary logistic regression model (glm with family = ``binomial'').
  \item
    \texttt{TRAIN} a data frame with the training data (usually \texttt{train}).
  \item
    \texttt{TEST} a data frame with the test/holdout data (usually \texttt{test}).
  \item
    \texttt{POSITIVE} is the level in the outcome variable representing the positive outcome.
  \end{itemize}
\end{itemize}

\begin{Shaded}
\begin{Highlighting}[]
\NormalTok{gl }\OtherTok{\textless{}{-}} \FunctionTok{gainlift\_logistic}\NormalTok{(}\AttributeTok{MOD =}\NormalTok{ mod, }\AttributeTok{TRAIN =}\NormalTok{ train, }\AttributeTok{TEST =}\NormalTok{ test, }\AttributeTok{POSITIVE =} \StringTok{"Yes"}\NormalTok{)}
\NormalTok{gl}\SpecialCharTok{$}\NormalTok{gainplot}
\end{Highlighting}
\end{Shaded}

\pandocbounded{\includegraphics[keepaspectratio]{MKT4320_R_Tutorial_files/figure-latex/gain-lift-1.pdf}}

\begin{Shaded}
\begin{Highlighting}[]
\NormalTok{gl}\SpecialCharTok{$}\NormalTok{gaintable}
\end{Highlighting}
\end{Shaded}

\begin{verbatim}
# A tibble: 20 x 3
   `% Sample` Training Holdout
        <dbl>    <dbl>   <dbl>
 1       0.05    0.130   0.114
 2       0.1     0.25    0.2  
 3       0.15    0.370   0.343
 4       0.2     0.481   0.486
 5       0.25    0.593   0.6  
 6       0.3     0.713   0.743
 7       0.35    0.778   0.771
 8       0.4     0.843   0.829
 9       0.45    0.907   0.914
10       0.5     0.935   0.971
11       0.55    0.981   1    
12       0.6     0.991   1    
13       0.65    0.991   1    
14       0.7     0.991   1    
15       0.75    1       1    
16       0.8     1       1    
17       0.85    1       1    
18       0.9     1       1    
19       0.95    1       1    
20       1       1       1    
\end{verbatim}

\begin{Shaded}
\begin{Highlighting}[]
\NormalTok{gl}\SpecialCharTok{$}\NormalTok{liftplot}
\end{Highlighting}
\end{Shaded}

\pandocbounded{\includegraphics[keepaspectratio]{MKT4320_R_Tutorial_files/figure-latex/gain-lift-2.pdf}}

\begin{Shaded}
\begin{Highlighting}[]
\NormalTok{gl}\SpecialCharTok{$}\NormalTok{lifttable}
\end{Highlighting}
\end{Shaded}

\begin{verbatim}
# A tibble: 20 x 3
   `% Sample` Training Holdout
        <dbl>    <dbl>   <dbl>
 1       0.05     2.60    2.83
 2       0.1      2.51    2.2 
 3       0.15     2.48    2.42
 4       0.2      2.42    2.53
 5       0.25     2.38    2.48
 6       0.3      2.38    2.54
 7       0.35     2.23    2.25
 8       0.4      2.11    2.10
 9       0.45     2.02    2.06
10       0.5      1.88    1.96
11       0.55     1.79    1.83
12       0.6      1.66    1.68
13       0.65     1.53    1.55
14       0.7      1.42    1.43
15       0.75     1.34    1.34
16       0.8      1.25    1.25
17       0.85     1.18    1.18
18       0.9      1.11    1.11
19       0.95     1.06    1.05
20       1        1       1   
\end{verbatim}

\begin{center}\rule{0.5\linewidth}{0.5pt}\end{center}

\section{\texorpdfstring{Margin Plots with \texttt{easy\_mp()}}{Margin Plots with easy\_mp()}}\label{margin-plots-with-easy_mp}

Coefficients are difficult to interpret directly. Marginal effects plots help visualize how predictors influence purchase probability. As with linear regression (see Section \ref{sec-margin-plots}), we use the \texttt{easy\_mp()} function from the \texttt{MKT4320BGSU} package to help create the margin plots.

\begin{Shaded}
\begin{Highlighting}[]
\NormalTok{mp\_age }\OtherTok{\textless{}{-}} \FunctionTok{easy\_mp}\NormalTok{(mod, }\AttributeTok{focal =} \StringTok{"age"}\NormalTok{)}
\NormalTok{mp\_age}\SpecialCharTok{$}\NormalTok{plot}
\end{Highlighting}
\end{Shaded}

\pandocbounded{\includegraphics[keepaspectratio]{MKT4320_R_Tutorial_files/figure-latex/marginal-age-1.pdf}}

We can also examine categorical predictors.

\begin{Shaded}
\begin{Highlighting}[]
\NormalTok{mp\_gender }\OtherTok{\textless{}{-}} \FunctionTok{easy\_mp}\NormalTok{(mod, }\AttributeTok{focal =} \StringTok{"gender"}\NormalTok{)}
\NormalTok{mp\_gender}\SpecialCharTok{$}\NormalTok{plot}
\end{Highlighting}
\end{Shaded}

\pandocbounded{\includegraphics[keepaspectratio]{MKT4320_R_Tutorial_files/figure-latex/marginal-gender-1.pdf}}

We can also examine interactions.

\begin{Shaded}
\begin{Highlighting}[]
\NormalTok{mod\_int }\OtherTok{\textless{}{-}} \FunctionTok{glm}\NormalTok{(buy }\SpecialCharTok{\textasciitilde{}}\NormalTok{ age }\SpecialCharTok{*}\NormalTok{ salary }\SpecialCharTok{+}\NormalTok{ gender, }\AttributeTok{data =}\NormalTok{ train, }\AttributeTok{family =} \StringTok{"binomial"}\NormalTok{)}
\NormalTok{mp\_gender }\OtherTok{\textless{}{-}} \FunctionTok{easy\_mp}\NormalTok{(mod\_int, }\AttributeTok{focal =} \StringTok{"age"}\NormalTok{, }\AttributeTok{int=}\StringTok{"salary"}\NormalTok{)}
\NormalTok{mp\_gender}\SpecialCharTok{$}\NormalTok{plot}
\end{Highlighting}
\end{Shaded}

\pandocbounded{\includegraphics[keepaspectratio]{MKT4320_R_Tutorial_files/figure-latex/marginal-interaction-1.pdf}}

\begin{center}\rule{0.5\linewidth}{0.5pt}\end{center}

\section{Putting It All Together}\label{putting-it-all-together}

Binary logistic regression provides a complete framework for:

\begin{itemize}
\tightlist
\item
  Predicting purchase probabilities
\item
  Classifying prospects
\item
  Evaluating models using multiple metrics
\item
  Supporting targeting decisions
\end{itemize}

No single metric tells the whole story. Marketing analysts must choose evaluation tools that align with business objectives.

\begin{center}\rule{0.5\linewidth}{0.5pt}\end{center}

\section{What's Next}\label{whats-next-9}

In the next chapter, we shift from prediction to segmentation. We will use cluster analysis to:

\begin{itemize}
\tightlist
\item
  Identify distinct customer segments based on observed attributes
\item
  Understand how customers naturally group together
\item
  Support positioning, targeting, and strategy development
\item
  Translate data-driven segments into actionable marketing insights
\end{itemize}

Where logistic regression answers questions like ``Who is likely to buy?'', cluster analysis addresses questions such as:

\begin{itemize}
\tightlist
\item
  What types of customers do we have?
\item
  How are customers meaningfully different from one another?
\item
  How many segments make sense from a managerial perspective?
\end{itemize}

The next chapter introduces several clustering approaches, discusses how to choose the number of clusters, and emphasizes interpretation and managerial usefulness over purely technical solutions.

\chapter{Cluster Analysis}\label{cluster-analysis}

\section{Why Cluster Analysis Matters in Marketing}\label{why-cluster-analysis-matters-in-marketing}

Cluster analysis is one of the most widely used tools in marketing analytics for market segmentation. Unlike regression or classification models, cluster analysis is an unsupervised learning technique: there is no dependent variable. Instead, the objective is to identify groups of customers who are similar to one another and meaningfully different from customers in other groups.

In marketing contexts, cluster analysis is commonly used to:

\begin{itemize}
\tightlist
\item
  Identify distinct customer segments
\item
  Support targeting and positioning decisions
\item
  Inform product design and messaging strategy
\item
  Provide structure for downstream analysis and managerial reporting
\end{itemize}

Because cluster analysis does not optimize a predictive objective, interpretation and managerial judgment play a central role in determining whether a solution is useful.

\begin{center}\rule{0.5\linewidth}{0.5pt}\end{center}

\section{\texorpdfstring{The \texttt{ffseg} Dataset}{The ffseg Dataset}}\label{the-ffseg-dataset}

This chapter uses the \texttt{ffseg} dataset, which contains survey responses related to fast-food consumption behaviors, attitudes, and demographics. Each row represents an individual consumer.

The dataset includes:

\begin{itemize}
\tightlist
\item
  Numeric attitudinal and behavioral variables suitable for segmentation
\item
  Demographic and categorical variables used for describing clusters after they are formed
\end{itemize}

\begin{center}\rule{0.5\linewidth}{0.5pt}\end{center}

\section{Hierarchical Agglomerative Clustering}\label{hierarchical-agglomerative-clustering}

\subsection{Conceptual Overview}\label{conceptual-overview}

Hierarchical agglomerative clustering builds clusters from the bottom up. Each observation starts as its own cluster, and clusters are successively merged based on similarity until all observations belong to a single cluster.

Key features of hierarchical clustering include:

\begin{itemize}
\tightlist
\item
  No need to specify the number of clusters in advance
\item
  A dendrogram that visualizes the clustering process
\item
  Strong transparency for exploratory segmentation
\end{itemize}

\subsection{Initial Hierarchical Clustering Fit}\label{initial-hierarchical-clustering-fit}

We begin by fitting a hierarchical clustering model using selected segmentation variables from the \texttt{ffseg} dataset. While base R can do this for us in a number of steps, we can more easily do the initial fit using the \texttt{easy\_hc\_fit()} function from the \texttt{MKT4320BGSU} package.

Usage:

\begin{itemize}
\tightlist
\item
  \texttt{easy\_hc\_fit(data,\ vars,\ dist\ =\ c("euc",\ "euc2",\ "max",\ "abs",\ "bin"),}\strut \\
  \texttt{method\ =\ c("ward",\ "single",\ "complete",\ "average"),\ k\_range\ =\ 1:10,}\strut \\
  \texttt{standardize\ =\ TRUE,\ show\_dend\ =\ TRUE,\ dend\_max\_n\ =\ 300)}
\item
  where:

  \begin{itemize}
  \tightlist
  \item
    \texttt{data} is a data frame containing the full dataset.
  \item
    \texttt{vars} is a character vector of numeric segmentation variable names.
  \item
    \texttt{dist} is a distance measure to use. One of: ``euc'', ``euc2'', ``max'', ``abs'', ``bin''.
  \item
    \texttt{method} is a linkage method to use. One of: ``ward'', ``single'', ``complete'', ``average''.
  \item
    \texttt{k\_range} is an integer vector of cluster solutions to evaluate (default 1:10; allowed 1--20).
  \item
    \texttt{standardize} is logical; if TRUE (default), standardizes segmentation variables prior to clustering.
  \item
    \texttt{show\_dend} is logical; if TRUE (default), plots a dendrogram (skipped if n \textgreater{} dend\_max\_n).
  \item
    \texttt{dend\_max\_n} is an integer for the maximum sample size for drawing a dendrogram (default 300).
  \end{itemize}
\end{itemize}

When using the \texttt{easy\_hc\_fit()} function, the results should be saved to an object. The \texttt{\$stop} diagnostics table saved in the object provides the necessary results to help decide on a (potential) final solution. The diagnostics table summarizes multiple stopping rules and balance checks across different values of \(k\). No single statistic should be used in isolation when choosing the number of clusters. You can also view the \texttt{\$size\_prop} table to see the cluster sizes for all potential solutions.

\begin{Shaded}
\begin{Highlighting}[]
\NormalTok{hc\_vars }\OtherTok{\textless{}{-}} \FunctionTok{c}\NormalTok{(}\StringTok{"quality"}\NormalTok{, }\StringTok{"price"}\NormalTok{, }\StringTok{"healthy"}\NormalTok{, }\StringTok{"variety"}\NormalTok{, }\StringTok{"speed"}\NormalTok{)}
\NormalTok{hc\_fit }\OtherTok{\textless{}{-}} \FunctionTok{easy\_hc\_fit}\NormalTok{(}\AttributeTok{data =}\NormalTok{ ffseg, }\AttributeTok{vars =}\NormalTok{ hc\_vars, }\AttributeTok{dist =} \StringTok{"euc"}\NormalTok{, }
                      \AttributeTok{method =} \StringTok{"ward"}\NormalTok{, }\AttributeTok{k\_range =} \DecValTok{2}\SpecialCharTok{:}\DecValTok{8}\NormalTok{)}
\end{Highlighting}
\end{Shaded}

\begin{verbatim}
Skipping dendrogram (n > dend_max_n).
\end{verbatim}

\begin{verbatim}
Clustering k = 1,2,..., K.max (= 8): .. done
Bootstrapping, b = 1,2,..., B (= 20)  [one "." per sample]:
.................... 20 
\end{verbatim}

\begin{Shaded}
\begin{Highlighting}[]
\NormalTok{hc\_fit}\SpecialCharTok{$}\NormalTok{stop}
\end{Highlighting}
\end{Shaded}

\global\setlength{\Oldarrayrulewidth}{\arrayrulewidth}

\global\setlength{\Oldtabcolsep}{\tabcolsep}

\setlength{\tabcolsep}{2pt}

\renewcommand*{\arraystretch}{1.5}



\providecommand{\ascline}[3]{\noalign{\global\arrayrulewidth #1}\arrayrulecolor[HTML]{#2}\cline{#3}}

\begin{longtable}[c]{|p{0.90in}|p{1.02in}|p{1.01in}|p{0.91in}|p{1.08in}|p{1.09in}|p{0.82in}}



\ascline{1.5pt}{666666}{1-7}

\multicolumn{7}{>{\centering}m{\dimexpr 6.82in+12\tabcolsep}}{\textcolor[HTML]{000000}{\fontsize{11}{11}\selectfont{\global\setmainfont{Arial}{\textbf{Cluster\ Diagnostics\ (Euclidean\ Distance\ /\ Ward's\ D\ Linkage)}}}}} \\

\ascline{1.5pt}{666666}{1-7}



\multicolumn{1}{>{\centering}m{\dimexpr 0.9in+0\tabcolsep}}{\textcolor[HTML]{000000}{\fontsize{11}{11}\selectfont{\global\setmainfont{Arial}{\textbf{Clusters}}}}} & \multicolumn{1}{>{\centering}m{\dimexpr 1.02in+0\tabcolsep}}{\textcolor[HTML]{000000}{\fontsize{11}{11}\selectfont{\global\setmainfont{Arial}{\textbf{Duda.Hart}}}}} & \multicolumn{1}{>{\centering}m{\dimexpr 1.01in+0\tabcolsep}}{\textcolor[HTML]{000000}{\fontsize{11}{11}\selectfont{\global\setmainfont{Arial}{\textbf{pseudo.t2}}}}} & \multicolumn{1}{>{\centering}m{\dimexpr 0.91in+0\tabcolsep}}{\textcolor[HTML]{000000}{\fontsize{11}{11}\selectfont{\global\setmainfont{Arial}{\textbf{Gap.Stat}}}}} & \multicolumn{1}{>{\centering}m{\dimexpr 1.08in+0\tabcolsep}}{\textcolor[HTML]{000000}{\fontsize{11}{11}\selectfont{\global\setmainfont{Arial}{\textbf{Small.Prop}}}}} & \multicolumn{1}{>{\centering}m{\dimexpr 1.09in+0\tabcolsep}}{\textcolor[HTML]{000000}{\fontsize{11}{11}\selectfont{\global\setmainfont{Arial}{\textbf{Large.Prop}}}}} & \multicolumn{1}{>{\centering}m{\dimexpr 0.82in+0\tabcolsep}}{\textcolor[HTML]{000000}{\fontsize{11}{11}\selectfont{\global\setmainfont{Arial}{\textbf{CV}}}}} \\

\ascline{1.5pt}{666666}{1-7}\endfirsthead 

\ascline{1.5pt}{666666}{1-7}

\multicolumn{7}{>{\centering}m{\dimexpr 6.82in+12\tabcolsep}}{\textcolor[HTML]{000000}{\fontsize{11}{11}\selectfont{\global\setmainfont{Arial}{\textbf{Cluster\ Diagnostics\ (Euclidean\ Distance\ /\ Ward's\ D\ Linkage)}}}}} \\

\ascline{1.5pt}{666666}{1-7}



\multicolumn{1}{>{\centering}m{\dimexpr 0.9in+0\tabcolsep}}{\textcolor[HTML]{000000}{\fontsize{11}{11}\selectfont{\global\setmainfont{Arial}{\textbf{Clusters}}}}} & \multicolumn{1}{>{\centering}m{\dimexpr 1.02in+0\tabcolsep}}{\textcolor[HTML]{000000}{\fontsize{11}{11}\selectfont{\global\setmainfont{Arial}{\textbf{Duda.Hart}}}}} & \multicolumn{1}{>{\centering}m{\dimexpr 1.01in+0\tabcolsep}}{\textcolor[HTML]{000000}{\fontsize{11}{11}\selectfont{\global\setmainfont{Arial}{\textbf{pseudo.t2}}}}} & \multicolumn{1}{>{\centering}m{\dimexpr 0.91in+0\tabcolsep}}{\textcolor[HTML]{000000}{\fontsize{11}{11}\selectfont{\global\setmainfont{Arial}{\textbf{Gap.Stat}}}}} & \multicolumn{1}{>{\centering}m{\dimexpr 1.08in+0\tabcolsep}}{\textcolor[HTML]{000000}{\fontsize{11}{11}\selectfont{\global\setmainfont{Arial}{\textbf{Small.Prop}}}}} & \multicolumn{1}{>{\centering}m{\dimexpr 1.09in+0\tabcolsep}}{\textcolor[HTML]{000000}{\fontsize{11}{11}\selectfont{\global\setmainfont{Arial}{\textbf{Large.Prop}}}}} & \multicolumn{1}{>{\centering}m{\dimexpr 0.82in+0\tabcolsep}}{\textcolor[HTML]{000000}{\fontsize{11}{11}\selectfont{\global\setmainfont{Arial}{\textbf{CV}}}}} \\

\ascline{1.5pt}{666666}{1-7}\endhead



\multicolumn{7}{>{\raggedright}m{\dimexpr 6.82in+12\tabcolsep}}{\textcolor[HTML]{000000}{\fontsize{11}{11}\selectfont{\global\setmainfont{Arial}{*\ 1-SE\ Gap\ rule.}}}} \\





\multicolumn{7}{>{\raggedright}m{\dimexpr 6.82in+12\tabcolsep}}{\textcolor[HTML]{000000}{\fontsize{11}{11}\selectfont{\global\setmainfont{Arial}{\\\textasciicircum \ Smallest\ cluster\ <\ 5\%\ of\ sample.}}}} \\





\multicolumn{7}{>{\raggedright}m{\dimexpr 6.82in+12\tabcolsep}}{\textcolor[HTML]{000000}{\fontsize{11}{11}\selectfont{\global\setmainfont{Arial}{◊\ Largest\ cluster\ >\ 50\%\ of\ sample.}}}} \\





\multicolumn{7}{>{\raggedright}m{\dimexpr 6.82in+12\tabcolsep}}{\textcolor[HTML]{000000}{\fontsize{11}{11}\selectfont{\global\setmainfont{Arial}{•\ CV\ <\ 0.5\ well\ balanced;\ ••\ moderately\ imbalanced.}}}} \\

\endlastfoot



\multicolumn{1}{>{\raggedleft}m{\dimexpr 0.9in+0\tabcolsep}}{\textcolor[HTML]{000000}{\fontsize{11}{11}\selectfont{\global\setmainfont{Arial}{2}}}} & \multicolumn{1}{>{\raggedleft}m{\dimexpr 1.02in+0\tabcolsep}}{\textcolor[HTML]{000000}{\fontsize{11}{11}\selectfont{\global\setmainfont{Arial}{0.837}}}} & \multicolumn{1}{>{\raggedleft}m{\dimexpr 1.01in+0\tabcolsep}}{\textcolor[HTML]{000000}{\fontsize{11}{11}\selectfont{\global\setmainfont{Arial}{88.38}}}} & \multicolumn{1}{>{\raggedleft}m{\dimexpr 0.91in+0\tabcolsep}}{\textcolor[HTML]{000000}{\fontsize{11}{11}\selectfont{\global\setmainfont{Arial}{0.6212*}}}} & \multicolumn{1}{>{\raggedleft}m{\dimexpr 1.08in+0\tabcolsep}}{\textcolor[HTML]{000000}{\fontsize{11}{11}\selectfont{\global\setmainfont{Arial}{0.3949}}}} & \multicolumn{1}{>{\raggedleft}m{\dimexpr 1.09in+0\tabcolsep}}{\textcolor[HTML]{000000}{\fontsize{11}{11}\selectfont{\global\setmainfont{Arial}{0.6051\ ◊}}}} & \multicolumn{1}{>{\raggedleft}m{\dimexpr 0.82in+0\tabcolsep}}{\textcolor[HTML]{000000}{\fontsize{11}{11}\selectfont{\global\setmainfont{Arial}{0.297\ •}}}} \\





\multicolumn{1}{>{\raggedleft}m{\dimexpr 0.9in+0\tabcolsep}}{\textcolor[HTML]{000000}{\fontsize{11}{11}\selectfont{\global\setmainfont{Arial}{3}}}} & \multicolumn{1}{>{\raggedleft}m{\dimexpr 1.02in+0\tabcolsep}}{\textcolor[HTML]{000000}{\fontsize{11}{11}\selectfont{\global\setmainfont{Arial}{0.844}}}} & \multicolumn{1}{>{\raggedleft}m{\dimexpr 1.01in+0\tabcolsep}}{\textcolor[HTML]{000000}{\fontsize{11}{11}\selectfont{\global\setmainfont{Arial}{54.74}}}} & \multicolumn{1}{>{\raggedleft}m{\dimexpr 0.91in+0\tabcolsep}}{\textcolor[HTML]{000000}{\fontsize{11}{11}\selectfont{\global\setmainfont{Arial}{0.6048}}}} & \multicolumn{1}{>{\raggedleft}m{\dimexpr 1.08in+0\tabcolsep}}{\textcolor[HTML]{000000}{\fontsize{11}{11}\selectfont{\global\setmainfont{Arial}{0.2247}}}} & \multicolumn{1}{>{\raggedleft}m{\dimexpr 1.09in+0\tabcolsep}}{\textcolor[HTML]{000000}{\fontsize{11}{11}\selectfont{\global\setmainfont{Arial}{0.3949}}}} & \multicolumn{1}{>{\raggedleft}m{\dimexpr 0.82in+0\tabcolsep}}{\textcolor[HTML]{000000}{\fontsize{11}{11}\selectfont{\global\setmainfont{Arial}{0.283\ •}}}} \\





\multicolumn{1}{>{\raggedleft}m{\dimexpr 0.9in+0\tabcolsep}}{\textcolor[HTML]{000000}{\fontsize{11}{11}\selectfont{\global\setmainfont{Arial}{4}}}} & \multicolumn{1}{>{\raggedleft}m{\dimexpr 1.02in+0\tabcolsep}}{\textcolor[HTML]{000000}{\fontsize{11}{11}\selectfont{\global\setmainfont{Arial}{0.849}}}} & \multicolumn{1}{>{\raggedleft}m{\dimexpr 1.01in+0\tabcolsep}}{\textcolor[HTML]{000000}{\fontsize{11}{11}\selectfont{\global\setmainfont{Arial}{50.52}}}} & \multicolumn{1}{>{\raggedleft}m{\dimexpr 0.91in+0\tabcolsep}}{\textcolor[HTML]{000000}{\fontsize{11}{11}\selectfont{\global\setmainfont{Arial}{0.5861}}}} & \multicolumn{1}{>{\raggedleft}m{\dimexpr 1.08in+0\tabcolsep}}{\textcolor[HTML]{000000}{\fontsize{11}{11}\selectfont{\global\setmainfont{Arial}{0.0465\\\textasciicircum }}}} & \multicolumn{1}{>{\raggedleft}m{\dimexpr 1.09in+0\tabcolsep}}{\textcolor[HTML]{000000}{\fontsize{11}{11}\selectfont{\global\setmainfont{Arial}{0.3803}}}} & \multicolumn{1}{>{\raggedleft}m{\dimexpr 0.82in+0\tabcolsep}}{\textcolor[HTML]{000000}{\fontsize{11}{11}\selectfont{\global\setmainfont{Arial}{0.605\ ••}}}} \\





\multicolumn{1}{>{\raggedleft}m{\dimexpr 0.9in+0\tabcolsep}}{\textcolor[HTML]{000000}{\fontsize{11}{11}\selectfont{\global\setmainfont{Arial}{5}}}} & \multicolumn{1}{>{\raggedleft}m{\dimexpr 1.02in+0\tabcolsep}}{\textcolor[HTML]{000000}{\fontsize{11}{11}\selectfont{\global\setmainfont{Arial}{0.745}}}} & \multicolumn{1}{>{\raggedleft}m{\dimexpr 1.01in+0\tabcolsep}}{\textcolor[HTML]{000000}{\fontsize{11}{11}\selectfont{\global\setmainfont{Arial}{58.58}}}} & \multicolumn{1}{>{\raggedleft}m{\dimexpr 0.91in+0\tabcolsep}}{\textcolor[HTML]{000000}{\fontsize{11}{11}\selectfont{\global\setmainfont{Arial}{0.5802}}}} & \multicolumn{1}{>{\raggedleft}m{\dimexpr 1.08in+0\tabcolsep}}{\textcolor[HTML]{000000}{\fontsize{11}{11}\selectfont{\global\setmainfont{Arial}{0.0465\\\textasciicircum }}}} & \multicolumn{1}{>{\raggedleft}m{\dimexpr 1.09in+0\tabcolsep}}{\textcolor[HTML]{000000}{\fontsize{11}{11}\selectfont{\global\setmainfont{Arial}{0.3484}}}} & \multicolumn{1}{>{\raggedleft}m{\dimexpr 0.82in+0\tabcolsep}}{\textcolor[HTML]{000000}{\fontsize{11}{11}\selectfont{\global\setmainfont{Arial}{0.557\ ••}}}} \\





\multicolumn{1}{>{\raggedleft}m{\dimexpr 0.9in+0\tabcolsep}}{\textcolor[HTML]{000000}{\fontsize{11}{11}\selectfont{\global\setmainfont{Arial}{6}}}} & \multicolumn{1}{>{\raggedleft}m{\dimexpr 1.02in+0\tabcolsep}}{\textcolor[HTML]{000000}{\fontsize{11}{11}\selectfont{\global\setmainfont{Arial}{0.863}}}} & \multicolumn{1}{>{\raggedleft}m{\dimexpr 1.01in+0\tabcolsep}}{\textcolor[HTML]{000000}{\fontsize{11}{11}\selectfont{\global\setmainfont{Arial}{41.36}}}} & \multicolumn{1}{>{\raggedleft}m{\dimexpr 0.91in+0\tabcolsep}}{\textcolor[HTML]{000000}{\fontsize{11}{11}\selectfont{\global\setmainfont{Arial}{0.5796}}}} & \multicolumn{1}{>{\raggedleft}m{\dimexpr 1.08in+0\tabcolsep}}{\textcolor[HTML]{000000}{\fontsize{11}{11}\selectfont{\global\setmainfont{Arial}{0.0465\\\textasciicircum }}}} & \multicolumn{1}{>{\raggedleft}m{\dimexpr 1.09in+0\tabcolsep}}{\textcolor[HTML]{000000}{\fontsize{11}{11}\selectfont{\global\setmainfont{Arial}{0.3484}}}} & \multicolumn{1}{>{\raggedleft}m{\dimexpr 0.82in+0\tabcolsep}}{\textcolor[HTML]{000000}{\fontsize{11}{11}\selectfont{\global\setmainfont{Arial}{0.640\ ••}}}} \\





\multicolumn{1}{>{\raggedleft}m{\dimexpr 0.9in+0\tabcolsep}}{\textcolor[HTML]{000000}{\fontsize{11}{11}\selectfont{\global\setmainfont{Arial}{7}}}} & \multicolumn{1}{>{\raggedleft}m{\dimexpr 1.02in+0\tabcolsep}}{\textcolor[HTML]{000000}{\fontsize{11}{11}\selectfont{\global\setmainfont{Arial}{0.822}}}} & \multicolumn{1}{>{\raggedleft}m{\dimexpr 1.01in+0\tabcolsep}}{\textcolor[HTML]{000000}{\fontsize{11}{11}\selectfont{\global\setmainfont{Arial}{36.11}}}} & \multicolumn{1}{>{\raggedleft}m{\dimexpr 0.91in+0\tabcolsep}}{\textcolor[HTML]{000000}{\fontsize{11}{11}\selectfont{\global\setmainfont{Arial}{0.5744}}}} & \multicolumn{1}{>{\raggedleft}m{\dimexpr 1.08in+0\tabcolsep}}{\textcolor[HTML]{000000}{\fontsize{11}{11}\selectfont{\global\setmainfont{Arial}{0.0465\\\textasciicircum }}}} & \multicolumn{1}{>{\raggedleft}m{\dimexpr 1.09in+0\tabcolsep}}{\textcolor[HTML]{000000}{\fontsize{11}{11}\selectfont{\global\setmainfont{Arial}{0.2460}}}} & \multicolumn{1}{>{\raggedleft}m{\dimexpr 0.82in+0\tabcolsep}}{\textcolor[HTML]{000000}{\fontsize{11}{11}\selectfont{\global\setmainfont{Arial}{0.497\ •}}}} \\





\multicolumn{1}{>{\raggedleft}m{\dimexpr 0.9in+0\tabcolsep}}{\textcolor[HTML]{000000}{\fontsize{11}{11}\selectfont{\global\setmainfont{Arial}{8}}}} & \multicolumn{1}{>{\raggedleft}m{\dimexpr 1.02in+0\tabcolsep}}{\textcolor[HTML]{000000}{\fontsize{11}{11}\selectfont{\global\setmainfont{Arial}{0.821}}}} & \multicolumn{1}{>{\raggedleft}m{\dimexpr 1.01in+0\tabcolsep}}{\textcolor[HTML]{000000}{\fontsize{11}{11}\selectfont{\global\setmainfont{Arial}{40.00}}}} & \multicolumn{1}{>{\raggedleft}m{\dimexpr 0.91in+0\tabcolsep}}{\textcolor[HTML]{000000}{\fontsize{11}{11}\selectfont{\global\setmainfont{Arial}{0.5747}}}} & \multicolumn{1}{>{\raggedleft}m{\dimexpr 1.08in+0\tabcolsep}}{\textcolor[HTML]{000000}{\fontsize{11}{11}\selectfont{\global\setmainfont{Arial}{0.0465\\\textasciicircum }}}} & \multicolumn{1}{>{\raggedleft}m{\dimexpr 1.09in+0\tabcolsep}}{\textcolor[HTML]{000000}{\fontsize{11}{11}\selectfont{\global\setmainfont{Arial}{0.2460}}}} & \multicolumn{1}{>{\raggedleft}m{\dimexpr 0.82in+0\tabcolsep}}{\textcolor[HTML]{000000}{\fontsize{11}{11}\selectfont{\global\setmainfont{Arial}{0.468\ •}}}} \\

\ascline{1.5pt}{666666}{1-7}



\end{longtable}



\arrayrulecolor[HTML]{000000}

\global\setlength{\arrayrulewidth}{\Oldarrayrulewidth}

\global\setlength{\tabcolsep}{\Oldtabcolsep}

\renewcommand*{\arraystretch}{1}

\begin{Shaded}
\begin{Highlighting}[]
\NormalTok{hc\_fit}\SpecialCharTok{$}\NormalTok{size\_prop}
\end{Highlighting}
\end{Shaded}

\global\setlength{\Oldarrayrulewidth}{\arrayrulewidth}

\global\setlength{\Oldtabcolsep}{\tabcolsep}

\setlength{\tabcolsep}{2pt}

\renewcommand*{\arraystretch}{1.5}



\providecommand{\ascline}[3]{\noalign{\global\arrayrulewidth #1}\arrayrulecolor[HTML]{#2}\cline{#3}}

\begin{longtable}[c]{|p{1.44in}|p{0.94in}|p{0.94in}|p{0.94in}|p{0.94in}|p{0.94in}|p{0.94in}|p{0.94in}|p{0.94in}}



\ascline{1.5pt}{666666}{1-9}

\multicolumn{9}{>{\centering}m{\dimexpr 8.95in+16\tabcolsep}}{\textcolor[HTML]{000000}{\fontsize{11}{11}\selectfont{\global\setmainfont{Arial}{\textbf{Cluster\ Size\ Proportions\ by\ Solution}}}}} \\

\ascline{1.5pt}{666666}{1-9}



\multicolumn{1}{>{\centering}m{\dimexpr 1.44in+0\tabcolsep}}{\textcolor[HTML]{000000}{\fontsize{11}{11}\selectfont{\global\setmainfont{Arial}{\textbf{Solution}}}}} & \multicolumn{1}{>{\centering}m{\dimexpr 0.94in+0\tabcolsep}}{\textcolor[HTML]{000000}{\fontsize{11}{11}\selectfont{\global\setmainfont{Arial}{\textbf{Cluster\ 1}}}}} & \multicolumn{1}{>{\centering}m{\dimexpr 0.94in+0\tabcolsep}}{\textcolor[HTML]{000000}{\fontsize{11}{11}\selectfont{\global\setmainfont{Arial}{\textbf{Cluster\ 2}}}}} & \multicolumn{1}{>{\centering}m{\dimexpr 0.94in+0\tabcolsep}}{\textcolor[HTML]{000000}{\fontsize{11}{11}\selectfont{\global\setmainfont{Arial}{\textbf{Cluster\ 3}}}}} & \multicolumn{1}{>{\centering}m{\dimexpr 0.94in+0\tabcolsep}}{\textcolor[HTML]{000000}{\fontsize{11}{11}\selectfont{\global\setmainfont{Arial}{\textbf{Cluster\ 4}}}}} & \multicolumn{1}{>{\centering}m{\dimexpr 0.94in+0\tabcolsep}}{\textcolor[HTML]{000000}{\fontsize{11}{11}\selectfont{\global\setmainfont{Arial}{\textbf{Cluster\ 5}}}}} & \multicolumn{1}{>{\centering}m{\dimexpr 0.94in+0\tabcolsep}}{\textcolor[HTML]{000000}{\fontsize{11}{11}\selectfont{\global\setmainfont{Arial}{\textbf{Cluster\ 6}}}}} & \multicolumn{1}{>{\centering}m{\dimexpr 0.94in+0\tabcolsep}}{\textcolor[HTML]{000000}{\fontsize{11}{11}\selectfont{\global\setmainfont{Arial}{\textbf{Cluster\ 7}}}}} & \multicolumn{1}{>{\centering}m{\dimexpr 0.94in+0\tabcolsep}}{\textcolor[HTML]{000000}{\fontsize{11}{11}\selectfont{\global\setmainfont{Arial}{\textbf{Cluster\ 8}}}}} \\

\ascline{1.5pt}{666666}{1-9}\endfirsthead 

\ascline{1.5pt}{666666}{1-9}

\multicolumn{9}{>{\centering}m{\dimexpr 8.95in+16\tabcolsep}}{\textcolor[HTML]{000000}{\fontsize{11}{11}\selectfont{\global\setmainfont{Arial}{\textbf{Cluster\ Size\ Proportions\ by\ Solution}}}}} \\

\ascline{1.5pt}{666666}{1-9}



\multicolumn{1}{>{\centering}m{\dimexpr 1.44in+0\tabcolsep}}{\textcolor[HTML]{000000}{\fontsize{11}{11}\selectfont{\global\setmainfont{Arial}{\textbf{Solution}}}}} & \multicolumn{1}{>{\centering}m{\dimexpr 0.94in+0\tabcolsep}}{\textcolor[HTML]{000000}{\fontsize{11}{11}\selectfont{\global\setmainfont{Arial}{\textbf{Cluster\ 1}}}}} & \multicolumn{1}{>{\centering}m{\dimexpr 0.94in+0\tabcolsep}}{\textcolor[HTML]{000000}{\fontsize{11}{11}\selectfont{\global\setmainfont{Arial}{\textbf{Cluster\ 2}}}}} & \multicolumn{1}{>{\centering}m{\dimexpr 0.94in+0\tabcolsep}}{\textcolor[HTML]{000000}{\fontsize{11}{11}\selectfont{\global\setmainfont{Arial}{\textbf{Cluster\ 3}}}}} & \multicolumn{1}{>{\centering}m{\dimexpr 0.94in+0\tabcolsep}}{\textcolor[HTML]{000000}{\fontsize{11}{11}\selectfont{\global\setmainfont{Arial}{\textbf{Cluster\ 4}}}}} & \multicolumn{1}{>{\centering}m{\dimexpr 0.94in+0\tabcolsep}}{\textcolor[HTML]{000000}{\fontsize{11}{11}\selectfont{\global\setmainfont{Arial}{\textbf{Cluster\ 5}}}}} & \multicolumn{1}{>{\centering}m{\dimexpr 0.94in+0\tabcolsep}}{\textcolor[HTML]{000000}{\fontsize{11}{11}\selectfont{\global\setmainfont{Arial}{\textbf{Cluster\ 6}}}}} & \multicolumn{1}{>{\centering}m{\dimexpr 0.94in+0\tabcolsep}}{\textcolor[HTML]{000000}{\fontsize{11}{11}\selectfont{\global\setmainfont{Arial}{\textbf{Cluster\ 7}}}}} & \multicolumn{1}{>{\centering}m{\dimexpr 0.94in+0\tabcolsep}}{\textcolor[HTML]{000000}{\fontsize{11}{11}\selectfont{\global\setmainfont{Arial}{\textbf{Cluster\ 8}}}}} \\

\ascline{1.5pt}{666666}{1-9}\endhead



\multicolumn{1}{>{\raggedleft}m{\dimexpr 1.44in+0\tabcolsep}}{\textcolor[HTML]{000000}{\fontsize{11}{11}\selectfont{\global\setmainfont{Arial}{2-cluster\ solution}}}} & \multicolumn{1}{>{\raggedleft}m{\dimexpr 0.94in+0\tabcolsep}}{\textcolor[HTML]{000000}{\fontsize{11}{11}\selectfont{\global\setmainfont{Arial}{0.3949}}}} & \multicolumn{1}{>{\raggedleft}m{\dimexpr 0.94in+0\tabcolsep}}{\textcolor[HTML]{000000}{\fontsize{11}{11}\selectfont{\global\setmainfont{Arial}{0.6051}}}} & \multicolumn{1}{>{\raggedleft}m{\dimexpr 0.94in+0\tabcolsep}}{\textcolor[HTML]{000000}{\fontsize{11}{11}\selectfont{\global\setmainfont{Arial}{}}}} & \multicolumn{1}{>{\raggedleft}m{\dimexpr 0.94in+0\tabcolsep}}{\textcolor[HTML]{000000}{\fontsize{11}{11}\selectfont{\global\setmainfont{Arial}{}}}} & \multicolumn{1}{>{\raggedleft}m{\dimexpr 0.94in+0\tabcolsep}}{\textcolor[HTML]{000000}{\fontsize{11}{11}\selectfont{\global\setmainfont{Arial}{}}}} & \multicolumn{1}{>{\raggedleft}m{\dimexpr 0.94in+0\tabcolsep}}{\textcolor[HTML]{000000}{\fontsize{11}{11}\selectfont{\global\setmainfont{Arial}{}}}} & \multicolumn{1}{>{\raggedleft}m{\dimexpr 0.94in+0\tabcolsep}}{\textcolor[HTML]{000000}{\fontsize{11}{11}\selectfont{\global\setmainfont{Arial}{}}}} & \multicolumn{1}{>{\raggedleft}m{\dimexpr 0.94in+0\tabcolsep}}{\textcolor[HTML]{000000}{\fontsize{11}{11}\selectfont{\global\setmainfont{Arial}{}}}} \\





\multicolumn{1}{>{\raggedleft}m{\dimexpr 1.44in+0\tabcolsep}}{\textcolor[HTML]{000000}{\fontsize{11}{11}\selectfont{\global\setmainfont{Arial}{3-cluster\ solution}}}} & \multicolumn{1}{>{\raggedleft}m{\dimexpr 0.94in+0\tabcolsep}}{\textcolor[HTML]{000000}{\fontsize{11}{11}\selectfont{\global\setmainfont{Arial}{0.3949}}}} & \multicolumn{1}{>{\raggedleft}m{\dimexpr 0.94in+0\tabcolsep}}{\textcolor[HTML]{000000}{\fontsize{11}{11}\selectfont{\global\setmainfont{Arial}{0.3803}}}} & \multicolumn{1}{>{\raggedleft}m{\dimexpr 0.94in+0\tabcolsep}}{\textcolor[HTML]{000000}{\fontsize{11}{11}\selectfont{\global\setmainfont{Arial}{0.2247}}}} & \multicolumn{1}{>{\raggedleft}m{\dimexpr 0.94in+0\tabcolsep}}{\textcolor[HTML]{000000}{\fontsize{11}{11}\selectfont{\global\setmainfont{Arial}{}}}} & \multicolumn{1}{>{\raggedleft}m{\dimexpr 0.94in+0\tabcolsep}}{\textcolor[HTML]{000000}{\fontsize{11}{11}\selectfont{\global\setmainfont{Arial}{}}}} & \multicolumn{1}{>{\raggedleft}m{\dimexpr 0.94in+0\tabcolsep}}{\textcolor[HTML]{000000}{\fontsize{11}{11}\selectfont{\global\setmainfont{Arial}{}}}} & \multicolumn{1}{>{\raggedleft}m{\dimexpr 0.94in+0\tabcolsep}}{\textcolor[HTML]{000000}{\fontsize{11}{11}\selectfont{\global\setmainfont{Arial}{}}}} & \multicolumn{1}{>{\raggedleft}m{\dimexpr 0.94in+0\tabcolsep}}{\textcolor[HTML]{000000}{\fontsize{11}{11}\selectfont{\global\setmainfont{Arial}{}}}} \\





\multicolumn{1}{>{\raggedleft}m{\dimexpr 1.44in+0\tabcolsep}}{\textcolor[HTML]{000000}{\fontsize{11}{11}\selectfont{\global\setmainfont{Arial}{4-cluster\ solution}}}} & \multicolumn{1}{>{\raggedleft}m{\dimexpr 0.94in+0\tabcolsep}}{\textcolor[HTML]{000000}{\fontsize{11}{11}\selectfont{\global\setmainfont{Arial}{0.3484}}}} & \multicolumn{1}{>{\raggedleft}m{\dimexpr 0.94in+0\tabcolsep}}{\textcolor[HTML]{000000}{\fontsize{11}{11}\selectfont{\global\setmainfont{Arial}{0.3803}}}} & \multicolumn{1}{>{\raggedleft}m{\dimexpr 0.94in+0\tabcolsep}}{\textcolor[HTML]{000000}{\fontsize{11}{11}\selectfont{\global\setmainfont{Arial}{0.2247}}}} & \multicolumn{1}{>{\raggedleft}m{\dimexpr 0.94in+0\tabcolsep}}{\textcolor[HTML]{000000}{\fontsize{11}{11}\selectfont{\global\setmainfont{Arial}{0.0465}}}} & \multicolumn{1}{>{\raggedleft}m{\dimexpr 0.94in+0\tabcolsep}}{\textcolor[HTML]{000000}{\fontsize{11}{11}\selectfont{\global\setmainfont{Arial}{}}}} & \multicolumn{1}{>{\raggedleft}m{\dimexpr 0.94in+0\tabcolsep}}{\textcolor[HTML]{000000}{\fontsize{11}{11}\selectfont{\global\setmainfont{Arial}{}}}} & \multicolumn{1}{>{\raggedleft}m{\dimexpr 0.94in+0\tabcolsep}}{\textcolor[HTML]{000000}{\fontsize{11}{11}\selectfont{\global\setmainfont{Arial}{}}}} & \multicolumn{1}{>{\raggedleft}m{\dimexpr 0.94in+0\tabcolsep}}{\textcolor[HTML]{000000}{\fontsize{11}{11}\selectfont{\global\setmainfont{Arial}{}}}} \\





\multicolumn{1}{>{\raggedleft}m{\dimexpr 1.44in+0\tabcolsep}}{\textcolor[HTML]{000000}{\fontsize{11}{11}\selectfont{\global\setmainfont{Arial}{5-cluster\ solution}}}} & \multicolumn{1}{>{\raggedleft}m{\dimexpr 0.94in+0\tabcolsep}}{\textcolor[HTML]{000000}{\fontsize{11}{11}\selectfont{\global\setmainfont{Arial}{0.3484}}}} & \multicolumn{1}{>{\raggedleft}m{\dimexpr 0.94in+0\tabcolsep}}{\textcolor[HTML]{000000}{\fontsize{11}{11}\selectfont{\global\setmainfont{Arial}{0.2301}}}} & \multicolumn{1}{>{\raggedleft}m{\dimexpr 0.94in+0\tabcolsep}}{\textcolor[HTML]{000000}{\fontsize{11}{11}\selectfont{\global\setmainfont{Arial}{0.2247}}}} & \multicolumn{1}{>{\raggedleft}m{\dimexpr 0.94in+0\tabcolsep}}{\textcolor[HTML]{000000}{\fontsize{11}{11}\selectfont{\global\setmainfont{Arial}{0.1503}}}} & \multicolumn{1}{>{\raggedleft}m{\dimexpr 0.94in+0\tabcolsep}}{\textcolor[HTML]{000000}{\fontsize{11}{11}\selectfont{\global\setmainfont{Arial}{0.0465}}}} & \multicolumn{1}{>{\raggedleft}m{\dimexpr 0.94in+0\tabcolsep}}{\textcolor[HTML]{000000}{\fontsize{11}{11}\selectfont{\global\setmainfont{Arial}{}}}} & \multicolumn{1}{>{\raggedleft}m{\dimexpr 0.94in+0\tabcolsep}}{\textcolor[HTML]{000000}{\fontsize{11}{11}\selectfont{\global\setmainfont{Arial}{}}}} & \multicolumn{1}{>{\raggedleft}m{\dimexpr 0.94in+0\tabcolsep}}{\textcolor[HTML]{000000}{\fontsize{11}{11}\selectfont{\global\setmainfont{Arial}{}}}} \\





\multicolumn{1}{>{\raggedleft}m{\dimexpr 1.44in+0\tabcolsep}}{\textcolor[HTML]{000000}{\fontsize{11}{11}\selectfont{\global\setmainfont{Arial}{6-cluster\ solution}}}} & \multicolumn{1}{>{\raggedleft}m{\dimexpr 0.94in+0\tabcolsep}}{\textcolor[HTML]{000000}{\fontsize{11}{11}\selectfont{\global\setmainfont{Arial}{0.3484}}}} & \multicolumn{1}{>{\raggedleft}m{\dimexpr 0.94in+0\tabcolsep}}{\textcolor[HTML]{000000}{\fontsize{11}{11}\selectfont{\global\setmainfont{Arial}{0.1290}}}} & \multicolumn{1}{>{\raggedleft}m{\dimexpr 0.94in+0\tabcolsep}}{\textcolor[HTML]{000000}{\fontsize{11}{11}\selectfont{\global\setmainfont{Arial}{0.1011}}}} & \multicolumn{1}{>{\raggedleft}m{\dimexpr 0.94in+0\tabcolsep}}{\textcolor[HTML]{000000}{\fontsize{11}{11}\selectfont{\global\setmainfont{Arial}{0.2247}}}} & \multicolumn{1}{>{\raggedleft}m{\dimexpr 0.94in+0\tabcolsep}}{\textcolor[HTML]{000000}{\fontsize{11}{11}\selectfont{\global\setmainfont{Arial}{0.1503}}}} & \multicolumn{1}{>{\raggedleft}m{\dimexpr 0.94in+0\tabcolsep}}{\textcolor[HTML]{000000}{\fontsize{11}{11}\selectfont{\global\setmainfont{Arial}{0.0465}}}} & \multicolumn{1}{>{\raggedleft}m{\dimexpr 0.94in+0\tabcolsep}}{\textcolor[HTML]{000000}{\fontsize{11}{11}\selectfont{\global\setmainfont{Arial}{}}}} & \multicolumn{1}{>{\raggedleft}m{\dimexpr 0.94in+0\tabcolsep}}{\textcolor[HTML]{000000}{\fontsize{11}{11}\selectfont{\global\setmainfont{Arial}{}}}} \\





\multicolumn{1}{>{\raggedleft}m{\dimexpr 1.44in+0\tabcolsep}}{\textcolor[HTML]{000000}{\fontsize{11}{11}\selectfont{\global\setmainfont{Arial}{7-cluster\ solution}}}} & \multicolumn{1}{>{\raggedleft}m{\dimexpr 0.94in+0\tabcolsep}}{\textcolor[HTML]{000000}{\fontsize{11}{11}\selectfont{\global\setmainfont{Arial}{0.2460}}}} & \multicolumn{1}{>{\raggedleft}m{\dimexpr 0.94in+0\tabcolsep}}{\textcolor[HTML]{000000}{\fontsize{11}{11}\selectfont{\global\setmainfont{Arial}{0.1290}}}} & \multicolumn{1}{>{\raggedleft}m{\dimexpr 0.94in+0\tabcolsep}}{\textcolor[HTML]{000000}{\fontsize{11}{11}\selectfont{\global\setmainfont{Arial}{0.1024}}}} & \multicolumn{1}{>{\raggedleft}m{\dimexpr 0.94in+0\tabcolsep}}{\textcolor[HTML]{000000}{\fontsize{11}{11}\selectfont{\global\setmainfont{Arial}{0.1011}}}} & \multicolumn{1}{>{\raggedleft}m{\dimexpr 0.94in+0\tabcolsep}}{\textcolor[HTML]{000000}{\fontsize{11}{11}\selectfont{\global\setmainfont{Arial}{0.2247}}}} & \multicolumn{1}{>{\raggedleft}m{\dimexpr 0.94in+0\tabcolsep}}{\textcolor[HTML]{000000}{\fontsize{11}{11}\selectfont{\global\setmainfont{Arial}{0.1503}}}} & \multicolumn{1}{>{\raggedleft}m{\dimexpr 0.94in+0\tabcolsep}}{\textcolor[HTML]{000000}{\fontsize{11}{11}\selectfont{\global\setmainfont{Arial}{0.0465}}}} & \multicolumn{1}{>{\raggedleft}m{\dimexpr 0.94in+0\tabcolsep}}{\textcolor[HTML]{000000}{\fontsize{11}{11}\selectfont{\global\setmainfont{Arial}{}}}} \\





\multicolumn{1}{>{\raggedleft}m{\dimexpr 1.44in+0\tabcolsep}}{\textcolor[HTML]{000000}{\fontsize{11}{11}\selectfont{\global\setmainfont{Arial}{8-cluster\ solution}}}} & \multicolumn{1}{>{\raggedleft}m{\dimexpr 0.94in+0\tabcolsep}}{\textcolor[HTML]{000000}{\fontsize{11}{11}\selectfont{\global\setmainfont{Arial}{0.2460}}}} & \multicolumn{1}{>{\raggedleft}m{\dimexpr 0.94in+0\tabcolsep}}{\textcolor[HTML]{000000}{\fontsize{11}{11}\selectfont{\global\setmainfont{Arial}{0.1290}}}} & \multicolumn{1}{>{\raggedleft}m{\dimexpr 0.94in+0\tabcolsep}}{\textcolor[HTML]{000000}{\fontsize{11}{11}\selectfont{\global\setmainfont{Arial}{0.1024}}}} & \multicolumn{1}{>{\raggedleft}m{\dimexpr 0.94in+0\tabcolsep}}{\textcolor[HTML]{000000}{\fontsize{11}{11}\selectfont{\global\setmainfont{Arial}{0.1011}}}} & \multicolumn{1}{>{\raggedleft}m{\dimexpr 0.94in+0\tabcolsep}}{\textcolor[HTML]{000000}{\fontsize{11}{11}\selectfont{\global\setmainfont{Arial}{0.1356}}}} & \multicolumn{1}{>{\raggedleft}m{\dimexpr 0.94in+0\tabcolsep}}{\textcolor[HTML]{000000}{\fontsize{11}{11}\selectfont{\global\setmainfont{Arial}{0.1503}}}} & \multicolumn{1}{>{\raggedleft}m{\dimexpr 0.94in+0\tabcolsep}}{\textcolor[HTML]{000000}{\fontsize{11}{11}\selectfont{\global\setmainfont{Arial}{0.0891}}}} & \multicolumn{1}{>{\raggedleft}m{\dimexpr 0.94in+0\tabcolsep}}{\textcolor[HTML]{000000}{\fontsize{11}{11}\selectfont{\global\setmainfont{Arial}{0.0465}}}} \\

\ascline{1.5pt}{666666}{1-9}



\end{longtable}



\arrayrulecolor[HTML]{000000}

\global\setlength{\arrayrulewidth}{\Oldarrayrulewidth}

\global\setlength{\tabcolsep}{\Oldtabcolsep}

\renewcommand*{\arraystretch}{1}

\subsection{Selecting the Number of Clusters}\label{selecting-the-number-of-clusters}

When selecting the number of clusters, consider:

\begin{itemize}
\tightlist
\item
  Statistical indicators such as the Gap statistic
\item
  Whether clusters are reasonably balanced in size
\item
  Whether the resulting segments are interpretable and actionable
\end{itemize}

Extremely small clusters or one dominant cluster are often warning signs in marketing segmentation.

\subsection{Final Hierarchical Clustering Solution}\label{final-hierarchical-clustering-solution}

Once a reasonable number of clusters has been selected, we finalize the hierarchical clustering solution and attach cluster membership back to the full dataset using the \texttt{easy\_hc\_final()} function from the \texttt{MKT4320BGSU} package.

Usage:

\begin{itemize}
\tightlist
\item
  \texttt{easy\_hc\_final(fit,\ data,\ k,\ cluster\_col\ =\ "cluster",\ conf\_level\ =\ 0.95,}\strut \\
  \texttt{auto\_print\ =\ TRUE)}
\item
  where:

  \begin{itemize}
  \tightlist
  \item
    \texttt{fit} is the object returned by \texttt{easy\_hc\_fit}.
  \item
    \texttt{data} is the original full dataset used to create fit.
  \item
    \texttt{k} is an integer representing the number of clusters to extract.
  \item
    \texttt{cluster\_col} is the name of the cluster column to append to \texttt{data} (default ``cluster''). Must not already exist in data.
  \item
    \texttt{conf\_level} confidence level for CI error bars when plotting (default 0.95).
  \item
    \texttt{auto\_print} is logical; if TRUE (default), prints props and profile and displays the plot (if available).
  \end{itemize}
\end{itemize}

Using the example from above with a 3-cluster solution, the cluster profile table reports mean values of the segmentation variables for each cluster. These means should be interpreted in relative terms across clusters. The cluster plot visualizes those means.

\begin{Shaded}
\begin{Highlighting}[]
\NormalTok{hc\_final }\OtherTok{\textless{}{-}} \FunctionTok{easy\_hc\_final}\NormalTok{(}\AttributeTok{fit =}\NormalTok{ hc\_fit, }\AttributeTok{data =}\NormalTok{ ffseg, }\AttributeTok{k =} \DecValTok{3}\NormalTok{, }
                          \AttributeTok{cluster\_col =} \StringTok{"hc\_cluster"}\NormalTok{, }\AttributeTok{auto\_print =} \ConstantTok{FALSE}\NormalTok{)}
\NormalTok{hc\_final}\SpecialCharTok{$}\NormalTok{props}
\end{Highlighting}
\end{Shaded}

\begin{verbatim}
  Cluster   N      Prop
1       1 297 0.3949468
2       2 286 0.3803191
3       3 169 0.2247340
\end{verbatim}

\begin{Shaded}
\begin{Highlighting}[]
\NormalTok{hc\_final}\SpecialCharTok{$}\NormalTok{profile}
\end{Highlighting}
\end{Shaded}

\begin{verbatim}
  Cluster  quality    price  healthy  variety    speed
1       1 4.737374 4.050505 3.996633 3.905724 3.936027
2       2 3.723776 4.520979 2.339161 2.979021 3.674825
3       3 3.905325 3.047337 2.568047 2.952663 3.112426
\end{verbatim}

\begin{Shaded}
\begin{Highlighting}[]
\NormalTok{hc\_final}\SpecialCharTok{$}\NormalTok{plot}
\end{Highlighting}
\end{Shaded}

\pandocbounded{\includegraphics[keepaspectratio]{MKT4320_R_Tutorial_files/figure-latex/hc_final-1.pdf}}

\begin{center}\rule{0.5\linewidth}{0.5pt}\end{center}

\section{Describing and Labeling Hierarchical Clusters}\label{sec-describe-clusters}

Segmentation variables alone rarely provide enough context to understand who the clusters represent. Additional variables are used to describe and label the clusters. For both hierarchical clustering and \emph{k}-means clustering (see below), use the \texttt{easy\_cluster\_decribe()} function from the \texttt{MKT4320BGSU} package to automate this process.

Usage:

\begin{itemize}
\tightlist
\item
  \texttt{easy\_cluster\_describe(data,\ cluster\_col\ =\ "cluster",\ var,\ alpha\ =\ 0.05,}\strut \\
  \texttt{drop\_missing\ =\ TRUE,\ auto\_print\ =\ TRUE),\ digits\ =\ 4}
\item
  where:

  \begin{itemize}
  \tightlist
  \item
    \texttt{data} is the data frame containing the cluster membership column and variables to describe that was saved to the \texttt{easy\_hc\_final} object; should be \texttt{objectname\$data}.
  \item
    \texttt{cluster\_col} is a character string naming the cluster membership column in data (default = ``cluster'').
  \item
    \texttt{var} is a character string naming the single variable to describe.
  \item
    \texttt{alpha} is the significance level for hypothesis tests (default = 0.05).
  \item
    \texttt{drop\_missing} is logical; if TRUE (default), rows with missing cluster membership are dropped prior to analysis.
  \item
    \texttt{auto\_print} is logical; if TRUE (default), prints summaries to the console; if FALSE returns results list
  \item
    \texttt{digits} is an integer for rounding/formatting of numeric output (default = 4).
  \end{itemize}
\end{itemize}

Using the results from above (i.e., the \texttt{hc\_final\$data} object) , the output below highlights which variables significantly differentiate clusters and provides detailed summaries only where differences are statistically meaningful.

\begin{Shaded}
\begin{Highlighting}[]
\FunctionTok{easy\_cluster\_describe}\NormalTok{(}\AttributeTok{data =}\NormalTok{ hc\_final}\SpecialCharTok{$}\NormalTok{data, }\AttributeTok{cluster\_col =} \StringTok{"hc\_cluster"}\NormalTok{,}
                      \AttributeTok{var=}\StringTok{"usertype"}\NormalTok{)}
\end{Highlighting}
\end{Shaded}

\begin{verbatim}

Variable: usertype (categorical)

Overall test
Variable  Test        p_value  Significant
--------  ----------  -------  -----------
usertype  Chi-square  0.05500  FALSE      

Not significant at alpha = 0.05; no post-hoc shown.
\end{verbatim}

\begin{Shaded}
\begin{Highlighting}[]
\FunctionTok{easy\_cluster\_describe}\NormalTok{(}\AttributeTok{data =}\NormalTok{ hc\_final}\SpecialCharTok{$}\NormalTok{data, }\AttributeTok{cluster\_col =} \StringTok{"hc\_cluster"}\NormalTok{,}
                      \AttributeTok{var=}\StringTok{"spend"}\NormalTok{)}
\end{Highlighting}
\end{Shaded}

\begin{verbatim}

Variable: spend (categorical)

Overall test
Variable  Test        p_value   Significant
--------  ----------  --------  -----------
spend     Chi-square  0.003500  TRUE       

Cross-tab (column %; clusters are columns)
Level         1      2      3    
------------  -----  -----  -----
$10 or more   25.25  17.48  26.63
$5 to $9      61.95  63.64  65.68
Less than $5  12.79  18.88  7.692

Post-hoc (Holm-adjusted p-values)
Row                       1:2      1:3     2:3     
------------------------  -------  ------  --------
$10 or more:$5 to $9      0.3758   1.000   0.5700  
$10 or more:Less than $5  0.06200  0.5700  0.002300
$5 to $9:Less than $5     0.5700   0.5700  0.03350 
\end{verbatim}

\begin{Shaded}
\begin{Highlighting}[]
\FunctionTok{easy\_cluster\_describe}\NormalTok{(}\AttributeTok{data =}\NormalTok{ hc\_final}\SpecialCharTok{$}\NormalTok{data, }\AttributeTok{cluster\_col =} \StringTok{"hc\_cluster"}\NormalTok{,}
                      \AttributeTok{var=}\StringTok{"hours"}\NormalTok{)}
\end{Highlighting}
\end{Shaded}

\begin{verbatim}

Variable: hours (continuous)

Overall test
Variable  Test   p_value  Significant
--------  -----  -------  -----------
hours     ANOVA  0        TRUE       

Cluster means (N, Mean, SD)
Cluster  N      Mean   SD    
-------  -----  -----  ------
1        297.0  3.710  0.9428
2        286.0  3.542  1.014 
3        169.0  3.065  1.070 

Significant differences: 1 > 3, 2 > 3

Post-hoc comparisons (Games-Howell)
.y.    group1  group2  estimate  conf.low  conf.high  p.adj            p.adj.signif
-----  ------  ------  --------  --------  ---------  ---------------  ------------
hours  1       2       -0.1685   -0.3592   0.02224    0.09600          ns          
hours  1       3       -0.6453   -0.8781   -0.4126    0.0000000007900  ****        
hours  2       3       -0.4769   -0.7166   -0.2372    0.00001220       ****        
\end{verbatim}

\begin{Shaded}
\begin{Highlighting}[]
\FunctionTok{easy\_cluster\_describe}\NormalTok{(}\AttributeTok{data =}\NormalTok{ hc\_final}\SpecialCharTok{$}\NormalTok{data, }\AttributeTok{cluster\_col =} \StringTok{"hc\_cluster"}\NormalTok{,}
                      \AttributeTok{var=}\StringTok{"eatin"}\NormalTok{)}
\end{Highlighting}
\end{Shaded}

\begin{verbatim}

Variable: eatin (continuous)

Overall test
Variable  Test   p_value  Significant
--------  -----  -------  -----------
eatin     ANOVA  0.02040  TRUE       

Cluster means (N, Mean, SD)
Cluster  N      Mean   SD   
-------  -----  -----  -----
1        297.0  3.986  2.000
2        286.0  4.441  1.978
3        169.0  4.160  1.903

Significant differences: 2 > 1

Post-hoc comparisons (Games-Howell)
.y.    group1  group2  estimate  conf.low  conf.high  p.adj    p.adj.signif
-----  ------  ------  --------  --------  ---------  -------  ------------
eatin  1       2       0.4540    0.06692   0.8411     0.01700  *           
eatin  1       3       0.1732    -0.2664   0.6129     0.6230   ns          
eatin  2       3       -0.2808   -0.7218   0.1602     0.2930   ns          
\end{verbatim}

\begin{Shaded}
\begin{Highlighting}[]
\FunctionTok{easy\_cluster\_describe}\NormalTok{(}\AttributeTok{data =}\NormalTok{ hc\_final}\SpecialCharTok{$}\NormalTok{data, }\AttributeTok{cluster\_col =} \StringTok{"hc\_cluster"}\NormalTok{,}
                      \AttributeTok{var=}\StringTok{"mealplan"}\NormalTok{)}
\end{Highlighting}
\end{Shaded}

\begin{verbatim}

Variable: mealplan (categorical)

Overall test
Variable  Test        p_value  Significant
--------  ----------  -------  -----------
mealplan  Chi-square  0.8882   FALSE      

Not significant at alpha = 0.05; no post-hoc shown.
\end{verbatim}

\begin{Shaded}
\begin{Highlighting}[]
\FunctionTok{easy\_cluster\_describe}\NormalTok{(}\AttributeTok{data =}\NormalTok{ hc\_final}\SpecialCharTok{$}\NormalTok{data, }\AttributeTok{cluster\_col =} \StringTok{"hc\_cluster"}\NormalTok{,}
                      \AttributeTok{var=}\StringTok{"gender"}\NormalTok{)}
\end{Highlighting}
\end{Shaded}

\begin{verbatim}

Variable: gender (categorical)

Overall test
Variable  Test        p_value  Significant
--------  ----------  -------  -----------
gender    Chi-square  0        TRUE       

Cross-tab (column %; clusters are columns)
Level   1      2      3    
------  -----  -----  -----
Female  81.82  67.83  59.17
Male    18.18  32.17  40.83

Post-hoc (Holm-adjusted p-values)
Row          1:2        1:3  2:3    
-----------  ---------  ---  -------
Female:Male  0.0002000  0    0.06810
\end{verbatim}

\begin{center}\rule{0.5\linewidth}{0.5pt}\end{center}

\section{K-Means Clustering}\label{k-means-clustering}

\subsection{Conceptual Overview}\label{conceptual-overview-1}

\emph{k}-means clustering is a partition-based method that assigns observations to a fixed number of clusters by minimizing within-cluster variation. Unlike hierarchical clustering, \emph{k}-means requires the analyst to specify the number of clusters in advance.

\emph{k}-means is often preferred when:

\begin{itemize}
\tightlist
\item
  Working with larger datasets
\item
  Refining a solution suggested by hierarchical clustering
\item
  Stability and computational efficiency are priorities
\end{itemize}

The process used for \emph{k}-means clustering follows the process used above in hierarchical agglomerative clustering. That is, we first initially fit a number of potential solutions, and then we anlayze a final solution.

\subsection{Initial K-Means Clustering Fit}\label{initial-k-means-clustering-fit}

We first fit \emph{k}-means clustering solutions across a range of cluster counts using the \texttt{easy\_km\_fit()} function from the \texttt{MKT4320BGSU} package.

Usage:

\begin{itemize}
\tightlist
\item
  \texttt{easy\_km\_fit(data,\ vars,\ k\_range\ =\ 1:10,\ standardize\ =\ TRUE,\ nstart\ =\ 25,}\strut \\
  \texttt{iter.max\ =\ 100,\ B\ =\ 20,\ seed\ =\ 4320)}
\item
  where:

  \begin{itemize}
  \tightlist
  \item
    \texttt{data} is a data frame containing the full dataset.
  \item
    \texttt{vars} is a character vector of numeric variable names used for clustering.
  \item
    \texttt{k\_range} is an integer vector of cluster counts to evaluate (default = 1:10; allowed values 1--20).
  \item
    \texttt{standardize} is logical; if TRUE (default), clustering variables are standardized before fitting k-means.
  \item
    \texttt{nstart} is an integer; number of random starts for each k-means solution (default = 25).
  \item
    \texttt{iter.max} is an integer; maximum number of iterations allowed for each k-means run (default = 100).
  \item
    \texttt{B} is an integer; number of Monte Carlo bootstrap samples used to compute the Gap statistic (default = 20).
  \item
    \texttt{seed} is an integer; random seed for reproducible results (default = 4320).
  \end{itemize}
\end{itemize}

We now fit the same variables we used above in hierarchical clustering. As before, the results should be saved to an object. The \texttt{\$diag} diagnostics table saved in the object provides the necessary results to help decide on a (potential) final solution. The diagnostics table summarizes multiple stopping rules and balance checks across different values of \(k\). No single statistic should be used in isolation when choosing the number of clusters. You can also view the \texttt{\$size\_prop} table to see the cluster sizes for all potential solutions.

\begin{Shaded}
\begin{Highlighting}[]
\NormalTok{km\_vars }\OtherTok{\textless{}{-}} \FunctionTok{c}\NormalTok{(}\StringTok{"quality"}\NormalTok{, }\StringTok{"price"}\NormalTok{, }\StringTok{"healthy"}\NormalTok{, }\StringTok{"variety"}\NormalTok{, }\StringTok{"speed"}\NormalTok{)}
\NormalTok{km\_fit }\OtherTok{\textless{}{-}} \FunctionTok{easy\_km\_fit}\NormalTok{(}\AttributeTok{data =}\NormalTok{ ffseg, }\AttributeTok{vars =}\NormalTok{ km\_vars, }\AttributeTok{k\_range =} \DecValTok{2}\SpecialCharTok{:}\DecValTok{8}\NormalTok{)}
\end{Highlighting}
\end{Shaded}

\begin{verbatim}
Clustering k = 1,2,..., K.max (= 8): .. done
Bootstrapping, b = 1,2,..., B (= 20)  [one "." per sample]:
.................... 20 
\end{verbatim}

\begin{Shaded}
\begin{Highlighting}[]
\NormalTok{km\_fit}\SpecialCharTok{$}\NormalTok{diag}
\end{Highlighting}
\end{Shaded}

\global\setlength{\Oldarrayrulewidth}{\arrayrulewidth}

\global\setlength{\Oldtabcolsep}{\tabcolsep}

\setlength{\tabcolsep}{2pt}

\renewcommand*{\arraystretch}{1.5}



\providecommand{\ascline}[3]{\noalign{\global\arrayrulewidth #1}\arrayrulecolor[HTML]{#2}\cline{#3}}

\begin{longtable}[c]{|p{0.90in}|p{1.02in}|p{0.91in}|p{1.08in}|p{1.09in}|p{0.76in}}



\ascline{1.5pt}{666666}{1-6}

\multicolumn{6}{>{\centering}m{\dimexpr 5.77in+10\tabcolsep}}{\textcolor[HTML]{000000}{\fontsize{11}{11}\selectfont{\global\setmainfont{Arial}{\textbf{K-Means\ Cluster\ Diagnostics\ (Standardized)}}}}} \\

\ascline{1.5pt}{666666}{1-6}



\multicolumn{1}{>{\centering}m{\dimexpr 0.9in+0\tabcolsep}}{\textcolor[HTML]{000000}{\fontsize{11}{11}\selectfont{\global\setmainfont{Arial}{\textbf{Clusters}}}}} & \multicolumn{1}{>{\centering}m{\dimexpr 1.02in+0\tabcolsep}}{\textcolor[HTML]{000000}{\fontsize{11}{11}\selectfont{\global\setmainfont{Arial}{\textbf{Silhouette}}}}} & \multicolumn{1}{>{\centering}m{\dimexpr 0.91in+0\tabcolsep}}{\textcolor[HTML]{000000}{\fontsize{11}{11}\selectfont{\global\setmainfont{Arial}{\textbf{Gap.Stat}}}}} & \multicolumn{1}{>{\centering}m{\dimexpr 1.08in+0\tabcolsep}}{\textcolor[HTML]{000000}{\fontsize{11}{11}\selectfont{\global\setmainfont{Arial}{\textbf{Small.Prop}}}}} & \multicolumn{1}{>{\centering}m{\dimexpr 1.09in+0\tabcolsep}}{\textcolor[HTML]{000000}{\fontsize{11}{11}\selectfont{\global\setmainfont{Arial}{\textbf{Large.Prop}}}}} & \multicolumn{1}{>{\centering}m{\dimexpr 0.76in+0\tabcolsep}}{\textcolor[HTML]{000000}{\fontsize{11}{11}\selectfont{\global\setmainfont{Arial}{\textbf{CV}}}}} \\

\ascline{1.5pt}{666666}{1-6}\endfirsthead 

\ascline{1.5pt}{666666}{1-6}

\multicolumn{6}{>{\centering}m{\dimexpr 5.77in+10\tabcolsep}}{\textcolor[HTML]{000000}{\fontsize{11}{11}\selectfont{\global\setmainfont{Arial}{\textbf{K-Means\ Cluster\ Diagnostics\ (Standardized)}}}}} \\

\ascline{1.5pt}{666666}{1-6}



\multicolumn{1}{>{\centering}m{\dimexpr 0.9in+0\tabcolsep}}{\textcolor[HTML]{000000}{\fontsize{11}{11}\selectfont{\global\setmainfont{Arial}{\textbf{Clusters}}}}} & \multicolumn{1}{>{\centering}m{\dimexpr 1.02in+0\tabcolsep}}{\textcolor[HTML]{000000}{\fontsize{11}{11}\selectfont{\global\setmainfont{Arial}{\textbf{Silhouette}}}}} & \multicolumn{1}{>{\centering}m{\dimexpr 0.91in+0\tabcolsep}}{\textcolor[HTML]{000000}{\fontsize{11}{11}\selectfont{\global\setmainfont{Arial}{\textbf{Gap.Stat}}}}} & \multicolumn{1}{>{\centering}m{\dimexpr 1.08in+0\tabcolsep}}{\textcolor[HTML]{000000}{\fontsize{11}{11}\selectfont{\global\setmainfont{Arial}{\textbf{Small.Prop}}}}} & \multicolumn{1}{>{\centering}m{\dimexpr 1.09in+0\tabcolsep}}{\textcolor[HTML]{000000}{\fontsize{11}{11}\selectfont{\global\setmainfont{Arial}{\textbf{Large.Prop}}}}} & \multicolumn{1}{>{\centering}m{\dimexpr 0.76in+0\tabcolsep}}{\textcolor[HTML]{000000}{\fontsize{11}{11}\selectfont{\global\setmainfont{Arial}{\textbf{CV}}}}} \\

\ascline{1.5pt}{666666}{1-6}\endhead



\multicolumn{6}{>{\raggedright}m{\dimexpr 5.77in+10\tabcolsep}}{\textcolor[HTML]{000000}{\fontsize{11}{11}\selectfont{\global\setmainfont{Arial}{*\ 1-SE\ Gap\ rule.}}}} \\





\multicolumn{6}{>{\raggedright}m{\dimexpr 5.77in+10\tabcolsep}}{\textcolor[HTML]{000000}{\fontsize{11}{11}\selectfont{\global\setmainfont{Arial}{\\\textasciicircum \ Smallest\ cluster\ <\ 5\%\ of\ sample.}}}} \\





\multicolumn{6}{>{\raggedright}m{\dimexpr 5.77in+10\tabcolsep}}{\textcolor[HTML]{000000}{\fontsize{11}{11}\selectfont{\global\setmainfont{Arial}{◊\ Largest\ cluster\ >\ 50\%\ of\ sample.}}}} \\





\multicolumn{6}{>{\raggedright}m{\dimexpr 5.77in+10\tabcolsep}}{\textcolor[HTML]{000000}{\fontsize{11}{11}\selectfont{\global\setmainfont{Arial}{•\ CV\ <\ 0.5\ well\ balanced;\ ••\ moderately\ imbalanced.}}}} \\





\multicolumn{6}{>{\raggedright}m{\dimexpr 5.77in+10\tabcolsep}}{\textcolor[HTML]{000000}{\fontsize{11}{11}\selectfont{\global\setmainfont{Arial}{Silhouette\ is\ average\ silhouette\ width\ (higher\ is\ better);\ defined\ for\ k\ >=\ 2.}}}} \\

\endlastfoot



\multicolumn{1}{>{\raggedleft}m{\dimexpr 0.9in+0\tabcolsep}}{\textcolor[HTML]{000000}{\fontsize{11}{11}\selectfont{\global\setmainfont{Arial}{2}}}} & \multicolumn{1}{>{\raggedleft}m{\dimexpr 1.02in+0\tabcolsep}}{\textcolor[HTML]{000000}{\fontsize{11}{11}\selectfont{\global\setmainfont{Arial}{0.2120}}}} & \multicolumn{1}{>{\raggedleft}m{\dimexpr 0.91in+0\tabcolsep}}{\textcolor[HTML]{000000}{\fontsize{11}{11}\selectfont{\global\setmainfont{Arial}{0.6307*}}}} & \multicolumn{1}{>{\raggedleft}m{\dimexpr 1.08in+0\tabcolsep}}{\textcolor[HTML]{000000}{\fontsize{11}{11}\selectfont{\global\setmainfont{Arial}{0.4761}}}} & \multicolumn{1}{>{\raggedleft}m{\dimexpr 1.09in+0\tabcolsep}}{\textcolor[HTML]{000000}{\fontsize{11}{11}\selectfont{\global\setmainfont{Arial}{0.5239\ ◊}}}} & \multicolumn{1}{>{\raggedleft}m{\dimexpr 0.76in+0\tabcolsep}}{\textcolor[HTML]{000000}{\fontsize{11}{11}\selectfont{\global\setmainfont{Arial}{0.068\ •}}}} \\





\multicolumn{1}{>{\raggedleft}m{\dimexpr 0.9in+0\tabcolsep}}{\textcolor[HTML]{000000}{\fontsize{11}{11}\selectfont{\global\setmainfont{Arial}{3}}}} & \multicolumn{1}{>{\raggedleft}m{\dimexpr 1.02in+0\tabcolsep}}{\textcolor[HTML]{000000}{\fontsize{11}{11}\selectfont{\global\setmainfont{Arial}{0.1973}}}} & \multicolumn{1}{>{\raggedleft}m{\dimexpr 0.91in+0\tabcolsep}}{\textcolor[HTML]{000000}{\fontsize{11}{11}\selectfont{\global\setmainfont{Arial}{0.6150}}}} & \multicolumn{1}{>{\raggedleft}m{\dimexpr 1.08in+0\tabcolsep}}{\textcolor[HTML]{000000}{\fontsize{11}{11}\selectfont{\global\setmainfont{Arial}{0.2739}}}} & \multicolumn{1}{>{\raggedleft}m{\dimexpr 1.09in+0\tabcolsep}}{\textcolor[HTML]{000000}{\fontsize{11}{11}\selectfont{\global\setmainfont{Arial}{0.4096}}}} & \multicolumn{1}{>{\raggedleft}m{\dimexpr 0.76in+0\tabcolsep}}{\textcolor[HTML]{000000}{\fontsize{11}{11}\selectfont{\global\setmainfont{Arial}{0.208\ •}}}} \\





\multicolumn{1}{>{\raggedleft}m{\dimexpr 0.9in+0\tabcolsep}}{\textcolor[HTML]{000000}{\fontsize{11}{11}\selectfont{\global\setmainfont{Arial}{4}}}} & \multicolumn{1}{>{\raggedleft}m{\dimexpr 1.02in+0\tabcolsep}}{\textcolor[HTML]{000000}{\fontsize{11}{11}\selectfont{\global\setmainfont{Arial}{0.1610}}}} & \multicolumn{1}{>{\raggedleft}m{\dimexpr 0.91in+0\tabcolsep}}{\textcolor[HTML]{000000}{\fontsize{11}{11}\selectfont{\global\setmainfont{Arial}{0.5953}}}} & \multicolumn{1}{>{\raggedleft}m{\dimexpr 1.08in+0\tabcolsep}}{\textcolor[HTML]{000000}{\fontsize{11}{11}\selectfont{\global\setmainfont{Arial}{0.2168}}}} & \multicolumn{1}{>{\raggedleft}m{\dimexpr 1.09in+0\tabcolsep}}{\textcolor[HTML]{000000}{\fontsize{11}{11}\selectfont{\global\setmainfont{Arial}{0.2806}}}} & \multicolumn{1}{>{\raggedleft}m{\dimexpr 0.76in+0\tabcolsep}}{\textcolor[HTML]{000000}{\fontsize{11}{11}\selectfont{\global\setmainfont{Arial}{0.105\ •}}}} \\





\multicolumn{1}{>{\raggedleft}m{\dimexpr 0.9in+0\tabcolsep}}{\textcolor[HTML]{000000}{\fontsize{11}{11}\selectfont{\global\setmainfont{Arial}{5}}}} & \multicolumn{1}{>{\raggedleft}m{\dimexpr 1.02in+0\tabcolsep}}{\textcolor[HTML]{000000}{\fontsize{11}{11}\selectfont{\global\setmainfont{Arial}{0.1679}}}} & \multicolumn{1}{>{\raggedleft}m{\dimexpr 0.91in+0\tabcolsep}}{\textcolor[HTML]{000000}{\fontsize{11}{11}\selectfont{\global\setmainfont{Arial}{0.5855}}}} & \multicolumn{1}{>{\raggedleft}m{\dimexpr 1.08in+0\tabcolsep}}{\textcolor[HTML]{000000}{\fontsize{11}{11}\selectfont{\global\setmainfont{Arial}{0.1569}}}} & \multicolumn{1}{>{\raggedleft}m{\dimexpr 1.09in+0\tabcolsep}}{\textcolor[HTML]{000000}{\fontsize{11}{11}\selectfont{\global\setmainfont{Arial}{0.2340}}}} & \multicolumn{1}{>{\raggedleft}m{\dimexpr 0.76in+0\tabcolsep}}{\textcolor[HTML]{000000}{\fontsize{11}{11}\selectfont{\global\setmainfont{Arial}{0.178\ •}}}} \\





\multicolumn{1}{>{\raggedleft}m{\dimexpr 0.9in+0\tabcolsep}}{\textcolor[HTML]{000000}{\fontsize{11}{11}\selectfont{\global\setmainfont{Arial}{6}}}} & \multicolumn{1}{>{\raggedleft}m{\dimexpr 1.02in+0\tabcolsep}}{\textcolor[HTML]{000000}{\fontsize{11}{11}\selectfont{\global\setmainfont{Arial}{0.1719}}}} & \multicolumn{1}{>{\raggedleft}m{\dimexpr 0.91in+0\tabcolsep}}{\textcolor[HTML]{000000}{\fontsize{11}{11}\selectfont{\global\setmainfont{Arial}{0.5827}}}} & \multicolumn{1}{>{\raggedleft}m{\dimexpr 1.08in+0\tabcolsep}}{\textcolor[HTML]{000000}{\fontsize{11}{11}\selectfont{\global\setmainfont{Arial}{0.1303}}}} & \multicolumn{1}{>{\raggedleft}m{\dimexpr 1.09in+0\tabcolsep}}{\textcolor[HTML]{000000}{\fontsize{11}{11}\selectfont{\global\setmainfont{Arial}{0.2354}}}} & \multicolumn{1}{>{\raggedleft}m{\dimexpr 0.76in+0\tabcolsep}}{\textcolor[HTML]{000000}{\fontsize{11}{11}\selectfont{\global\setmainfont{Arial}{0.224\ •}}}} \\





\multicolumn{1}{>{\raggedleft}m{\dimexpr 0.9in+0\tabcolsep}}{\textcolor[HTML]{000000}{\fontsize{11}{11}\selectfont{\global\setmainfont{Arial}{7}}}} & \multicolumn{1}{>{\raggedleft}m{\dimexpr 1.02in+0\tabcolsep}}{\textcolor[HTML]{000000}{\fontsize{11}{11}\selectfont{\global\setmainfont{Arial}{0.1803}}}} & \multicolumn{1}{>{\raggedleft}m{\dimexpr 0.91in+0\tabcolsep}}{\textcolor[HTML]{000000}{\fontsize{11}{11}\selectfont{\global\setmainfont{Arial}{0.5741}}}} & \multicolumn{1}{>{\raggedleft}m{\dimexpr 1.08in+0\tabcolsep}}{\textcolor[HTML]{000000}{\fontsize{11}{11}\selectfont{\global\setmainfont{Arial}{0.1184}}}} & \multicolumn{1}{>{\raggedleft}m{\dimexpr 1.09in+0\tabcolsep}}{\textcolor[HTML]{000000}{\fontsize{11}{11}\selectfont{\global\setmainfont{Arial}{0.1835}}}} & \multicolumn{1}{>{\raggedleft}m{\dimexpr 0.76in+0\tabcolsep}}{\textcolor[HTML]{000000}{\fontsize{11}{11}\selectfont{\global\setmainfont{Arial}{0.195\ •}}}} \\





\multicolumn{1}{>{\raggedleft}m{\dimexpr 0.9in+0\tabcolsep}}{\textcolor[HTML]{000000}{\fontsize{11}{11}\selectfont{\global\setmainfont{Arial}{8}}}} & \multicolumn{1}{>{\raggedleft}m{\dimexpr 1.02in+0\tabcolsep}}{\textcolor[HTML]{000000}{\fontsize{11}{11}\selectfont{\global\setmainfont{Arial}{0.1804}}}} & \multicolumn{1}{>{\raggedleft}m{\dimexpr 0.91in+0\tabcolsep}}{\textcolor[HTML]{000000}{\fontsize{11}{11}\selectfont{\global\setmainfont{Arial}{0.5747}}}} & \multicolumn{1}{>{\raggedleft}m{\dimexpr 1.08in+0\tabcolsep}}{\textcolor[HTML]{000000}{\fontsize{11}{11}\selectfont{\global\setmainfont{Arial}{0.0851}}}} & \multicolumn{1}{>{\raggedleft}m{\dimexpr 1.09in+0\tabcolsep}}{\textcolor[HTML]{000000}{\fontsize{11}{11}\selectfont{\global\setmainfont{Arial}{0.1582}}}} & \multicolumn{1}{>{\raggedleft}m{\dimexpr 0.76in+0\tabcolsep}}{\textcolor[HTML]{000000}{\fontsize{11}{11}\selectfont{\global\setmainfont{Arial}{0.185\ •}}}} \\

\ascline{1.5pt}{666666}{1-6}



\end{longtable}



\arrayrulecolor[HTML]{000000}

\global\setlength{\arrayrulewidth}{\Oldarrayrulewidth}

\global\setlength{\tabcolsep}{\Oldtabcolsep}

\renewcommand*{\arraystretch}{1}

\begin{Shaded}
\begin{Highlighting}[]
\NormalTok{km\_fit}\SpecialCharTok{$}\NormalTok{size\_prop}
\end{Highlighting}
\end{Shaded}

\global\setlength{\Oldarrayrulewidth}{\arrayrulewidth}

\global\setlength{\Oldtabcolsep}{\tabcolsep}

\setlength{\tabcolsep}{2pt}

\renewcommand*{\arraystretch}{1.5}



\providecommand{\ascline}[3]{\noalign{\global\arrayrulewidth #1}\arrayrulecolor[HTML]{#2}\cline{#3}}

\begin{longtable}[c]{|p{1.44in}|p{0.94in}|p{0.94in}|p{0.94in}|p{0.94in}|p{0.94in}|p{0.94in}|p{0.94in}|p{0.94in}}



\ascline{1.5pt}{666666}{1-9}

\multicolumn{9}{>{\centering}m{\dimexpr 8.95in+16\tabcolsep}}{\textcolor[HTML]{000000}{\fontsize{11}{11}\selectfont{\global\setmainfont{Arial}{\textbf{Cluster\ Size\ Proportions\ by\ Solution}}}}} \\

\ascline{1.5pt}{666666}{1-9}



\multicolumn{1}{>{\centering}m{\dimexpr 1.44in+0\tabcolsep}}{\textcolor[HTML]{000000}{\fontsize{11}{11}\selectfont{\global\setmainfont{Arial}{\textbf{Solution}}}}} & \multicolumn{1}{>{\centering}m{\dimexpr 0.94in+0\tabcolsep}}{\textcolor[HTML]{000000}{\fontsize{11}{11}\selectfont{\global\setmainfont{Arial}{\textbf{Cluster\ 1}}}}} & \multicolumn{1}{>{\centering}m{\dimexpr 0.94in+0\tabcolsep}}{\textcolor[HTML]{000000}{\fontsize{11}{11}\selectfont{\global\setmainfont{Arial}{\textbf{Cluster\ 2}}}}} & \multicolumn{1}{>{\centering}m{\dimexpr 0.94in+0\tabcolsep}}{\textcolor[HTML]{000000}{\fontsize{11}{11}\selectfont{\global\setmainfont{Arial}{\textbf{Cluster\ 3}}}}} & \multicolumn{1}{>{\centering}m{\dimexpr 0.94in+0\tabcolsep}}{\textcolor[HTML]{000000}{\fontsize{11}{11}\selectfont{\global\setmainfont{Arial}{\textbf{Cluster\ 4}}}}} & \multicolumn{1}{>{\centering}m{\dimexpr 0.94in+0\tabcolsep}}{\textcolor[HTML]{000000}{\fontsize{11}{11}\selectfont{\global\setmainfont{Arial}{\textbf{Cluster\ 5}}}}} & \multicolumn{1}{>{\centering}m{\dimexpr 0.94in+0\tabcolsep}}{\textcolor[HTML]{000000}{\fontsize{11}{11}\selectfont{\global\setmainfont{Arial}{\textbf{Cluster\ 6}}}}} & \multicolumn{1}{>{\centering}m{\dimexpr 0.94in+0\tabcolsep}}{\textcolor[HTML]{000000}{\fontsize{11}{11}\selectfont{\global\setmainfont{Arial}{\textbf{Cluster\ 7}}}}} & \multicolumn{1}{>{\centering}m{\dimexpr 0.94in+0\tabcolsep}}{\textcolor[HTML]{000000}{\fontsize{11}{11}\selectfont{\global\setmainfont{Arial}{\textbf{Cluster\ 8}}}}} \\

\ascline{1.5pt}{666666}{1-9}\endfirsthead 

\ascline{1.5pt}{666666}{1-9}

\multicolumn{9}{>{\centering}m{\dimexpr 8.95in+16\tabcolsep}}{\textcolor[HTML]{000000}{\fontsize{11}{11}\selectfont{\global\setmainfont{Arial}{\textbf{Cluster\ Size\ Proportions\ by\ Solution}}}}} \\

\ascline{1.5pt}{666666}{1-9}



\multicolumn{1}{>{\centering}m{\dimexpr 1.44in+0\tabcolsep}}{\textcolor[HTML]{000000}{\fontsize{11}{11}\selectfont{\global\setmainfont{Arial}{\textbf{Solution}}}}} & \multicolumn{1}{>{\centering}m{\dimexpr 0.94in+0\tabcolsep}}{\textcolor[HTML]{000000}{\fontsize{11}{11}\selectfont{\global\setmainfont{Arial}{\textbf{Cluster\ 1}}}}} & \multicolumn{1}{>{\centering}m{\dimexpr 0.94in+0\tabcolsep}}{\textcolor[HTML]{000000}{\fontsize{11}{11}\selectfont{\global\setmainfont{Arial}{\textbf{Cluster\ 2}}}}} & \multicolumn{1}{>{\centering}m{\dimexpr 0.94in+0\tabcolsep}}{\textcolor[HTML]{000000}{\fontsize{11}{11}\selectfont{\global\setmainfont{Arial}{\textbf{Cluster\ 3}}}}} & \multicolumn{1}{>{\centering}m{\dimexpr 0.94in+0\tabcolsep}}{\textcolor[HTML]{000000}{\fontsize{11}{11}\selectfont{\global\setmainfont{Arial}{\textbf{Cluster\ 4}}}}} & \multicolumn{1}{>{\centering}m{\dimexpr 0.94in+0\tabcolsep}}{\textcolor[HTML]{000000}{\fontsize{11}{11}\selectfont{\global\setmainfont{Arial}{\textbf{Cluster\ 5}}}}} & \multicolumn{1}{>{\centering}m{\dimexpr 0.94in+0\tabcolsep}}{\textcolor[HTML]{000000}{\fontsize{11}{11}\selectfont{\global\setmainfont{Arial}{\textbf{Cluster\ 6}}}}} & \multicolumn{1}{>{\centering}m{\dimexpr 0.94in+0\tabcolsep}}{\textcolor[HTML]{000000}{\fontsize{11}{11}\selectfont{\global\setmainfont{Arial}{\textbf{Cluster\ 7}}}}} & \multicolumn{1}{>{\centering}m{\dimexpr 0.94in+0\tabcolsep}}{\textcolor[HTML]{000000}{\fontsize{11}{11}\selectfont{\global\setmainfont{Arial}{\textbf{Cluster\ 8}}}}} \\

\ascline{1.5pt}{666666}{1-9}\endhead



\multicolumn{1}{>{\raggedleft}m{\dimexpr 1.44in+0\tabcolsep}}{\textcolor[HTML]{000000}{\fontsize{11}{11}\selectfont{\global\setmainfont{Arial}{2-cluster\ solution}}}} & \multicolumn{1}{>{\raggedleft}m{\dimexpr 0.94in+0\tabcolsep}}{\textcolor[HTML]{000000}{\fontsize{11}{11}\selectfont{\global\setmainfont{Arial}{0.4761}}}} & \multicolumn{1}{>{\raggedleft}m{\dimexpr 0.94in+0\tabcolsep}}{\textcolor[HTML]{000000}{\fontsize{11}{11}\selectfont{\global\setmainfont{Arial}{0.5239}}}} & \multicolumn{1}{>{\raggedleft}m{\dimexpr 0.94in+0\tabcolsep}}{\textcolor[HTML]{000000}{\fontsize{11}{11}\selectfont{\global\setmainfont{Arial}{}}}} & \multicolumn{1}{>{\raggedleft}m{\dimexpr 0.94in+0\tabcolsep}}{\textcolor[HTML]{000000}{\fontsize{11}{11}\selectfont{\global\setmainfont{Arial}{}}}} & \multicolumn{1}{>{\raggedleft}m{\dimexpr 0.94in+0\tabcolsep}}{\textcolor[HTML]{000000}{\fontsize{11}{11}\selectfont{\global\setmainfont{Arial}{}}}} & \multicolumn{1}{>{\raggedleft}m{\dimexpr 0.94in+0\tabcolsep}}{\textcolor[HTML]{000000}{\fontsize{11}{11}\selectfont{\global\setmainfont{Arial}{}}}} & \multicolumn{1}{>{\raggedleft}m{\dimexpr 0.94in+0\tabcolsep}}{\textcolor[HTML]{000000}{\fontsize{11}{11}\selectfont{\global\setmainfont{Arial}{}}}} & \multicolumn{1}{>{\raggedleft}m{\dimexpr 0.94in+0\tabcolsep}}{\textcolor[HTML]{000000}{\fontsize{11}{11}\selectfont{\global\setmainfont{Arial}{}}}} \\





\multicolumn{1}{>{\raggedleft}m{\dimexpr 1.44in+0\tabcolsep}}{\textcolor[HTML]{000000}{\fontsize{11}{11}\selectfont{\global\setmainfont{Arial}{3-cluster\ solution}}}} & \multicolumn{1}{>{\raggedleft}m{\dimexpr 0.94in+0\tabcolsep}}{\textcolor[HTML]{000000}{\fontsize{11}{11}\selectfont{\global\setmainfont{Arial}{0.3165}}}} & \multicolumn{1}{>{\raggedleft}m{\dimexpr 0.94in+0\tabcolsep}}{\textcolor[HTML]{000000}{\fontsize{11}{11}\selectfont{\global\setmainfont{Arial}{0.4096}}}} & \multicolumn{1}{>{\raggedleft}m{\dimexpr 0.94in+0\tabcolsep}}{\textcolor[HTML]{000000}{\fontsize{11}{11}\selectfont{\global\setmainfont{Arial}{0.2739}}}} & \multicolumn{1}{>{\raggedleft}m{\dimexpr 0.94in+0\tabcolsep}}{\textcolor[HTML]{000000}{\fontsize{11}{11}\selectfont{\global\setmainfont{Arial}{}}}} & \multicolumn{1}{>{\raggedleft}m{\dimexpr 0.94in+0\tabcolsep}}{\textcolor[HTML]{000000}{\fontsize{11}{11}\selectfont{\global\setmainfont{Arial}{}}}} & \multicolumn{1}{>{\raggedleft}m{\dimexpr 0.94in+0\tabcolsep}}{\textcolor[HTML]{000000}{\fontsize{11}{11}\selectfont{\global\setmainfont{Arial}{}}}} & \multicolumn{1}{>{\raggedleft}m{\dimexpr 0.94in+0\tabcolsep}}{\textcolor[HTML]{000000}{\fontsize{11}{11}\selectfont{\global\setmainfont{Arial}{}}}} & \multicolumn{1}{>{\raggedleft}m{\dimexpr 0.94in+0\tabcolsep}}{\textcolor[HTML]{000000}{\fontsize{11}{11}\selectfont{\global\setmainfont{Arial}{}}}} \\





\multicolumn{1}{>{\raggedleft}m{\dimexpr 1.44in+0\tabcolsep}}{\textcolor[HTML]{000000}{\fontsize{11}{11}\selectfont{\global\setmainfont{Arial}{4-cluster\ solution}}}} & \multicolumn{1}{>{\raggedleft}m{\dimexpr 0.94in+0\tabcolsep}}{\textcolor[HTML]{000000}{\fontsize{11}{11}\selectfont{\global\setmainfont{Arial}{0.2168}}}} & \multicolumn{1}{>{\raggedleft}m{\dimexpr 0.94in+0\tabcolsep}}{\textcolor[HTML]{000000}{\fontsize{11}{11}\selectfont{\global\setmainfont{Arial}{0.2487}}}} & \multicolumn{1}{>{\raggedleft}m{\dimexpr 0.94in+0\tabcolsep}}{\textcolor[HTML]{000000}{\fontsize{11}{11}\selectfont{\global\setmainfont{Arial}{0.2540}}}} & \multicolumn{1}{>{\raggedleft}m{\dimexpr 0.94in+0\tabcolsep}}{\textcolor[HTML]{000000}{\fontsize{11}{11}\selectfont{\global\setmainfont{Arial}{0.2806}}}} & \multicolumn{1}{>{\raggedleft}m{\dimexpr 0.94in+0\tabcolsep}}{\textcolor[HTML]{000000}{\fontsize{11}{11}\selectfont{\global\setmainfont{Arial}{}}}} & \multicolumn{1}{>{\raggedleft}m{\dimexpr 0.94in+0\tabcolsep}}{\textcolor[HTML]{000000}{\fontsize{11}{11}\selectfont{\global\setmainfont{Arial}{}}}} & \multicolumn{1}{>{\raggedleft}m{\dimexpr 0.94in+0\tabcolsep}}{\textcolor[HTML]{000000}{\fontsize{11}{11}\selectfont{\global\setmainfont{Arial}{}}}} & \multicolumn{1}{>{\raggedleft}m{\dimexpr 0.94in+0\tabcolsep}}{\textcolor[HTML]{000000}{\fontsize{11}{11}\selectfont{\global\setmainfont{Arial}{}}}} \\





\multicolumn{1}{>{\raggedleft}m{\dimexpr 1.44in+0\tabcolsep}}{\textcolor[HTML]{000000}{\fontsize{11}{11}\selectfont{\global\setmainfont{Arial}{5-cluster\ solution}}}} & \multicolumn{1}{>{\raggedleft}m{\dimexpr 0.94in+0\tabcolsep}}{\textcolor[HTML]{000000}{\fontsize{11}{11}\selectfont{\global\setmainfont{Arial}{0.2128}}}} & \multicolumn{1}{>{\raggedleft}m{\dimexpr 0.94in+0\tabcolsep}}{\textcolor[HTML]{000000}{\fontsize{11}{11}\selectfont{\global\setmainfont{Arial}{0.1569}}}} & \multicolumn{1}{>{\raggedleft}m{\dimexpr 0.94in+0\tabcolsep}}{\textcolor[HTML]{000000}{\fontsize{11}{11}\selectfont{\global\setmainfont{Arial}{0.2340}}}} & \multicolumn{1}{>{\raggedleft}m{\dimexpr 0.94in+0\tabcolsep}}{\textcolor[HTML]{000000}{\fontsize{11}{11}\selectfont{\global\setmainfont{Arial}{0.2287}}}} & \multicolumn{1}{>{\raggedleft}m{\dimexpr 0.94in+0\tabcolsep}}{\textcolor[HTML]{000000}{\fontsize{11}{11}\selectfont{\global\setmainfont{Arial}{0.1676}}}} & \multicolumn{1}{>{\raggedleft}m{\dimexpr 0.94in+0\tabcolsep}}{\textcolor[HTML]{000000}{\fontsize{11}{11}\selectfont{\global\setmainfont{Arial}{}}}} & \multicolumn{1}{>{\raggedleft}m{\dimexpr 0.94in+0\tabcolsep}}{\textcolor[HTML]{000000}{\fontsize{11}{11}\selectfont{\global\setmainfont{Arial}{}}}} & \multicolumn{1}{>{\raggedleft}m{\dimexpr 0.94in+0\tabcolsep}}{\textcolor[HTML]{000000}{\fontsize{11}{11}\selectfont{\global\setmainfont{Arial}{}}}} \\





\multicolumn{1}{>{\raggedleft}m{\dimexpr 1.44in+0\tabcolsep}}{\textcolor[HTML]{000000}{\fontsize{11}{11}\selectfont{\global\setmainfont{Arial}{6-cluster\ solution}}}} & \multicolumn{1}{>{\raggedleft}m{\dimexpr 0.94in+0\tabcolsep}}{\textcolor[HTML]{000000}{\fontsize{11}{11}\selectfont{\global\setmainfont{Arial}{0.1423}}}} & \multicolumn{1}{>{\raggedleft}m{\dimexpr 0.94in+0\tabcolsep}}{\textcolor[HTML]{000000}{\fontsize{11}{11}\selectfont{\global\setmainfont{Arial}{0.1503}}}} & \multicolumn{1}{>{\raggedleft}m{\dimexpr 0.94in+0\tabcolsep}}{\textcolor[HTML]{000000}{\fontsize{11}{11}\selectfont{\global\setmainfont{Arial}{0.1303}}}} & \multicolumn{1}{>{\raggedleft}m{\dimexpr 0.94in+0\tabcolsep}}{\textcolor[HTML]{000000}{\fontsize{11}{11}\selectfont{\global\setmainfont{Arial}{0.1715}}}} & \multicolumn{1}{>{\raggedleft}m{\dimexpr 0.94in+0\tabcolsep}}{\textcolor[HTML]{000000}{\fontsize{11}{11}\selectfont{\global\setmainfont{Arial}{0.1702}}}} & \multicolumn{1}{>{\raggedleft}m{\dimexpr 0.94in+0\tabcolsep}}{\textcolor[HTML]{000000}{\fontsize{11}{11}\selectfont{\global\setmainfont{Arial}{0.2354}}}} & \multicolumn{1}{>{\raggedleft}m{\dimexpr 0.94in+0\tabcolsep}}{\textcolor[HTML]{000000}{\fontsize{11}{11}\selectfont{\global\setmainfont{Arial}{}}}} & \multicolumn{1}{>{\raggedleft}m{\dimexpr 0.94in+0\tabcolsep}}{\textcolor[HTML]{000000}{\fontsize{11}{11}\selectfont{\global\setmainfont{Arial}{}}}} \\





\multicolumn{1}{>{\raggedleft}m{\dimexpr 1.44in+0\tabcolsep}}{\textcolor[HTML]{000000}{\fontsize{11}{11}\selectfont{\global\setmainfont{Arial}{7-cluster\ solution}}}} & \multicolumn{1}{>{\raggedleft}m{\dimexpr 0.94in+0\tabcolsep}}{\textcolor[HTML]{000000}{\fontsize{11}{11}\selectfont{\global\setmainfont{Arial}{0.1263}}}} & \multicolumn{1}{>{\raggedleft}m{\dimexpr 0.94in+0\tabcolsep}}{\textcolor[HTML]{000000}{\fontsize{11}{11}\selectfont{\global\setmainfont{Arial}{0.1184}}}} & \multicolumn{1}{>{\raggedleft}m{\dimexpr 0.94in+0\tabcolsep}}{\textcolor[HTML]{000000}{\fontsize{11}{11}\selectfont{\global\setmainfont{Arial}{0.1516}}}} & \multicolumn{1}{>{\raggedleft}m{\dimexpr 0.94in+0\tabcolsep}}{\textcolor[HTML]{000000}{\fontsize{11}{11}\selectfont{\global\setmainfont{Arial}{0.1835}}}} & \multicolumn{1}{>{\raggedleft}m{\dimexpr 0.94in+0\tabcolsep}}{\textcolor[HTML]{000000}{\fontsize{11}{11}\selectfont{\global\setmainfont{Arial}{0.1769}}}} & \multicolumn{1}{>{\raggedleft}m{\dimexpr 0.94in+0\tabcolsep}}{\textcolor[HTML]{000000}{\fontsize{11}{11}\selectfont{\global\setmainfont{Arial}{0.1210}}}} & \multicolumn{1}{>{\raggedleft}m{\dimexpr 0.94in+0\tabcolsep}}{\textcolor[HTML]{000000}{\fontsize{11}{11}\selectfont{\global\setmainfont{Arial}{0.1223}}}} & \multicolumn{1}{>{\raggedleft}m{\dimexpr 0.94in+0\tabcolsep}}{\textcolor[HTML]{000000}{\fontsize{11}{11}\selectfont{\global\setmainfont{Arial}{}}}} \\





\multicolumn{1}{>{\raggedleft}m{\dimexpr 1.44in+0\tabcolsep}}{\textcolor[HTML]{000000}{\fontsize{11}{11}\selectfont{\global\setmainfont{Arial}{8-cluster\ solution}}}} & \multicolumn{1}{>{\raggedleft}m{\dimexpr 0.94in+0\tabcolsep}}{\textcolor[HTML]{000000}{\fontsize{11}{11}\selectfont{\global\setmainfont{Arial}{0.1330}}}} & \multicolumn{1}{>{\raggedleft}m{\dimexpr 0.94in+0\tabcolsep}}{\textcolor[HTML]{000000}{\fontsize{11}{11}\selectfont{\global\setmainfont{Arial}{0.1436}}}} & \multicolumn{1}{>{\raggedleft}m{\dimexpr 0.94in+0\tabcolsep}}{\textcolor[HTML]{000000}{\fontsize{11}{11}\selectfont{\global\setmainfont{Arial}{0.1396}}}} & \multicolumn{1}{>{\raggedleft}m{\dimexpr 0.94in+0\tabcolsep}}{\textcolor[HTML]{000000}{\fontsize{11}{11}\selectfont{\global\setmainfont{Arial}{0.1582}}}} & \multicolumn{1}{>{\raggedleft}m{\dimexpr 0.94in+0\tabcolsep}}{\textcolor[HTML]{000000}{\fontsize{11}{11}\selectfont{\global\setmainfont{Arial}{0.1117}}}} & \multicolumn{1}{>{\raggedleft}m{\dimexpr 0.94in+0\tabcolsep}}{\textcolor[HTML]{000000}{\fontsize{11}{11}\selectfont{\global\setmainfont{Arial}{0.0851}}}} & \multicolumn{1}{>{\raggedleft}m{\dimexpr 0.94in+0\tabcolsep}}{\textcolor[HTML]{000000}{\fontsize{11}{11}\selectfont{\global\setmainfont{Arial}{0.1157}}}} & \multicolumn{1}{>{\raggedleft}m{\dimexpr 0.94in+0\tabcolsep}}{\textcolor[HTML]{000000}{\fontsize{11}{11}\selectfont{\global\setmainfont{Arial}{0.1130}}}} \\

\ascline{1.5pt}{666666}{1-9}



\end{longtable}



\arrayrulecolor[HTML]{000000}

\global\setlength{\arrayrulewidth}{\Oldarrayrulewidth}

\global\setlength{\tabcolsep}{\Oldtabcolsep}

\renewcommand*{\arraystretch}{1}

\subsection{Final K-Means Clustering Solution}\label{final-k-means-clustering-solution}

After selecting a value for \(k\), we finalize the \emph{k}-means solution using the \texttt{easy\_km\_final()} function from the \texttt{MKT4320BGSU} package.

Usage:

\begin{itemize}
\tightlist
\item
  \texttt{easy\_km\_final(fit,\ data,\ k,\ cluster\_col\ =\ "cluster",\ conf\_level\ =\ 0.95,}
  \texttt{auto\_print\ =\ TRUE)}
\item
  where:

  \begin{itemize}
  \tightlist
  \item
    \texttt{fit} is an object returned by \texttt{easy\_km\_fit()}.
  \item
    \texttt{data} is the original full dataset used in \texttt{easy\_km\_fit()}.
  \item
    \texttt{k} is an integer; number of clusters to extract.
  \item
    \texttt{cluster\_col} is a character string; name of the cluster column to append to data (default = ``cluster'').
  \item
    \texttt{conf\_level} is numeric; confidence level for CI error bars (default = 0.95).
  \item
    \texttt{auto\_print} is logical; if TRUE (default), prints selected outputs and displays the plot when the function is run.
  \end{itemize}
\end{itemize}

Using the example from above with a 3-cluster solution, the cluster profile table reports mean values of the segmentation variables for each cluster. These means should be interpreted in relative terms across clusters. The cluster plot visualizes those means.

\begin{Shaded}
\begin{Highlighting}[]
\NormalTok{km\_final }\OtherTok{\textless{}{-}} \FunctionTok{easy\_km\_final}\NormalTok{(}\AttributeTok{fit =}\NormalTok{ km\_fit, }\AttributeTok{data =}\NormalTok{ ffseg, }\AttributeTok{k =} \DecValTok{3}\NormalTok{, }
                          \AttributeTok{cluster\_col =} \StringTok{"km\_cluster"}\NormalTok{, }\AttributeTok{auto\_print =} \ConstantTok{FALSE}\NormalTok{)}
\NormalTok{km\_final}\SpecialCharTok{$}\NormalTok{props}
\end{Highlighting}
\end{Shaded}

\begin{verbatim}
  Cluster   N      Prop
1       1 238 0.3164894
2       2 308 0.4095745
3       3 206 0.2739362
\end{verbatim}

\begin{Shaded}
\begin{Highlighting}[]
\NormalTok{km\_final}\SpecialCharTok{$}\NormalTok{plot}
\end{Highlighting}
\end{Shaded}

\pandocbounded{\includegraphics[keepaspectratio]{MKT4320_R_Tutorial_files/figure-latex/km_final-1.pdf}}

\begin{Shaded}
\begin{Highlighting}[]
\NormalTok{km\_final}\SpecialCharTok{$}\NormalTok{profile}
\end{Highlighting}
\end{Shaded}

\begin{verbatim}
  Cluster  quality    price  healthy  variety    speed
1       1 3.483193 4.361345 2.117647 2.907563 3.680672
2       2 4.714286 4.399351 3.762987 4.022727 4.000000
3       3 4.131068 3.000000 3.043689 2.815534 3.097087
\end{verbatim}

\begin{center}\rule{0.5\linewidth}{0.5pt}\end{center}

\section{\texorpdfstring{Describing and Labeling \emph{k}-Means Clusters}{Describing and Labeling k-Means Clusters}}\label{describing-and-labeling-k-means-clusters}

As with hierarchical clustering, segmentation variables alone rarely provide enough context to understand who the clusters represent. Additional variables are used to describe and label the clusters. As with hierarchical clustering, use the \texttt{easy\_cluster\_decribe()} function from the \texttt{MKT4320BGSU} package to automate this process. See Section \ref{sec-describe-clusters}

\begin{center}\rule{0.5\linewidth}{0.5pt}\end{center}

\section{Comparing Clustering Approaches}\label{comparing-clustering-approaches}

Hierarchical and \emph{k}-means clustering often produce similar but not identical segmentation results. Differences between solutions can be informative and may reveal alternative ways to think about the market.

In practice, analysts often:

\begin{itemize}
\tightlist
\item
  Use hierarchical clustering to explore the structure of the data
\item
  Use k-means clustering to refine and stabilize the final solution
\end{itemize}

\begin{center}\rule{0.5\linewidth}{0.5pt}\end{center}

\section{Chapter Summary}\label{chapter-summary}

In this chapter, we:

\begin{itemize}
\tightlist
\item
  Introduced cluster analysis as a tool for marketing segmentation
\item
  Applied hierarchical and k-means clustering to the \texttt{ffseg} dataset
\item
  Used diagnostics to guide the choice of the number of clusters
\item
  Described and interpreted clusters in a marketing context
\end{itemize}

\begin{center}\rule{0.5\linewidth}{0.5pt}\end{center}

\section{What's Next}\label{whats-next-10}

In the next chapter, we shift from grouping customers to summarizing and visualizing variables. Specifically, we will introduce Principal Components Analysis (PCA) and then extend it to PCA perceptual maps. PCA is a dimensionality-reduction technique that helps simplify complex datasets by transforming many correlated variables into a smaller set of components that capture the most important patterns in the data.

You will learn how PCA can be used to:

\begin{itemize}
\tightlist
\item
  Reduce a large set of variables into a smaller number of interpretable dimensions
\item
  Identify underlying structures in consumer perceptions and evaluations
\item
  Prepare data for visualization and communication
\end{itemize}

We will then use these components to create perceptual maps, which are widely used in marketing to:

\begin{itemize}
\tightlist
\item
  Visualize brand or product positions
\item
  Understand competitive structure
\item
  Support positioning and differentiation decisions
\end{itemize}

\chapter{PCA and Perceptual Maps}\label{pca-and-perceptual-maps}

\section{Introduction: Why PCA Matters in Marketing Analytics}\label{introduction-why-pca-matters-in-marketing-analytics}

Marketing datasets often contain many related variables that describe how
consumers perceive brands, products, or services. When these attributes are
highly correlated, interpretation becomes difficult and redundancy increases.
Principal Components Analysis (PCA) is a dimension-reduction technique that
helps uncover the underlying structure in such data.

In marketing analytics, PCA is commonly used to:

\begin{itemize}
\tightlist
\item
  Summarize brand image and positioning data
\item
  Reduce large attribute batteries into interpretable dimensions
\item
  Serve as the foundation for perceptual maps
\end{itemize}

In this chapter, we focus on applying PCA for \textbf{interpretation and insight},
not mathematical derivation. We will:

\begin{enumerate}
\def\labelenumi{\arabic{enumi}.}
\tightlist
\item
  Fit a PCA model and evaluate diagnostics
\item
  Choose an appropriate number of components
\item
  Interpret component loadings
\item
  Use PCA results to construct perceptual maps
\end{enumerate}

As a high-level overview, PCA transforms a set of correlated variables into a smaller set of new variables
called \emph{principal components}. Each component is a weighted combination of the
original variables.

Key ideas:

\begin{itemize}
\tightlist
\item
  Components are ordered by how much variance they explain
\item
  The first component explains the most variance, the second explains the most
  remaining variance, and so on
\item
  Components are uncorrelated with one another
\end{itemize}

From a marketing perspective, PCA helps answer:
``What are the main dimensions consumers use to differentiate brands?''

\begin{center}\rule{0.5\linewidth}{0.5pt}\end{center}

\section{\texorpdfstring{The \texttt{greekbrands} Dataset}{The greekbrands Dataset}}\label{the-greekbrands-dataset}

This chapter uses the \texttt{greekbrands} dataset, which contains simulated attribute ratings and brand preference data for ten fictional technology brands. Each observation
corresponds to a respondent-brand evaluation.

The dataset includes:

\begin{itemize}
\tightlist
\item
  A brand identifier
\item
  Multiple numeric attribute ratings describing brand perceptions
\end{itemize}

This type of brand image data is well suited for PCA because many of the
attributes tend to be correlated and may reflect a smaller number of latent
dimensions.

\begin{center}\rule{0.5\linewidth}{0.5pt}\end{center}

\section{Preparing for PCA in a Marketing Context}\label{preparing-for-pca-in-a-marketing-context}

Before fitting a PCA model, it is important to:

\begin{itemize}
\tightlist
\item
  Use only numeric perceptual attributes
\item
  Exclude identifiers (e.g., brand names, respondent IDs)
\item
  Consider whether PCA should be run at the individual or brand level
\end{itemize}

For perceptual mapping, brand-level aggregation is typically preferred so that
brands (not respondents) appear as points in the map.

\begin{center}\rule{0.5\linewidth}{0.5pt}\end{center}

\section{PCA Modeling}\label{pca-modeling}

\subsection{Fitting an Initial PCA Model}\label{fitting-an-initial-pca-model}

We begin with an initial PCA fit to evaluate how many components should be
retained. This step focuses on diagnostics rather than interpretation. Although PCA is not difficult in base R using the \texttt{prcomp()} function, we'll use the \texttt{easy\_pca\_fit()} function from the \texttt{MKT4320BGSU} to automate the process for both fitting and a separate function for the final model.

Usage:

\begin{itemize}
\tightlist
\item
  \texttt{easy\_pca\_fit(data,\ vars,\ group\ =\ NULL,\ ft\ =\ TRUE)}
\item
  where:

  \begin{itemize}
  \tightlist
  \item
    \texttt{data} is a data frame containing the full dataset.
  \item
    \texttt{vars} is a character vector of variable names to use in PCA (required). All variables must be numeric.
  \item
    \texttt{group} is an optional character sting of a single variable name to aggregate by before PCA.
  \item
    \texttt{ft} is logical; if TRUE, return \$table as a flextable (default = TRUE).
  \end{itemize}
\end{itemize}

In the example below, we do not use the \texttt{group} option.

\begin{Shaded}
\begin{Highlighting}[]
\NormalTok{attr\_vars }\OtherTok{\textless{}{-}} \FunctionTok{c}\NormalTok{(}\StringTok{"perform"}\NormalTok{, }\StringTok{"leader"}\NormalTok{, }\StringTok{"fun"}\NormalTok{, }\StringTok{"serious"}\NormalTok{, }\StringTok{"bargain"}\NormalTok{, }\StringTok{"value"}\NormalTok{)}
\NormalTok{pca\_fit }\OtherTok{\textless{}{-}} \FunctionTok{easy\_pca\_fit}\NormalTok{(}\AttributeTok{data =}\NormalTok{ greekbrands,  }\AttributeTok{vars =}\NormalTok{ attr\_vars, }\AttributeTok{ft=}\ConstantTok{TRUE}\NormalTok{)}
\NormalTok{pca\_fit}\SpecialCharTok{$}\NormalTok{table}
\end{Highlighting}
\end{Shaded}

\global\setlength{\Oldarrayrulewidth}{\arrayrulewidth}

\global\setlength{\Oldtabcolsep}{\tabcolsep}

\setlength{\tabcolsep}{2pt}

\renewcommand*{\arraystretch}{1.5}



\providecommand{\ascline}[3]{\noalign{\global\arrayrulewidth #1}\arrayrulecolor[HTML]{#2}\cline{#3}}

\begin{longtable}[c]{|p{1.08in}|p{1.04in}|p{0.98in}|p{0.99in}|p{1.05in}}



\ascline{1.5pt}{666666}{1-5}

\multicolumn{1}{>{\raggedleft}m{\dimexpr 1.08in+0\tabcolsep}}{\textcolor[HTML]{000000}{\fontsize{11}{11}\selectfont{\global\setmainfont{Arial}{Component}}}} & \multicolumn{1}{>{\raggedleft}m{\dimexpr 1.04in+0\tabcolsep}}{\textcolor[HTML]{000000}{\fontsize{11}{11}\selectfont{\global\setmainfont{Arial}{Eigenvalue}}}} & \multicolumn{1}{>{\raggedleft}m{\dimexpr 0.98in+0\tabcolsep}}{\textcolor[HTML]{000000}{\fontsize{11}{11}\selectfont{\global\setmainfont{Arial}{Difference}}}} & \multicolumn{1}{>{\raggedleft}m{\dimexpr 0.99in+0\tabcolsep}}{\textcolor[HTML]{000000}{\fontsize{11}{11}\selectfont{\global\setmainfont{Arial}{Proportion}}}} & \multicolumn{1}{>{\raggedleft}m{\dimexpr 1.05in+0\tabcolsep}}{\textcolor[HTML]{000000}{\fontsize{11}{11}\selectfont{\global\setmainfont{Arial}{Cumulative}}}} \\

\ascline{1.5pt}{666666}{1-5}\endfirsthead 

\ascline{1.5pt}{666666}{1-5}

\multicolumn{1}{>{\raggedleft}m{\dimexpr 1.08in+0\tabcolsep}}{\textcolor[HTML]{000000}{\fontsize{11}{11}\selectfont{\global\setmainfont{Arial}{Component}}}} & \multicolumn{1}{>{\raggedleft}m{\dimexpr 1.04in+0\tabcolsep}}{\textcolor[HTML]{000000}{\fontsize{11}{11}\selectfont{\global\setmainfont{Arial}{Eigenvalue}}}} & \multicolumn{1}{>{\raggedleft}m{\dimexpr 0.98in+0\tabcolsep}}{\textcolor[HTML]{000000}{\fontsize{11}{11}\selectfont{\global\setmainfont{Arial}{Difference}}}} & \multicolumn{1}{>{\raggedleft}m{\dimexpr 0.99in+0\tabcolsep}}{\textcolor[HTML]{000000}{\fontsize{11}{11}\selectfont{\global\setmainfont{Arial}{Proportion}}}} & \multicolumn{1}{>{\raggedleft}m{\dimexpr 1.05in+0\tabcolsep}}{\textcolor[HTML]{000000}{\fontsize{11}{11}\selectfont{\global\setmainfont{Arial}{Cumulative}}}} \\

\ascline{1.5pt}{666666}{1-5}\endhead



\multicolumn{1}{>{\raggedleft}m{\dimexpr 1.08in+0\tabcolsep}}{\textcolor[HTML]{000000}{\fontsize{11}{11}\selectfont{\global\setmainfont{Arial}{1}}}} & \multicolumn{1}{>{\raggedleft}m{\dimexpr 1.04in+0\tabcolsep}}{\textcolor[HTML]{000000}{\fontsize{11}{11}\selectfont{\global\setmainfont{Arial}{2.2293}}}} & \multicolumn{1}{>{\raggedleft}m{\dimexpr 0.98in+0\tabcolsep}}{\textcolor[HTML]{000000}{\fontsize{11}{11}\selectfont{\global\setmainfont{Arial}{0.5454}}}} & \multicolumn{1}{>{\raggedleft}m{\dimexpr 0.99in+0\tabcolsep}}{\textcolor[HTML]{000000}{\fontsize{11}{11}\selectfont{\global\setmainfont{Arial}{0.3716}}}} & \multicolumn{1}{>{\raggedleft}m{\dimexpr 1.05in+0\tabcolsep}}{\textcolor[HTML]{000000}{\fontsize{11}{11}\selectfont{\global\setmainfont{Arial}{0.3716}}}} \\





\multicolumn{1}{>{\raggedleft}m{\dimexpr 1.08in+0\tabcolsep}}{\textcolor[HTML]{000000}{\fontsize{11}{11}\selectfont{\global\setmainfont{Arial}{2}}}} & \multicolumn{1}{>{\raggedleft}m{\dimexpr 1.04in+0\tabcolsep}}{\textcolor[HTML]{000000}{\fontsize{11}{11}\selectfont{\global\setmainfont{Arial}{1.6839}}}} & \multicolumn{1}{>{\raggedleft}m{\dimexpr 0.98in+0\tabcolsep}}{\textcolor[HTML]{000000}{\fontsize{11}{11}\selectfont{\global\setmainfont{Arial}{0.8876}}}} & \multicolumn{1}{>{\raggedleft}m{\dimexpr 0.99in+0\tabcolsep}}{\textcolor[HTML]{000000}{\fontsize{11}{11}\selectfont{\global\setmainfont{Arial}{0.2806}}}} & \multicolumn{1}{>{\raggedleft}m{\dimexpr 1.05in+0\tabcolsep}}{\textcolor[HTML]{000000}{\fontsize{11}{11}\selectfont{\global\setmainfont{Arial}{0.6522}}}} \\





\multicolumn{1}{>{\raggedleft}m{\dimexpr 1.08in+0\tabcolsep}}{\textcolor[HTML]{000000}{\fontsize{11}{11}\selectfont{\global\setmainfont{Arial}{3}}}} & \multicolumn{1}{>{\raggedleft}m{\dimexpr 1.04in+0\tabcolsep}}{\textcolor[HTML]{000000}{\fontsize{11}{11}\selectfont{\global\setmainfont{Arial}{0.7963}}}} & \multicolumn{1}{>{\raggedleft}m{\dimexpr 0.98in+0\tabcolsep}}{\textcolor[HTML]{000000}{\fontsize{11}{11}\selectfont{\global\setmainfont{Arial}{0.1545}}}} & \multicolumn{1}{>{\raggedleft}m{\dimexpr 0.99in+0\tabcolsep}}{\textcolor[HTML]{000000}{\fontsize{11}{11}\selectfont{\global\setmainfont{Arial}{0.1327}}}} & \multicolumn{1}{>{\raggedleft}m{\dimexpr 1.05in+0\tabcolsep}}{\textcolor[HTML]{000000}{\fontsize{11}{11}\selectfont{\global\setmainfont{Arial}{0.7849}}}} \\





\multicolumn{1}{>{\raggedleft}m{\dimexpr 1.08in+0\tabcolsep}}{\textcolor[HTML]{000000}{\fontsize{11}{11}\selectfont{\global\setmainfont{Arial}{4}}}} & \multicolumn{1}{>{\raggedleft}m{\dimexpr 1.04in+0\tabcolsep}}{\textcolor[HTML]{000000}{\fontsize{11}{11}\selectfont{\global\setmainfont{Arial}{0.6418}}}} & \multicolumn{1}{>{\raggedleft}m{\dimexpr 0.98in+0\tabcolsep}}{\textcolor[HTML]{000000}{\fontsize{11}{11}\selectfont{\global\setmainfont{Arial}{0.2433}}}} & \multicolumn{1}{>{\raggedleft}m{\dimexpr 0.99in+0\tabcolsep}}{\textcolor[HTML]{000000}{\fontsize{11}{11}\selectfont{\global\setmainfont{Arial}{0.1070}}}} & \multicolumn{1}{>{\raggedleft}m{\dimexpr 1.05in+0\tabcolsep}}{\textcolor[HTML]{000000}{\fontsize{11}{11}\selectfont{\global\setmainfont{Arial}{0.8919}}}} \\





\multicolumn{1}{>{\raggedleft}m{\dimexpr 1.08in+0\tabcolsep}}{\textcolor[HTML]{000000}{\fontsize{11}{11}\selectfont{\global\setmainfont{Arial}{5}}}} & \multicolumn{1}{>{\raggedleft}m{\dimexpr 1.04in+0\tabcolsep}}{\textcolor[HTML]{000000}{\fontsize{11}{11}\selectfont{\global\setmainfont{Arial}{0.3985}}}} & \multicolumn{1}{>{\raggedleft}m{\dimexpr 0.98in+0\tabcolsep}}{\textcolor[HTML]{000000}{\fontsize{11}{11}\selectfont{\global\setmainfont{Arial}{0.1484}}}} & \multicolumn{1}{>{\raggedleft}m{\dimexpr 0.99in+0\tabcolsep}}{\textcolor[HTML]{000000}{\fontsize{11}{11}\selectfont{\global\setmainfont{Arial}{0.0664}}}} & \multicolumn{1}{>{\raggedleft}m{\dimexpr 1.05in+0\tabcolsep}}{\textcolor[HTML]{000000}{\fontsize{11}{11}\selectfont{\global\setmainfont{Arial}{0.9583}}}} \\





\multicolumn{1}{>{\raggedleft}m{\dimexpr 1.08in+0\tabcolsep}}{\textcolor[HTML]{000000}{\fontsize{11}{11}\selectfont{\global\setmainfont{Arial}{6}}}} & \multicolumn{1}{>{\raggedleft}m{\dimexpr 1.04in+0\tabcolsep}}{\textcolor[HTML]{000000}{\fontsize{11}{11}\selectfont{\global\setmainfont{Arial}{0.2501}}}} & \multicolumn{1}{>{\raggedleft}m{\dimexpr 0.98in+0\tabcolsep}}{\textcolor[HTML]{000000}{\fontsize{11}{11}\selectfont{\global\setmainfont{Arial}{}}}} & \multicolumn{1}{>{\raggedleft}m{\dimexpr 0.99in+0\tabcolsep}}{\textcolor[HTML]{000000}{\fontsize{11}{11}\selectfont{\global\setmainfont{Arial}{0.0417}}}} & \multicolumn{1}{>{\raggedleft}m{\dimexpr 1.05in+0\tabcolsep}}{\textcolor[HTML]{000000}{\fontsize{11}{11}\selectfont{\global\setmainfont{Arial}{1.0000}}}} \\

\ascline{1.5pt}{666666}{1-5}



\end{longtable}



\arrayrulecolor[HTML]{000000}

\global\setlength{\arrayrulewidth}{\Oldarrayrulewidth}

\global\setlength{\tabcolsep}{\Oldtabcolsep}

\renewcommand*{\arraystretch}{1}

\begin{Shaded}
\begin{Highlighting}[]
\NormalTok{pca\_fit}\SpecialCharTok{$}\NormalTok{plot}
\end{Highlighting}
\end{Shaded}

\pandocbounded{\includegraphics[keepaspectratio]{MKT4320_R_Tutorial_files/figure-latex/pca-fit-1.pdf}}

The eigenvalue table and scree plot summarize how much variance each component explains. Important columns in the eigenvalue table include:

\begin{itemize}
\tightlist
\item
  \textbf{Eigenvalue}: total variance explained by each component
\item
  \textbf{Proportion}: share of total variance explained
\item
  \textbf{Cumulative}: cumulative proportion of variance explained
\end{itemize}

Common decision rules:

\begin{itemize}
\tightlist
\item
  Retain components with eigenvalues greater than 1
\item
  Look for an ``elbow'' where additional components add little explanatory power
\item
  Aim for a solution that balances parsimony and interpretability
\end{itemize}

There is no single correct answer. Component retention should be guided by
marketing judgment as well as statistics.

\subsection{Final PCA Solution}\label{final-pca-solution}

After deciding how many components to retain, we refit the PCA model and examine
the loadings using the \texttt{easy\_pca\_final()} function from the \texttt{MKT4320BGSU} package.

Usage:

\begin{itemize}
\tightlist
\item
  \texttt{easy\_pca\_final(data,\ vars,\ comp,\ group\ =\ NULL,\ ft\ =\ TRUE)}
\item
  where:

  \begin{itemize}
  \tightlist
  \item
    \texttt{data} is a data frame containing the full dataset.
  \item
    \texttt{vars} is a character vector of variable names to use in PCA (required). All variables must be numeric.
  \item
    \texttt{comp} is an integer representing the number of components to retain.
  \item
    \texttt{group} is an optional character sting of a single variable name to aggregate by before PCA.
  \item
    \texttt{ft} is logical; if TRUE, return \$table as a flextable (default = TRUE).
  \end{itemize}
\end{itemize}

In the example below, we again choose not to use the \texttt{group} option.

\begin{Shaded}
\begin{Highlighting}[]
\NormalTok{pca\_final }\OtherTok{\textless{}{-}} \FunctionTok{easy\_pca\_final}\NormalTok{(}\AttributeTok{data =}\NormalTok{ greekbrands, }\AttributeTok{vars =}\NormalTok{ attr\_vars, }
                            \AttributeTok{comp =} \DecValTok{2}\NormalTok{, }\AttributeTok{ft=}\ConstantTok{TRUE}\NormalTok{)}
\NormalTok{pca\_final}\SpecialCharTok{$}\NormalTok{rotated}
\end{Highlighting}
\end{Shaded}

\global\setlength{\Oldarrayrulewidth}{\arrayrulewidth}

\global\setlength{\Oldtabcolsep}{\tabcolsep}

\setlength{\tabcolsep}{2pt}

\renewcommand*{\arraystretch}{1.5}



\providecommand{\ascline}[3]{\noalign{\global\arrayrulewidth #1}\arrayrulecolor[HTML]{#2}\cline{#3}}

\begin{longtable}[c]{|p{0.88in}|p{0.89in}|p{0.89in}|p{1.19in}}



\ascline{1.5pt}{666666}{1-4}

\multicolumn{4}{>{\raggedright}m{\dimexpr 3.85in+6\tabcolsep}}{\textcolor[HTML]{000000}{\fontsize{11}{11}\selectfont{\global\setmainfont{Arial}{\textbf{Varimax-Rotated\ PCA\ Loadings}}}}} \\

\ascline{1.5pt}{666666}{1-4}



\multicolumn{1}{>{\raggedright}m{\dimexpr 0.88in+0\tabcolsep}}{\textcolor[HTML]{000000}{\fontsize{11}{11}\selectfont{\global\setmainfont{Arial}{\textbf{Variable}}}}} & \multicolumn{1}{>{\raggedleft}m{\dimexpr 0.89in+0\tabcolsep}}{\textcolor[HTML]{000000}{\fontsize{11}{11}\selectfont{\global\setmainfont{Arial}{\textbf{Comp\_1}}}}} & \multicolumn{1}{>{\raggedleft}m{\dimexpr 0.89in+0\tabcolsep}}{\textcolor[HTML]{000000}{\fontsize{11}{11}\selectfont{\global\setmainfont{Arial}{\textbf{Comp\_2}}}}} & \multicolumn{1}{>{\raggedleft}m{\dimexpr 1.19in+0\tabcolsep}}{\textcolor[HTML]{000000}{\fontsize{11}{11}\selectfont{\global\setmainfont{Arial}{\textbf{Unexplained}}}}} \\

\ascline{1.5pt}{666666}{1-4}\endfirsthead 

\ascline{1.5pt}{666666}{1-4}

\multicolumn{4}{>{\raggedright}m{\dimexpr 3.85in+6\tabcolsep}}{\textcolor[HTML]{000000}{\fontsize{11}{11}\selectfont{\global\setmainfont{Arial}{\textbf{Varimax-Rotated\ PCA\ Loadings}}}}} \\

\ascline{1.5pt}{666666}{1-4}



\multicolumn{1}{>{\raggedright}m{\dimexpr 0.88in+0\tabcolsep}}{\textcolor[HTML]{000000}{\fontsize{11}{11}\selectfont{\global\setmainfont{Arial}{\textbf{Variable}}}}} & \multicolumn{1}{>{\raggedleft}m{\dimexpr 0.89in+0\tabcolsep}}{\textcolor[HTML]{000000}{\fontsize{11}{11}\selectfont{\global\setmainfont{Arial}{\textbf{Comp\_1}}}}} & \multicolumn{1}{>{\raggedleft}m{\dimexpr 0.89in+0\tabcolsep}}{\textcolor[HTML]{000000}{\fontsize{11}{11}\selectfont{\global\setmainfont{Arial}{\textbf{Comp\_2}}}}} & \multicolumn{1}{>{\raggedleft}m{\dimexpr 1.19in+0\tabcolsep}}{\textcolor[HTML]{000000}{\fontsize{11}{11}\selectfont{\global\setmainfont{Arial}{\textbf{Unexplained}}}}} \\

\ascline{1.5pt}{666666}{1-4}\endhead



\multicolumn{1}{>{\raggedright}m{\dimexpr 0.88in+0\tabcolsep}}{\textcolor[HTML]{000000}{\fontsize{11}{11}\selectfont{\global\setmainfont{Arial}{perform}}}} & \multicolumn{1}{>{\raggedleft}m{\dimexpr 0.89in+0\tabcolsep}}{\textcolor[HTML]{000000}{\fontsize{11}{11}\selectfont{\global\setmainfont{Arial}{0.7243}}}} & \multicolumn{1}{>{\raggedleft}m{\dimexpr 0.89in+0\tabcolsep}}{\textcolor[HTML]{000000}{\fontsize{11}{11}\selectfont{\global\setmainfont{Arial}{-0.0750}}}} & \multicolumn{1}{>{\raggedleft}m{\dimexpr 1.19in+0\tabcolsep}}{\textcolor[HTML]{000000}{\fontsize{11}{11}\selectfont{\global\setmainfont{Arial}{0.4697}}}} \\





\multicolumn{1}{>{\raggedright}m{\dimexpr 0.88in+0\tabcolsep}}{\textcolor[HTML]{000000}{\fontsize{11}{11}\selectfont{\global\setmainfont{Arial}{leader}}}} & \multicolumn{1}{>{\raggedleft}m{\dimexpr 0.89in+0\tabcolsep}}{\textcolor[HTML]{000000}{\fontsize{11}{11}\selectfont{\global\setmainfont{Arial}{0.8404}}}} & \multicolumn{1}{>{\raggedleft}m{\dimexpr 0.89in+0\tabcolsep}}{\textcolor[HTML]{000000}{\fontsize{11}{11}\selectfont{\global\setmainfont{Arial}{-0.0486}}}} & \multicolumn{1}{>{\raggedleft}m{\dimexpr 1.19in+0\tabcolsep}}{\textcolor[HTML]{000000}{\fontsize{11}{11}\selectfont{\global\setmainfont{Arial}{0.2914}}}} \\





\multicolumn{1}{>{\raggedright}m{\dimexpr 0.88in+0\tabcolsep}}{\textcolor[HTML]{000000}{\fontsize{11}{11}\selectfont{\global\setmainfont{Arial}{fun}}}} & \multicolumn{1}{>{\raggedleft}m{\dimexpr 0.89in+0\tabcolsep}}{\textcolor[HTML]{000000}{\fontsize{11}{11}\selectfont{\global\setmainfont{Arial}{-0.5475}}}} & \multicolumn{1}{>{\raggedleft}m{\dimexpr 0.89in+0\tabcolsep}}{\textcolor[HTML]{000000}{\fontsize{11}{11}\selectfont{\global\setmainfont{Arial}{0.1496}}}} & \multicolumn{1}{>{\raggedleft}m{\dimexpr 1.19in+0\tabcolsep}}{\textcolor[HTML]{000000}{\fontsize{11}{11}\selectfont{\global\setmainfont{Arial}{0.6778}}}} \\





\multicolumn{1}{>{\raggedright}m{\dimexpr 0.88in+0\tabcolsep}}{\textcolor[HTML]{000000}{\fontsize{11}{11}\selectfont{\global\setmainfont{Arial}{serious}}}} & \multicolumn{1}{>{\raggedleft}m{\dimexpr 0.89in+0\tabcolsep}}{\textcolor[HTML]{000000}{\fontsize{11}{11}\selectfont{\global\setmainfont{Arial}{0.7849}}}} & \multicolumn{1}{>{\raggedleft}m{\dimexpr 0.89in+0\tabcolsep}}{\textcolor[HTML]{000000}{\fontsize{11}{11}\selectfont{\global\setmainfont{Arial}{0.0434}}}} & \multicolumn{1}{>{\raggedleft}m{\dimexpr 1.19in+0\tabcolsep}}{\textcolor[HTML]{000000}{\fontsize{11}{11}\selectfont{\global\setmainfont{Arial}{0.3821}}}} \\





\multicolumn{1}{>{\raggedright}m{\dimexpr 0.88in+0\tabcolsep}}{\textcolor[HTML]{000000}{\fontsize{11}{11}\selectfont{\global\setmainfont{Arial}{bargain}}}} & \multicolumn{1}{>{\raggedleft}m{\dimexpr 0.89in+0\tabcolsep}}{\textcolor[HTML]{000000}{\fontsize{11}{11}\selectfont{\global\setmainfont{Arial}{-0.0189}}}} & \multicolumn{1}{>{\raggedleft}m{\dimexpr 0.89in+0\tabcolsep}}{\textcolor[HTML]{000000}{\fontsize{11}{11}\selectfont{\global\setmainfont{Arial}{-0.9294}}}} & \multicolumn{1}{>{\raggedleft}m{\dimexpr 1.19in+0\tabcolsep}}{\textcolor[HTML]{000000}{\fontsize{11}{11}\selectfont{\global\setmainfont{Arial}{0.1359}}}} \\





\multicolumn{1}{>{\raggedright}m{\dimexpr 0.88in+0\tabcolsep}}{\textcolor[HTML]{000000}{\fontsize{11}{11}\selectfont{\global\setmainfont{Arial}{value}}}} & \multicolumn{1}{>{\raggedleft}m{\dimexpr 0.89in+0\tabcolsep}}{\textcolor[HTML]{000000}{\fontsize{11}{11}\selectfont{\global\setmainfont{Arial}{0.0698}}}} & \multicolumn{1}{>{\raggedleft}m{\dimexpr 0.89in+0\tabcolsep}}{\textcolor[HTML]{000000}{\fontsize{11}{11}\selectfont{\global\setmainfont{Arial}{-0.9302}}}} & \multicolumn{1}{>{\raggedleft}m{\dimexpr 1.19in+0\tabcolsep}}{\textcolor[HTML]{000000}{\fontsize{11}{11}\selectfont{\global\setmainfont{Arial}{0.1298}}}} \\

\ascline{1.5pt}{666666}{1-4}



\end{longtable}



\arrayrulecolor[HTML]{000000}

\global\setlength{\arrayrulewidth}{\Oldarrayrulewidth}

\global\setlength{\tabcolsep}{\Oldtabcolsep}

\renewcommand*{\arraystretch}{1}

We focus on the \textbf{varimax-rotated} loadings because they are easier to interpret. A loading represents the relationship between an original attribute and a component:

\begin{itemize}
\tightlist
\item
  Larger absolute values indicate stronger relationships
\item
  Attributes with high loadings on the same component tend to reflect a common
  underlying dimension
\end{itemize}

When interpreting loadings:

\begin{itemize}
\tightlist
\item
  Look for patterns across attributes
\item
  Identify which attributes define each component
\item
  Assign descriptive, managerially meaningful names to components
\end{itemize}

For example:

\begin{itemize}
\tightlist
\item
  A component with high loadings on \emph{perform}, \emph{leader}, and \emph{serious} might be
  labeled \textbf{Performance}
\item
  A component with high loadings on \emph{bargain} and \emph{value} might be labeled
  \textbf{Value Orientation}
\end{itemize}

\begin{center}\rule{0.5\linewidth}{0.5pt}\end{center}

\section{From PCA to Perceptual Maps}\label{from-pca-to-perceptual-maps}

PCA components can be used as axes in perceptual maps. Each brand's position on a
component reflects how strongly it scores on that underlying dimension.

Perceptual maps translate statistical results into a visual format that is easy
to communicate to managers and decision-makers.

\begin{center}\rule{0.5\linewidth}{0.5pt}\end{center}

\section{Attribute-Based Perceptual Maps Using PCA}\label{attribute-based-perceptual-maps-using-pca}

\subsection{Creating PCA-Based Maps}\label{creating-pca-based-maps}

We now use the retained PCA solution to create perceptual maps. We use the \texttt{easy\_pca\_maps()} function from the \texttt{MKT4320BGSU} package to automate the process.

Usage:

\begin{itemize}
\tightlist
\item
  \texttt{easy\_pca\_maps(data,\ vars,\ group,\ comp,\ pref\ =\ NULL,\ rotate\ =\ TRUE,}\strut \\
  \texttt{arrow\_scale\ =\ 0.75,label\_pad\ =\ 0.04)}
\item
  where:

  \begin{itemize}
  \tightlist
  \item
    \texttt{data} is a data frame containing individual-level observations.
  \item
    \texttt{vars} is a character vector of numeric attribute variable names used in PCA.
  \item
    \texttt{group} is a single string specifying the grouping variable (e.g., brand or product name).
  \item
    \texttt{comp} is an integer specifying the number of components to retain (must be \(≥2\)).
  \item
    \texttt{pref} is an optional single string specifying a numeric preference variable name to produce a joint space map with a preference vector (if data is available)
  \item
    \texttt{rotate} is logical; if TRUE (default), apply varimax rotation to the retained component space before creating perceptual maps.
  \item
    \texttt{arrow\_scale} is numeric in (0, 1{]}; scales arrow lengths relative to the object range (default = 0.75).
  \item
    \texttt{label\_pad} is numeric; distance (as a proportion of the axis range) used to push attribute arrow labels beyond arrow tips (default = 0.04) for easier viewing of the map.
  \end{itemize}
\end{itemize}

We'll use the same variables as before, but add in the required \texttt{group} and an optional \texttt{pref} to create a joint space map.

\begin{Shaded}
\begin{Highlighting}[]
\NormalTok{pca\_maps }\OtherTok{\textless{}{-}} \FunctionTok{easy\_pca\_maps}\NormalTok{(}\AttributeTok{data =}\NormalTok{ greekbrands, }\AttributeTok{vars =}\NormalTok{ attr\_vars, }
                          \AttributeTok{group =} \StringTok{"brand"}\NormalTok{, }\AttributeTok{pref =} \StringTok{"pref"}\NormalTok{, }\AttributeTok{comp =} \DecValTok{2}\NormalTok{)}
\NormalTok{pca\_maps}\SpecialCharTok{$}\NormalTok{plots}
\end{Highlighting}
\end{Shaded}

\begin{verbatim}
$Comp_1_vs_Comp_2
\end{verbatim}

\pandocbounded{\includegraphics[keepaspectratio]{MKT4320_R_Tutorial_files/figure-latex/pca-maps-1.pdf}}

The map displays:

\begin{itemize}
\tightlist
\item
  Brands as points
\item
  Attribute vectors showing how attributes align with the components
\end{itemize}

Key interpretation guidelines:

\begin{itemize}
\tightlist
\item
  Brands close together are perceived similarly
\item
  Brands far apart are perceived differently
\item
  Attribute vectors indicate the direction of increasing attribute values
\item
  Brands in the direction of an attribute vector score higher on that attribute
\item
  Attribute vectors more parallel with the preference vector (if available) a stronger drivers of preference
\end{itemize}

Distances are relative and should be interpreted qualitatively rather than
precisely.

\begin{center}\rule{0.5\linewidth}{0.5pt}\end{center}

\section{Managerial Interpretation and Strategic Insights}\label{managerial-interpretation-and-strategic-insights}

PCA-based perceptual maps can help managers:

\begin{itemize}
\tightlist
\item
  Identify direct competitors
\item
  Detect market clusters and white space
\item
  Evaluate whether a brand's positioning matches strategic intent
\end{itemize}

These insights can inform:

\begin{itemize}
\tightlist
\item
  Positioning statements
\item
  Advertising and messaging strategy
\item
  Product reformulation decisions
\end{itemize}

Common Pitfalls and Best Practices

\begin{itemize}
\tightlist
\item
  Do not over-interpret small loadings
\item
  Avoid retaining too many components
\item
  Remember that PCA alone reflects perceptions, not preferences
\item
  Always explain components in clear, non-technical language
\end{itemize}

\begin{center}\rule{0.5\linewidth}{0.5pt}\end{center}

\section{Chapter Summary}\label{chapter-summary-1}

In this chapter, we:

\begin{itemize}
\tightlist
\item
  Used PCA to reduce and interpret brand perception data
\item
  Applied diagnostics to choose the number of components
\item
  Interpreted rotated component loadings
\item
  Created perceptual maps to visualize brand positioning
\end{itemize}

PCA is a powerful bridge between data analysis and strategic insight in
marketing analytics.

\begin{center}\rule{0.5\linewidth}{0.5pt}\end{center}

\section{What's Next}\label{whats-next-11}

In the next chapter, we shift from describing perceptions to testing causal impact.Specifically, we will study A/B testing and uplift modeling, which are tools used to answer questions such as:

\begin{itemize}
\tightlist
\item
  Does a new message, offer, or design actually change behavior?
\item
  How large is the effect of a treatment compared to a control?
\item
  Are some customers more responsive to an intervention than others?
\end{itemize}

Where PCA and perceptual maps help us understand how consumers see brands, A/B testing helps us evaluate what actions work, and uplift modeling helps us determine for whom they work best. These methods are central to modern data-driven marketing in areas such as digital advertising, pricing experiments, promotions, and personalization.

\chapter{A/B Testing and Uplift Modeling}\label{ab-testing-and-uplift-modeling}

\section{Introduction: From Average Effects to Targeted Marketing}\label{introduction-from-average-effects-to-targeted-marketing}

A/B testing is one of the most widely used tools in marketing analytics. Whether testing
email subject lines, promotional offers, website layouts, or pricing messages, marketers
frequently rely on randomized experiments to measure causal effects.

In this chapter, we begin with \textbf{average treatment effects (ATEs)}, which is the traditional goal
of A/B testing,and then move beyond averages to \textbf{uplift modeling}, which focuses on
identifying \emph{who} is most likely to be influenced by a marketing intervention.

Throughout the chapter, we use data from an email marketing experiment contained in the
\texttt{email.camp.w} dataset.

\begin{center}\rule{0.5\linewidth}{0.5pt}\end{center}

\section{The Email Campaign Experiment}\label{the-email-campaign-experiment}

The dataset \texttt{email.camp.w} comes from a randomized email campaign experiment. Customers
were randomly assigned to receive either:

\begin{itemize}
\tightlist
\item
  a \textbf{promotional email} (treatment group), or
\item
  \textbf{no promotional email} (control group).
\end{itemize}

The primary outcome of interest is whether the customer responded (e.g., clicked or
converted). In addition, the dataset contains several customer characteristics
(covariates) such as demographics and prior behavior.

Because treatment assignment was randomized, differences in outcomes between the two
groups can be interpreted causally.

\begin{center}\rule{0.5\linewidth}{0.5pt}\end{center}

\section{Checking the Randomization Assumption}\label{checking-the-randomization-assumption}

\subsection{Why Balance Checks Matter}\label{why-balance-checks-matter}

Randomization ensures that treatment and control groups are similar \emph{on average}.
However, especially in applied settings, it is good practice to verify that observable
covariates are balanced across groups.

Large imbalances may signal problems such as implementation errors or data issues.

\subsection{\texorpdfstring{Randomization Check Using \texttt{rand\_check()}}{Randomization Check Using rand\_check()}}\label{randomization-check-using-rand_check}

The \texttt{rand\_check()} function from the \texttt{MKT4320BGSU} package compares the distribution of selected covariates across
treatment groups and automatically applies appropriate statistical tests.

Usage:

\begin{itemize}
\tightlist
\item
  \texttt{rand\_check(data,\ treatment,\ covariates,\ ft\ =\ TRUE,\ digits\ =\ 3)}
\item
  where:

  \begin{itemize}
  \tightlist
  \item
    \texttt{data} is a data frame containing the treatment indicator and covariates.
  \item
    \texttt{treatment} is a character string giving the name of the treatment variable. Must identify two or more groups.
  \item
    \texttt{covariates} is a character vector of covariate names to include in the randomization check.
  \item
    \texttt{ft} is logical; if TRUE (default), return results as a flextable. If FALSE, return a data frame.
  \item
    \texttt{digits} is an integer; number of decimal places to display in the output (default = 3).
  \end{itemize}
\end{itemize}

An example is provided below. When reviewing the output, focus on:

\begin{itemize}
\tightlist
\item
  \textbf{Scaled mean differences}: values close to zero indicate good balance.
\item
  \textbf{p-values}: large p-values suggest no systematic differences.
\end{itemize}

Well-balanced covariates support the validity of the experiment.

\begin{Shaded}
\begin{Highlighting}[]
\FunctionTok{rand\_check}\NormalTok{(}\AttributeTok{data =}\NormalTok{ email.camp.w, }\AttributeTok{treatment =} \StringTok{"promotion"}\NormalTok{,}
  \AttributeTok{covariates =} \FunctionTok{c}\NormalTok{(}\StringTok{"recency"}\NormalTok{, }\StringTok{"history"}\NormalTok{, }\StringTok{"womens"}\NormalTok{, }\StringTok{"zip"}\NormalTok{),}
  \AttributeTok{ft =} \ConstantTok{TRUE}\NormalTok{)}
\end{Highlighting}
\end{Shaded}

\global\setlength{\Oldarrayrulewidth}{\arrayrulewidth}

\global\setlength{\Oldtabcolsep}{\tabcolsep}

\setlength{\tabcolsep}{2pt}

\renewcommand*{\arraystretch}{1.5}



\providecommand{\ascline}[3]{\noalign{\global\arrayrulewidth #1}\arrayrulecolor[HTML]{#2}\cline{#3}}

\begin{longtable}[c]{|p{0.75in}|p{0.75in}|p{0.75in}|p{0.75in}|p{0.75in}|p{0.75in}}



\ascline{1.5pt}{666666}{1-6}

\multicolumn{1}{>{\centering}b{\dimexpr 0.75in+0\tabcolsep}}{} & \multicolumn{2}{>{\centering}b{\dimexpr 1.5in+2\tabcolsep}}{\textcolor[HTML]{000000}{\fontsize{11}{11}\selectfont{\global\setmainfont{Arial}{\textbf{Mean}}}}} & \multicolumn{1}{>{\centering}b{\dimexpr 0.75in+0\tabcolsep}}{} & \multicolumn{1}{>{\centering}b{\dimexpr 0.75in+0\tabcolsep}}{} & \multicolumn{1}{>{\centering}b{\dimexpr 0.75in+0\tabcolsep}}{} \\

\ascline{1.5pt}{666666}{2-3}



\multicolumn{1}{>{\centering}b{\dimexpr 0.75in+0\tabcolsep}}{\multirow[b]{-2}{*}{\parbox{0.75in}{\centering \textcolor[HTML]{000000}{\fontsize{11}{11}\selectfont{\global\setmainfont{Arial}{\textbf{Variable}}}}}}} & \multicolumn{1}{>{\centering}b{\dimexpr 0.75in+0\tabcolsep}}{\textcolor[HTML]{000000}{\fontsize{11}{11}\selectfont{\global\setmainfont{Arial}{\textbf{Treatment}}}}} & \multicolumn{1}{>{\centering}b{\dimexpr 0.75in+0\tabcolsep}}{\textcolor[HTML]{000000}{\fontsize{11}{11}\selectfont{\global\setmainfont{Arial}{\textbf{Control}}}}} & \multicolumn{1}{>{\centering}b{\dimexpr 0.75in+0\tabcolsep}}{\multirow[b]{-2}{*}{\parbox{0.75in}{\centering \textcolor[HTML]{000000}{\fontsize{11}{11}\selectfont{\global\setmainfont{Arial}{\textbf{SD}}}}}}} & \multicolumn{1}{>{\centering}b{\dimexpr 0.75in+0\tabcolsep}}{\multirow[b]{-2}{*}{\parbox{0.75in}{\centering \textcolor[HTML]{000000}{\fontsize{11}{11}\selectfont{\global\setmainfont{Arial}{\textbf{Scaled\ Mean\ Difference}}}}}}} & \multicolumn{1}{>{\centering}b{\dimexpr 0.75in+0\tabcolsep}}{\multirow[b]{-2}{*}{\parbox{0.75in}{\centering \textcolor[HTML]{000000}{\fontsize{11}{11}\selectfont{\global\setmainfont{Arial}{\textbf{p-value}}}}}}} \\

\ascline{1.5pt}{666666}{1-6}\endfirsthead 

\ascline{1.5pt}{666666}{1-6}

\multicolumn{1}{>{\centering}b{\dimexpr 0.75in+0\tabcolsep}}{} & \multicolumn{2}{>{\centering}b{\dimexpr 1.5in+2\tabcolsep}}{\textcolor[HTML]{000000}{\fontsize{11}{11}\selectfont{\global\setmainfont{Arial}{\textbf{Mean}}}}} & \multicolumn{1}{>{\centering}b{\dimexpr 0.75in+0\tabcolsep}}{} & \multicolumn{1}{>{\centering}b{\dimexpr 0.75in+0\tabcolsep}}{} & \multicolumn{1}{>{\centering}b{\dimexpr 0.75in+0\tabcolsep}}{} \\

\ascline{1.5pt}{666666}{2-3}



\multicolumn{1}{>{\centering}b{\dimexpr 0.75in+0\tabcolsep}}{\multirow[b]{-2}{*}{\parbox{0.75in}{\centering \textcolor[HTML]{000000}{\fontsize{11}{11}\selectfont{\global\setmainfont{Arial}{\textbf{Variable}}}}}}} & \multicolumn{1}{>{\centering}b{\dimexpr 0.75in+0\tabcolsep}}{\textcolor[HTML]{000000}{\fontsize{11}{11}\selectfont{\global\setmainfont{Arial}{\textbf{Treatment}}}}} & \multicolumn{1}{>{\centering}b{\dimexpr 0.75in+0\tabcolsep}}{\textcolor[HTML]{000000}{\fontsize{11}{11}\selectfont{\global\setmainfont{Arial}{\textbf{Control}}}}} & \multicolumn{1}{>{\centering}b{\dimexpr 0.75in+0\tabcolsep}}{\multirow[b]{-2}{*}{\parbox{0.75in}{\centering \textcolor[HTML]{000000}{\fontsize{11}{11}\selectfont{\global\setmainfont{Arial}{\textbf{SD}}}}}}} & \multicolumn{1}{>{\centering}b{\dimexpr 0.75in+0\tabcolsep}}{\multirow[b]{-2}{*}{\parbox{0.75in}{\centering \textcolor[HTML]{000000}{\fontsize{11}{11}\selectfont{\global\setmainfont{Arial}{\textbf{Scaled\ Mean\ Difference}}}}}}} & \multicolumn{1}{>{\centering}b{\dimexpr 0.75in+0\tabcolsep}}{\multirow[b]{-2}{*}{\parbox{0.75in}{\centering \textcolor[HTML]{000000}{\fontsize{11}{11}\selectfont{\global\setmainfont{Arial}{\textbf{p-value}}}}}}} \\

\ascline{1.5pt}{666666}{1-6}\endhead



\multicolumn{1}{>{\raggedright}m{\dimexpr 0.75in+0\tabcolsep}}{\textcolor[HTML]{000000}{\fontsize{11}{11}\selectfont{\global\setmainfont{Arial}{recency}}}} & \multicolumn{1}{>{\raggedleft}m{\dimexpr 0.75in+0\tabcolsep}}{\textcolor[HTML]{000000}{\fontsize{11}{11}\selectfont{\global\setmainfont{Arial}{5.810}}}} & \multicolumn{1}{>{\raggedleft}m{\dimexpr 0.75in+0\tabcolsep}}{\textcolor[HTML]{000000}{\fontsize{11}{11}\selectfont{\global\setmainfont{Arial}{5.725}}}} & \multicolumn{1}{>{\raggedleft}m{\dimexpr 0.75in+0\tabcolsep}}{\textcolor[HTML]{000000}{\fontsize{11}{11}\selectfont{\global\setmainfont{Arial}{3.504}}}} & \multicolumn{1}{>{\raggedleft}m{\dimexpr 0.75in+0\tabcolsep}}{\textcolor[HTML]{000000}{\fontsize{11}{11}\selectfont{\global\setmainfont{Arial}{0.024}}}} & \multicolumn{1}{>{\raggedleft}m{\dimexpr 0.75in+0\tabcolsep}}{\textcolor[HTML]{000000}{\fontsize{11}{11}\selectfont{\global\setmainfont{Arial}{0.227}}}} \\





\multicolumn{1}{>{\raggedright}m{\dimexpr 0.75in+0\tabcolsep}}{\textcolor[HTML]{000000}{\fontsize{11}{11}\selectfont{\global\setmainfont{Arial}{history}}}} & \multicolumn{1}{>{\raggedleft}m{\dimexpr 0.75in+0\tabcolsep}}{\textcolor[HTML]{000000}{\fontsize{11}{11}\selectfont{\global\setmainfont{Arial}{245.995}}}} & \multicolumn{1}{>{\raggedleft}m{\dimexpr 0.75in+0\tabcolsep}}{\textcolor[HTML]{000000}{\fontsize{11}{11}\selectfont{\global\setmainfont{Arial}{242.539}}}} & \multicolumn{1}{>{\raggedleft}m{\dimexpr 0.75in+0\tabcolsep}}{\textcolor[HTML]{000000}{\fontsize{11}{11}\selectfont{\global\setmainfont{Arial}{253.384}}}} & \multicolumn{1}{>{\raggedleft}m{\dimexpr 0.75in+0\tabcolsep}}{\textcolor[HTML]{000000}{\fontsize{11}{11}\selectfont{\global\setmainfont{Arial}{0.014}}}} & \multicolumn{1}{>{\raggedleft}m{\dimexpr 0.75in+0\tabcolsep}}{\textcolor[HTML]{000000}{\fontsize{11}{11}\selectfont{\global\setmainfont{Arial}{0.495}}}} \\





\multicolumn{1}{>{\raggedright}m{\dimexpr 0.75in+0\tabcolsep}}{\textcolor[HTML]{000000}{\fontsize{11}{11}\selectfont{\global\setmainfont{Arial}{womens}}}} & \multicolumn{1}{>{\raggedleft}m{\dimexpr 0.75in+0\tabcolsep}}{\textcolor[HTML]{000000}{\fontsize{11}{11}\selectfont{\global\setmainfont{Arial}{0.545}}}} & \multicolumn{1}{>{\raggedleft}m{\dimexpr 0.75in+0\tabcolsep}}{\textcolor[HTML]{000000}{\fontsize{11}{11}\selectfont{\global\setmainfont{Arial}{0.539}}}} & \multicolumn{1}{>{\raggedleft}m{\dimexpr 0.75in+0\tabcolsep}}{\textcolor[HTML]{000000}{\fontsize{11}{11}\selectfont{\global\setmainfont{Arial}{0.498}}}} & \multicolumn{1}{>{\raggedleft}m{\dimexpr 0.75in+0\tabcolsep}}{\textcolor[HTML]{000000}{\fontsize{11}{11}\selectfont{\global\setmainfont{Arial}{0.011}}}} & \multicolumn{1}{>{\raggedleft}m{\dimexpr 0.75in+0\tabcolsep}}{\textcolor[HTML]{000000}{\fontsize{11}{11}\selectfont{\global\setmainfont{Arial}{0.574}}}} \\





\multicolumn{1}{>{\raggedright}m{\dimexpr 0.75in+0\tabcolsep}}{\textcolor[HTML]{000000}{\fontsize{11}{11}\selectfont{\global\setmainfont{Arial}{zip:Rural}}}} & \multicolumn{1}{>{\raggedleft}m{\dimexpr 0.75in+0\tabcolsep}}{\textcolor[HTML]{000000}{\fontsize{11}{11}\selectfont{\global\setmainfont{Arial}{0.143}}}} & \multicolumn{1}{>{\raggedleft}m{\dimexpr 0.75in+0\tabcolsep}}{\textcolor[HTML]{000000}{\fontsize{11}{11}\selectfont{\global\setmainfont{Arial}{0.148}}}} & \multicolumn{1}{>{\raggedleft}m{\dimexpr 0.75in+0\tabcolsep}}{\textcolor[HTML]{000000}{\fontsize{11}{11}\selectfont{\global\setmainfont{Arial}{0.353}}}} & \multicolumn{1}{>{\raggedleft}m{\dimexpr 0.75in+0\tabcolsep}}{\textcolor[HTML]{000000}{\fontsize{11}{11}\selectfont{\global\setmainfont{Arial}{-0.014}}}} & \multicolumn{1}{>{\raggedleft}m{\dimexpr 0.75in+0\tabcolsep}}{\textcolor[HTML]{000000}{\fontsize{11}{11}\selectfont{\global\setmainfont{Arial}{0.395}}}} \\





\multicolumn{1}{>{\raggedright}m{\dimexpr 0.75in+0\tabcolsep}}{\textcolor[HTML]{000000}{\fontsize{11}{11}\selectfont{\global\setmainfont{Arial}{zip:Surburban}}}} & \multicolumn{1}{>{\raggedleft}m{\dimexpr 0.75in+0\tabcolsep}}{\textcolor[HTML]{000000}{\fontsize{11}{11}\selectfont{\global\setmainfont{Arial}{0.459}}}} & \multicolumn{1}{>{\raggedleft}m{\dimexpr 0.75in+0\tabcolsep}}{\textcolor[HTML]{000000}{\fontsize{11}{11}\selectfont{\global\setmainfont{Arial}{0.445}}}} & \multicolumn{1}{>{\raggedleft}m{\dimexpr 0.75in+0\tabcolsep}}{\textcolor[HTML]{000000}{\fontsize{11}{11}\selectfont{\global\setmainfont{Arial}{0.498}}}} & \multicolumn{1}{>{\raggedleft}m{\dimexpr 0.75in+0\tabcolsep}}{\textcolor[HTML]{000000}{\fontsize{11}{11}\selectfont{\global\setmainfont{Arial}{0.027}}}} & \multicolumn{1}{>{\raggedleft}m{\dimexpr 0.75in+0\tabcolsep}}{\textcolor[HTML]{000000}{\fontsize{11}{11}\selectfont{\global\setmainfont{Arial}{}}}} \\





\multicolumn{1}{>{\raggedright}m{\dimexpr 0.75in+0\tabcolsep}}{\textcolor[HTML]{000000}{\fontsize{11}{11}\selectfont{\global\setmainfont{Arial}{zip:Urban}}}} & \multicolumn{1}{>{\raggedleft}m{\dimexpr 0.75in+0\tabcolsep}}{\textcolor[HTML]{000000}{\fontsize{11}{11}\selectfont{\global\setmainfont{Arial}{0.398}}}} & \multicolumn{1}{>{\raggedleft}m{\dimexpr 0.75in+0\tabcolsep}}{\textcolor[HTML]{000000}{\fontsize{11}{11}\selectfont{\global\setmainfont{Arial}{0.406}}}} & \multicolumn{1}{>{\raggedleft}m{\dimexpr 0.75in+0\tabcolsep}}{\textcolor[HTML]{000000}{\fontsize{11}{11}\selectfont{\global\setmainfont{Arial}{0.490}}}} & \multicolumn{1}{>{\raggedleft}m{\dimexpr 0.75in+0\tabcolsep}}{\textcolor[HTML]{000000}{\fontsize{11}{11}\selectfont{\global\setmainfont{Arial}{-0.017}}}} & \multicolumn{1}{>{\raggedleft}m{\dimexpr 0.75in+0\tabcolsep}}{\textcolor[HTML]{000000}{\fontsize{11}{11}\selectfont{\global\setmainfont{Arial}{}}}} \\

\ascline{1.5pt}{666666}{1-6}



\end{longtable}



\arrayrulecolor[HTML]{000000}

\global\setlength{\arrayrulewidth}{\Oldarrayrulewidth}

\global\setlength{\tabcolsep}{\Oldtabcolsep}

\renewcommand*{\arraystretch}{1}

\begin{center}\rule{0.5\linewidth}{0.5pt}\end{center}

\section{Estimating the Average Treatment Effect (ATE)}\label{estimating-the-average-treatment-effect-ate}

\subsection{What Is the ATE?}\label{what-is-the-ate}

The \textbf{average treatment effect} measures the average impact of the promotion across all
customers. In an email campaign, this answers the question:
Did sending the promotion increase response rates overall?

\subsection{ATE via Regression}\label{ate-via-regression}

For binary outcomes, a linear regression with a treatment indicator is equivalent to a
difference-in-means estimator. When the outcome is binary, this is known as a \textbf{linear
probability model (LPM)}.

\subsection{\texorpdfstring{Using \texttt{easy\_ab\_ate()}}{Using easy\_ab\_ate()}}\label{using-easy_ab_ate}

We estimate the ATE using a regression model that includes the treatment indicator and
optionally adjusts for covariates by using the \texttt{easy\_ab\_ate()} function from the \texttt{MKT4320BGSU} package.

Usage:

\begin{itemize}
\tightlist
\item
  \texttt{easy\_ab\_ate(model,\ treatment,\ ft\ =\ TRUE)}
\item
  where:

  \begin{itemize}
  \tightlist
  \item
    \texttt{model} is a fitted linear regression model of class lm. This model should include the treatment variable and (optionally) covariates.
  \item
    \texttt{treatment} is a character string with the name of the treatment variable (in quotes).
  \item
    \texttt{ft} is logical; if TRUE (default) return a flextable. If FALSE, print full regression results to the console.
  \end{itemize}
\end{itemize}

Note that to use this function, you must first create a linear regression model using the \texttt{lm()} function. The model should have a response variable as the dependent variable, the treatment variable as an independent variable, and any additional covariates as additional independent variables. The results should be saved to an object. For example:

\begin{itemize}
\tightlist
\item
  \texttt{object\ \textless{}-\ lm(response\ \textasciitilde{}\ treatment\ +\ cov\_1\ +\ cov\_2\ +\ ...\ +\ cov\_k,\ data\ =\ data)}
\end{itemize}

\begin{Shaded}
\begin{Highlighting}[]
\NormalTok{m\_ab\_visit }\OtherTok{\textless{}{-}} \FunctionTok{lm}\NormalTok{(visit }\SpecialCharTok{\textasciitilde{}}\NormalTok{ promotion }\SpecialCharTok{+}\NormalTok{ recency }\SpecialCharTok{+}\NormalTok{ history }\SpecialCharTok{+}\NormalTok{ womens }\SpecialCharTok{+}\NormalTok{ zip, }
                 \AttributeTok{data =}\NormalTok{ email.camp.w)}
\FunctionTok{easy\_ab\_ate}\NormalTok{(}\AttributeTok{model =}\NormalTok{ m\_ab\_visit, }\AttributeTok{treatment =} \StringTok{"promotion"}\NormalTok{, }\AttributeTok{ft =} \ConstantTok{TRUE}\NormalTok{)}
\end{Highlighting}
\end{Shaded}

\global\setlength{\Oldarrayrulewidth}{\arrayrulewidth}

\global\setlength{\Oldtabcolsep}{\tabcolsep}

\setlength{\tabcolsep}{2pt}

\renewcommand*{\arraystretch}{1.5}



\providecommand{\ascline}[3]{\noalign{\global\arrayrulewidth #1}\arrayrulecolor[HTML]{#2}\cline{#3}}

\begin{longtable}[c]{|p{1.30in}|p{0.76in}|p{0.82in}|p{0.76in}|p{0.82in}}



\ascline{1pt}{000000}{1-5}

\multicolumn{1}{>{\raggedright}m{\dimexpr 1.3in+0\tabcolsep}}{\textcolor[HTML]{000000}{\fontsize{14}{14}\selectfont{\global\setmainfont{Arial}{\textbf{\ }}}}} & \multicolumn{2}{>{\centering}m{\dimexpr 1.58in+2\tabcolsep}}{\textcolor[HTML]{000000}{\fontsize{14}{14}\selectfont{\global\setmainfont{Arial}{\textbf{Without}}}}\textcolor[HTML]{000000}{\fontsize{14}{14}\selectfont{\global\setmainfont{Arial}{\textbf{\linebreak }}}}\textcolor[HTML]{000000}{\fontsize{14}{14}\selectfont{\global\setmainfont{Arial}{\textbf{Covariates}}}}} & \multicolumn{2}{>{\centering}m{\dimexpr 1.58in+2\tabcolsep}}{\textcolor[HTML]{000000}{\fontsize{14}{14}\selectfont{\global\setmainfont{Arial}{\textbf{With}}}}\textcolor[HTML]{000000}{\fontsize{14}{14}\selectfont{\global\setmainfont{Arial}{\textbf{\linebreak }}}}\textcolor[HTML]{000000}{\fontsize{14}{14}\selectfont{\global\setmainfont{Arial}{\textbf{Covariates}}}}} \\

\ascline{1pt}{000000}{1-5}



\multicolumn{1}{>{\raggedright}m{\dimexpr 1.3in+0\tabcolsep}}{\textcolor[HTML]{000000}{\fontsize{11}{11}\selectfont{\global\setmainfont{Arial}{\textbf{Characteristic}}}}} & \multicolumn{1}{>{\centering}m{\dimexpr 0.76in+0\tabcolsep}}{\textcolor[HTML]{000000}{\fontsize{11}{11}\selectfont{\global\setmainfont{Arial}{\textbf{Beta}}}}} & \multicolumn{1}{>{\centering}m{\dimexpr 0.82in+0\tabcolsep}}{\textcolor[HTML]{000000}{\fontsize{11}{11}\selectfont{\global\setmainfont{Arial}{\textbf{p-value}}}}} & \multicolumn{1}{>{\centering}m{\dimexpr 0.76in+0\tabcolsep}}{\textcolor[HTML]{000000}{\fontsize{11}{11}\selectfont{\global\setmainfont{Arial}{\textbf{Beta}}}}} & \multicolumn{1}{>{\centering}m{\dimexpr 0.82in+0\tabcolsep}}{\textcolor[HTML]{000000}{\fontsize{11}{11}\selectfont{\global\setmainfont{Arial}{\textbf{p-value}}}}} \\

\ascline{1pt}{000000}{1-5}\endfirsthead 

\ascline{1pt}{000000}{1-5}

\multicolumn{1}{>{\raggedright}m{\dimexpr 1.3in+0\tabcolsep}}{\textcolor[HTML]{000000}{\fontsize{14}{14}\selectfont{\global\setmainfont{Arial}{\textbf{\ }}}}} & \multicolumn{2}{>{\centering}m{\dimexpr 1.58in+2\tabcolsep}}{\textcolor[HTML]{000000}{\fontsize{14}{14}\selectfont{\global\setmainfont{Arial}{\textbf{Without}}}}\textcolor[HTML]{000000}{\fontsize{14}{14}\selectfont{\global\setmainfont{Arial}{\textbf{\linebreak }}}}\textcolor[HTML]{000000}{\fontsize{14}{14}\selectfont{\global\setmainfont{Arial}{\textbf{Covariates}}}}} & \multicolumn{2}{>{\centering}m{\dimexpr 1.58in+2\tabcolsep}}{\textcolor[HTML]{000000}{\fontsize{14}{14}\selectfont{\global\setmainfont{Arial}{\textbf{With}}}}\textcolor[HTML]{000000}{\fontsize{14}{14}\selectfont{\global\setmainfont{Arial}{\textbf{\linebreak }}}}\textcolor[HTML]{000000}{\fontsize{14}{14}\selectfont{\global\setmainfont{Arial}{\textbf{Covariates}}}}} \\

\ascline{1pt}{000000}{1-5}



\multicolumn{1}{>{\raggedright}m{\dimexpr 1.3in+0\tabcolsep}}{\textcolor[HTML]{000000}{\fontsize{11}{11}\selectfont{\global\setmainfont{Arial}{\textbf{Characteristic}}}}} & \multicolumn{1}{>{\centering}m{\dimexpr 0.76in+0\tabcolsep}}{\textcolor[HTML]{000000}{\fontsize{11}{11}\selectfont{\global\setmainfont{Arial}{\textbf{Beta}}}}} & \multicolumn{1}{>{\centering}m{\dimexpr 0.82in+0\tabcolsep}}{\textcolor[HTML]{000000}{\fontsize{11}{11}\selectfont{\global\setmainfont{Arial}{\textbf{p-value}}}}} & \multicolumn{1}{>{\centering}m{\dimexpr 0.76in+0\tabcolsep}}{\textcolor[HTML]{000000}{\fontsize{11}{11}\selectfont{\global\setmainfont{Arial}{\textbf{Beta}}}}} & \multicolumn{1}{>{\centering}m{\dimexpr 0.82in+0\tabcolsep}}{\textcolor[HTML]{000000}{\fontsize{11}{11}\selectfont{\global\setmainfont{Arial}{\textbf{p-value}}}}} \\

\ascline{1pt}{000000}{1-5}\endhead



\multicolumn{1}{>{\raggedright}p{\dimexpr 1.3in+0\tabcolsep}}{\textcolor[HTML]{000000}{\fontsize{12}{12}\selectfont{\global\setmainfont{Arial}{(Intercept)}}}} & \multicolumn{1}{>{\centering}p{\dimexpr 0.76in+0\tabcolsep}}{\textcolor[HTML]{000000}{\fontsize{12}{12}\selectfont{\global\setmainfont{Arial}{0.106}}}} & \multicolumn{1}{>{\centering}p{\dimexpr 0.82in+0\tabcolsep}}{\textcolor[HTML]{000000}{\fontsize{12}{12}\selectfont{\global\setmainfont{Arial}{<0.001}}}} & \multicolumn{1}{>{\centering}p{\dimexpr 0.76in+0\tabcolsep}}{\textcolor[HTML]{000000}{\fontsize{12}{12}\selectfont{\global\setmainfont{Arial}{0.151}}}} & \multicolumn{1}{>{\centering}p{\dimexpr 0.82in+0\tabcolsep}}{\textcolor[HTML]{000000}{\fontsize{12}{12}\selectfont{\global\setmainfont{Arial}{<0.001}}}} \\





\multicolumn{1}{>{\raggedright}p{\dimexpr 1.3in+0\tabcolsep}}{\textcolor[HTML]{000000}{\fontsize{12}{12}\selectfont{\global\setmainfont{Arial}{promotion}}}} & \multicolumn{1}{>{\centering}p{\dimexpr 0.76in+0\tabcolsep}}{\textcolor[HTML]{000000}{\fontsize{12}{12}\selectfont{\global\setmainfont{Arial}{0.049}}}} & \multicolumn{1}{>{\centering}p{\dimexpr 0.82in+0\tabcolsep}}{\textcolor[HTML]{000000}{\fontsize{12}{12}\selectfont{\global\setmainfont{Arial}{<0.001}}}} & \multicolumn{1}{>{\centering}p{\dimexpr 0.76in+0\tabcolsep}}{\textcolor[HTML]{000000}{\fontsize{12}{12}\selectfont{\global\setmainfont{Arial}{0.050}}}} & \multicolumn{1}{>{\centering}p{\dimexpr 0.82in+0\tabcolsep}}{\textcolor[HTML]{000000}{\fontsize{12}{12}\selectfont{\global\setmainfont{Arial}{<0.001}}}} \\





\multicolumn{1}{>{\raggedright}p{\dimexpr 1.3in+0\tabcolsep}}{\textcolor[HTML]{000000}{\fontsize{12}{12}\selectfont{\global\setmainfont{Arial}{recency}}}} & \multicolumn{1}{>{\centering}p{\dimexpr 0.76in+0\tabcolsep}}{\textcolor[HTML]{000000}{\fontsize{12}{12}\selectfont{\global\setmainfont{Arial}{}}}} & \multicolumn{1}{>{\centering}p{\dimexpr 0.82in+0\tabcolsep}}{\textcolor[HTML]{000000}{\fontsize{12}{12}\selectfont{\global\setmainfont{Arial}{}}}} & \multicolumn{1}{>{\centering}p{\dimexpr 0.76in+0\tabcolsep}}{\textcolor[HTML]{000000}{\fontsize{12}{12}\selectfont{\global\setmainfont{Arial}{-0.006}}}} & \multicolumn{1}{>{\centering}p{\dimexpr 0.82in+0\tabcolsep}}{\textcolor[HTML]{000000}{\fontsize{12}{12}\selectfont{\global\setmainfont{Arial}{<0.001}}}} \\





\multicolumn{1}{>{\raggedright}p{\dimexpr 1.3in+0\tabcolsep}}{\textcolor[HTML]{000000}{\fontsize{12}{12}\selectfont{\global\setmainfont{Arial}{history}}}} & \multicolumn{1}{>{\centering}p{\dimexpr 0.76in+0\tabcolsep}}{\textcolor[HTML]{000000}{\fontsize{12}{12}\selectfont{\global\setmainfont{Arial}{}}}} & \multicolumn{1}{>{\centering}p{\dimexpr 0.82in+0\tabcolsep}}{\textcolor[HTML]{000000}{\fontsize{12}{12}\selectfont{\global\setmainfont{Arial}{}}}} & \multicolumn{1}{>{\centering}p{\dimexpr 0.76in+0\tabcolsep}}{\textcolor[HTML]{000000}{\fontsize{12}{12}\selectfont{\global\setmainfont{Arial}{0.000}}}} & \multicolumn{1}{>{\centering}p{\dimexpr 0.82in+0\tabcolsep}}{\textcolor[HTML]{000000}{\fontsize{12}{12}\selectfont{\global\setmainfont{Arial}{<0.001}}}} \\





\multicolumn{1}{>{\raggedright}p{\dimexpr 1.3in+0\tabcolsep}}{\textcolor[HTML]{000000}{\fontsize{12}{12}\selectfont{\global\setmainfont{Arial}{womens}}}} & \multicolumn{1}{>{\centering}p{\dimexpr 0.76in+0\tabcolsep}}{\textcolor[HTML]{000000}{\fontsize{12}{12}\selectfont{\global\setmainfont{Arial}{}}}} & \multicolumn{1}{>{\centering}p{\dimexpr 0.82in+0\tabcolsep}}{\textcolor[HTML]{000000}{\fontsize{12}{12}\selectfont{\global\setmainfont{Arial}{}}}} & \multicolumn{1}{>{\centering}p{\dimexpr 0.76in+0\tabcolsep}}{\textcolor[HTML]{000000}{\fontsize{12}{12}\selectfont{\global\setmainfont{Arial}{0.046}}}} & \multicolumn{1}{>{\centering}p{\dimexpr 0.82in+0\tabcolsep}}{\textcolor[HTML]{000000}{\fontsize{12}{12}\selectfont{\global\setmainfont{Arial}{<0.001}}}} \\





\multicolumn{1}{>{\raggedright}p{\dimexpr 1.3in+0\tabcolsep}}{\textcolor[HTML]{000000}{\fontsize{12}{12}\selectfont{\global\setmainfont{Arial}{zip}}}} & \multicolumn{1}{>{\centering}p{\dimexpr 0.76in+0\tabcolsep}}{\textcolor[HTML]{000000}{\fontsize{12}{12}\selectfont{\global\setmainfont{Arial}{}}}} & \multicolumn{1}{>{\centering}p{\dimexpr 0.82in+0\tabcolsep}}{\textcolor[HTML]{000000}{\fontsize{12}{12}\selectfont{\global\setmainfont{Arial}{}}}} & \multicolumn{1}{>{\centering}p{\dimexpr 0.76in+0\tabcolsep}}{\textcolor[HTML]{000000}{\fontsize{12}{12}\selectfont{\global\setmainfont{Arial}{}}}} & \multicolumn{1}{>{\centering}p{\dimexpr 0.82in+0\tabcolsep}}{\textcolor[HTML]{000000}{\fontsize{12}{12}\selectfont{\global\setmainfont{Arial}{}}}} \\





\multicolumn{1}{>{\raggedright}p{\dimexpr 1.3in+0\tabcolsep}}{\textcolor[HTML]{000000}{\fontsize{12}{12}\selectfont{\global\setmainfont{Arial}{Rural}}}} & \multicolumn{1}{>{\centering}p{\dimexpr 0.76in+0\tabcolsep}}{\textcolor[HTML]{000000}{\fontsize{12}{12}\selectfont{\global\setmainfont{Arial}{}}}} & \multicolumn{1}{>{\centering}p{\dimexpr 0.82in+0\tabcolsep}}{\textcolor[HTML]{000000}{\fontsize{12}{12}\selectfont{\global\setmainfont{Arial}{}}}} & \multicolumn{1}{>{\centering}p{\dimexpr 0.76in+0\tabcolsep}}{\textcolor[HTML]{000000}{\fontsize{12}{12}\selectfont{\global\setmainfont{Arial}{—}}}} & \multicolumn{1}{>{\centering}p{\dimexpr 0.82in+0\tabcolsep}}{\textcolor[HTML]{000000}{\fontsize{12}{12}\selectfont{\global\setmainfont{Arial}{}}}} \\





\multicolumn{1}{>{\raggedright}p{\dimexpr 1.3in+0\tabcolsep}}{\textcolor[HTML]{000000}{\fontsize{12}{12}\selectfont{\global\setmainfont{Arial}{Surburban}}}} & \multicolumn{1}{>{\centering}p{\dimexpr 0.76in+0\tabcolsep}}{\textcolor[HTML]{000000}{\fontsize{12}{12}\selectfont{\global\setmainfont{Arial}{}}}} & \multicolumn{1}{>{\centering}p{\dimexpr 0.82in+0\tabcolsep}}{\textcolor[HTML]{000000}{\fontsize{12}{12}\selectfont{\global\setmainfont{Arial}{}}}} & \multicolumn{1}{>{\centering}p{\dimexpr 0.76in+0\tabcolsep}}{\textcolor[HTML]{000000}{\fontsize{12}{12}\selectfont{\global\setmainfont{Arial}{-0.053}}}} & \multicolumn{1}{>{\centering}p{\dimexpr 0.82in+0\tabcolsep}}{\textcolor[HTML]{000000}{\fontsize{12}{12}\selectfont{\global\setmainfont{Arial}{<0.001}}}} \\





\multicolumn{1}{>{\raggedright}p{\dimexpr 1.3in+0\tabcolsep}}{\textcolor[HTML]{000000}{\fontsize{12}{12}\selectfont{\global\setmainfont{Arial}{Urban}}}} & \multicolumn{1}{>{\centering}p{\dimexpr 0.76in+0\tabcolsep}}{\textcolor[HTML]{000000}{\fontsize{12}{12}\selectfont{\global\setmainfont{Arial}{}}}} & \multicolumn{1}{>{\centering}p{\dimexpr 0.82in+0\tabcolsep}}{\textcolor[HTML]{000000}{\fontsize{12}{12}\selectfont{\global\setmainfont{Arial}{}}}} & \multicolumn{1}{>{\centering}p{\dimexpr 0.76in+0\tabcolsep}}{\textcolor[HTML]{000000}{\fontsize{12}{12}\selectfont{\global\setmainfont{Arial}{-0.065}}}} & \multicolumn{1}{>{\centering}p{\dimexpr 0.82in+0\tabcolsep}}{\textcolor[HTML]{000000}{\fontsize{12}{12}\selectfont{\global\setmainfont{Arial}{<0.001}}}} \\

\ascline{1pt}{000000}{1-5}



\multicolumn{1}{>{\raggedright}p{\dimexpr 1.3in+0\tabcolsep}}{\textcolor[HTML]{000000}{\fontsize{12}{12}\selectfont{\global\setmainfont{Arial}{p-value}}}} & \multicolumn{1}{>{\centering}p{\dimexpr 0.76in+0\tabcolsep}}{\textcolor[HTML]{000000}{\fontsize{12}{12}\selectfont{\global\setmainfont{Arial}{<0.001}}}} & \multicolumn{1}{>{\centering}p{\dimexpr 0.82in+0\tabcolsep}}{\textcolor[HTML]{000000}{\fontsize{12}{12}\selectfont{\global\setmainfont{Arial}{}}}} & \multicolumn{1}{>{\centering}p{\dimexpr 0.76in+0\tabcolsep}}{\textcolor[HTML]{000000}{\fontsize{12}{12}\selectfont{\global\setmainfont{Arial}{<0.001}}}} & \multicolumn{1}{>{\centering}p{\dimexpr 0.82in+0\tabcolsep}}{\textcolor[HTML]{000000}{\fontsize{12}{12}\selectfont{\global\setmainfont{Arial}{}}}} \\





\multicolumn{1}{>{\raggedright}p{\dimexpr 1.3in+0\tabcolsep}}{\textcolor[HTML]{000000}{\fontsize{12}{12}\selectfont{\global\setmainfont{Arial}{R²}}}} & \multicolumn{1}{>{\centering}p{\dimexpr 0.76in+0\tabcolsep}}{\textcolor[HTML]{000000}{\fontsize{12}{12}\selectfont{\global\setmainfont{Arial}{0.005}}}} & \multicolumn{1}{>{\centering}p{\dimexpr 0.82in+0\tabcolsep}}{\textcolor[HTML]{000000}{\fontsize{12}{12}\selectfont{\global\setmainfont{Arial}{}}}} & \multicolumn{1}{>{\centering}p{\dimexpr 0.76in+0\tabcolsep}}{\textcolor[HTML]{000000}{\fontsize{12}{12}\selectfont{\global\setmainfont{Arial}{0.024}}}} & \multicolumn{1}{>{\centering}p{\dimexpr 0.82in+0\tabcolsep}}{\textcolor[HTML]{000000}{\fontsize{12}{12}\selectfont{\global\setmainfont{Arial}{}}}} \\

\ascline{1pt}{000000}{1-5}



\end{longtable}



\arrayrulecolor[HTML]{000000}

\global\setlength{\arrayrulewidth}{\Oldarrayrulewidth}

\global\setlength{\tabcolsep}{\Oldtabcolsep}

\renewcommand*{\arraystretch}{1}

\begin{Shaded}
\begin{Highlighting}[]
\NormalTok{m\_ab\_spend }\OtherTok{\textless{}{-}} \FunctionTok{lm}\NormalTok{(spend }\SpecialCharTok{\textasciitilde{}}\NormalTok{ promotion }\SpecialCharTok{+}\NormalTok{ recency }\SpecialCharTok{+}\NormalTok{ history }\SpecialCharTok{+}\NormalTok{ womens }\SpecialCharTok{+}\NormalTok{ zip,}
                 \AttributeTok{data =}\NormalTok{ email.camp.w)}
\FunctionTok{easy\_ab\_ate}\NormalTok{(}\AttributeTok{model =}\NormalTok{ m\_ab\_spend, }\AttributeTok{treatment =} \StringTok{"promotion"}\NormalTok{, }\AttributeTok{ft =} \ConstantTok{TRUE}\NormalTok{)}
\end{Highlighting}
\end{Shaded}

\global\setlength{\Oldarrayrulewidth}{\arrayrulewidth}

\global\setlength{\Oldtabcolsep}{\tabcolsep}

\setlength{\tabcolsep}{2pt}

\renewcommand*{\arraystretch}{1.5}



\providecommand{\ascline}[3]{\noalign{\global\arrayrulewidth #1}\arrayrulecolor[HTML]{#2}\cline{#3}}

\begin{longtable}[c]{|p{1.30in}|p{0.67in}|p{0.82in}|p{0.72in}|p{0.82in}}



\ascline{1pt}{000000}{1-5}

\multicolumn{1}{>{\raggedright}m{\dimexpr 1.3in+0\tabcolsep}}{\textcolor[HTML]{000000}{\fontsize{14}{14}\selectfont{\global\setmainfont{Arial}{\textbf{\ }}}}} & \multicolumn{2}{>{\centering}m{\dimexpr 1.49in+2\tabcolsep}}{\textcolor[HTML]{000000}{\fontsize{14}{14}\selectfont{\global\setmainfont{Arial}{\textbf{Without}}}}\textcolor[HTML]{000000}{\fontsize{14}{14}\selectfont{\global\setmainfont{Arial}{\textbf{\linebreak }}}}\textcolor[HTML]{000000}{\fontsize{14}{14}\selectfont{\global\setmainfont{Arial}{\textbf{Covariates}}}}} & \multicolumn{2}{>{\centering}m{\dimexpr 1.54in+2\tabcolsep}}{\textcolor[HTML]{000000}{\fontsize{14}{14}\selectfont{\global\setmainfont{Arial}{\textbf{With}}}}\textcolor[HTML]{000000}{\fontsize{14}{14}\selectfont{\global\setmainfont{Arial}{\textbf{\linebreak }}}}\textcolor[HTML]{000000}{\fontsize{14}{14}\selectfont{\global\setmainfont{Arial}{\textbf{Covariates}}}}} \\

\ascline{1pt}{000000}{1-5}



\multicolumn{1}{>{\raggedright}m{\dimexpr 1.3in+0\tabcolsep}}{\textcolor[HTML]{000000}{\fontsize{11}{11}\selectfont{\global\setmainfont{Arial}{\textbf{Characteristic}}}}} & \multicolumn{1}{>{\centering}m{\dimexpr 0.67in+0\tabcolsep}}{\textcolor[HTML]{000000}{\fontsize{11}{11}\selectfont{\global\setmainfont{Arial}{\textbf{Beta}}}}} & \multicolumn{1}{>{\centering}m{\dimexpr 0.82in+0\tabcolsep}}{\textcolor[HTML]{000000}{\fontsize{11}{11}\selectfont{\global\setmainfont{Arial}{\textbf{p-value}}}}} & \multicolumn{1}{>{\centering}m{\dimexpr 0.72in+0\tabcolsep}}{\textcolor[HTML]{000000}{\fontsize{11}{11}\selectfont{\global\setmainfont{Arial}{\textbf{Beta}}}}} & \multicolumn{1}{>{\centering}m{\dimexpr 0.82in+0\tabcolsep}}{\textcolor[HTML]{000000}{\fontsize{11}{11}\selectfont{\global\setmainfont{Arial}{\textbf{p-value}}}}} \\

\ascline{1pt}{000000}{1-5}\endfirsthead 

\ascline{1pt}{000000}{1-5}

\multicolumn{1}{>{\raggedright}m{\dimexpr 1.3in+0\tabcolsep}}{\textcolor[HTML]{000000}{\fontsize{14}{14}\selectfont{\global\setmainfont{Arial}{\textbf{\ }}}}} & \multicolumn{2}{>{\centering}m{\dimexpr 1.49in+2\tabcolsep}}{\textcolor[HTML]{000000}{\fontsize{14}{14}\selectfont{\global\setmainfont{Arial}{\textbf{Without}}}}\textcolor[HTML]{000000}{\fontsize{14}{14}\selectfont{\global\setmainfont{Arial}{\textbf{\linebreak }}}}\textcolor[HTML]{000000}{\fontsize{14}{14}\selectfont{\global\setmainfont{Arial}{\textbf{Covariates}}}}} & \multicolumn{2}{>{\centering}m{\dimexpr 1.54in+2\tabcolsep}}{\textcolor[HTML]{000000}{\fontsize{14}{14}\selectfont{\global\setmainfont{Arial}{\textbf{With}}}}\textcolor[HTML]{000000}{\fontsize{14}{14}\selectfont{\global\setmainfont{Arial}{\textbf{\linebreak }}}}\textcolor[HTML]{000000}{\fontsize{14}{14}\selectfont{\global\setmainfont{Arial}{\textbf{Covariates}}}}} \\

\ascline{1pt}{000000}{1-5}



\multicolumn{1}{>{\raggedright}m{\dimexpr 1.3in+0\tabcolsep}}{\textcolor[HTML]{000000}{\fontsize{11}{11}\selectfont{\global\setmainfont{Arial}{\textbf{Characteristic}}}}} & \multicolumn{1}{>{\centering}m{\dimexpr 0.67in+0\tabcolsep}}{\textcolor[HTML]{000000}{\fontsize{11}{11}\selectfont{\global\setmainfont{Arial}{\textbf{Beta}}}}} & \multicolumn{1}{>{\centering}m{\dimexpr 0.82in+0\tabcolsep}}{\textcolor[HTML]{000000}{\fontsize{11}{11}\selectfont{\global\setmainfont{Arial}{\textbf{p-value}}}}} & \multicolumn{1}{>{\centering}m{\dimexpr 0.72in+0\tabcolsep}}{\textcolor[HTML]{000000}{\fontsize{11}{11}\selectfont{\global\setmainfont{Arial}{\textbf{Beta}}}}} & \multicolumn{1}{>{\centering}m{\dimexpr 0.82in+0\tabcolsep}}{\textcolor[HTML]{000000}{\fontsize{11}{11}\selectfont{\global\setmainfont{Arial}{\textbf{p-value}}}}} \\

\ascline{1pt}{000000}{1-5}\endhead



\multicolumn{1}{>{\raggedright}p{\dimexpr 1.3in+0\tabcolsep}}{\textcolor[HTML]{000000}{\fontsize{12}{12}\selectfont{\global\setmainfont{Arial}{(Intercept)}}}} & \multicolumn{1}{>{\centering}p{\dimexpr 0.67in+0\tabcolsep}}{\textcolor[HTML]{000000}{\fontsize{12}{12}\selectfont{\global\setmainfont{Arial}{0.651}}}} & \multicolumn{1}{>{\centering}p{\dimexpr 0.82in+0\tabcolsep}}{\textcolor[HTML]{000000}{\fontsize{12}{12}\selectfont{\global\setmainfont{Arial}{<0.001}}}} & \multicolumn{1}{>{\centering}p{\dimexpr 0.72in+0\tabcolsep}}{\textcolor[HTML]{000000}{\fontsize{12}{12}\selectfont{\global\setmainfont{Arial}{1.265}}}} & \multicolumn{1}{>{\centering}p{\dimexpr 0.82in+0\tabcolsep}}{\textcolor[HTML]{000000}{\fontsize{12}{12}\selectfont{\global\setmainfont{Arial}{0.011}}}} \\





\multicolumn{1}{>{\raggedright}p{\dimexpr 1.3in+0\tabcolsep}}{\textcolor[HTML]{000000}{\fontsize{12}{12}\selectfont{\global\setmainfont{Arial}{promotion}}}} & \multicolumn{1}{>{\centering}p{\dimexpr 0.67in+0\tabcolsep}}{\textcolor[HTML]{000000}{\fontsize{12}{12}\selectfont{\global\setmainfont{Arial}{0.436}}}} & \multicolumn{1}{>{\centering}p{\dimexpr 0.82in+0\tabcolsep}}{\textcolor[HTML]{000000}{\fontsize{12}{12}\selectfont{\global\setmainfont{Arial}{0.108}}}} & \multicolumn{1}{>{\centering}p{\dimexpr 0.72in+0\tabcolsep}}{\textcolor[HTML]{000000}{\fontsize{12}{12}\selectfont{\global\setmainfont{Arial}{0.450}}}} & \multicolumn{1}{>{\centering}p{\dimexpr 0.82in+0\tabcolsep}}{\textcolor[HTML]{000000}{\fontsize{12}{12}\selectfont{\global\setmainfont{Arial}{0.097}}}} \\





\multicolumn{1}{>{\raggedright}p{\dimexpr 1.3in+0\tabcolsep}}{\textcolor[HTML]{000000}{\fontsize{12}{12}\selectfont{\global\setmainfont{Arial}{recency}}}} & \multicolumn{1}{>{\centering}p{\dimexpr 0.67in+0\tabcolsep}}{\textcolor[HTML]{000000}{\fontsize{12}{12}\selectfont{\global\setmainfont{Arial}{}}}} & \multicolumn{1}{>{\centering}p{\dimexpr 0.82in+0\tabcolsep}}{\textcolor[HTML]{000000}{\fontsize{12}{12}\selectfont{\global\setmainfont{Arial}{}}}} & \multicolumn{1}{>{\centering}p{\dimexpr 0.72in+0\tabcolsep}}{\textcolor[HTML]{000000}{\fontsize{12}{12}\selectfont{\global\setmainfont{Arial}{-0.081}}}} & \multicolumn{1}{>{\centering}p{\dimexpr 0.82in+0\tabcolsep}}{\textcolor[HTML]{000000}{\fontsize{12}{12}\selectfont{\global\setmainfont{Arial}{0.042}}}} \\





\multicolumn{1}{>{\raggedright}p{\dimexpr 1.3in+0\tabcolsep}}{\textcolor[HTML]{000000}{\fontsize{12}{12}\selectfont{\global\setmainfont{Arial}{history}}}} & \multicolumn{1}{>{\centering}p{\dimexpr 0.67in+0\tabcolsep}}{\textcolor[HTML]{000000}{\fontsize{12}{12}\selectfont{\global\setmainfont{Arial}{}}}} & \multicolumn{1}{>{\centering}p{\dimexpr 0.82in+0\tabcolsep}}{\textcolor[HTML]{000000}{\fontsize{12}{12}\selectfont{\global\setmainfont{Arial}{}}}} & \multicolumn{1}{>{\centering}p{\dimexpr 0.72in+0\tabcolsep}}{\textcolor[HTML]{000000}{\fontsize{12}{12}\selectfont{\global\setmainfont{Arial}{0.000}}}} & \multicolumn{1}{>{\centering}p{\dimexpr 0.82in+0\tabcolsep}}{\textcolor[HTML]{000000}{\fontsize{12}{12}\selectfont{\global\setmainfont{Arial}{0.703}}}} \\





\multicolumn{1}{>{\raggedright}p{\dimexpr 1.3in+0\tabcolsep}}{\textcolor[HTML]{000000}{\fontsize{12}{12}\selectfont{\global\setmainfont{Arial}{womens}}}} & \multicolumn{1}{>{\centering}p{\dimexpr 0.67in+0\tabcolsep}}{\textcolor[HTML]{000000}{\fontsize{12}{12}\selectfont{\global\setmainfont{Arial}{}}}} & \multicolumn{1}{>{\centering}p{\dimexpr 0.82in+0\tabcolsep}}{\textcolor[HTML]{000000}{\fontsize{12}{12}\selectfont{\global\setmainfont{Arial}{}}}} & \multicolumn{1}{>{\centering}p{\dimexpr 0.72in+0\tabcolsep}}{\textcolor[HTML]{000000}{\fontsize{12}{12}\selectfont{\global\setmainfont{Arial}{0.049}}}} & \multicolumn{1}{>{\centering}p{\dimexpr 0.82in+0\tabcolsep}}{\textcolor[HTML]{000000}{\fontsize{12}{12}\selectfont{\global\setmainfont{Arial}{0.858}}}} \\





\multicolumn{1}{>{\raggedright}p{\dimexpr 1.3in+0\tabcolsep}}{\textcolor[HTML]{000000}{\fontsize{12}{12}\selectfont{\global\setmainfont{Arial}{zip}}}} & \multicolumn{1}{>{\centering}p{\dimexpr 0.67in+0\tabcolsep}}{\textcolor[HTML]{000000}{\fontsize{12}{12}\selectfont{\global\setmainfont{Arial}{}}}} & \multicolumn{1}{>{\centering}p{\dimexpr 0.82in+0\tabcolsep}}{\textcolor[HTML]{000000}{\fontsize{12}{12}\selectfont{\global\setmainfont{Arial}{}}}} & \multicolumn{1}{>{\centering}p{\dimexpr 0.72in+0\tabcolsep}}{\textcolor[HTML]{000000}{\fontsize{12}{12}\selectfont{\global\setmainfont{Arial}{}}}} & \multicolumn{1}{>{\centering}p{\dimexpr 0.82in+0\tabcolsep}}{\textcolor[HTML]{000000}{\fontsize{12}{12}\selectfont{\global\setmainfont{Arial}{}}}} \\





\multicolumn{1}{>{\raggedright}p{\dimexpr 1.3in+0\tabcolsep}}{\textcolor[HTML]{000000}{\fontsize{12}{12}\selectfont{\global\setmainfont{Arial}{Rural}}}} & \multicolumn{1}{>{\centering}p{\dimexpr 0.67in+0\tabcolsep}}{\textcolor[HTML]{000000}{\fontsize{12}{12}\selectfont{\global\setmainfont{Arial}{}}}} & \multicolumn{1}{>{\centering}p{\dimexpr 0.82in+0\tabcolsep}}{\textcolor[HTML]{000000}{\fontsize{12}{12}\selectfont{\global\setmainfont{Arial}{}}}} & \multicolumn{1}{>{\centering}p{\dimexpr 0.72in+0\tabcolsep}}{\textcolor[HTML]{000000}{\fontsize{12}{12}\selectfont{\global\setmainfont{Arial}{—}}}} & \multicolumn{1}{>{\centering}p{\dimexpr 0.82in+0\tabcolsep}}{\textcolor[HTML]{000000}{\fontsize{12}{12}\selectfont{\global\setmainfont{Arial}{}}}} \\





\multicolumn{1}{>{\raggedright}p{\dimexpr 1.3in+0\tabcolsep}}{\textcolor[HTML]{000000}{\fontsize{12}{12}\selectfont{\global\setmainfont{Arial}{Surburban}}}} & \multicolumn{1}{>{\centering}p{\dimexpr 0.67in+0\tabcolsep}}{\textcolor[HTML]{000000}{\fontsize{12}{12}\selectfont{\global\setmainfont{Arial}{}}}} & \multicolumn{1}{>{\centering}p{\dimexpr 0.82in+0\tabcolsep}}{\textcolor[HTML]{000000}{\fontsize{12}{12}\selectfont{\global\setmainfont{Arial}{}}}} & \multicolumn{1}{>{\centering}p{\dimexpr 0.72in+0\tabcolsep}}{\textcolor[HTML]{000000}{\fontsize{12}{12}\selectfont{\global\setmainfont{Arial}{-0.596}}}} & \multicolumn{1}{>{\centering}p{\dimexpr 0.82in+0\tabcolsep}}{\textcolor[HTML]{000000}{\fontsize{12}{12}\selectfont{\global\setmainfont{Arial}{0.144}}}} \\





\multicolumn{1}{>{\raggedright}p{\dimexpr 1.3in+0\tabcolsep}}{\textcolor[HTML]{000000}{\fontsize{12}{12}\selectfont{\global\setmainfont{Arial}{Urban}}}} & \multicolumn{1}{>{\centering}p{\dimexpr 0.67in+0\tabcolsep}}{\textcolor[HTML]{000000}{\fontsize{12}{12}\selectfont{\global\setmainfont{Arial}{}}}} & \multicolumn{1}{>{\centering}p{\dimexpr 0.82in+0\tabcolsep}}{\textcolor[HTML]{000000}{\fontsize{12}{12}\selectfont{\global\setmainfont{Arial}{}}}} & \multicolumn{1}{>{\centering}p{\dimexpr 0.72in+0\tabcolsep}}{\textcolor[HTML]{000000}{\fontsize{12}{12}\selectfont{\global\setmainfont{Arial}{0.098}}}} & \multicolumn{1}{>{\centering}p{\dimexpr 0.82in+0\tabcolsep}}{\textcolor[HTML]{000000}{\fontsize{12}{12}\selectfont{\global\setmainfont{Arial}{0.814}}}} \\

\ascline{1pt}{000000}{1-5}



\multicolumn{1}{>{\raggedright}p{\dimexpr 1.3in+0\tabcolsep}}{\textcolor[HTML]{000000}{\fontsize{12}{12}\selectfont{\global\setmainfont{Arial}{p-value}}}} & \multicolumn{1}{>{\centering}p{\dimexpr 0.67in+0\tabcolsep}}{\textcolor[HTML]{000000}{\fontsize{12}{12}\selectfont{\global\setmainfont{Arial}{0.11}}}} & \multicolumn{1}{>{\centering}p{\dimexpr 0.82in+0\tabcolsep}}{\textcolor[HTML]{000000}{\fontsize{12}{12}\selectfont{\global\setmainfont{Arial}{}}}} & \multicolumn{1}{>{\centering}p{\dimexpr 0.72in+0\tabcolsep}}{\textcolor[HTML]{000000}{\fontsize{12}{12}\selectfont{\global\setmainfont{Arial}{0.032}}}} & \multicolumn{1}{>{\centering}p{\dimexpr 0.82in+0\tabcolsep}}{\textcolor[HTML]{000000}{\fontsize{12}{12}\selectfont{\global\setmainfont{Arial}{}}}} \\





\multicolumn{1}{>{\raggedright}p{\dimexpr 1.3in+0\tabcolsep}}{\textcolor[HTML]{000000}{\fontsize{12}{12}\selectfont{\global\setmainfont{Arial}{R²}}}} & \multicolumn{1}{>{\centering}p{\dimexpr 0.67in+0\tabcolsep}}{\textcolor[HTML]{000000}{\fontsize{12}{12}\selectfont{\global\setmainfont{Arial}{0.000}}}} & \multicolumn{1}{>{\centering}p{\dimexpr 0.82in+0\tabcolsep}}{\textcolor[HTML]{000000}{\fontsize{12}{12}\selectfont{\global\setmainfont{Arial}{}}}} & \multicolumn{1}{>{\centering}p{\dimexpr 0.72in+0\tabcolsep}}{\textcolor[HTML]{000000}{\fontsize{12}{12}\selectfont{\global\setmainfont{Arial}{0.001}}}} & \multicolumn{1}{>{\centering}p{\dimexpr 0.82in+0\tabcolsep}}{\textcolor[HTML]{000000}{\fontsize{12}{12}\selectfont{\global\setmainfont{Arial}{}}}} \\

\ascline{1pt}{000000}{1-5}



\end{longtable}



\arrayrulecolor[HTML]{000000}

\global\setlength{\arrayrulewidth}{\Oldarrayrulewidth}

\global\setlength{\tabcolsep}{\Oldtabcolsep}

\renewcommand*{\arraystretch}{1}

The table compares two models:

\begin{itemize}
\tightlist
\item
  \textbf{Without covariates}: a pure A/B comparison.
\item
  \textbf{With covariates}: a regression-adjusted estimate.
\end{itemize}

The treatment coefficient represents the average change in response probability caused
by the promotion in isolation (in the without covariates column) or when controlling for other variables (in the with covariates column).

\subsection{Why Average Effects Are Not Enough}\label{why-average-effects-are-not-enough}

While the ATE is useful, it hides important heterogeneity:

\begin{itemize}
\tightlist
\item
  Some customers may respond very positively.
\item
  Others may be unaffected or even respond negatively.
\end{itemize}

From a managerial perspective, sending promotions to everyone may be inefficient or
costly. This motivates \textbf{uplift modeling}, which focuses on targeting customers who are
most likely to be influenced.

\begin{center}\rule{0.5\linewidth}{0.5pt}\end{center}

\section{Introduction to Uplift Modeling}\label{introduction-to-uplift-modeling}

\subsection{What Is Uplift?}\label{what-is-uplift}

\textbf{Uplift} measures the \emph{incremental} effect of treatment on an individual. That is, how much more likely is this customer to respond \emph{because} they received the promotion?

This differs from standard prediction, which focuses on response likelihood regardless
of treatment.

\subsection{Two-Model (Indirect) Approach}\label{two-model-indirect-approach}

The two-model approach estimates:

\begin{itemize}
\tightlist
\item
  one model estimated on treated customers, and
\item
  one model estimated on control customers.
\end{itemize}

The difference between their predicted outcomes for a given customer is the estimated uplift.

\begin{center}\rule{0.5\linewidth}{0.5pt}\end{center}

\section{\texorpdfstring{Estimating Uplift with \texttt{easy\_uplift()}}{Estimating Uplift with easy\_uplift()}}\label{estimating-uplift-with-easy_uplift}

\subsection{Model Specification}\label{model-specification}

The outcome variable can be either continuous, like amount spent (outcome) after a promotion (treatment), or it can be binary, like if they visited or not (outcome) after a promotion (treatment). To perform a uplift modeling using regression, we will use the \texttt{easy\_uplift()} function from the \texttt{MKT4320BGSU} package. This function performs uplift modeling based on either logistic regression (for binary outcomes) or linear regression (for continuous outcomes). The function uses the two-model, indirect modeling approach.

Usage:

\begin{itemize}
\tightlist
\item
  \texttt{easy\_uplift(model,\ treatment,\ newdata\ =\ NULL,\ bins\ =\ 10,\ aspect\_ratio\ =\ NULL)}
\item
  where:

  \begin{itemize}
  \tightlist
  \item
    \texttt{model} is a fitted regression model of class glm (binary logit) or lm.
  \item
    \texttt{treatment} is a character string giving the name of the treatment variable. The variable must have exactly two levels and be coded as (0/1), logical, or (``Yes'', ``No'').
  \item
    \texttt{newdata} is an optional data frame on which to compute uplift (e.g., holdout or test data). If NULL, uplift is computed on the model data.
  \item
    \texttt{bins} is an integer; number of groups used for the uplift tables and plots. Must be between 5 and 20. Default is 10.
  \item
    \texttt{aspect\_ratio} is an optional numeric aspect ratio applied to all plots. Default is NULL.
  \end{itemize}
\end{itemize}

In order to use the function, we must first create our base model:

\begin{itemize}
\tightlist
\item
  The base model is usually a model with no interactions included, along with the treatment variable. But if known interactions are to be used, the base model can include the interactions also.
\item
  The base model must contain the treatment variable.
\end{itemize}

Base model examples:

\begin{Shaded}
\begin{Highlighting}[]
\NormalTok{email\_visit }\OtherTok{\textless{}{-}} \FunctionTok{glm}\NormalTok{(visit }\SpecialCharTok{\textasciitilde{}}\NormalTok{ promotion }\SpecialCharTok{+}\NormalTok{ recency }\SpecialCharTok{+}\NormalTok{ history }\SpecialCharTok{+}\NormalTok{ zip }\SpecialCharTok{+}\NormalTok{ womens,}
                   \AttributeTok{data=}\NormalTok{email.camp.w, }\AttributeTok{family=}\StringTok{"binomial"}\NormalTok{)}
\NormalTok{email\_spend }\OtherTok{\textless{}{-}} \FunctionTok{lm}\NormalTok{(spend }\SpecialCharTok{\textasciitilde{}}\NormalTok{ promotion }\SpecialCharTok{+}\NormalTok{ recency }\SpecialCharTok{+}\NormalTok{ history }\SpecialCharTok{+}\NormalTok{ zip }\SpecialCharTok{+}\NormalTok{ womens,}
                  \AttributeTok{data=}\NormalTok{email.camp.w)}
\end{Highlighting}
\end{Shaded}

Once the base model is created, we are aready to use the \texttt{easy\_uplift()} function:

\begin{Shaded}
\begin{Highlighting}[]
\NormalTok{visit\_uplift }\OtherTok{\textless{}{-}} \FunctionTok{easy\_uplift}\NormalTok{(}\AttributeTok{model =}\NormalTok{ email\_visit, }\AttributeTok{treatment =} \StringTok{"promotion"}\NormalTok{)}
\NormalTok{visit\_uplift}\SpecialCharTok{$}\NormalTok{plots}
\end{Highlighting}
\end{Shaded}

\begin{verbatim}
$qini
\end{verbatim}

\pandocbounded{\includegraphics[keepaspectratio]{MKT4320_R_Tutorial_files/figure-latex/uplift-model-1.pdf}}

\begin{verbatim}

$uplift
\end{verbatim}

\pandocbounded{\includegraphics[keepaspectratio]{MKT4320_R_Tutorial_files/figure-latex/uplift-model-2.pdf}}

\begin{verbatim}

$c.gain
\end{verbatim}

\pandocbounded{\includegraphics[keepaspectratio]{MKT4320_R_Tutorial_files/figure-latex/uplift-model-3.pdf}}

\begin{Shaded}
\begin{Highlighting}[]
\NormalTok{spend\_uplift }\OtherTok{\textless{}{-}} \FunctionTok{easy\_uplift}\NormalTok{(}\AttributeTok{model =}\NormalTok{ email\_spend, }\AttributeTok{treatment =} \StringTok{"promotion"}\NormalTok{)}
\NormalTok{spend\_uplift}\SpecialCharTok{$}\NormalTok{plots}
\end{Highlighting}
\end{Shaded}

\begin{verbatim}
$qini
\end{verbatim}

\pandocbounded{\includegraphics[keepaspectratio]{MKT4320_R_Tutorial_files/figure-latex/uplift-model-4.pdf}}

\begin{verbatim}

$uplift
\end{verbatim}

\pandocbounded{\includegraphics[keepaspectratio]{MKT4320_R_Tutorial_files/figure-latex/uplift-model-5.pdf}}

\begin{verbatim}

$c.gain
\end{verbatim}

\pandocbounded{\includegraphics[keepaspectratio]{MKT4320_R_Tutorial_files/figure-latex/uplift-model-6.pdf}}

\subsection{Interpreting Core Outputs}\label{interpreting-core-outputs}

The uplift object includes:

\begin{itemize}
\tightlist
\item
  predicted individual-level lift appended to original data (\texttt{\$all}),
\item
  uplift table by percentile group (\texttt{\$group}),
\item
  diagnostic plots, including:

  \begin{itemize}
  \tightlist
  \item
    uplift by group (\texttt{\$plots\$uplift}),
  \item
    cumulative gain (\texttt{\$plots\$c.gain}),
  \item
    Qini curve (\texttt{\$plots\$qini}).
  \end{itemize}
\end{itemize}

Customers in the top-ranked groups should exhibit the largest incremental gains from treatment.

\begin{center}\rule{0.5\linewidth}{0.5pt}\end{center}

\section{Diagnosing Uplift with Lift Plots}\label{diagnosing-uplift-with-lift-plots}

\subsection{Why Lift Diagnostics Matter}\label{why-lift-diagnostics-matter}

Lift diagnostics help explain \emph{why} uplift varies across customers and which covariates
drive heterogeneity.

\subsection{\texorpdfstring{Using \texttt{easy\_liftplots()}}{Using easy\_liftplots()}}\label{using-easy_liftplots}

We use the \texttt{easy\_liftplots()} function from the \texttt{MKT4320BGSU} package to easily create lift plots.

Usage:

\begin{itemize}
\tightlist
\item
  \texttt{easy\_liftplots(x,\ vars\ =\ "all",\ pairs\ =\ NULL,\ ar\ =\ NULL,\ ci\ =\ 0.95,}\strut \\
  \texttt{bins\ =\ 30,\ numeric\_bins\ =\ 5,\ by\_numeric\_bins\ =\ 3,\ grid\ =\ TRUE,}\strut \\
  \texttt{top\ =\ NULL,\ ft\ =\ TRUE)}
\item
  where:

  \begin{itemize}
  \tightlist
  \item
    \texttt{x} is an object returned by \texttt{easy\_uplift()} (must include x\(all and x\)covariates or x\(spec\)covariates).
  \item
    \texttt{vars} is a character vector of covariate names to plot. Default is ``all'' (uses x\(covariates / x\)spec\$covariates).
  \item
    \texttt{pairs} is an optional list of length-2 character vectors specifying interaction-style plots to create, e.g., list(c(``recency'',``zip''), c(``gender'',``income'')).
  \item
    \texttt{ar} is an optional aspect ratio passed to theme(aspect.ratio = ar). Default is NULL.
  \item
    \texttt{ci} affects the error-bar style. Use 0 for \(\pm 1\) SD error bars, or one of c(0.90, 0.95, 0.975, 0.99) for normal-approximation confidence intervals. Default is 0.95.
  \item
    \texttt{bins} is an integer; number of bins for the histogram. Default is 30.
  \item
    \texttt{numeric\_bins} is an integer; number of quantile bins for numeric covariates. Default is 5.
  \item
    \texttt{by\_numeric\_bins} is an integer; number of quantile bins to use for the second variable in a pair when it is numeric. Default is 3.
  \item
    \texttt{grid} is logical; if TRUE, also return paginated cowplot grids of plots. Default is TRUE.
  \item
    \texttt{top} is an optional integer. If provided, only the top \texttt{top} covariates (by score\_wmae) are included in plots\_main/pages\_main. Rankings are still computed for all covariates.
  \item
    \texttt{ft} is logical; if TRUE (default), return ranking tables as flextable objects.
  \end{itemize}
\end{itemize}

A histogram is always produced, which shows the distribution of predicted uplift across customers. It is saved as \texttt{\$hist}.

\begin{Shaded}
\begin{Highlighting}[]
\NormalTok{lift\_out }\OtherTok{\textless{}{-}} \FunctionTok{easy\_liftplots}\NormalTok{(visit\_uplift, }\AttributeTok{vars =} \StringTok{"all"}\NormalTok{, }\AttributeTok{ci =} \FloatTok{0.99}\NormalTok{)}
\NormalTok{lift\_out}\SpecialCharTok{$}\NormalTok{hist}
\end{Highlighting}
\end{Shaded}

\pandocbounded{\includegraphics[keepaspectratio]{MKT4320_R_Tutorial_files/figure-latex/liftplots-1.pdf}}

The main lift plots are saved in \texttt{\$plots\_main}. Lift-by-covariate plots display how average uplift varies across customer segments.

\begin{Shaded}
\begin{Highlighting}[]
\NormalTok{lift\_out}\SpecialCharTok{$}\NormalTok{plots\_main}
\end{Highlighting}
\end{Shaded}

\begin{verbatim}
$recency
\end{verbatim}

\pandocbounded{\includegraphics[keepaspectratio]{MKT4320_R_Tutorial_files/figure-latex/lift-by-covariate-1.pdf}}

\begin{verbatim}

$history
\end{verbatim}

\pandocbounded{\includegraphics[keepaspectratio]{MKT4320_R_Tutorial_files/figure-latex/lift-by-covariate-2.pdf}}

\begin{verbatim}

$zip
\end{verbatim}

\pandocbounded{\includegraphics[keepaspectratio]{MKT4320_R_Tutorial_files/figure-latex/lift-by-covariate-3.pdf}}

\begin{verbatim}

$womens
\end{verbatim}

\pandocbounded{\includegraphics[keepaspectratio]{MKT4320_R_Tutorial_files/figure-latex/lift-by-covariate-4.pdf}}

The \texttt{pairs} option is exteremely valuable when known interactions are includedin the model. The option can also be useful to help identify if an interaction may be warranted. If \texttt{pairs} is provided, the plots are saved in \texttt{plots\_pairs}.

\begin{Shaded}
\begin{Highlighting}[]
\NormalTok{lift\_out\_pairs }\OtherTok{\textless{}{-}} \FunctionTok{easy\_liftplots}\NormalTok{(visit\_uplift, }\AttributeTok{vars =} \StringTok{"all"}\NormalTok{, }
                                 \AttributeTok{pairs =} \FunctionTok{list}\NormalTok{(}\FunctionTok{c}\NormalTok{(}\StringTok{"recency"}\NormalTok{,}\StringTok{"zip"}\NormalTok{), }
                                              \FunctionTok{c}\NormalTok{(}\StringTok{"history"}\NormalTok{,}\StringTok{"womens"}\NormalTok{)), }
                                 \AttributeTok{ci =} \FloatTok{0.99}\NormalTok{)}
\NormalTok{lift\_out\_pairs}\SpecialCharTok{$}\NormalTok{plots\_pairs}
\end{Highlighting}
\end{Shaded}

\begin{verbatim}
$`recency × zip`
\end{verbatim}

\pandocbounded{\includegraphics[keepaspectratio]{MKT4320_R_Tutorial_files/figure-latex/lift-pairs-1.pdf}}

\begin{verbatim}

$`history × womens`
\end{verbatim}

\pandocbounded{\includegraphics[keepaspectratio]{MKT4320_R_Tutorial_files/figure-latex/lift-pairs-2.pdf}}

\begin{center}\rule{0.5\linewidth}{0.5pt}\end{center}

\section{From Analysis to Action}\label{from-analysis-to-action}

Uplift modeling enables smarter targeting strategies:

\begin{itemize}
\tightlist
\item
  Send promotions only to customers with positive uplift.
\item
  Prioritize customers in the highest uplift deciles.
\item
  Avoid over-targeting customers unlikely to respond.
\end{itemize}

These strategies can improve campaign profitability and customer experience.

\begin{center}\rule{0.5\linewidth}{0.5pt}\end{center}

\section{Summary}\label{summary-1}

In this chapter, you learned how to:

\begin{itemize}
\tightlist
\item
  validate randomization in A/B tests,
\item
  estimate average treatment effects,
\item
  move beyond averages using uplift modeling,
\item
  interpret uplift diagnostics for targeting decisions.
\end{itemize}

A/B testing answers \emph{whether} a campaign works. Uplift modeling answers \emph{for whom} it works.

\begin{center}\rule{0.5\linewidth}{0.5pt}\end{center}

\section{What's Next}\label{whats-next-12}

In many real-world marketing problems, managers face a different challenge: customers are not choosing between respond and not respond, but among multiple competing alternatives. Examples include:

\begin{itemize}
\tightlist
\item
  Which brand a customer purchases
\item
  Which product variant is selected
\item
  Which service tier is chosen
\end{itemize}

In the next chapter, we introduce standard multinomial logistic regression, a workhorse model for analyzing and predicting choice among more than two options. You will learn how to:

\begin{itemize}
\tightlist
\item
  Model customer choice across multiple alternatives
\item
  Interpret coefficients and predicted choice probabilities
\item
  Evaluate model fit and classification performance
\item
  Use multinomial logit models for applied marketing decisions
\end{itemize}

This next step shifts our focus from experimental treatment effects to choice modeling, setting the foundation for more advanced models of consumer decision-making used throughout marketing analytics and research.

\chapter{Standard Multinomial Logit Models}\label{standard-multinomial-logit-models}

\section{Introduction to Multinomial Choice in Marketing}\label{introduction-to-multinomial-choice-in-marketing}

Many marketing decisions involve choices among more than two discrete
alternatives. Consumers may choose among competing brands, subscription
plans, service providers, or product variants. When the outcome variable
has more than two unordered categories, linear regression and binary
logistic regression are no longer appropriate.

The \textbf{standard multinomial logit (MNL)} model is the most common baseline
model for analyzing and predicting such outcomes. In marketing analytics,
it is widely used for brand choice, product selection, and competitive
response analysis. The focus of this chapter is on \textbf{applied interpretation}
rather than mathematical derivation.

\begin{center}\rule{0.5\linewidth}{0.5pt}\end{center}

\section{\texorpdfstring{The \texttt{bfast} Dataset}{The bfast Dataset}}\label{the-bfast-dataset}

In this chapter, we use the \texttt{bfast} dataset, which contains data on
breakfast food preferences. Each observation represents a consumer choice
occasion in which one type of food was selected from a competitive set.

The outcome variable records the chosen type, while predictor variables
capture marketing mix and consumer characteristics that may influence
choice. Our core marketing question is:
Which factors increase or decrease the probability that a consumer
chooses a particular fast-food brand?

\begin{center}\rule{0.5\linewidth}{0.5pt}\end{center}

\section{Training and Test Samples}\label{training-and-test-samples}

To evaluate predictive performance, we split the data into training and
test samples. As with binary logistic regression, we use the \texttt{splitsample()} function from the \texttt{MKT4320BGSU} package. This function creates reproducible partitions and supports stratification on the outcome variable.

Usage:

\begin{itemize}
\tightlist
\item
  \texttt{splitsample(data,\ outcome\ =\ NULL,\ group\ =\ NULL,\ choice\ =\ NULL,\ alt\ =\ NULL,}\strut \\
  \texttt{p\ =\ 0.75,\ seed\ =\ 4320)}
\item
  where:

  \begin{itemize}
  \tightlist
  \item
    \texttt{data} is the data frame to split.
  \item
    \texttt{outcome} is the outcome variable in quotes used for stratification. Required when group is \texttt{NULL}. Optional when group is provided. For standard MNL, it is required.
  \item
    \texttt{group} is NOT USED FOR STANDARD MNL
  \item
    \texttt{choice} is NOT USED FOR STANDARD MNL
  \item
    \texttt{alt} is NOT USED FOR STANDARD MNL
  \item
    \texttt{p} is the proportion of observations to place in the training set. Must be strictly between 0 and 1. Default is 0.75.
  \item
    \texttt{seed} is the random seed for reproducibility. Default is 4320.
  \end{itemize}
\end{itemize}

Below, we create are training and test samples. We also check the outcome variable in the two samples to ensure they are similar proportions in each.

\begin{Shaded}
\begin{Highlighting}[]
\NormalTok{sp }\OtherTok{\textless{}{-}} \FunctionTok{splitsample}\NormalTok{(}\AttributeTok{data =}\NormalTok{ bfast, }\AttributeTok{outcome =} \StringTok{"bfast"}\NormalTok{)}
\NormalTok{train }\OtherTok{\textless{}{-}}\NormalTok{ sp}\SpecialCharTok{$}\NormalTok{train}
\NormalTok{test  }\OtherTok{\textless{}{-}}\NormalTok{ sp}\SpecialCharTok{$}\NormalTok{test}

\FunctionTok{proportions}\NormalTok{(}\FunctionTok{table}\NormalTok{(train}\SpecialCharTok{$}\NormalTok{bfast))}
\end{Highlighting}
\end{Shaded}

\begin{verbatim}

   Cereal       Bar   Oatmeal 
0.3851964 0.2628399 0.3519637 
\end{verbatim}

\begin{Shaded}
\begin{Highlighting}[]
\FunctionTok{proportions}\NormalTok{(}\FunctionTok{table}\NormalTok{(test}\SpecialCharTok{$}\NormalTok{bfast))}
\end{Highlighting}
\end{Shaded}

\begin{verbatim}

   Cereal       Bar   Oatmeal 
0.3853211 0.2614679 0.3532110 
\end{verbatim}

\begin{center}\rule{0.5\linewidth}{0.5pt}\end{center}

\section{Estimating a Standard Multinomial Logit Model}\label{estimating-a-standard-multinomial-logit-model}

We estimate the standard multinomial logit model using
\texttt{nnet::multinom()}. It is important to include \texttt{model\ =\ TRUE} so that
model diagnostics and classification results can be computed later.

Using the \texttt{summary()} function in base R will provide the raw coefficients from the model. The estimated coefficients describe how each predictor affects the relative log-odds of choosing one product versus the reference product.

\begin{Shaded}
\begin{Highlighting}[]
\FunctionTok{library}\NormalTok{(nnet)}
\NormalTok{mnl\_fit }\OtherTok{\textless{}{-}} \FunctionTok{multinom}\NormalTok{(bfast }\SpecialCharTok{\textasciitilde{}}\NormalTok{ gender }\SpecialCharTok{+}\NormalTok{ marital }\SpecialCharTok{+}\NormalTok{ lifestyle }\SpecialCharTok{+}\NormalTok{ age, }
                    \AttributeTok{model =} \ConstantTok{TRUE}\NormalTok{, }\AttributeTok{data=}\NormalTok{train)}
\end{Highlighting}
\end{Shaded}

\begin{verbatim}
# weights:  18 (10 variable)
initial  value 727.281335 
iter  10 value 579.014122
final  value 574.997631 
converged
\end{verbatim}

\begin{Shaded}
\begin{Highlighting}[]
\FunctionTok{summary}\NormalTok{(mnl\_fit)}
\end{Highlighting}
\end{Shaded}

\begin{verbatim}
Call:
multinom(formula = bfast ~ gender + marital + lifestyle + age, 
    data = train, model = TRUE)

Coefficients:
        (Intercept)  genderMale maritalUnmarried lifestyleInactive         age
Bar       0.8832457 -0.21298963        0.6126977        -0.7865772 -0.02532866
Oatmeal  -4.4920408 -0.02262325       -0.3897362         0.3187473  0.07996475

Std. Errors:
        (Intercept) genderMale maritalUnmarried lifestyleInactive         age
Bar       0.3256994  0.2064320        0.2123832         0.2090460 0.006655803
Oatmeal   0.4596750  0.2094666        0.2366511         0.2156992 0.007755708

Residual Deviance: 1149.995 
AIC: 1169.995 
\end{verbatim}

\begin{center}\rule{0.5\linewidth}{0.5pt}\end{center}

\section{Evaluating Model Fit}\label{evaluating-model-fit}

Raw coefficients alone do not indicate whether a model performs well.
We use \texttt{eval\_std\_mnl()} from the \texttt{MKT4320BGSU} package to compute model-fit statistics and diagnostics.

Usage:

\begin{itemize}
\tightlist
\item
  \texttt{eval\_std\_mnl(OBJ,\ exp\ =\ FALSE,\ digits\ =\ 4,\ ft\ =\ FALSE,\ newdata\ =\ NULL,}\strut \\
  \texttt{label\_model\ =\ "Model\ data",\ label\_newdata\ =\ "New\ data",\ class\_digits\ =\ 3)}
\item
  where:

  \begin{itemize}
  \tightlist
  \item
    \texttt{model} is a fitted \texttt{multinom} model.
  \item
    \texttt{exp} is logical; if TRUE, return relative risk ratios (exp(beta)). If FALSE, return log-odds coefficients (default = FALSE).
  \item
    \texttt{digits} is an integer; number of decimals used to round coefficient and model-fit results (default = 4).
  \item
    \texttt{ft} is logical; if TRUE, return coefficient and classification tables as flextable objects (default = FALSE).
  \item
    \texttt{newdata} is an optional data frame for an additional classification matrix (e.g., a holdout or test set). If NULL, only the model-data classification is produced.
  \item
    \texttt{label\_model} is a character string; label for the model-data classification output (default = ``Model data'').
  \item
    \texttt{label\_newdata} is a character string; label for the newdata classification output (default = ``New data'').
  \item
    \texttt{class\_digits} is an integer; number of decimals used to round classification statistics (default = 3).
  \end{itemize}
\end{itemize}

Key outputs include:

\begin{itemize}
\tightlist
\item
  A likelihood-ratio test comparing the fitted model to an intercept-only model
\item
  McFadden's pseudo R-squared
\item
  Classification accuracy and diagnostics
\end{itemize}

In applied marketing contexts, even modest pseudo R-squared values can
indicate meaningful improvements over random choice.

\begin{Shaded}
\begin{Highlighting}[]
\NormalTok{mnl\_eval }\OtherTok{\textless{}{-}} \FunctionTok{eval\_std\_mnl}\NormalTok{(}\AttributeTok{model =}\NormalTok{ mnl\_fit, }\AttributeTok{newdata =}\NormalTok{ test, }\AttributeTok{ft =} \ConstantTok{TRUE}\NormalTok{)}
\NormalTok{mnl\_eval}
\end{Highlighting}
\end{Shaded}

\global\setlength{\Oldarrayrulewidth}{\arrayrulewidth}

\global\setlength{\Oldtabcolsep}{\tabcolsep}

\setlength{\tabcolsep}{2pt}

\renewcommand*{\arraystretch}{1.5}



\providecommand{\ascline}[3]{\noalign{\global\arrayrulewidth #1}\arrayrulecolor[HTML]{#2}\cline{#3}}

\begin{longtable}[c]{|p{0.86in}|p{1.46in}|p{0.82in}|p{0.85in}|p{0.84in}|p{0.78in}}



\ascline{1.5pt}{666666}{1-6}

\multicolumn{6}{>{\raggedright}m{\dimexpr 5.61in+10\tabcolsep}}{\textcolor[HTML]{000000}{\fontsize{11}{11}\selectfont{\global\setmainfont{Arial}{LR\ chi2\ (8)\ =\ 288.1568;\ p\ <\ 0.0001}}}} \\

\ascline{1.5pt}{666666}{1-6}



\multicolumn{6}{>{\raggedright}m{\dimexpr 5.61in+10\tabcolsep}}{\textcolor[HTML]{000000}{\fontsize{11}{11}\selectfont{\global\setmainfont{Arial}{McFadden's\ Pseudo\ R-square\ =\ 0.2004}}}} \\

\ascline{1.5pt}{666666}{1-6}



\multicolumn{1}{>{\raggedright}m{\dimexpr 0.86in+0\tabcolsep}}{\textcolor[HTML]{000000}{\fontsize{11}{11}\selectfont{\global\setmainfont{Arial}{y.level}}}} & \multicolumn{1}{>{\raggedright}m{\dimexpr 1.46in+0\tabcolsep}}{\textcolor[HTML]{000000}{\fontsize{11}{11}\selectfont{\global\setmainfont{Arial}{term}}}} & \multicolumn{1}{>{\raggedleft}m{\dimexpr 0.82in+0\tabcolsep}}{\textcolor[HTML]{000000}{\fontsize{11}{11}\selectfont{\global\setmainfont{Arial}{logodds}}}} & \multicolumn{1}{>{\raggedleft}m{\dimexpr 0.85in+0\tabcolsep}}{\textcolor[HTML]{000000}{\fontsize{11}{11}\selectfont{\global\setmainfont{Arial}{std.error}}}} & \multicolumn{1}{>{\raggedleft}m{\dimexpr 0.84in+0\tabcolsep}}{\textcolor[HTML]{000000}{\fontsize{11}{11}\selectfont{\global\setmainfont{Arial}{statistic}}}} & \multicolumn{1}{>{\raggedleft}m{\dimexpr 0.78in+0\tabcolsep}}{\textcolor[HTML]{000000}{\fontsize{11}{11}\selectfont{\global\setmainfont{Arial}{p.value}}}} \\

\ascline{1.5pt}{666666}{1-6}\endfirsthead 

\ascline{1.5pt}{666666}{1-6}

\multicolumn{6}{>{\raggedright}m{\dimexpr 5.61in+10\tabcolsep}}{\textcolor[HTML]{000000}{\fontsize{11}{11}\selectfont{\global\setmainfont{Arial}{LR\ chi2\ (8)\ =\ 288.1568;\ p\ <\ 0.0001}}}} \\

\ascline{1.5pt}{666666}{1-6}



\multicolumn{6}{>{\raggedright}m{\dimexpr 5.61in+10\tabcolsep}}{\textcolor[HTML]{000000}{\fontsize{11}{11}\selectfont{\global\setmainfont{Arial}{McFadden's\ Pseudo\ R-square\ =\ 0.2004}}}} \\

\ascline{1.5pt}{666666}{1-6}



\multicolumn{1}{>{\raggedright}m{\dimexpr 0.86in+0\tabcolsep}}{\textcolor[HTML]{000000}{\fontsize{11}{11}\selectfont{\global\setmainfont{Arial}{y.level}}}} & \multicolumn{1}{>{\raggedright}m{\dimexpr 1.46in+0\tabcolsep}}{\textcolor[HTML]{000000}{\fontsize{11}{11}\selectfont{\global\setmainfont{Arial}{term}}}} & \multicolumn{1}{>{\raggedleft}m{\dimexpr 0.82in+0\tabcolsep}}{\textcolor[HTML]{000000}{\fontsize{11}{11}\selectfont{\global\setmainfont{Arial}{logodds}}}} & \multicolumn{1}{>{\raggedleft}m{\dimexpr 0.85in+0\tabcolsep}}{\textcolor[HTML]{000000}{\fontsize{11}{11}\selectfont{\global\setmainfont{Arial}{std.error}}}} & \multicolumn{1}{>{\raggedleft}m{\dimexpr 0.84in+0\tabcolsep}}{\textcolor[HTML]{000000}{\fontsize{11}{11}\selectfont{\global\setmainfont{Arial}{statistic}}}} & \multicolumn{1}{>{\raggedleft}m{\dimexpr 0.78in+0\tabcolsep}}{\textcolor[HTML]{000000}{\fontsize{11}{11}\selectfont{\global\setmainfont{Arial}{p.value}}}} \\

\ascline{1.5pt}{666666}{1-6}\endhead



\multicolumn{1}{>{\raggedright}m{\dimexpr 0.86in+0\tabcolsep}}{\textcolor[HTML]{000000}{\fontsize{11}{11}\selectfont{\global\setmainfont{Arial}{Bar}}}} & \multicolumn{1}{>{\raggedright}m{\dimexpr 1.46in+0\tabcolsep}}{\textcolor[HTML]{000000}{\fontsize{11}{11}\selectfont{\global\setmainfont{Arial}{(Intercept)}}}} & \multicolumn{1}{>{\raggedleft}m{\dimexpr 0.82in+0\tabcolsep}}{\textcolor[HTML]{000000}{\fontsize{11}{11}\selectfont{\global\setmainfont{Arial}{0.8832}}}} & \multicolumn{1}{>{\raggedleft}m{\dimexpr 0.85in+0\tabcolsep}}{\textcolor[HTML]{000000}{\fontsize{11}{11}\selectfont{\global\setmainfont{Arial}{0.3257}}}} & \multicolumn{1}{>{\raggedleft}m{\dimexpr 0.84in+0\tabcolsep}}{\textcolor[HTML]{000000}{\fontsize{11}{11}\selectfont{\global\setmainfont{Arial}{2.7118}}}} & \multicolumn{1}{>{\raggedleft}m{\dimexpr 0.78in+0\tabcolsep}}{\textcolor[HTML]{000000}{\fontsize{11}{11}\selectfont{\global\setmainfont{Arial}{0.0067}}}} \\





\multicolumn{1}{>{\raggedright}m{\dimexpr 0.86in+0\tabcolsep}}{\textcolor[HTML]{000000}{\fontsize{11}{11}\selectfont{\global\setmainfont{Arial}{Bar}}}} & \multicolumn{1}{>{\raggedright}m{\dimexpr 1.46in+0\tabcolsep}}{\textcolor[HTML]{000000}{\fontsize{11}{11}\selectfont{\global\setmainfont{Arial}{genderMale}}}} & \multicolumn{1}{>{\raggedleft}m{\dimexpr 0.82in+0\tabcolsep}}{\textcolor[HTML]{000000}{\fontsize{11}{11}\selectfont{\global\setmainfont{Arial}{-0.2130}}}} & \multicolumn{1}{>{\raggedleft}m{\dimexpr 0.85in+0\tabcolsep}}{\textcolor[HTML]{000000}{\fontsize{11}{11}\selectfont{\global\setmainfont{Arial}{0.2064}}}} & \multicolumn{1}{>{\raggedleft}m{\dimexpr 0.84in+0\tabcolsep}}{\textcolor[HTML]{000000}{\fontsize{11}{11}\selectfont{\global\setmainfont{Arial}{-1.0318}}}} & \multicolumn{1}{>{\raggedleft}m{\dimexpr 0.78in+0\tabcolsep}}{\textcolor[HTML]{000000}{\fontsize{11}{11}\selectfont{\global\setmainfont{Arial}{0.3022}}}} \\





\multicolumn{1}{>{\raggedright}m{\dimexpr 0.86in+0\tabcolsep}}{\textcolor[HTML]{000000}{\fontsize{11}{11}\selectfont{\global\setmainfont{Arial}{Bar}}}} & \multicolumn{1}{>{\raggedright}m{\dimexpr 1.46in+0\tabcolsep}}{\textcolor[HTML]{000000}{\fontsize{11}{11}\selectfont{\global\setmainfont{Arial}{maritalUnmarried}}}} & \multicolumn{1}{>{\raggedleft}m{\dimexpr 0.82in+0\tabcolsep}}{\textcolor[HTML]{000000}{\fontsize{11}{11}\selectfont{\global\setmainfont{Arial}{0.6127}}}} & \multicolumn{1}{>{\raggedleft}m{\dimexpr 0.85in+0\tabcolsep}}{\textcolor[HTML]{000000}{\fontsize{11}{11}\selectfont{\global\setmainfont{Arial}{0.2124}}}} & \multicolumn{1}{>{\raggedleft}m{\dimexpr 0.84in+0\tabcolsep}}{\textcolor[HTML]{000000}{\fontsize{11}{11}\selectfont{\global\setmainfont{Arial}{2.8849}}}} & \multicolumn{1}{>{\raggedleft}m{\dimexpr 0.78in+0\tabcolsep}}{\textcolor[HTML]{000000}{\fontsize{11}{11}\selectfont{\global\setmainfont{Arial}{0.0039}}}} \\





\multicolumn{1}{>{\raggedright}m{\dimexpr 0.86in+0\tabcolsep}}{\textcolor[HTML]{000000}{\fontsize{11}{11}\selectfont{\global\setmainfont{Arial}{Bar}}}} & \multicolumn{1}{>{\raggedright}m{\dimexpr 1.46in+0\tabcolsep}}{\textcolor[HTML]{000000}{\fontsize{11}{11}\selectfont{\global\setmainfont{Arial}{lifestyleInactive}}}} & \multicolumn{1}{>{\raggedleft}m{\dimexpr 0.82in+0\tabcolsep}}{\textcolor[HTML]{000000}{\fontsize{11}{11}\selectfont{\global\setmainfont{Arial}{-0.7866}}}} & \multicolumn{1}{>{\raggedleft}m{\dimexpr 0.85in+0\tabcolsep}}{\textcolor[HTML]{000000}{\fontsize{11}{11}\selectfont{\global\setmainfont{Arial}{0.2090}}}} & \multicolumn{1}{>{\raggedleft}m{\dimexpr 0.84in+0\tabcolsep}}{\textcolor[HTML]{000000}{\fontsize{11}{11}\selectfont{\global\setmainfont{Arial}{-3.7627}}}} & \multicolumn{1}{>{\raggedleft}m{\dimexpr 0.78in+0\tabcolsep}}{\textcolor[HTML]{000000}{\fontsize{11}{11}\selectfont{\global\setmainfont{Arial}{0.0002}}}} \\





\multicolumn{1}{>{\raggedright}m{\dimexpr 0.86in+0\tabcolsep}}{\textcolor[HTML]{000000}{\fontsize{11}{11}\selectfont{\global\setmainfont{Arial}{Bar}}}} & \multicolumn{1}{>{\raggedright}m{\dimexpr 1.46in+0\tabcolsep}}{\textcolor[HTML]{000000}{\fontsize{11}{11}\selectfont{\global\setmainfont{Arial}{age}}}} & \multicolumn{1}{>{\raggedleft}m{\dimexpr 0.82in+0\tabcolsep}}{\textcolor[HTML]{000000}{\fontsize{11}{11}\selectfont{\global\setmainfont{Arial}{-0.0253}}}} & \multicolumn{1}{>{\raggedleft}m{\dimexpr 0.85in+0\tabcolsep}}{\textcolor[HTML]{000000}{\fontsize{11}{11}\selectfont{\global\setmainfont{Arial}{0.0067}}}} & \multicolumn{1}{>{\raggedleft}m{\dimexpr 0.84in+0\tabcolsep}}{\textcolor[HTML]{000000}{\fontsize{11}{11}\selectfont{\global\setmainfont{Arial}{-3.8055}}}} & \multicolumn{1}{>{\raggedleft}m{\dimexpr 0.78in+0\tabcolsep}}{\textcolor[HTML]{000000}{\fontsize{11}{11}\selectfont{\global\setmainfont{Arial}{0.0001}}}} \\





\multicolumn{1}{>{\raggedright}m{\dimexpr 0.86in+0\tabcolsep}}{\textcolor[HTML]{000000}{\fontsize{11}{11}\selectfont{\global\setmainfont{Arial}{Oatmeal}}}} & \multicolumn{1}{>{\raggedright}m{\dimexpr 1.46in+0\tabcolsep}}{\textcolor[HTML]{000000}{\fontsize{11}{11}\selectfont{\global\setmainfont{Arial}{(Intercept)}}}} & \multicolumn{1}{>{\raggedleft}m{\dimexpr 0.82in+0\tabcolsep}}{\textcolor[HTML]{000000}{\fontsize{11}{11}\selectfont{\global\setmainfont{Arial}{-4.4920}}}} & \multicolumn{1}{>{\raggedleft}m{\dimexpr 0.85in+0\tabcolsep}}{\textcolor[HTML]{000000}{\fontsize{11}{11}\selectfont{\global\setmainfont{Arial}{0.4597}}}} & \multicolumn{1}{>{\raggedleft}m{\dimexpr 0.84in+0\tabcolsep}}{\textcolor[HTML]{000000}{\fontsize{11}{11}\selectfont{\global\setmainfont{Arial}{-9.7722}}}} & \multicolumn{1}{>{\raggedleft}m{\dimexpr 0.78in+0\tabcolsep}}{\textcolor[HTML]{000000}{\fontsize{11}{11}\selectfont{\global\setmainfont{Arial}{0.0000}}}} \\





\multicolumn{1}{>{\raggedright}m{\dimexpr 0.86in+0\tabcolsep}}{\textcolor[HTML]{000000}{\fontsize{11}{11}\selectfont{\global\setmainfont{Arial}{Oatmeal}}}} & \multicolumn{1}{>{\raggedright}m{\dimexpr 1.46in+0\tabcolsep}}{\textcolor[HTML]{000000}{\fontsize{11}{11}\selectfont{\global\setmainfont{Arial}{genderMale}}}} & \multicolumn{1}{>{\raggedleft}m{\dimexpr 0.82in+0\tabcolsep}}{\textcolor[HTML]{000000}{\fontsize{11}{11}\selectfont{\global\setmainfont{Arial}{-0.0226}}}} & \multicolumn{1}{>{\raggedleft}m{\dimexpr 0.85in+0\tabcolsep}}{\textcolor[HTML]{000000}{\fontsize{11}{11}\selectfont{\global\setmainfont{Arial}{0.2095}}}} & \multicolumn{1}{>{\raggedleft}m{\dimexpr 0.84in+0\tabcolsep}}{\textcolor[HTML]{000000}{\fontsize{11}{11}\selectfont{\global\setmainfont{Arial}{-0.1080}}}} & \multicolumn{1}{>{\raggedleft}m{\dimexpr 0.78in+0\tabcolsep}}{\textcolor[HTML]{000000}{\fontsize{11}{11}\selectfont{\global\setmainfont{Arial}{0.9140}}}} \\





\multicolumn{1}{>{\raggedright}m{\dimexpr 0.86in+0\tabcolsep}}{\textcolor[HTML]{000000}{\fontsize{11}{11}\selectfont{\global\setmainfont{Arial}{Oatmeal}}}} & \multicolumn{1}{>{\raggedright}m{\dimexpr 1.46in+0\tabcolsep}}{\textcolor[HTML]{000000}{\fontsize{11}{11}\selectfont{\global\setmainfont{Arial}{maritalUnmarried}}}} & \multicolumn{1}{>{\raggedleft}m{\dimexpr 0.82in+0\tabcolsep}}{\textcolor[HTML]{000000}{\fontsize{11}{11}\selectfont{\global\setmainfont{Arial}{-0.3897}}}} & \multicolumn{1}{>{\raggedleft}m{\dimexpr 0.85in+0\tabcolsep}}{\textcolor[HTML]{000000}{\fontsize{11}{11}\selectfont{\global\setmainfont{Arial}{0.2367}}}} & \multicolumn{1}{>{\raggedleft}m{\dimexpr 0.84in+0\tabcolsep}}{\textcolor[HTML]{000000}{\fontsize{11}{11}\selectfont{\global\setmainfont{Arial}{-1.6469}}}} & \multicolumn{1}{>{\raggedleft}m{\dimexpr 0.78in+0\tabcolsep}}{\textcolor[HTML]{000000}{\fontsize{11}{11}\selectfont{\global\setmainfont{Arial}{0.0996}}}} \\





\multicolumn{1}{>{\raggedright}m{\dimexpr 0.86in+0\tabcolsep}}{\textcolor[HTML]{000000}{\fontsize{11}{11}\selectfont{\global\setmainfont{Arial}{Oatmeal}}}} & \multicolumn{1}{>{\raggedright}m{\dimexpr 1.46in+0\tabcolsep}}{\textcolor[HTML]{000000}{\fontsize{11}{11}\selectfont{\global\setmainfont{Arial}{lifestyleInactive}}}} & \multicolumn{1}{>{\raggedleft}m{\dimexpr 0.82in+0\tabcolsep}}{\textcolor[HTML]{000000}{\fontsize{11}{11}\selectfont{\global\setmainfont{Arial}{0.3187}}}} & \multicolumn{1}{>{\raggedleft}m{\dimexpr 0.85in+0\tabcolsep}}{\textcolor[HTML]{000000}{\fontsize{11}{11}\selectfont{\global\setmainfont{Arial}{0.2157}}}} & \multicolumn{1}{>{\raggedleft}m{\dimexpr 0.84in+0\tabcolsep}}{\textcolor[HTML]{000000}{\fontsize{11}{11}\selectfont{\global\setmainfont{Arial}{1.4777}}}} & \multicolumn{1}{>{\raggedleft}m{\dimexpr 0.78in+0\tabcolsep}}{\textcolor[HTML]{000000}{\fontsize{11}{11}\selectfont{\global\setmainfont{Arial}{0.1395}}}} \\





\multicolumn{1}{>{\raggedright}m{\dimexpr 0.86in+0\tabcolsep}}{\textcolor[HTML]{000000}{\fontsize{11}{11}\selectfont{\global\setmainfont{Arial}{Oatmeal}}}} & \multicolumn{1}{>{\raggedright}m{\dimexpr 1.46in+0\tabcolsep}}{\textcolor[HTML]{000000}{\fontsize{11}{11}\selectfont{\global\setmainfont{Arial}{age}}}} & \multicolumn{1}{>{\raggedleft}m{\dimexpr 0.82in+0\tabcolsep}}{\textcolor[HTML]{000000}{\fontsize{11}{11}\selectfont{\global\setmainfont{Arial}{0.0800}}}} & \multicolumn{1}{>{\raggedleft}m{\dimexpr 0.85in+0\tabcolsep}}{\textcolor[HTML]{000000}{\fontsize{11}{11}\selectfont{\global\setmainfont{Arial}{0.0078}}}} & \multicolumn{1}{>{\raggedleft}m{\dimexpr 0.84in+0\tabcolsep}}{\textcolor[HTML]{000000}{\fontsize{11}{11}\selectfont{\global\setmainfont{Arial}{10.3104}}}} & \multicolumn{1}{>{\raggedleft}m{\dimexpr 0.78in+0\tabcolsep}}{\textcolor[HTML]{000000}{\fontsize{11}{11}\selectfont{\global\setmainfont{Arial}{0.0000}}}} \\

\ascline{1.5pt}{666666}{1-6}



\end{longtable}



\arrayrulecolor[HTML]{000000}

\global\setlength{\arrayrulewidth}{\Oldarrayrulewidth}

\global\setlength{\tabcolsep}{\Oldtabcolsep}

\renewcommand*{\arraystretch}{1}

\global\setlength{\Oldarrayrulewidth}{\arrayrulewidth}

\global\setlength{\Oldtabcolsep}{\tabcolsep}

\setlength{\tabcolsep}{2pt}

\renewcommand*{\arraystretch}{1.5}



\providecommand{\ascline}[3]{\noalign{\global\arrayrulewidth #1}\arrayrulecolor[HTML]{#2}\cline{#3}}

\begin{longtable}[c]{|p{1.57in}|p{0.74in}|p{0.67in}|p{0.86in}|p{0.62in}}



\ascline{1.5pt}{666666}{1-5}

\multicolumn{5}{>{\raggedright}m{\dimexpr 4.46in+8\tabcolsep}}{\textcolor[HTML]{000000}{\fontsize{11}{11}\selectfont{\global\setmainfont{Arial}{Classification\ Matrix\ -\ Model\ data}}}} \\

\ascline{1.5pt}{666666}{1-5}



\multicolumn{5}{>{\raggedright}m{\dimexpr 4.46in+8\tabcolsep}}{\textcolor[HTML]{000000}{\fontsize{11}{11}\selectfont{\global\setmainfont{Arial}{Accuracy\ =\ 0.562}}}} \\

\ascline{1.5pt}{666666}{1-5}



\multicolumn{5}{>{\raggedright}m{\dimexpr 4.46in+8\tabcolsep}}{\textcolor[HTML]{000000}{\fontsize{11}{11}\selectfont{\global\setmainfont{Arial}{PCC\ =\ 0.341}}}} \\

\ascline{1.5pt}{666666}{1-5}



\multicolumn{1}{>{\raggedright}m{\dimexpr 1.57in+0\tabcolsep}}{\textcolor[HTML]{000000}{\fontsize{11}{11}\selectfont{\global\setmainfont{Arial}{}}}} & \multicolumn{3}{>{\raggedright}m{\dimexpr 2.27in+4\tabcolsep}}{\textcolor[HTML]{000000}{\fontsize{11}{11}\selectfont{\global\setmainfont{Arial}{Reference}}}} & \multicolumn{1}{>{\raggedright}m{\dimexpr 0.62in+0\tabcolsep}}{\textcolor[HTML]{000000}{\fontsize{11}{11}\selectfont{\global\setmainfont{Arial}{}}}} \\

\ascline{1.5pt}{666666}{1-5}



\multicolumn{1}{>{\raggedright}m{\dimexpr 1.57in+0\tabcolsep}}{\textcolor[HTML]{000000}{\fontsize{11}{11}\selectfont{\global\setmainfont{Arial}{Predicted}}}} & \multicolumn{1}{>{\raggedright}m{\dimexpr 0.74in+0\tabcolsep}}{\textcolor[HTML]{000000}{\fontsize{11}{11}\selectfont{\global\setmainfont{Arial}{Cereal}}}} & \multicolumn{1}{>{\raggedright}m{\dimexpr 0.67in+0\tabcolsep}}{\textcolor[HTML]{000000}{\fontsize{11}{11}\selectfont{\global\setmainfont{Arial}{Bar}}}} & \multicolumn{1}{>{\raggedright}m{\dimexpr 0.86in+0\tabcolsep}}{\textcolor[HTML]{000000}{\fontsize{11}{11}\selectfont{\global\setmainfont{Arial}{Oatmeal}}}} & \multicolumn{1}{>{\raggedright}m{\dimexpr 0.62in+0\tabcolsep}}{\textcolor[HTML]{000000}{\fontsize{11}{11}\selectfont{\global\setmainfont{Arial}{Total}}}} \\

\ascline{1.5pt}{666666}{1-5}\endfirsthead 

\ascline{1.5pt}{666666}{1-5}

\multicolumn{5}{>{\raggedright}m{\dimexpr 4.46in+8\tabcolsep}}{\textcolor[HTML]{000000}{\fontsize{11}{11}\selectfont{\global\setmainfont{Arial}{Classification\ Matrix\ -\ Model\ data}}}} \\

\ascline{1.5pt}{666666}{1-5}



\multicolumn{5}{>{\raggedright}m{\dimexpr 4.46in+8\tabcolsep}}{\textcolor[HTML]{000000}{\fontsize{11}{11}\selectfont{\global\setmainfont{Arial}{Accuracy\ =\ 0.562}}}} \\

\ascline{1.5pt}{666666}{1-5}



\multicolumn{5}{>{\raggedright}m{\dimexpr 4.46in+8\tabcolsep}}{\textcolor[HTML]{000000}{\fontsize{11}{11}\selectfont{\global\setmainfont{Arial}{PCC\ =\ 0.341}}}} \\

\ascline{1.5pt}{666666}{1-5}



\multicolumn{1}{>{\raggedright}m{\dimexpr 1.57in+0\tabcolsep}}{\textcolor[HTML]{000000}{\fontsize{11}{11}\selectfont{\global\setmainfont{Arial}{}}}} & \multicolumn{3}{>{\raggedright}m{\dimexpr 2.27in+4\tabcolsep}}{\textcolor[HTML]{000000}{\fontsize{11}{11}\selectfont{\global\setmainfont{Arial}{Reference}}}} & \multicolumn{1}{>{\raggedright}m{\dimexpr 0.62in+0\tabcolsep}}{\textcolor[HTML]{000000}{\fontsize{11}{11}\selectfont{\global\setmainfont{Arial}{}}}} \\

\ascline{1.5pt}{666666}{1-5}



\multicolumn{1}{>{\raggedright}m{\dimexpr 1.57in+0\tabcolsep}}{\textcolor[HTML]{000000}{\fontsize{11}{11}\selectfont{\global\setmainfont{Arial}{Predicted}}}} & \multicolumn{1}{>{\raggedright}m{\dimexpr 0.74in+0\tabcolsep}}{\textcolor[HTML]{000000}{\fontsize{11}{11}\selectfont{\global\setmainfont{Arial}{Cereal}}}} & \multicolumn{1}{>{\raggedright}m{\dimexpr 0.67in+0\tabcolsep}}{\textcolor[HTML]{000000}{\fontsize{11}{11}\selectfont{\global\setmainfont{Arial}{Bar}}}} & \multicolumn{1}{>{\raggedright}m{\dimexpr 0.86in+0\tabcolsep}}{\textcolor[HTML]{000000}{\fontsize{11}{11}\selectfont{\global\setmainfont{Arial}{Oatmeal}}}} & \multicolumn{1}{>{\raggedright}m{\dimexpr 0.62in+0\tabcolsep}}{\textcolor[HTML]{000000}{\fontsize{11}{11}\selectfont{\global\setmainfont{Arial}{Total}}}} \\

\ascline{1.5pt}{666666}{1-5}\endhead



\multicolumn{1}{>{\raggedright}m{\dimexpr 1.57in+0\tabcolsep}}{\textcolor[HTML]{000000}{\fontsize{11}{11}\selectfont{\global\setmainfont{Arial}{Cereal}}}} & \multicolumn{1}{>{\raggedright}m{\dimexpr 0.74in+0\tabcolsep}}{\textcolor[HTML]{000000}{\fontsize{11}{11}\selectfont{\global\setmainfont{Arial}{124}}}} & \multicolumn{1}{>{\raggedright}m{\dimexpr 0.67in+0\tabcolsep}}{\textcolor[HTML]{000000}{\fontsize{11}{11}\selectfont{\global\setmainfont{Arial}{85}}}} & \multicolumn{1}{>{\raggedright}m{\dimexpr 0.86in+0\tabcolsep}}{\textcolor[HTML]{000000}{\fontsize{11}{11}\selectfont{\global\setmainfont{Arial}{46}}}} & \multicolumn{1}{>{\raggedright}m{\dimexpr 0.62in+0\tabcolsep}}{\textcolor[HTML]{000000}{\fontsize{11}{11}\selectfont{\global\setmainfont{Arial}{255}}}} \\





\multicolumn{1}{>{\raggedright}m{\dimexpr 1.57in+0\tabcolsep}}{\textcolor[HTML]{000000}{\fontsize{11}{11}\selectfont{\global\setmainfont{Arial}{Bar}}}} & \multicolumn{1}{>{\raggedright}m{\dimexpr 0.74in+0\tabcolsep}}{\textcolor[HTML]{000000}{\fontsize{11}{11}\selectfont{\global\setmainfont{Arial}{52}}}} & \multicolumn{1}{>{\raggedright}m{\dimexpr 0.67in+0\tabcolsep}}{\textcolor[HTML]{000000}{\fontsize{11}{11}\selectfont{\global\setmainfont{Arial}{68}}}} & \multicolumn{1}{>{\raggedright}m{\dimexpr 0.86in+0\tabcolsep}}{\textcolor[HTML]{000000}{\fontsize{11}{11}\selectfont{\global\setmainfont{Arial}{7}}}} & \multicolumn{1}{>{\raggedright}m{\dimexpr 0.62in+0\tabcolsep}}{\textcolor[HTML]{000000}{\fontsize{11}{11}\selectfont{\global\setmainfont{Arial}{127}}}} \\





\multicolumn{1}{>{\raggedright}m{\dimexpr 1.57in+0\tabcolsep}}{\textcolor[HTML]{000000}{\fontsize{11}{11}\selectfont{\global\setmainfont{Arial}{Oatmeal}}}} & \multicolumn{1}{>{\raggedright}m{\dimexpr 0.74in+0\tabcolsep}}{\textcolor[HTML]{000000}{\fontsize{11}{11}\selectfont{\global\setmainfont{Arial}{79}}}} & \multicolumn{1}{>{\raggedright}m{\dimexpr 0.67in+0\tabcolsep}}{\textcolor[HTML]{000000}{\fontsize{11}{11}\selectfont{\global\setmainfont{Arial}{21}}}} & \multicolumn{1}{>{\raggedright}m{\dimexpr 0.86in+0\tabcolsep}}{\textcolor[HTML]{000000}{\fontsize{11}{11}\selectfont{\global\setmainfont{Arial}{180}}}} & \multicolumn{1}{>{\raggedright}m{\dimexpr 0.62in+0\tabcolsep}}{\textcolor[HTML]{000000}{\fontsize{11}{11}\selectfont{\global\setmainfont{Arial}{280}}}} \\





\multicolumn{1}{>{\raggedright}m{\dimexpr 1.57in+0\tabcolsep}}{\textcolor[HTML]{000000}{\fontsize{11}{11}\selectfont{\global\setmainfont{Arial}{Total}}}} & \multicolumn{1}{>{\raggedright}m{\dimexpr 0.74in+0\tabcolsep}}{\textcolor[HTML]{000000}{\fontsize{11}{11}\selectfont{\global\setmainfont{Arial}{255}}}} & \multicolumn{1}{>{\raggedright}m{\dimexpr 0.67in+0\tabcolsep}}{\textcolor[HTML]{000000}{\fontsize{11}{11}\selectfont{\global\setmainfont{Arial}{174}}}} & \multicolumn{1}{>{\raggedright}m{\dimexpr 0.86in+0\tabcolsep}}{\textcolor[HTML]{000000}{\fontsize{11}{11}\selectfont{\global\setmainfont{Arial}{233}}}} & \multicolumn{1}{>{\raggedright}m{\dimexpr 0.62in+0\tabcolsep}}{\textcolor[HTML]{000000}{\fontsize{11}{11}\selectfont{\global\setmainfont{Arial}{662}}}} \\





\multicolumn{1}{>{\raggedright}m{\dimexpr 1.57in+0\tabcolsep}}{\textcolor[HTML]{000000}{\fontsize{11}{11}\selectfont{\global\setmainfont{Arial}{Statistics\ by\ Class:}}}} & \multicolumn{1}{>{\raggedright}m{\dimexpr 0.74in+0\tabcolsep}}{\textcolor[HTML]{000000}{\fontsize{11}{11}\selectfont{\global\setmainfont{Arial}{}}}} & \multicolumn{1}{>{\raggedright}m{\dimexpr 0.67in+0\tabcolsep}}{\textcolor[HTML]{000000}{\fontsize{11}{11}\selectfont{\global\setmainfont{Arial}{}}}} & \multicolumn{1}{>{\raggedright}m{\dimexpr 0.86in+0\tabcolsep}}{\textcolor[HTML]{000000}{\fontsize{11}{11}\selectfont{\global\setmainfont{Arial}{}}}} & \multicolumn{1}{>{\raggedright}m{\dimexpr 0.62in+0\tabcolsep}}{\textcolor[HTML]{000000}{\fontsize{11}{11}\selectfont{\global\setmainfont{Arial}{}}}} \\





\multicolumn{1}{>{\raggedright}m{\dimexpr 1.57in+0\tabcolsep}}{\textcolor[HTML]{000000}{\fontsize{11}{11}\selectfont{\global\setmainfont{Arial}{Sensitivity}}}} & \multicolumn{1}{>{\raggedright}m{\dimexpr 0.74in+0\tabcolsep}}{\textcolor[HTML]{000000}{\fontsize{11}{11}\selectfont{\global\setmainfont{Arial}{0.486}}}} & \multicolumn{1}{>{\raggedright}m{\dimexpr 0.67in+0\tabcolsep}}{\textcolor[HTML]{000000}{\fontsize{11}{11}\selectfont{\global\setmainfont{Arial}{0.391}}}} & \multicolumn{1}{>{\raggedright}m{\dimexpr 0.86in+0\tabcolsep}}{\textcolor[HTML]{000000}{\fontsize{11}{11}\selectfont{\global\setmainfont{Arial}{0.773}}}} & \multicolumn{1}{>{\raggedright}m{\dimexpr 0.62in+0\tabcolsep}}{\textcolor[HTML]{000000}{\fontsize{11}{11}\selectfont{\global\setmainfont{Arial}{}}}} \\





\multicolumn{1}{>{\raggedright}m{\dimexpr 1.57in+0\tabcolsep}}{\textcolor[HTML]{000000}{\fontsize{11}{11}\selectfont{\global\setmainfont{Arial}{Specificity}}}} & \multicolumn{1}{>{\raggedright}m{\dimexpr 0.74in+0\tabcolsep}}{\textcolor[HTML]{000000}{\fontsize{11}{11}\selectfont{\global\setmainfont{Arial}{0.678}}}} & \multicolumn{1}{>{\raggedright}m{\dimexpr 0.67in+0\tabcolsep}}{\textcolor[HTML]{000000}{\fontsize{11}{11}\selectfont{\global\setmainfont{Arial}{0.879}}}} & \multicolumn{1}{>{\raggedright}m{\dimexpr 0.86in+0\tabcolsep}}{\textcolor[HTML]{000000}{\fontsize{11}{11}\selectfont{\global\setmainfont{Arial}{0.767}}}} & \multicolumn{1}{>{\raggedright}m{\dimexpr 0.62in+0\tabcolsep}}{\textcolor[HTML]{000000}{\fontsize{11}{11}\selectfont{\global\setmainfont{Arial}{}}}} \\





\multicolumn{1}{>{\raggedright}m{\dimexpr 1.57in+0\tabcolsep}}{\textcolor[HTML]{000000}{\fontsize{11}{11}\selectfont{\global\setmainfont{Arial}{Precision}}}} & \multicolumn{1}{>{\raggedright}m{\dimexpr 0.74in+0\tabcolsep}}{\textcolor[HTML]{000000}{\fontsize{11}{11}\selectfont{\global\setmainfont{Arial}{0.486}}}} & \multicolumn{1}{>{\raggedright}m{\dimexpr 0.67in+0\tabcolsep}}{\textcolor[HTML]{000000}{\fontsize{11}{11}\selectfont{\global\setmainfont{Arial}{0.535}}}} & \multicolumn{1}{>{\raggedright}m{\dimexpr 0.86in+0\tabcolsep}}{\textcolor[HTML]{000000}{\fontsize{11}{11}\selectfont{\global\setmainfont{Arial}{0.643}}}} & \multicolumn{1}{>{\raggedright}m{\dimexpr 0.62in+0\tabcolsep}}{\textcolor[HTML]{000000}{\fontsize{11}{11}\selectfont{\global\setmainfont{Arial}{}}}} \\

\ascline{1.5pt}{666666}{1-5}



\end{longtable}



\arrayrulecolor[HTML]{000000}

\global\setlength{\arrayrulewidth}{\Oldarrayrulewidth}

\global\setlength{\tabcolsep}{\Oldtabcolsep}

\renewcommand*{\arraystretch}{1}

\global\setlength{\Oldarrayrulewidth}{\arrayrulewidth}

\global\setlength{\Oldtabcolsep}{\tabcolsep}

\setlength{\tabcolsep}{2pt}

\renewcommand*{\arraystretch}{1.5}



\providecommand{\ascline}[3]{\noalign{\global\arrayrulewidth #1}\arrayrulecolor[HTML]{#2}\cline{#3}}

\begin{longtable}[c]{|p{1.57in}|p{0.74in}|p{0.67in}|p{0.86in}|p{0.62in}}



\ascline{1.5pt}{666666}{1-5}

\multicolumn{5}{>{\raggedright}m{\dimexpr 4.46in+8\tabcolsep}}{\textcolor[HTML]{000000}{\fontsize{11}{11}\selectfont{\global\setmainfont{Arial}{Classification\ Matrix\ -\ New\ data}}}} \\

\ascline{1.5pt}{666666}{1-5}



\multicolumn{5}{>{\raggedright}m{\dimexpr 4.46in+8\tabcolsep}}{\textcolor[HTML]{000000}{\fontsize{11}{11}\selectfont{\global\setmainfont{Arial}{Accuracy\ =\ 0.583}}}} \\

\ascline{1.5pt}{666666}{1-5}



\multicolumn{5}{>{\raggedright}m{\dimexpr 4.46in+8\tabcolsep}}{\textcolor[HTML]{000000}{\fontsize{11}{11}\selectfont{\global\setmainfont{Arial}{PCC\ =\ 0.342}}}} \\

\ascline{1.5pt}{666666}{1-5}



\multicolumn{1}{>{\raggedright}m{\dimexpr 1.57in+0\tabcolsep}}{\textcolor[HTML]{000000}{\fontsize{11}{11}\selectfont{\global\setmainfont{Arial}{}}}} & \multicolumn{3}{>{\raggedright}m{\dimexpr 2.27in+4\tabcolsep}}{\textcolor[HTML]{000000}{\fontsize{11}{11}\selectfont{\global\setmainfont{Arial}{Reference}}}} & \multicolumn{1}{>{\raggedright}m{\dimexpr 0.62in+0\tabcolsep}}{\textcolor[HTML]{000000}{\fontsize{11}{11}\selectfont{\global\setmainfont{Arial}{}}}} \\

\ascline{1.5pt}{666666}{1-5}



\multicolumn{1}{>{\raggedright}m{\dimexpr 1.57in+0\tabcolsep}}{\textcolor[HTML]{000000}{\fontsize{11}{11}\selectfont{\global\setmainfont{Arial}{Predicted}}}} & \multicolumn{1}{>{\raggedright}m{\dimexpr 0.74in+0\tabcolsep}}{\textcolor[HTML]{000000}{\fontsize{11}{11}\selectfont{\global\setmainfont{Arial}{Cereal}}}} & \multicolumn{1}{>{\raggedright}m{\dimexpr 0.67in+0\tabcolsep}}{\textcolor[HTML]{000000}{\fontsize{11}{11}\selectfont{\global\setmainfont{Arial}{Bar}}}} & \multicolumn{1}{>{\raggedright}m{\dimexpr 0.86in+0\tabcolsep}}{\textcolor[HTML]{000000}{\fontsize{11}{11}\selectfont{\global\setmainfont{Arial}{Oatmeal}}}} & \multicolumn{1}{>{\raggedright}m{\dimexpr 0.62in+0\tabcolsep}}{\textcolor[HTML]{000000}{\fontsize{11}{11}\selectfont{\global\setmainfont{Arial}{Total}}}} \\

\ascline{1.5pt}{666666}{1-5}\endfirsthead 

\ascline{1.5pt}{666666}{1-5}

\multicolumn{5}{>{\raggedright}m{\dimexpr 4.46in+8\tabcolsep}}{\textcolor[HTML]{000000}{\fontsize{11}{11}\selectfont{\global\setmainfont{Arial}{Classification\ Matrix\ -\ New\ data}}}} \\

\ascline{1.5pt}{666666}{1-5}



\multicolumn{5}{>{\raggedright}m{\dimexpr 4.46in+8\tabcolsep}}{\textcolor[HTML]{000000}{\fontsize{11}{11}\selectfont{\global\setmainfont{Arial}{Accuracy\ =\ 0.583}}}} \\

\ascline{1.5pt}{666666}{1-5}



\multicolumn{5}{>{\raggedright}m{\dimexpr 4.46in+8\tabcolsep}}{\textcolor[HTML]{000000}{\fontsize{11}{11}\selectfont{\global\setmainfont{Arial}{PCC\ =\ 0.342}}}} \\

\ascline{1.5pt}{666666}{1-5}



\multicolumn{1}{>{\raggedright}m{\dimexpr 1.57in+0\tabcolsep}}{\textcolor[HTML]{000000}{\fontsize{11}{11}\selectfont{\global\setmainfont{Arial}{}}}} & \multicolumn{3}{>{\raggedright}m{\dimexpr 2.27in+4\tabcolsep}}{\textcolor[HTML]{000000}{\fontsize{11}{11}\selectfont{\global\setmainfont{Arial}{Reference}}}} & \multicolumn{1}{>{\raggedright}m{\dimexpr 0.62in+0\tabcolsep}}{\textcolor[HTML]{000000}{\fontsize{11}{11}\selectfont{\global\setmainfont{Arial}{}}}} \\

\ascline{1.5pt}{666666}{1-5}



\multicolumn{1}{>{\raggedright}m{\dimexpr 1.57in+0\tabcolsep}}{\textcolor[HTML]{000000}{\fontsize{11}{11}\selectfont{\global\setmainfont{Arial}{Predicted}}}} & \multicolumn{1}{>{\raggedright}m{\dimexpr 0.74in+0\tabcolsep}}{\textcolor[HTML]{000000}{\fontsize{11}{11}\selectfont{\global\setmainfont{Arial}{Cereal}}}} & \multicolumn{1}{>{\raggedright}m{\dimexpr 0.67in+0\tabcolsep}}{\textcolor[HTML]{000000}{\fontsize{11}{11}\selectfont{\global\setmainfont{Arial}{Bar}}}} & \multicolumn{1}{>{\raggedright}m{\dimexpr 0.86in+0\tabcolsep}}{\textcolor[HTML]{000000}{\fontsize{11}{11}\selectfont{\global\setmainfont{Arial}{Oatmeal}}}} & \multicolumn{1}{>{\raggedright}m{\dimexpr 0.62in+0\tabcolsep}}{\textcolor[HTML]{000000}{\fontsize{11}{11}\selectfont{\global\setmainfont{Arial}{Total}}}} \\

\ascline{1.5pt}{666666}{1-5}\endhead



\multicolumn{1}{>{\raggedright}m{\dimexpr 1.57in+0\tabcolsep}}{\textcolor[HTML]{000000}{\fontsize{11}{11}\selectfont{\global\setmainfont{Arial}{Cereal}}}} & \multicolumn{1}{>{\raggedright}m{\dimexpr 0.74in+0\tabcolsep}}{\textcolor[HTML]{000000}{\fontsize{11}{11}\selectfont{\global\setmainfont{Arial}{45}}}} & \multicolumn{1}{>{\raggedright}m{\dimexpr 0.67in+0\tabcolsep}}{\textcolor[HTML]{000000}{\fontsize{11}{11}\selectfont{\global\setmainfont{Arial}{24}}}} & \multicolumn{1}{>{\raggedright}m{\dimexpr 0.86in+0\tabcolsep}}{\textcolor[HTML]{000000}{\fontsize{11}{11}\selectfont{\global\setmainfont{Arial}{20}}}} & \multicolumn{1}{>{\raggedright}m{\dimexpr 0.62in+0\tabcolsep}}{\textcolor[HTML]{000000}{\fontsize{11}{11}\selectfont{\global\setmainfont{Arial}{89}}}} \\





\multicolumn{1}{>{\raggedright}m{\dimexpr 1.57in+0\tabcolsep}}{\textcolor[HTML]{000000}{\fontsize{11}{11}\selectfont{\global\setmainfont{Arial}{Bar}}}} & \multicolumn{1}{>{\raggedright}m{\dimexpr 0.74in+0\tabcolsep}}{\textcolor[HTML]{000000}{\fontsize{11}{11}\selectfont{\global\setmainfont{Arial}{18}}}} & \multicolumn{1}{>{\raggedright}m{\dimexpr 0.67in+0\tabcolsep}}{\textcolor[HTML]{000000}{\fontsize{11}{11}\selectfont{\global\setmainfont{Arial}{25}}}} & \multicolumn{1}{>{\raggedright}m{\dimexpr 0.86in+0\tabcolsep}}{\textcolor[HTML]{000000}{\fontsize{11}{11}\selectfont{\global\setmainfont{Arial}{0}}}} & \multicolumn{1}{>{\raggedright}m{\dimexpr 0.62in+0\tabcolsep}}{\textcolor[HTML]{000000}{\fontsize{11}{11}\selectfont{\global\setmainfont{Arial}{43}}}} \\





\multicolumn{1}{>{\raggedright}m{\dimexpr 1.57in+0\tabcolsep}}{\textcolor[HTML]{000000}{\fontsize{11}{11}\selectfont{\global\setmainfont{Arial}{Oatmeal}}}} & \multicolumn{1}{>{\raggedright}m{\dimexpr 0.74in+0\tabcolsep}}{\textcolor[HTML]{000000}{\fontsize{11}{11}\selectfont{\global\setmainfont{Arial}{21}}}} & \multicolumn{1}{>{\raggedright}m{\dimexpr 0.67in+0\tabcolsep}}{\textcolor[HTML]{000000}{\fontsize{11}{11}\selectfont{\global\setmainfont{Arial}{8}}}} & \multicolumn{1}{>{\raggedright}m{\dimexpr 0.86in+0\tabcolsep}}{\textcolor[HTML]{000000}{\fontsize{11}{11}\selectfont{\global\setmainfont{Arial}{57}}}} & \multicolumn{1}{>{\raggedright}m{\dimexpr 0.62in+0\tabcolsep}}{\textcolor[HTML]{000000}{\fontsize{11}{11}\selectfont{\global\setmainfont{Arial}{86}}}} \\





\multicolumn{1}{>{\raggedright}m{\dimexpr 1.57in+0\tabcolsep}}{\textcolor[HTML]{000000}{\fontsize{11}{11}\selectfont{\global\setmainfont{Arial}{Total}}}} & \multicolumn{1}{>{\raggedright}m{\dimexpr 0.74in+0\tabcolsep}}{\textcolor[HTML]{000000}{\fontsize{11}{11}\selectfont{\global\setmainfont{Arial}{84}}}} & \multicolumn{1}{>{\raggedright}m{\dimexpr 0.67in+0\tabcolsep}}{\textcolor[HTML]{000000}{\fontsize{11}{11}\selectfont{\global\setmainfont{Arial}{57}}}} & \multicolumn{1}{>{\raggedright}m{\dimexpr 0.86in+0\tabcolsep}}{\textcolor[HTML]{000000}{\fontsize{11}{11}\selectfont{\global\setmainfont{Arial}{77}}}} & \multicolumn{1}{>{\raggedright}m{\dimexpr 0.62in+0\tabcolsep}}{\textcolor[HTML]{000000}{\fontsize{11}{11}\selectfont{\global\setmainfont{Arial}{218}}}} \\





\multicolumn{1}{>{\raggedright}m{\dimexpr 1.57in+0\tabcolsep}}{\textcolor[HTML]{000000}{\fontsize{11}{11}\selectfont{\global\setmainfont{Arial}{Statistics\ by\ Class:}}}} & \multicolumn{1}{>{\raggedright}m{\dimexpr 0.74in+0\tabcolsep}}{\textcolor[HTML]{000000}{\fontsize{11}{11}\selectfont{\global\setmainfont{Arial}{}}}} & \multicolumn{1}{>{\raggedright}m{\dimexpr 0.67in+0\tabcolsep}}{\textcolor[HTML]{000000}{\fontsize{11}{11}\selectfont{\global\setmainfont{Arial}{}}}} & \multicolumn{1}{>{\raggedright}m{\dimexpr 0.86in+0\tabcolsep}}{\textcolor[HTML]{000000}{\fontsize{11}{11}\selectfont{\global\setmainfont{Arial}{}}}} & \multicolumn{1}{>{\raggedright}m{\dimexpr 0.62in+0\tabcolsep}}{\textcolor[HTML]{000000}{\fontsize{11}{11}\selectfont{\global\setmainfont{Arial}{}}}} \\





\multicolumn{1}{>{\raggedright}m{\dimexpr 1.57in+0\tabcolsep}}{\textcolor[HTML]{000000}{\fontsize{11}{11}\selectfont{\global\setmainfont{Arial}{Sensitivity}}}} & \multicolumn{1}{>{\raggedright}m{\dimexpr 0.74in+0\tabcolsep}}{\textcolor[HTML]{000000}{\fontsize{11}{11}\selectfont{\global\setmainfont{Arial}{0.536}}}} & \multicolumn{1}{>{\raggedright}m{\dimexpr 0.67in+0\tabcolsep}}{\textcolor[HTML]{000000}{\fontsize{11}{11}\selectfont{\global\setmainfont{Arial}{0.439}}}} & \multicolumn{1}{>{\raggedright}m{\dimexpr 0.86in+0\tabcolsep}}{\textcolor[HTML]{000000}{\fontsize{11}{11}\selectfont{\global\setmainfont{Arial}{0.740}}}} & \multicolumn{1}{>{\raggedright}m{\dimexpr 0.62in+0\tabcolsep}}{\textcolor[HTML]{000000}{\fontsize{11}{11}\selectfont{\global\setmainfont{Arial}{}}}} \\





\multicolumn{1}{>{\raggedright}m{\dimexpr 1.57in+0\tabcolsep}}{\textcolor[HTML]{000000}{\fontsize{11}{11}\selectfont{\global\setmainfont{Arial}{Specificity}}}} & \multicolumn{1}{>{\raggedright}m{\dimexpr 0.74in+0\tabcolsep}}{\textcolor[HTML]{000000}{\fontsize{11}{11}\selectfont{\global\setmainfont{Arial}{0.672}}}} & \multicolumn{1}{>{\raggedright}m{\dimexpr 0.67in+0\tabcolsep}}{\textcolor[HTML]{000000}{\fontsize{11}{11}\selectfont{\global\setmainfont{Arial}{0.888}}}} & \multicolumn{1}{>{\raggedright}m{\dimexpr 0.86in+0\tabcolsep}}{\textcolor[HTML]{000000}{\fontsize{11}{11}\selectfont{\global\setmainfont{Arial}{0.794}}}} & \multicolumn{1}{>{\raggedright}m{\dimexpr 0.62in+0\tabcolsep}}{\textcolor[HTML]{000000}{\fontsize{11}{11}\selectfont{\global\setmainfont{Arial}{}}}} \\





\multicolumn{1}{>{\raggedright}m{\dimexpr 1.57in+0\tabcolsep}}{\textcolor[HTML]{000000}{\fontsize{11}{11}\selectfont{\global\setmainfont{Arial}{Precision}}}} & \multicolumn{1}{>{\raggedright}m{\dimexpr 0.74in+0\tabcolsep}}{\textcolor[HTML]{000000}{\fontsize{11}{11}\selectfont{\global\setmainfont{Arial}{0.506}}}} & \multicolumn{1}{>{\raggedright}m{\dimexpr 0.67in+0\tabcolsep}}{\textcolor[HTML]{000000}{\fontsize{11}{11}\selectfont{\global\setmainfont{Arial}{0.581}}}} & \multicolumn{1}{>{\raggedright}m{\dimexpr 0.86in+0\tabcolsep}}{\textcolor[HTML]{000000}{\fontsize{11}{11}\selectfont{\global\setmainfont{Arial}{0.663}}}} & \multicolumn{1}{>{\raggedright}m{\dimexpr 0.62in+0\tabcolsep}}{\textcolor[HTML]{000000}{\fontsize{11}{11}\selectfont{\global\setmainfont{Arial}{}}}} \\

\ascline{1.5pt}{666666}{1-5}



\end{longtable}



\arrayrulecolor[HTML]{000000}

\global\setlength{\arrayrulewidth}{\Oldarrayrulewidth}

\global\setlength{\tabcolsep}{\Oldtabcolsep}

\renewcommand*{\arraystretch}{1}

\subsection{Interpreting Coefficients}\label{interpreting-coefficients}

Coefficient estimates in a standard MNL model are interpreted \textbf{relative
to the reference brand}. A positive coefficient means that higher values
of the predictor increase the likelihood of choosing that brand relative
to the baseline.

To aid interpretation, coefficients can also be expressed as \textbf{relative
risk ratios (RRRs)}. RRRs greater than 1 indicate increased relative likelihood, while values
below 1 indicate decreased likelihood. These interpretations are often
more intuitive for managerial audiences.

\begin{Shaded}
\begin{Highlighting}[]
\NormalTok{mnl\_eval\_rrr }\OtherTok{\textless{}{-}} \FunctionTok{eval\_std\_mnl}\NormalTok{(}\AttributeTok{model =}\NormalTok{ mnl\_fit, }\AttributeTok{exp =} \ConstantTok{TRUE}\NormalTok{, }
                             \AttributeTok{newdata =}\NormalTok{ test, }\AttributeTok{ft=}\ConstantTok{TRUE}\NormalTok{)}
\NormalTok{mnl\_eval\_rrr}\SpecialCharTok{$}\NormalTok{coef\_table}
\end{Highlighting}
\end{Shaded}

\global\setlength{\Oldarrayrulewidth}{\arrayrulewidth}

\global\setlength{\Oldtabcolsep}{\tabcolsep}

\setlength{\tabcolsep}{2pt}

\renewcommand*{\arraystretch}{1.5}



\providecommand{\ascline}[3]{\noalign{\global\arrayrulewidth #1}\arrayrulecolor[HTML]{#2}\cline{#3}}

\begin{longtable}[c]{|p{0.86in}|p{1.46in}|p{0.75in}|p{0.85in}|p{0.84in}|p{0.78in}}



\ascline{1.5pt}{666666}{1-6}

\multicolumn{6}{>{\raggedright}m{\dimexpr 5.54in+10\tabcolsep}}{\textcolor[HTML]{000000}{\fontsize{11}{11}\selectfont{\global\setmainfont{Arial}{LR\ chi2\ (8)\ =\ 288.1568;\ p\ <\ 0.0001}}}} \\

\ascline{1.5pt}{666666}{1-6}



\multicolumn{6}{>{\raggedright}m{\dimexpr 5.54in+10\tabcolsep}}{\textcolor[HTML]{000000}{\fontsize{11}{11}\selectfont{\global\setmainfont{Arial}{McFadden's\ Pseudo\ R-square\ =\ 0.2004}}}} \\

\ascline{1.5pt}{666666}{1-6}



\multicolumn{1}{>{\raggedright}m{\dimexpr 0.86in+0\tabcolsep}}{\textcolor[HTML]{000000}{\fontsize{11}{11}\selectfont{\global\setmainfont{Arial}{y.level}}}} & \multicolumn{1}{>{\raggedright}m{\dimexpr 1.46in+0\tabcolsep}}{\textcolor[HTML]{000000}{\fontsize{11}{11}\selectfont{\global\setmainfont{Arial}{term}}}} & \multicolumn{1}{>{\raggedleft}m{\dimexpr 0.75in+0\tabcolsep}}{\textcolor[HTML]{000000}{\fontsize{11}{11}\selectfont{\global\setmainfont{Arial}{RRR}}}} & \multicolumn{1}{>{\raggedleft}m{\dimexpr 0.85in+0\tabcolsep}}{\textcolor[HTML]{000000}{\fontsize{11}{11}\selectfont{\global\setmainfont{Arial}{std.error}}}} & \multicolumn{1}{>{\raggedleft}m{\dimexpr 0.84in+0\tabcolsep}}{\textcolor[HTML]{000000}{\fontsize{11}{11}\selectfont{\global\setmainfont{Arial}{statistic}}}} & \multicolumn{1}{>{\raggedleft}m{\dimexpr 0.78in+0\tabcolsep}}{\textcolor[HTML]{000000}{\fontsize{11}{11}\selectfont{\global\setmainfont{Arial}{p.value}}}} \\

\ascline{1.5pt}{666666}{1-6}\endfirsthead 

\ascline{1.5pt}{666666}{1-6}

\multicolumn{6}{>{\raggedright}m{\dimexpr 5.54in+10\tabcolsep}}{\textcolor[HTML]{000000}{\fontsize{11}{11}\selectfont{\global\setmainfont{Arial}{LR\ chi2\ (8)\ =\ 288.1568;\ p\ <\ 0.0001}}}} \\

\ascline{1.5pt}{666666}{1-6}



\multicolumn{6}{>{\raggedright}m{\dimexpr 5.54in+10\tabcolsep}}{\textcolor[HTML]{000000}{\fontsize{11}{11}\selectfont{\global\setmainfont{Arial}{McFadden's\ Pseudo\ R-square\ =\ 0.2004}}}} \\

\ascline{1.5pt}{666666}{1-6}



\multicolumn{1}{>{\raggedright}m{\dimexpr 0.86in+0\tabcolsep}}{\textcolor[HTML]{000000}{\fontsize{11}{11}\selectfont{\global\setmainfont{Arial}{y.level}}}} & \multicolumn{1}{>{\raggedright}m{\dimexpr 1.46in+0\tabcolsep}}{\textcolor[HTML]{000000}{\fontsize{11}{11}\selectfont{\global\setmainfont{Arial}{term}}}} & \multicolumn{1}{>{\raggedleft}m{\dimexpr 0.75in+0\tabcolsep}}{\textcolor[HTML]{000000}{\fontsize{11}{11}\selectfont{\global\setmainfont{Arial}{RRR}}}} & \multicolumn{1}{>{\raggedleft}m{\dimexpr 0.85in+0\tabcolsep}}{\textcolor[HTML]{000000}{\fontsize{11}{11}\selectfont{\global\setmainfont{Arial}{std.error}}}} & \multicolumn{1}{>{\raggedleft}m{\dimexpr 0.84in+0\tabcolsep}}{\textcolor[HTML]{000000}{\fontsize{11}{11}\selectfont{\global\setmainfont{Arial}{statistic}}}} & \multicolumn{1}{>{\raggedleft}m{\dimexpr 0.78in+0\tabcolsep}}{\textcolor[HTML]{000000}{\fontsize{11}{11}\selectfont{\global\setmainfont{Arial}{p.value}}}} \\

\ascline{1.5pt}{666666}{1-6}\endhead



\multicolumn{1}{>{\raggedright}m{\dimexpr 0.86in+0\tabcolsep}}{\textcolor[HTML]{000000}{\fontsize{11}{11}\selectfont{\global\setmainfont{Arial}{Bar}}}} & \multicolumn{1}{>{\raggedright}m{\dimexpr 1.46in+0\tabcolsep}}{\textcolor[HTML]{000000}{\fontsize{11}{11}\selectfont{\global\setmainfont{Arial}{(Intercept)}}}} & \multicolumn{1}{>{\raggedleft}m{\dimexpr 0.75in+0\tabcolsep}}{\textcolor[HTML]{000000}{\fontsize{11}{11}\selectfont{\global\setmainfont{Arial}{2.4187}}}} & \multicolumn{1}{>{\raggedleft}m{\dimexpr 0.85in+0\tabcolsep}}{\textcolor[HTML]{000000}{\fontsize{11}{11}\selectfont{\global\setmainfont{Arial}{0.3257}}}} & \multicolumn{1}{>{\raggedleft}m{\dimexpr 0.84in+0\tabcolsep}}{\textcolor[HTML]{000000}{\fontsize{11}{11}\selectfont{\global\setmainfont{Arial}{2.7118}}}} & \multicolumn{1}{>{\raggedleft}m{\dimexpr 0.78in+0\tabcolsep}}{\textcolor[HTML]{000000}{\fontsize{11}{11}\selectfont{\global\setmainfont{Arial}{0.0067}}}} \\





\multicolumn{1}{>{\raggedright}m{\dimexpr 0.86in+0\tabcolsep}}{\textcolor[HTML]{000000}{\fontsize{11}{11}\selectfont{\global\setmainfont{Arial}{Bar}}}} & \multicolumn{1}{>{\raggedright}m{\dimexpr 1.46in+0\tabcolsep}}{\textcolor[HTML]{000000}{\fontsize{11}{11}\selectfont{\global\setmainfont{Arial}{genderMale}}}} & \multicolumn{1}{>{\raggedleft}m{\dimexpr 0.75in+0\tabcolsep}}{\textcolor[HTML]{000000}{\fontsize{11}{11}\selectfont{\global\setmainfont{Arial}{0.8082}}}} & \multicolumn{1}{>{\raggedleft}m{\dimexpr 0.85in+0\tabcolsep}}{\textcolor[HTML]{000000}{\fontsize{11}{11}\selectfont{\global\setmainfont{Arial}{0.2064}}}} & \multicolumn{1}{>{\raggedleft}m{\dimexpr 0.84in+0\tabcolsep}}{\textcolor[HTML]{000000}{\fontsize{11}{11}\selectfont{\global\setmainfont{Arial}{-1.0318}}}} & \multicolumn{1}{>{\raggedleft}m{\dimexpr 0.78in+0\tabcolsep}}{\textcolor[HTML]{000000}{\fontsize{11}{11}\selectfont{\global\setmainfont{Arial}{0.3022}}}} \\





\multicolumn{1}{>{\raggedright}m{\dimexpr 0.86in+0\tabcolsep}}{\textcolor[HTML]{000000}{\fontsize{11}{11}\selectfont{\global\setmainfont{Arial}{Bar}}}} & \multicolumn{1}{>{\raggedright}m{\dimexpr 1.46in+0\tabcolsep}}{\textcolor[HTML]{000000}{\fontsize{11}{11}\selectfont{\global\setmainfont{Arial}{maritalUnmarried}}}} & \multicolumn{1}{>{\raggedleft}m{\dimexpr 0.75in+0\tabcolsep}}{\textcolor[HTML]{000000}{\fontsize{11}{11}\selectfont{\global\setmainfont{Arial}{1.8454}}}} & \multicolumn{1}{>{\raggedleft}m{\dimexpr 0.85in+0\tabcolsep}}{\textcolor[HTML]{000000}{\fontsize{11}{11}\selectfont{\global\setmainfont{Arial}{0.2124}}}} & \multicolumn{1}{>{\raggedleft}m{\dimexpr 0.84in+0\tabcolsep}}{\textcolor[HTML]{000000}{\fontsize{11}{11}\selectfont{\global\setmainfont{Arial}{2.8849}}}} & \multicolumn{1}{>{\raggedleft}m{\dimexpr 0.78in+0\tabcolsep}}{\textcolor[HTML]{000000}{\fontsize{11}{11}\selectfont{\global\setmainfont{Arial}{0.0039}}}} \\





\multicolumn{1}{>{\raggedright}m{\dimexpr 0.86in+0\tabcolsep}}{\textcolor[HTML]{000000}{\fontsize{11}{11}\selectfont{\global\setmainfont{Arial}{Bar}}}} & \multicolumn{1}{>{\raggedright}m{\dimexpr 1.46in+0\tabcolsep}}{\textcolor[HTML]{000000}{\fontsize{11}{11}\selectfont{\global\setmainfont{Arial}{lifestyleInactive}}}} & \multicolumn{1}{>{\raggedleft}m{\dimexpr 0.75in+0\tabcolsep}}{\textcolor[HTML]{000000}{\fontsize{11}{11}\selectfont{\global\setmainfont{Arial}{0.4554}}}} & \multicolumn{1}{>{\raggedleft}m{\dimexpr 0.85in+0\tabcolsep}}{\textcolor[HTML]{000000}{\fontsize{11}{11}\selectfont{\global\setmainfont{Arial}{0.2090}}}} & \multicolumn{1}{>{\raggedleft}m{\dimexpr 0.84in+0\tabcolsep}}{\textcolor[HTML]{000000}{\fontsize{11}{11}\selectfont{\global\setmainfont{Arial}{-3.7627}}}} & \multicolumn{1}{>{\raggedleft}m{\dimexpr 0.78in+0\tabcolsep}}{\textcolor[HTML]{000000}{\fontsize{11}{11}\selectfont{\global\setmainfont{Arial}{0.0002}}}} \\





\multicolumn{1}{>{\raggedright}m{\dimexpr 0.86in+0\tabcolsep}}{\textcolor[HTML]{000000}{\fontsize{11}{11}\selectfont{\global\setmainfont{Arial}{Bar}}}} & \multicolumn{1}{>{\raggedright}m{\dimexpr 1.46in+0\tabcolsep}}{\textcolor[HTML]{000000}{\fontsize{11}{11}\selectfont{\global\setmainfont{Arial}{age}}}} & \multicolumn{1}{>{\raggedleft}m{\dimexpr 0.75in+0\tabcolsep}}{\textcolor[HTML]{000000}{\fontsize{11}{11}\selectfont{\global\setmainfont{Arial}{0.9750}}}} & \multicolumn{1}{>{\raggedleft}m{\dimexpr 0.85in+0\tabcolsep}}{\textcolor[HTML]{000000}{\fontsize{11}{11}\selectfont{\global\setmainfont{Arial}{0.0067}}}} & \multicolumn{1}{>{\raggedleft}m{\dimexpr 0.84in+0\tabcolsep}}{\textcolor[HTML]{000000}{\fontsize{11}{11}\selectfont{\global\setmainfont{Arial}{-3.8055}}}} & \multicolumn{1}{>{\raggedleft}m{\dimexpr 0.78in+0\tabcolsep}}{\textcolor[HTML]{000000}{\fontsize{11}{11}\selectfont{\global\setmainfont{Arial}{0.0001}}}} \\





\multicolumn{1}{>{\raggedright}m{\dimexpr 0.86in+0\tabcolsep}}{\textcolor[HTML]{000000}{\fontsize{11}{11}\selectfont{\global\setmainfont{Arial}{Oatmeal}}}} & \multicolumn{1}{>{\raggedright}m{\dimexpr 1.46in+0\tabcolsep}}{\textcolor[HTML]{000000}{\fontsize{11}{11}\selectfont{\global\setmainfont{Arial}{(Intercept)}}}} & \multicolumn{1}{>{\raggedleft}m{\dimexpr 0.75in+0\tabcolsep}}{\textcolor[HTML]{000000}{\fontsize{11}{11}\selectfont{\global\setmainfont{Arial}{0.0112}}}} & \multicolumn{1}{>{\raggedleft}m{\dimexpr 0.85in+0\tabcolsep}}{\textcolor[HTML]{000000}{\fontsize{11}{11}\selectfont{\global\setmainfont{Arial}{0.4597}}}} & \multicolumn{1}{>{\raggedleft}m{\dimexpr 0.84in+0\tabcolsep}}{\textcolor[HTML]{000000}{\fontsize{11}{11}\selectfont{\global\setmainfont{Arial}{-9.7722}}}} & \multicolumn{1}{>{\raggedleft}m{\dimexpr 0.78in+0\tabcolsep}}{\textcolor[HTML]{000000}{\fontsize{11}{11}\selectfont{\global\setmainfont{Arial}{0.0000}}}} \\





\multicolumn{1}{>{\raggedright}m{\dimexpr 0.86in+0\tabcolsep}}{\textcolor[HTML]{000000}{\fontsize{11}{11}\selectfont{\global\setmainfont{Arial}{Oatmeal}}}} & \multicolumn{1}{>{\raggedright}m{\dimexpr 1.46in+0\tabcolsep}}{\textcolor[HTML]{000000}{\fontsize{11}{11}\selectfont{\global\setmainfont{Arial}{genderMale}}}} & \multicolumn{1}{>{\raggedleft}m{\dimexpr 0.75in+0\tabcolsep}}{\textcolor[HTML]{000000}{\fontsize{11}{11}\selectfont{\global\setmainfont{Arial}{0.9776}}}} & \multicolumn{1}{>{\raggedleft}m{\dimexpr 0.85in+0\tabcolsep}}{\textcolor[HTML]{000000}{\fontsize{11}{11}\selectfont{\global\setmainfont{Arial}{0.2095}}}} & \multicolumn{1}{>{\raggedleft}m{\dimexpr 0.84in+0\tabcolsep}}{\textcolor[HTML]{000000}{\fontsize{11}{11}\selectfont{\global\setmainfont{Arial}{-0.1080}}}} & \multicolumn{1}{>{\raggedleft}m{\dimexpr 0.78in+0\tabcolsep}}{\textcolor[HTML]{000000}{\fontsize{11}{11}\selectfont{\global\setmainfont{Arial}{0.9140}}}} \\





\multicolumn{1}{>{\raggedright}m{\dimexpr 0.86in+0\tabcolsep}}{\textcolor[HTML]{000000}{\fontsize{11}{11}\selectfont{\global\setmainfont{Arial}{Oatmeal}}}} & \multicolumn{1}{>{\raggedright}m{\dimexpr 1.46in+0\tabcolsep}}{\textcolor[HTML]{000000}{\fontsize{11}{11}\selectfont{\global\setmainfont{Arial}{maritalUnmarried}}}} & \multicolumn{1}{>{\raggedleft}m{\dimexpr 0.75in+0\tabcolsep}}{\textcolor[HTML]{000000}{\fontsize{11}{11}\selectfont{\global\setmainfont{Arial}{0.6772}}}} & \multicolumn{1}{>{\raggedleft}m{\dimexpr 0.85in+0\tabcolsep}}{\textcolor[HTML]{000000}{\fontsize{11}{11}\selectfont{\global\setmainfont{Arial}{0.2367}}}} & \multicolumn{1}{>{\raggedleft}m{\dimexpr 0.84in+0\tabcolsep}}{\textcolor[HTML]{000000}{\fontsize{11}{11}\selectfont{\global\setmainfont{Arial}{-1.6469}}}} & \multicolumn{1}{>{\raggedleft}m{\dimexpr 0.78in+0\tabcolsep}}{\textcolor[HTML]{000000}{\fontsize{11}{11}\selectfont{\global\setmainfont{Arial}{0.0996}}}} \\





\multicolumn{1}{>{\raggedright}m{\dimexpr 0.86in+0\tabcolsep}}{\textcolor[HTML]{000000}{\fontsize{11}{11}\selectfont{\global\setmainfont{Arial}{Oatmeal}}}} & \multicolumn{1}{>{\raggedright}m{\dimexpr 1.46in+0\tabcolsep}}{\textcolor[HTML]{000000}{\fontsize{11}{11}\selectfont{\global\setmainfont{Arial}{lifestyleInactive}}}} & \multicolumn{1}{>{\raggedleft}m{\dimexpr 0.75in+0\tabcolsep}}{\textcolor[HTML]{000000}{\fontsize{11}{11}\selectfont{\global\setmainfont{Arial}{1.3754}}}} & \multicolumn{1}{>{\raggedleft}m{\dimexpr 0.85in+0\tabcolsep}}{\textcolor[HTML]{000000}{\fontsize{11}{11}\selectfont{\global\setmainfont{Arial}{0.2157}}}} & \multicolumn{1}{>{\raggedleft}m{\dimexpr 0.84in+0\tabcolsep}}{\textcolor[HTML]{000000}{\fontsize{11}{11}\selectfont{\global\setmainfont{Arial}{1.4777}}}} & \multicolumn{1}{>{\raggedleft}m{\dimexpr 0.78in+0\tabcolsep}}{\textcolor[HTML]{000000}{\fontsize{11}{11}\selectfont{\global\setmainfont{Arial}{0.1395}}}} \\





\multicolumn{1}{>{\raggedright}m{\dimexpr 0.86in+0\tabcolsep}}{\textcolor[HTML]{000000}{\fontsize{11}{11}\selectfont{\global\setmainfont{Arial}{Oatmeal}}}} & \multicolumn{1}{>{\raggedright}m{\dimexpr 1.46in+0\tabcolsep}}{\textcolor[HTML]{000000}{\fontsize{11}{11}\selectfont{\global\setmainfont{Arial}{age}}}} & \multicolumn{1}{>{\raggedleft}m{\dimexpr 0.75in+0\tabcolsep}}{\textcolor[HTML]{000000}{\fontsize{11}{11}\selectfont{\global\setmainfont{Arial}{1.0832}}}} & \multicolumn{1}{>{\raggedleft}m{\dimexpr 0.85in+0\tabcolsep}}{\textcolor[HTML]{000000}{\fontsize{11}{11}\selectfont{\global\setmainfont{Arial}{0.0078}}}} & \multicolumn{1}{>{\raggedleft}m{\dimexpr 0.84in+0\tabcolsep}}{\textcolor[HTML]{000000}{\fontsize{11}{11}\selectfont{\global\setmainfont{Arial}{10.3104}}}} & \multicolumn{1}{>{\raggedleft}m{\dimexpr 0.78in+0\tabcolsep}}{\textcolor[HTML]{000000}{\fontsize{11}{11}\selectfont{\global\setmainfont{Arial}{0.0000}}}} \\

\ascline{1.5pt}{666666}{1-6}



\end{longtable}



\arrayrulecolor[HTML]{000000}

\global\setlength{\arrayrulewidth}{\Oldarrayrulewidth}

\global\setlength{\tabcolsep}{\Oldtabcolsep}

\renewcommand*{\arraystretch}{1}

\subsection{Classification Performance}\label{classification-performance}

Beyond fit statistics, classification results help assess how well the
model predicts observed choices.

The output includes:

\begin{itemize}
\tightlist
\item
  Overall accuracy
\item
  Proportional Chance Criterion (PCC)
\item
  Product-specific sensitivity, specificity, and precision
\end{itemize}

These metrics help identify which brands are easier or harder to predict
based on observed covariates.

\subsection{Holdout Sample Evaluation}\label{holdout-sample-evaluation}

Evaluating the model on a test sample provides insight into how well it
generalizes to new data. Large discrepancies between training and test
performance may indicate overfitting.

In practice, marketing data often contain substantial noise, so perfect
prediction is neither expected nor required for managerial usefulness.

\begin{center}\rule{0.5\linewidth}{0.5pt}\end{center}

\section{Predicted Probabilities}\label{predicted-probabilities-1}

Coefficients and classification tables are not always the most intuitive
outputs for decision-makers. Predicted probabilities translate model
results into directly interpretable quantities.

We use the \texttt{pp\_std\_mnl()} function from the \texttt{MKT4320BGSU} package to compute and visualize average predicted
probabilities for a focal predictor.

Usage:

\begin{itemize}
\tightlist
\item
  \texttt{pp\_std\_mnl(model,\ focal,\ xlab\ =\ NULL,\ ft\_table\ =\ FALSE)}
\item
  where:

  \begin{itemize}
  \tightlist
  \item
    \texttt{model} is a fitted \texttt{multinom} model.
  \item
    \texttt{focal} is a character string; name of the focal predictor variable.
  \item
    \texttt{xlab} is an optional character string; label for the x-axis in the plot.
  \item
    \texttt{ft\_table} is logical; if TRUE, return the probability table as a flextable (default = FALSE).
  \end{itemize}
\end{itemize}

Below, examples are provided for \texttt{age} (a continuous variable) and \texttt{lifestyle} (a categorical variable).

\begin{Shaded}
\begin{Highlighting}[]
\NormalTok{pp\_age }\OtherTok{\textless{}{-}} \FunctionTok{pp\_std\_mnl}\NormalTok{(}\AttributeTok{model =}\NormalTok{ mnl\_fit, }\AttributeTok{focal =} \StringTok{"age"}\NormalTok{, }\AttributeTok{xlab  =} \StringTok{"Age"}\NormalTok{)}
\NormalTok{pp\_age}\SpecialCharTok{$}\NormalTok{table}
\end{Highlighting}
\end{Shaded}

\begin{verbatim}
# A tibble: 9 x 5
    age bfast   p.prob lower.CI upper.CI
  <dbl> <chr>    <dbl>    <dbl>    <dbl>
1    31 Cereal  0.516    0.459     0.573
2    31 Bar     0.413    0.357     0.472
3    31 Oatmeal 0.0708   0.0473    0.105
4    49 Cereal  0.479    0.431     0.527
5    49 Bar     0.243    0.204     0.287
6    49 Oatmeal 0.277    0.234     0.326
7    67 Cereal  0.266    0.218     0.320
8    67 Bar     0.0856   0.0604    0.120
9    67 Oatmeal 0.649    0.590     0.704
\end{verbatim}

\begin{Shaded}
\begin{Highlighting}[]
\NormalTok{pp\_age}\SpecialCharTok{$}\NormalTok{plot}
\end{Highlighting}
\end{Shaded}

\pandocbounded{\includegraphics[keepaspectratio]{MKT4320_R_Tutorial_files/figure-latex/pp-price-1.pdf}}

\begin{Shaded}
\begin{Highlighting}[]
\NormalTok{pp\_lifestyle }\OtherTok{\textless{}{-}} \FunctionTok{pp\_std\_mnl}\NormalTok{(}\AttributeTok{model =}\NormalTok{ mnl\_fit, }\AttributeTok{focal=} \StringTok{"lifestyle"}\NormalTok{, }
                           \AttributeTok{xlab =} \StringTok{"Lifestyle"}\NormalTok{)}
\NormalTok{pp\_lifestyle}\SpecialCharTok{$}\NormalTok{table}
\end{Highlighting}
\end{Shaded}

\begin{verbatim}
# A tibble: 6 x 5
  lifestyle bfast   p.prob lower.CI upper.CI
  <fct>     <chr>    <dbl>    <dbl>    <dbl>
1 Active    Cereal   0.442    0.378    0.507
2 Active    Bar      0.345    0.285    0.410
3 Active    Oatmeal  0.213    0.162    0.275
4 Inactive  Cereal   0.495    0.435    0.556
5 Inactive  Bar      0.176    0.135    0.226
6 Inactive  Oatmeal  0.329    0.271    0.392
\end{verbatim}

\begin{Shaded}
\begin{Highlighting}[]
\NormalTok{pp\_lifestyle}\SpecialCharTok{$}\NormalTok{plot}
\end{Highlighting}
\end{Shaded}

\pandocbounded{\includegraphics[keepaspectratio]{MKT4320_R_Tutorial_files/figure-latex/pp-price-2.pdf}}

Predicted probabilities help answer questions such as:

\begin{itemize}
\tightlist
\item
  How does increasing price shift brand choice probabilities?
\item
  Which brands are most sensitive to changes in a given variable?
\end{itemize}

\begin{center}\rule{0.5\linewidth}{0.5pt}\end{center}

\section{Marketing Interpretation}\label{marketing-interpretation}

The standard multinomial logit model provides a powerful yet accessible
framework for understanding brand choice. It allows marketers to:

\begin{itemize}
\tightlist
\item
  Compare competitive positioning across brands
\item
  Assess price and promotion sensitivity
\item
  Translate statistical estimates into actionable probabilities
\end{itemize}

However, it also has limitations, including restrictive substitution
patterns across alternatives.

\section{Summary}\label{summary-2}

In this chapter, you learned how to:

\begin{itemize}
\tightlist
\item
  Estimate a standard multinomial logit model
\item
  Evaluate model fit and predictive performance
\item
  Interpret coefficients and relative risk ratios
\item
  Use predicted probabilities for marketing insight
\end{itemize}

The standard MNL model serves as a foundational tool in marketing
analytics and provides a benchmark against which more advanced choice
models can be compared.

\section{What's Next}\label{whats-next-13}

In this chapter, we treated brand choice as a function of consumer-level characteristics and marketing variables that affect all alternatives in the same way. This approach works well as a baseline, but it imposes an important limitation: it assumes that predictors influence every brand symmetrically.

In the next chapter, we relax this restriction by introducing the alternative-specific multinomial logit (AS-MNL) model. This framework allows predictors---such as price, promotions, or product attributes---to vary by brand, more closely reflecting how consumers actually evaluate competing options.

You will learn how to:
- Specify predictors that differ across alternatives
- Interpret coefficients that are specific to each brand
- Compare alternative-specific results to the standard MNL model
- Gain deeper insight into competitive positioning and substitution patterns

Alternative-specific MNL models provide a major step forward in realism and interpretability, especially in marketing settings where attributes like price, availability, or features differ meaningfully across brands.

\chapter{Alternative-Specific Multinomial Logit Models}\label{alternative-specific-multinomial-logit-models}

\section{Introduction: Why Alternative-Specific MNL?}\label{introduction-why-alternative-specific-mnl}

In the last chapter, we modeled brand choice using standard multinomial logit (MNL) models, where all predictors were \textbf{case-specific}. That is, they described the consumer or choice situation and took the same value for all brands in a given choice set.

In many real marketing applications, however, the most important predictors vary \textbf{by brand}. Examples include:

\begin{itemize}
\tightlist
\item
  Price of each brand
\item
  Package size
\item
  Sugar content or nutritional attributes
\item
  Promotional indicators
\item
  Brand-specific features
\end{itemize}

Alternative-specific multinomial logit (AS-MNL) models allow us to include these variables directly, providing richer managerial insight into how brand attributes drive choice.

In this chapter, you will learn how to:

\begin{itemize}
\tightlist
\item
  Work with long-format choice data
\item
  Split alternative-specific data correctly into training and test samples
\item
  Estimate an alternative-specific MNL model
\item
  Evaluate model fit and classification performance
\item
  Interpret predicted probabilities and marginal effects in a marketing context
\end{itemize}

Throughout the chapter, we will use the \textbf{yogurt} dataset.

\begin{center}\rule{0.5\linewidth}{0.5pt}\end{center}

\section{The Yogurt Choice Data}\label{the-yogurt-choice-data}

The yogurt dataset records consumer brand choices in repeated choice situations. Each row represents \textbf{one alternative within one choice situation}, not a single consumer.

Key implications:

\begin{itemize}
\tightlist
\item
  Each choice situation appears multiple times (once per brand)
\item
  Exactly one alternative is chosen per choice set
\item
  Many predictors vary across brands within the same choice set
\end{itemize}

This ``long'' structure is required for alternative-specific MNL models and differs from the wide-format data used earlier in the course.

\begin{center}\rule{0.5\linewidth}{0.5pt}\end{center}

\section{Preparing the Data for Modeling}\label{preparing-the-data-for-modeling}

\subsection{Why Splitting Is Different for Choice Data}\label{why-splitting-is-different-for-choice-data}

With alternative-specific data, we \textbf{cannot} randomly split rows into training and test sets. Doing so would break apart choice sets and contaminate model evaluation.

Instead, we must split at the \textbf{choice-set level}, ensuring that all rows belonging to the same choice situation stay together.

\subsection{Creating Training and Test Samples}\label{creating-training-and-test-samples}

We still use the \texttt{splitsample()} function from the \texttt{MKT4320BGSU} package, which supports group-level splitting. Whereas before we didn't use several parameters, will will use them for alternative specific MNL.

Usage:

\begin{itemize}
\tightlist
\item
  \texttt{splitsample(data,\ outcome\ =\ NULL,\ group\ =\ NULL,\ choice\ =\ NULL,\ alt\ =\ NULL,}\strut \\
  \texttt{p\ =\ 0.75,\ seed\ =\ 4320)}
\item
  where:

  \begin{itemize}
  \tightlist
  \item
    \texttt{data} is the data frame to split, in long-format.
  \item
    \texttt{outcome} is NOT (USUALLY) USED FOR ALTERNATIVE SPECIFIC MNL
  \item
    \texttt{group} is the grouping variable (e.g., choice situation id or respondent id). If provided, splitting is done at the group level. Required for alternative specific MNL.
  \item
    \texttt{choice} is the 0/1 (or TRUE/FALSE) indicator for the chosen alternative. Used only when \texttt{group} is provided. Required for alternative specific MNL.
  \item
    \texttt{alt} is the optional alternative label/ID. Used with \texttt{choice} to stratify at the group level. Required for alternative specific MNL.
  \item
    \texttt{p} is the proportion of observations to place in the training set. Must be strictly between 0 and 1. Default is 0.75.
  \item
    \texttt{seed} is the random seed for reproducibility. Default is 4320.
  \end{itemize}
\end{itemize}

Before, we were interested in the \texttt{\$train} and \texttt{\$test} data frames. Now, we are interested in the \texttt{train.mdata} and \texttt{test.mdata} objects that are saved. They are in the format needed for the using \texttt{mlogit} (see below). However, to avoid a console error, you'll access the a slightly different way.

\begin{Shaded}
\begin{Highlighting}[]
\NormalTok{sp }\OtherTok{\textless{}{-}} \FunctionTok{splitsample}\NormalTok{(}\AttributeTok{data =}\NormalTok{ yogurt, }\AttributeTok{group =} \StringTok{"id"}\NormalTok{, }\AttributeTok{choice =} \StringTok{"choice"}\NormalTok{, }\AttributeTok{alt =} \StringTok{"brand"}\NormalTok{)}

\NormalTok{train }\OtherTok{\textless{}{-}}\NormalTok{ sp[[}\StringTok{"train.mdata"}\NormalTok{]]}
\NormalTok{test  }\OtherTok{\textless{}{-}}\NormalTok{ sp[[}\StringTok{"test.mdata"}\NormalTok{]]}
\end{Highlighting}
\end{Shaded}

At this point:

\begin{itemize}
\tightlist
\item
  \texttt{train} contains complete choice sets for model estimation
\item
  \texttt{test} contains unseen choice sets for out-of-sample evaluation
\end{itemize}

\begin{center}\rule{0.5\linewidth}{0.5pt}\end{center}

\section{Specifying an Alternative-Specific MNL Model}\label{specifying-an-alternative-specific-mnl-model}

In an alternative-specific MNL model:

\begin{itemize}
\tightlist
\item
  Case-specific variables enter once
\item
  Alternative-specific variables enter as brand-varying predictors
\end{itemize}

We use the \texttt{mlogit} function from the \texttt{mlogit} package to estimate the model. We separate the alternative specific from the case specific variables with a \texttt{\textbar{}}. Alternative specific come first, then the case specific. We can use the base R \texttt{summary()} function to get the raw log-odds estimates.

\begin{Shaded}
\begin{Highlighting}[]
\FunctionTok{library}\NormalTok{(mlogit)}
\NormalTok{as\_mnl\_fit }\OtherTok{\textless{}{-}} \FunctionTok{mlogit}\NormalTok{(choice }\SpecialCharTok{\textasciitilde{}}\NormalTok{ price }\SpecialCharTok{+}\NormalTok{ feat }\SpecialCharTok{|}\NormalTok{ income, }\AttributeTok{data =}\NormalTok{ train)}
\FunctionTok{summary}\NormalTok{(as\_mnl\_fit)}
\end{Highlighting}
\end{Shaded}

\begin{verbatim}

Call:
mlogit(formula = choice ~ price + feat | income, data = train, 
    method = "nr")

Frequencies of alternatives:choice
  Dannon   Hiland   Weight  Yoplait 
0.401988 0.029818 0.229155 0.339039 

nr method
8 iterations, 0h:0m:0s 
g'(-H)^-1g = 0.000171 
successive function values within tolerance limits 

Coefficients :
                      Estimate Std. Error  z-value  Pr(>|z|)    
(Intercept):Hiland   0.7587200  0.5677111   1.3365  0.181401    
(Intercept):Weight  -0.0263906  0.2078931  -0.1269  0.898986    
(Intercept):Yoplait -3.9886941  0.2679762 -14.8845 < 2.2e-16 ***
price               -0.4424450  0.0295572 -14.9691 < 2.2e-16 ***
feat                 0.4230830  0.1491240   2.8371  0.004552 ** 
income:Hiland       -0.1081164  0.0149201  -7.2464 4.281e-13 ***
income:Weight       -0.0114764  0.0037707  -3.0436  0.002338 ** 
income:Yoplait       0.0729207  0.0040281  18.1030 < 2.2e-16 ***
---
Signif. codes:  0 '***' 0.001 '**' 0.01 '*' 0.05 '.' 0.1 ' ' 1

Log-Likelihood: -1618.4
McFadden R^2:  0.23972 
Likelihood ratio test : chisq = 1020.6 (p.value = < 2.22e-16)
\end{verbatim}

Interpretation notes:

\begin{itemize}
\tightlist
\item
  Coefficients reflect changes in \textbf{relative utility}
\item
  Signs and magnitudes should be interpreted in marketing terms
\item
  Alternative-specific variables capture within-choice substitution effects
\end{itemize}

\begin{center}\rule{0.5\linewidth}{0.5pt}\end{center}

\section{Evaluating Model Performance}\label{evaluating-model-performance}

\subsection{Model Fit and Coefficients}\label{model-fit-and-coefficients}

We use the \texttt{eval\_as\_mnl()} function from the \texttt{MKT4320BGSU} package to obtain fit statistics, coefficients (both log-odds and odds ratio), and classification diagnostics.

Usage:

\begin{itemize}
\tightlist
\item
  \texttt{eval\_as\_mnl(model,\ digits\ =\ 4,\ ft\ =\ FALSE,\ newdata\ =\ NULL,}\strut \\
  \texttt{label\_model\ =\ "Model\ data",\ label\_newdata\ =\ "New\ data",\ class\_digits\ =\ 3)}
\item
  where:

  \begin{itemize}
  \tightlist
  \item
    \texttt{model} is a fitted mlogit model.
  \item
    \texttt{digits} is an integer; decimals to round coefficient and fit results (default 4).
  \item
    \texttt{ft} is logical; if TRUE, return coefficient and classification tables as flextable objects (default FALSE).
  \item
    \texttt{newdata} is an optional \texttt{dfidx} object (e.g., \texttt{test.mdata}) for an additional classification matrix. If NULL, only the training-data matrix is produced.
  \item
    \texttt{label\_model} is a character string label for the training-data classification matrix (default ``Model data'').
  \item
    \texttt{label\_newdata} is a character string label for the newdata classification matrix (default ``New data'').
  \item
    \texttt{class\_digits} is an integer; decimals to round classification results (default 3).
  \end{itemize}
\end{itemize}

Key outputs include:

\begin{itemize}
\tightlist
\item
  Log-likelihood \(\chi^2\) test
\item
  McFadden's pseudo \(R^2\)
\item
  Odds ratios for interpretation
\item
  Classification accuracy and diagnostics
\end{itemize}

\begin{Shaded}
\begin{Highlighting}[]
\NormalTok{as\_eval }\OtherTok{\textless{}{-}} \FunctionTok{eval\_as\_mnl}\NormalTok{(as\_mnl\_fit, }\AttributeTok{ft =} \ConstantTok{TRUE}\NormalTok{, }\AttributeTok{newdata =}\NormalTok{ test)}
\NormalTok{as\_eval}\SpecialCharTok{$}\NormalTok{coef\_table}
\end{Highlighting}
\end{Shaded}

\global\setlength{\Oldarrayrulewidth}{\arrayrulewidth}

\global\setlength{\Oldtabcolsep}{\tabcolsep}

\setlength{\tabcolsep}{2pt}

\renewcommand*{\arraystretch}{1.5}



\providecommand{\ascline}[3]{\noalign{\global\arrayrulewidth #1}\arrayrulecolor[HTML]{#2}\cline{#3}}

\begin{longtable}[c]{|p{1.50in}|p{0.82in}|p{0.75in}|p{0.85in}|p{0.89in}|p{0.78in}}



\ascline{1.5pt}{666666}{1-6}

\multicolumn{6}{>{\raggedright}m{\dimexpr 5.59in+10\tabcolsep}}{\textcolor[HTML]{000000}{\fontsize{11}{11}\selectfont{\global\setmainfont{Arial}{LR\ chi2\ (5)\ =\ 1020.5649;\ p\ <\ 0.0001}}}} \\

\ascline{1.5pt}{666666}{1-6}



\multicolumn{6}{>{\raggedright}m{\dimexpr 5.59in+10\tabcolsep}}{\textcolor[HTML]{000000}{\fontsize{11}{11}\selectfont{\global\setmainfont{Arial}{McFadden's\ Pseudo\ R-square\ =\ 0.2397}}}} \\

\ascline{1.5pt}{666666}{1-6}



\multicolumn{1}{>{\raggedright}m{\dimexpr 1.5in+0\tabcolsep}}{\textcolor[HTML]{000000}{\fontsize{11}{11}\selectfont{\global\setmainfont{Arial}{term}}}} & \multicolumn{1}{>{\raggedleft}m{\dimexpr 0.82in+0\tabcolsep}}{\textcolor[HTML]{000000}{\fontsize{11}{11}\selectfont{\global\setmainfont{Arial}{logodds}}}} & \multicolumn{1}{>{\raggedleft}m{\dimexpr 0.75in+0\tabcolsep}}{\textcolor[HTML]{000000}{\fontsize{11}{11}\selectfont{\global\setmainfont{Arial}{OR}}}} & \multicolumn{1}{>{\raggedleft}m{\dimexpr 0.85in+0\tabcolsep}}{\textcolor[HTML]{000000}{\fontsize{11}{11}\selectfont{\global\setmainfont{Arial}{std.error}}}} & \multicolumn{1}{>{\raggedleft}m{\dimexpr 0.89in+0\tabcolsep}}{\textcolor[HTML]{000000}{\fontsize{11}{11}\selectfont{\global\setmainfont{Arial}{statistic}}}} & \multicolumn{1}{>{\raggedleft}m{\dimexpr 0.78in+0\tabcolsep}}{\textcolor[HTML]{000000}{\fontsize{11}{11}\selectfont{\global\setmainfont{Arial}{p.value}}}} \\

\ascline{1.5pt}{666666}{1-6}\endfirsthead 

\ascline{1.5pt}{666666}{1-6}

\multicolumn{6}{>{\raggedright}m{\dimexpr 5.59in+10\tabcolsep}}{\textcolor[HTML]{000000}{\fontsize{11}{11}\selectfont{\global\setmainfont{Arial}{LR\ chi2\ (5)\ =\ 1020.5649;\ p\ <\ 0.0001}}}} \\

\ascline{1.5pt}{666666}{1-6}



\multicolumn{6}{>{\raggedright}m{\dimexpr 5.59in+10\tabcolsep}}{\textcolor[HTML]{000000}{\fontsize{11}{11}\selectfont{\global\setmainfont{Arial}{McFadden's\ Pseudo\ R-square\ =\ 0.2397}}}} \\

\ascline{1.5pt}{666666}{1-6}



\multicolumn{1}{>{\raggedright}m{\dimexpr 1.5in+0\tabcolsep}}{\textcolor[HTML]{000000}{\fontsize{11}{11}\selectfont{\global\setmainfont{Arial}{term}}}} & \multicolumn{1}{>{\raggedleft}m{\dimexpr 0.82in+0\tabcolsep}}{\textcolor[HTML]{000000}{\fontsize{11}{11}\selectfont{\global\setmainfont{Arial}{logodds}}}} & \multicolumn{1}{>{\raggedleft}m{\dimexpr 0.75in+0\tabcolsep}}{\textcolor[HTML]{000000}{\fontsize{11}{11}\selectfont{\global\setmainfont{Arial}{OR}}}} & \multicolumn{1}{>{\raggedleft}m{\dimexpr 0.85in+0\tabcolsep}}{\textcolor[HTML]{000000}{\fontsize{11}{11}\selectfont{\global\setmainfont{Arial}{std.error}}}} & \multicolumn{1}{>{\raggedleft}m{\dimexpr 0.89in+0\tabcolsep}}{\textcolor[HTML]{000000}{\fontsize{11}{11}\selectfont{\global\setmainfont{Arial}{statistic}}}} & \multicolumn{1}{>{\raggedleft}m{\dimexpr 0.78in+0\tabcolsep}}{\textcolor[HTML]{000000}{\fontsize{11}{11}\selectfont{\global\setmainfont{Arial}{p.value}}}} \\

\ascline{1.5pt}{666666}{1-6}\endhead



\multicolumn{1}{>{\raggedright}m{\dimexpr 1.5in+0\tabcolsep}}{\textcolor[HTML]{000000}{\fontsize{11}{11}\selectfont{\global\setmainfont{Arial}{(Intercept):Hiland}}}} & \multicolumn{1}{>{\raggedleft}m{\dimexpr 0.82in+0\tabcolsep}}{\textcolor[HTML]{000000}{\fontsize{11}{11}\selectfont{\global\setmainfont{Arial}{0.7587}}}} & \multicolumn{1}{>{\raggedleft}m{\dimexpr 0.75in+0\tabcolsep}}{\textcolor[HTML]{000000}{\fontsize{11}{11}\selectfont{\global\setmainfont{Arial}{2.1355}}}} & \multicolumn{1}{>{\raggedleft}m{\dimexpr 0.85in+0\tabcolsep}}{\textcolor[HTML]{000000}{\fontsize{11}{11}\selectfont{\global\setmainfont{Arial}{0.5677}}}} & \multicolumn{1}{>{\raggedleft}m{\dimexpr 0.89in+0\tabcolsep}}{\textcolor[HTML]{000000}{\fontsize{11}{11}\selectfont{\global\setmainfont{Arial}{1.3365}}}} & \multicolumn{1}{>{\raggedleft}m{\dimexpr 0.78in+0\tabcolsep}}{\textcolor[HTML]{000000}{\fontsize{11}{11}\selectfont{\global\setmainfont{Arial}{0.1814}}}} \\





\multicolumn{1}{>{\raggedright}m{\dimexpr 1.5in+0\tabcolsep}}{\textcolor[HTML]{000000}{\fontsize{11}{11}\selectfont{\global\setmainfont{Arial}{(Intercept):Weight}}}} & \multicolumn{1}{>{\raggedleft}m{\dimexpr 0.82in+0\tabcolsep}}{\textcolor[HTML]{000000}{\fontsize{11}{11}\selectfont{\global\setmainfont{Arial}{-0.0264}}}} & \multicolumn{1}{>{\raggedleft}m{\dimexpr 0.75in+0\tabcolsep}}{\textcolor[HTML]{000000}{\fontsize{11}{11}\selectfont{\global\setmainfont{Arial}{0.9740}}}} & \multicolumn{1}{>{\raggedleft}m{\dimexpr 0.85in+0\tabcolsep}}{\textcolor[HTML]{000000}{\fontsize{11}{11}\selectfont{\global\setmainfont{Arial}{0.2079}}}} & \multicolumn{1}{>{\raggedleft}m{\dimexpr 0.89in+0\tabcolsep}}{\textcolor[HTML]{000000}{\fontsize{11}{11}\selectfont{\global\setmainfont{Arial}{-0.1269}}}} & \multicolumn{1}{>{\raggedleft}m{\dimexpr 0.78in+0\tabcolsep}}{\textcolor[HTML]{000000}{\fontsize{11}{11}\selectfont{\global\setmainfont{Arial}{0.8990}}}} \\





\multicolumn{1}{>{\raggedright}m{\dimexpr 1.5in+0\tabcolsep}}{\textcolor[HTML]{000000}{\fontsize{11}{11}\selectfont{\global\setmainfont{Arial}{(Intercept):Yoplait}}}} & \multicolumn{1}{>{\raggedleft}m{\dimexpr 0.82in+0\tabcolsep}}{\textcolor[HTML]{000000}{\fontsize{11}{11}\selectfont{\global\setmainfont{Arial}{-3.9887}}}} & \multicolumn{1}{>{\raggedleft}m{\dimexpr 0.75in+0\tabcolsep}}{\textcolor[HTML]{000000}{\fontsize{11}{11}\selectfont{\global\setmainfont{Arial}{0.0185}}}} & \multicolumn{1}{>{\raggedleft}m{\dimexpr 0.85in+0\tabcolsep}}{\textcolor[HTML]{000000}{\fontsize{11}{11}\selectfont{\global\setmainfont{Arial}{0.2680}}}} & \multicolumn{1}{>{\raggedleft}m{\dimexpr 0.89in+0\tabcolsep}}{\textcolor[HTML]{000000}{\fontsize{11}{11}\selectfont{\global\setmainfont{Arial}{-14.8845}}}} & \multicolumn{1}{>{\raggedleft}m{\dimexpr 0.78in+0\tabcolsep}}{\textcolor[HTML]{000000}{\fontsize{11}{11}\selectfont{\global\setmainfont{Arial}{0.0000}}}} \\





\multicolumn{1}{>{\raggedright}m{\dimexpr 1.5in+0\tabcolsep}}{\textcolor[HTML]{000000}{\fontsize{11}{11}\selectfont{\global\setmainfont{Arial}{price}}}} & \multicolumn{1}{>{\raggedleft}m{\dimexpr 0.82in+0\tabcolsep}}{\textcolor[HTML]{000000}{\fontsize{11}{11}\selectfont{\global\setmainfont{Arial}{-0.4424}}}} & \multicolumn{1}{>{\raggedleft}m{\dimexpr 0.75in+0\tabcolsep}}{\textcolor[HTML]{000000}{\fontsize{11}{11}\selectfont{\global\setmainfont{Arial}{0.6425}}}} & \multicolumn{1}{>{\raggedleft}m{\dimexpr 0.85in+0\tabcolsep}}{\textcolor[HTML]{000000}{\fontsize{11}{11}\selectfont{\global\setmainfont{Arial}{0.0296}}}} & \multicolumn{1}{>{\raggedleft}m{\dimexpr 0.89in+0\tabcolsep}}{\textcolor[HTML]{000000}{\fontsize{11}{11}\selectfont{\global\setmainfont{Arial}{-14.9691}}}} & \multicolumn{1}{>{\raggedleft}m{\dimexpr 0.78in+0\tabcolsep}}{\textcolor[HTML]{000000}{\fontsize{11}{11}\selectfont{\global\setmainfont{Arial}{0.0000}}}} \\





\multicolumn{1}{>{\raggedright}m{\dimexpr 1.5in+0\tabcolsep}}{\textcolor[HTML]{000000}{\fontsize{11}{11}\selectfont{\global\setmainfont{Arial}{feat}}}} & \multicolumn{1}{>{\raggedleft}m{\dimexpr 0.82in+0\tabcolsep}}{\textcolor[HTML]{000000}{\fontsize{11}{11}\selectfont{\global\setmainfont{Arial}{0.4231}}}} & \multicolumn{1}{>{\raggedleft}m{\dimexpr 0.75in+0\tabcolsep}}{\textcolor[HTML]{000000}{\fontsize{11}{11}\selectfont{\global\setmainfont{Arial}{1.5267}}}} & \multicolumn{1}{>{\raggedleft}m{\dimexpr 0.85in+0\tabcolsep}}{\textcolor[HTML]{000000}{\fontsize{11}{11}\selectfont{\global\setmainfont{Arial}{0.1491}}}} & \multicolumn{1}{>{\raggedleft}m{\dimexpr 0.89in+0\tabcolsep}}{\textcolor[HTML]{000000}{\fontsize{11}{11}\selectfont{\global\setmainfont{Arial}{2.8371}}}} & \multicolumn{1}{>{\raggedleft}m{\dimexpr 0.78in+0\tabcolsep}}{\textcolor[HTML]{000000}{\fontsize{11}{11}\selectfont{\global\setmainfont{Arial}{0.0046}}}} \\





\multicolumn{1}{>{\raggedright}m{\dimexpr 1.5in+0\tabcolsep}}{\textcolor[HTML]{000000}{\fontsize{11}{11}\selectfont{\global\setmainfont{Arial}{income:Hiland}}}} & \multicolumn{1}{>{\raggedleft}m{\dimexpr 0.82in+0\tabcolsep}}{\textcolor[HTML]{000000}{\fontsize{11}{11}\selectfont{\global\setmainfont{Arial}{-0.1081}}}} & \multicolumn{1}{>{\raggedleft}m{\dimexpr 0.75in+0\tabcolsep}}{\textcolor[HTML]{000000}{\fontsize{11}{11}\selectfont{\global\setmainfont{Arial}{0.8975}}}} & \multicolumn{1}{>{\raggedleft}m{\dimexpr 0.85in+0\tabcolsep}}{\textcolor[HTML]{000000}{\fontsize{11}{11}\selectfont{\global\setmainfont{Arial}{0.0149}}}} & \multicolumn{1}{>{\raggedleft}m{\dimexpr 0.89in+0\tabcolsep}}{\textcolor[HTML]{000000}{\fontsize{11}{11}\selectfont{\global\setmainfont{Arial}{-7.2464}}}} & \multicolumn{1}{>{\raggedleft}m{\dimexpr 0.78in+0\tabcolsep}}{\textcolor[HTML]{000000}{\fontsize{11}{11}\selectfont{\global\setmainfont{Arial}{0.0000}}}} \\





\multicolumn{1}{>{\raggedright}m{\dimexpr 1.5in+0\tabcolsep}}{\textcolor[HTML]{000000}{\fontsize{11}{11}\selectfont{\global\setmainfont{Arial}{income:Weight}}}} & \multicolumn{1}{>{\raggedleft}m{\dimexpr 0.82in+0\tabcolsep}}{\textcolor[HTML]{000000}{\fontsize{11}{11}\selectfont{\global\setmainfont{Arial}{-0.0115}}}} & \multicolumn{1}{>{\raggedleft}m{\dimexpr 0.75in+0\tabcolsep}}{\textcolor[HTML]{000000}{\fontsize{11}{11}\selectfont{\global\setmainfont{Arial}{0.9886}}}} & \multicolumn{1}{>{\raggedleft}m{\dimexpr 0.85in+0\tabcolsep}}{\textcolor[HTML]{000000}{\fontsize{11}{11}\selectfont{\global\setmainfont{Arial}{0.0038}}}} & \multicolumn{1}{>{\raggedleft}m{\dimexpr 0.89in+0\tabcolsep}}{\textcolor[HTML]{000000}{\fontsize{11}{11}\selectfont{\global\setmainfont{Arial}{-3.0436}}}} & \multicolumn{1}{>{\raggedleft}m{\dimexpr 0.78in+0\tabcolsep}}{\textcolor[HTML]{000000}{\fontsize{11}{11}\selectfont{\global\setmainfont{Arial}{0.0023}}}} \\





\multicolumn{1}{>{\raggedright}m{\dimexpr 1.5in+0\tabcolsep}}{\textcolor[HTML]{000000}{\fontsize{11}{11}\selectfont{\global\setmainfont{Arial}{income:Yoplait}}}} & \multicolumn{1}{>{\raggedleft}m{\dimexpr 0.82in+0\tabcolsep}}{\textcolor[HTML]{000000}{\fontsize{11}{11}\selectfont{\global\setmainfont{Arial}{0.0729}}}} & \multicolumn{1}{>{\raggedleft}m{\dimexpr 0.75in+0\tabcolsep}}{\textcolor[HTML]{000000}{\fontsize{11}{11}\selectfont{\global\setmainfont{Arial}{1.0756}}}} & \multicolumn{1}{>{\raggedleft}m{\dimexpr 0.85in+0\tabcolsep}}{\textcolor[HTML]{000000}{\fontsize{11}{11}\selectfont{\global\setmainfont{Arial}{0.0040}}}} & \multicolumn{1}{>{\raggedleft}m{\dimexpr 0.89in+0\tabcolsep}}{\textcolor[HTML]{000000}{\fontsize{11}{11}\selectfont{\global\setmainfont{Arial}{18.1030}}}} & \multicolumn{1}{>{\raggedleft}m{\dimexpr 0.78in+0\tabcolsep}}{\textcolor[HTML]{000000}{\fontsize{11}{11}\selectfont{\global\setmainfont{Arial}{0.0000}}}} \\

\ascline{1.5pt}{666666}{1-6}



\end{longtable}



\arrayrulecolor[HTML]{000000}

\global\setlength{\arrayrulewidth}{\Oldarrayrulewidth}

\global\setlength{\tabcolsep}{\Oldtabcolsep}

\renewcommand*{\arraystretch}{1}

\begin{Shaded}
\begin{Highlighting}[]
\NormalTok{as\_eval}\SpecialCharTok{$}\NormalTok{classify\_model}
\end{Highlighting}
\end{Shaded}

\global\setlength{\Oldarrayrulewidth}{\arrayrulewidth}

\global\setlength{\Oldtabcolsep}{\tabcolsep}

\setlength{\tabcolsep}{2pt}

\renewcommand*{\arraystretch}{1.5}



\providecommand{\ascline}[3]{\noalign{\global\arrayrulewidth #1}\arrayrulecolor[HTML]{#2}\cline{#3}}

\begin{longtable}[c]{|p{1.57in}|p{0.82in}|p{0.72in}|p{0.76in}|p{0.75in}|p{0.63in}}



\ascline{1.5pt}{666666}{1-6}

\multicolumn{6}{>{\raggedright}m{\dimexpr 5.24in+10\tabcolsep}}{\textcolor[HTML]{000000}{\fontsize{11}{11}\selectfont{\global\setmainfont{Arial}{Classification\ Matrix\ -\ Model\ data}}}} \\

\ascline{1.5pt}{666666}{1-6}



\multicolumn{6}{>{\raggedright}m{\dimexpr 5.24in+10\tabcolsep}}{\textcolor[HTML]{000000}{\fontsize{11}{11}\selectfont{\global\setmainfont{Arial}{Accuracy\ =\ 0.621}}}} \\

\ascline{1.5pt}{666666}{1-6}



\multicolumn{6}{>{\raggedright}m{\dimexpr 5.24in+10\tabcolsep}}{\textcolor[HTML]{000000}{\fontsize{11}{11}\selectfont{\global\setmainfont{Arial}{PCC\ =\ 0.330}}}} \\

\ascline{1.5pt}{666666}{1-6}



\multicolumn{1}{>{\raggedright}m{\dimexpr 1.57in+0\tabcolsep}}{\textcolor[HTML]{000000}{\fontsize{11}{11}\selectfont{\global\setmainfont{Arial}{}}}} & \multicolumn{4}{>{\raggedright}m{\dimexpr 3.05in+6\tabcolsep}}{\textcolor[HTML]{000000}{\fontsize{11}{11}\selectfont{\global\setmainfont{Arial}{Reference}}}} & \multicolumn{1}{>{\raggedright}m{\dimexpr 0.63in+0\tabcolsep}}{\textcolor[HTML]{000000}{\fontsize{11}{11}\selectfont{\global\setmainfont{Arial}{}}}} \\

\ascline{1.5pt}{666666}{1-6}



\multicolumn{1}{>{\raggedright}m{\dimexpr 1.57in+0\tabcolsep}}{\textcolor[HTML]{000000}{\fontsize{11}{11}\selectfont{\global\setmainfont{Arial}{Predicted}}}} & \multicolumn{1}{>{\raggedright}m{\dimexpr 0.82in+0\tabcolsep}}{\textcolor[HTML]{000000}{\fontsize{11}{11}\selectfont{\global\setmainfont{Arial}{Dannon}}}} & \multicolumn{1}{>{\raggedright}m{\dimexpr 0.72in+0\tabcolsep}}{\textcolor[HTML]{000000}{\fontsize{11}{11}\selectfont{\global\setmainfont{Arial}{Hiland}}}} & \multicolumn{1}{>{\raggedright}m{\dimexpr 0.76in+0\tabcolsep}}{\textcolor[HTML]{000000}{\fontsize{11}{11}\selectfont{\global\setmainfont{Arial}{Weight}}}} & \multicolumn{1}{>{\raggedright}m{\dimexpr 0.75in+0\tabcolsep}}{\textcolor[HTML]{000000}{\fontsize{11}{11}\selectfont{\global\setmainfont{Arial}{Yoplait}}}} & \multicolumn{1}{>{\raggedright}m{\dimexpr 0.63in+0\tabcolsep}}{\textcolor[HTML]{000000}{\fontsize{11}{11}\selectfont{\global\setmainfont{Arial}{Total}}}} \\

\ascline{1.5pt}{666666}{1-6}\endfirsthead 

\ascline{1.5pt}{666666}{1-6}

\multicolumn{6}{>{\raggedright}m{\dimexpr 5.24in+10\tabcolsep}}{\textcolor[HTML]{000000}{\fontsize{11}{11}\selectfont{\global\setmainfont{Arial}{Classification\ Matrix\ -\ Model\ data}}}} \\

\ascline{1.5pt}{666666}{1-6}



\multicolumn{6}{>{\raggedright}m{\dimexpr 5.24in+10\tabcolsep}}{\textcolor[HTML]{000000}{\fontsize{11}{11}\selectfont{\global\setmainfont{Arial}{Accuracy\ =\ 0.621}}}} \\

\ascline{1.5pt}{666666}{1-6}



\multicolumn{6}{>{\raggedright}m{\dimexpr 5.24in+10\tabcolsep}}{\textcolor[HTML]{000000}{\fontsize{11}{11}\selectfont{\global\setmainfont{Arial}{PCC\ =\ 0.330}}}} \\

\ascline{1.5pt}{666666}{1-6}



\multicolumn{1}{>{\raggedright}m{\dimexpr 1.57in+0\tabcolsep}}{\textcolor[HTML]{000000}{\fontsize{11}{11}\selectfont{\global\setmainfont{Arial}{}}}} & \multicolumn{4}{>{\raggedright}m{\dimexpr 3.05in+6\tabcolsep}}{\textcolor[HTML]{000000}{\fontsize{11}{11}\selectfont{\global\setmainfont{Arial}{Reference}}}} & \multicolumn{1}{>{\raggedright}m{\dimexpr 0.63in+0\tabcolsep}}{\textcolor[HTML]{000000}{\fontsize{11}{11}\selectfont{\global\setmainfont{Arial}{}}}} \\

\ascline{1.5pt}{666666}{1-6}



\multicolumn{1}{>{\raggedright}m{\dimexpr 1.57in+0\tabcolsep}}{\textcolor[HTML]{000000}{\fontsize{11}{11}\selectfont{\global\setmainfont{Arial}{Predicted}}}} & \multicolumn{1}{>{\raggedright}m{\dimexpr 0.82in+0\tabcolsep}}{\textcolor[HTML]{000000}{\fontsize{11}{11}\selectfont{\global\setmainfont{Arial}{Dannon}}}} & \multicolumn{1}{>{\raggedright}m{\dimexpr 0.72in+0\tabcolsep}}{\textcolor[HTML]{000000}{\fontsize{11}{11}\selectfont{\global\setmainfont{Arial}{Hiland}}}} & \multicolumn{1}{>{\raggedright}m{\dimexpr 0.76in+0\tabcolsep}}{\textcolor[HTML]{000000}{\fontsize{11}{11}\selectfont{\global\setmainfont{Arial}{Weight}}}} & \multicolumn{1}{>{\raggedright}m{\dimexpr 0.75in+0\tabcolsep}}{\textcolor[HTML]{000000}{\fontsize{11}{11}\selectfont{\global\setmainfont{Arial}{Yoplait}}}} & \multicolumn{1}{>{\raggedright}m{\dimexpr 0.63in+0\tabcolsep}}{\textcolor[HTML]{000000}{\fontsize{11}{11}\selectfont{\global\setmainfont{Arial}{Total}}}} \\

\ascline{1.5pt}{666666}{1-6}\endhead



\multicolumn{1}{>{\raggedright}m{\dimexpr 1.57in+0\tabcolsep}}{\textcolor[HTML]{000000}{\fontsize{11}{11}\selectfont{\global\setmainfont{Arial}{Dannon}}}} & \multicolumn{1}{>{\raggedright}m{\dimexpr 0.82in+0\tabcolsep}}{\textcolor[HTML]{000000}{\fontsize{11}{11}\selectfont{\global\setmainfont{Arial}{577}}}} & \multicolumn{1}{>{\raggedright}m{\dimexpr 0.72in+0\tabcolsep}}{\textcolor[HTML]{000000}{\fontsize{11}{11}\selectfont{\global\setmainfont{Arial}{39}}}} & \multicolumn{1}{>{\raggedright}m{\dimexpr 0.76in+0\tabcolsep}}{\textcolor[HTML]{000000}{\fontsize{11}{11}\selectfont{\global\setmainfont{Arial}{324}}}} & \multicolumn{1}{>{\raggedright}m{\dimexpr 0.75in+0\tabcolsep}}{\textcolor[HTML]{000000}{\fontsize{11}{11}\selectfont{\global\setmainfont{Arial}{97}}}} & \multicolumn{1}{>{\raggedright}m{\dimexpr 0.63in+0\tabcolsep}}{\textcolor[HTML]{000000}{\fontsize{11}{11}\selectfont{\global\setmainfont{Arial}{1037}}}} \\





\multicolumn{1}{>{\raggedright}m{\dimexpr 1.57in+0\tabcolsep}}{\textcolor[HTML]{000000}{\fontsize{11}{11}\selectfont{\global\setmainfont{Arial}{Hiland}}}} & \multicolumn{1}{>{\raggedright}m{\dimexpr 0.82in+0\tabcolsep}}{\textcolor[HTML]{000000}{\fontsize{11}{11}\selectfont{\global\setmainfont{Arial}{1}}}} & \multicolumn{1}{>{\raggedright}m{\dimexpr 0.72in+0\tabcolsep}}{\textcolor[HTML]{000000}{\fontsize{11}{11}\selectfont{\global\setmainfont{Arial}{12}}}} & \multicolumn{1}{>{\raggedright}m{\dimexpr 0.76in+0\tabcolsep}}{\textcolor[HTML]{000000}{\fontsize{11}{11}\selectfont{\global\setmainfont{Arial}{0}}}} & \multicolumn{1}{>{\raggedright}m{\dimexpr 0.75in+0\tabcolsep}}{\textcolor[HTML]{000000}{\fontsize{11}{11}\selectfont{\global\setmainfont{Arial}{2}}}} & \multicolumn{1}{>{\raggedright}m{\dimexpr 0.63in+0\tabcolsep}}{\textcolor[HTML]{000000}{\fontsize{11}{11}\selectfont{\global\setmainfont{Arial}{15}}}} \\





\multicolumn{1}{>{\raggedright}m{\dimexpr 1.57in+0\tabcolsep}}{\textcolor[HTML]{000000}{\fontsize{11}{11}\selectfont{\global\setmainfont{Arial}{Weight}}}} & \multicolumn{1}{>{\raggedright}m{\dimexpr 0.82in+0\tabcolsep}}{\textcolor[HTML]{000000}{\fontsize{11}{11}\selectfont{\global\setmainfont{Arial}{18}}}} & \multicolumn{1}{>{\raggedright}m{\dimexpr 0.72in+0\tabcolsep}}{\textcolor[HTML]{000000}{\fontsize{11}{11}\selectfont{\global\setmainfont{Arial}{2}}}} & \multicolumn{1}{>{\raggedright}m{\dimexpr 0.76in+0\tabcolsep}}{\textcolor[HTML]{000000}{\fontsize{11}{11}\selectfont{\global\setmainfont{Arial}{38}}}} & \multicolumn{1}{>{\raggedright}m{\dimexpr 0.75in+0\tabcolsep}}{\textcolor[HTML]{000000}{\fontsize{11}{11}\selectfont{\global\setmainfont{Arial}{18}}}} & \multicolumn{1}{>{\raggedright}m{\dimexpr 0.63in+0\tabcolsep}}{\textcolor[HTML]{000000}{\fontsize{11}{11}\selectfont{\global\setmainfont{Arial}{76}}}} \\





\multicolumn{1}{>{\raggedright}m{\dimexpr 1.57in+0\tabcolsep}}{\textcolor[HTML]{000000}{\fontsize{11}{11}\selectfont{\global\setmainfont{Arial}{Yoplait}}}} & \multicolumn{1}{>{\raggedright}m{\dimexpr 0.82in+0\tabcolsep}}{\textcolor[HTML]{000000}{\fontsize{11}{11}\selectfont{\global\setmainfont{Arial}{132}}}} & \multicolumn{1}{>{\raggedright}m{\dimexpr 0.72in+0\tabcolsep}}{\textcolor[HTML]{000000}{\fontsize{11}{11}\selectfont{\global\setmainfont{Arial}{1}}}} & \multicolumn{1}{>{\raggedright}m{\dimexpr 0.76in+0\tabcolsep}}{\textcolor[HTML]{000000}{\fontsize{11}{11}\selectfont{\global\setmainfont{Arial}{53}}}} & \multicolumn{1}{>{\raggedright}m{\dimexpr 0.75in+0\tabcolsep}}{\textcolor[HTML]{000000}{\fontsize{11}{11}\selectfont{\global\setmainfont{Arial}{497}}}} & \multicolumn{1}{>{\raggedright}m{\dimexpr 0.63in+0\tabcolsep}}{\textcolor[HTML]{000000}{\fontsize{11}{11}\selectfont{\global\setmainfont{Arial}{683}}}} \\





\multicolumn{1}{>{\raggedright}m{\dimexpr 1.57in+0\tabcolsep}}{\textcolor[HTML]{000000}{\fontsize{11}{11}\selectfont{\global\setmainfont{Arial}{Total}}}} & \multicolumn{1}{>{\raggedright}m{\dimexpr 0.82in+0\tabcolsep}}{\textcolor[HTML]{000000}{\fontsize{11}{11}\selectfont{\global\setmainfont{Arial}{728}}}} & \multicolumn{1}{>{\raggedright}m{\dimexpr 0.72in+0\tabcolsep}}{\textcolor[HTML]{000000}{\fontsize{11}{11}\selectfont{\global\setmainfont{Arial}{54}}}} & \multicolumn{1}{>{\raggedright}m{\dimexpr 0.76in+0\tabcolsep}}{\textcolor[HTML]{000000}{\fontsize{11}{11}\selectfont{\global\setmainfont{Arial}{415}}}} & \multicolumn{1}{>{\raggedright}m{\dimexpr 0.75in+0\tabcolsep}}{\textcolor[HTML]{000000}{\fontsize{11}{11}\selectfont{\global\setmainfont{Arial}{614}}}} & \multicolumn{1}{>{\raggedright}m{\dimexpr 0.63in+0\tabcolsep}}{\textcolor[HTML]{000000}{\fontsize{11}{11}\selectfont{\global\setmainfont{Arial}{1811}}}} \\





\multicolumn{1}{>{\raggedright}m{\dimexpr 1.57in+0\tabcolsep}}{\textcolor[HTML]{000000}{\fontsize{11}{11}\selectfont{\global\setmainfont{Arial}{Statistics\ by\ Class:}}}} & \multicolumn{1}{>{\raggedright}m{\dimexpr 0.82in+0\tabcolsep}}{\textcolor[HTML]{000000}{\fontsize{11}{11}\selectfont{\global\setmainfont{Arial}{}}}} & \multicolumn{1}{>{\raggedright}m{\dimexpr 0.72in+0\tabcolsep}}{\textcolor[HTML]{000000}{\fontsize{11}{11}\selectfont{\global\setmainfont{Arial}{}}}} & \multicolumn{1}{>{\raggedright}m{\dimexpr 0.76in+0\tabcolsep}}{\textcolor[HTML]{000000}{\fontsize{11}{11}\selectfont{\global\setmainfont{Arial}{}}}} & \multicolumn{1}{>{\raggedright}m{\dimexpr 0.75in+0\tabcolsep}}{\textcolor[HTML]{000000}{\fontsize{11}{11}\selectfont{\global\setmainfont{Arial}{}}}} & \multicolumn{1}{>{\raggedright}m{\dimexpr 0.63in+0\tabcolsep}}{\textcolor[HTML]{000000}{\fontsize{11}{11}\selectfont{\global\setmainfont{Arial}{}}}} \\





\multicolumn{1}{>{\raggedright}m{\dimexpr 1.57in+0\tabcolsep}}{\textcolor[HTML]{000000}{\fontsize{11}{11}\selectfont{\global\setmainfont{Arial}{Sensitivity}}}} & \multicolumn{1}{>{\raggedright}m{\dimexpr 0.82in+0\tabcolsep}}{\textcolor[HTML]{000000}{\fontsize{11}{11}\selectfont{\global\setmainfont{Arial}{0.793}}}} & \multicolumn{1}{>{\raggedright}m{\dimexpr 0.72in+0\tabcolsep}}{\textcolor[HTML]{000000}{\fontsize{11}{11}\selectfont{\global\setmainfont{Arial}{0.222}}}} & \multicolumn{1}{>{\raggedright}m{\dimexpr 0.76in+0\tabcolsep}}{\textcolor[HTML]{000000}{\fontsize{11}{11}\selectfont{\global\setmainfont{Arial}{0.092}}}} & \multicolumn{1}{>{\raggedright}m{\dimexpr 0.75in+0\tabcolsep}}{\textcolor[HTML]{000000}{\fontsize{11}{11}\selectfont{\global\setmainfont{Arial}{0.809}}}} & \multicolumn{1}{>{\raggedright}m{\dimexpr 0.63in+0\tabcolsep}}{\textcolor[HTML]{000000}{\fontsize{11}{11}\selectfont{\global\setmainfont{Arial}{}}}} \\





\multicolumn{1}{>{\raggedright}m{\dimexpr 1.57in+0\tabcolsep}}{\textcolor[HTML]{000000}{\fontsize{11}{11}\selectfont{\global\setmainfont{Arial}{Specificity}}}} & \multicolumn{1}{>{\raggedright}m{\dimexpr 0.82in+0\tabcolsep}}{\textcolor[HTML]{000000}{\fontsize{11}{11}\selectfont{\global\setmainfont{Arial}{0.575}}}} & \multicolumn{1}{>{\raggedright}m{\dimexpr 0.72in+0\tabcolsep}}{\textcolor[HTML]{000000}{\fontsize{11}{11}\selectfont{\global\setmainfont{Arial}{0.998}}}} & \multicolumn{1}{>{\raggedright}m{\dimexpr 0.76in+0\tabcolsep}}{\textcolor[HTML]{000000}{\fontsize{11}{11}\selectfont{\global\setmainfont{Arial}{0.973}}}} & \multicolumn{1}{>{\raggedright}m{\dimexpr 0.75in+0\tabcolsep}}{\textcolor[HTML]{000000}{\fontsize{11}{11}\selectfont{\global\setmainfont{Arial}{0.845}}}} & \multicolumn{1}{>{\raggedright}m{\dimexpr 0.63in+0\tabcolsep}}{\textcolor[HTML]{000000}{\fontsize{11}{11}\selectfont{\global\setmainfont{Arial}{}}}} \\





\multicolumn{1}{>{\raggedright}m{\dimexpr 1.57in+0\tabcolsep}}{\textcolor[HTML]{000000}{\fontsize{11}{11}\selectfont{\global\setmainfont{Arial}{Precision}}}} & \multicolumn{1}{>{\raggedright}m{\dimexpr 0.82in+0\tabcolsep}}{\textcolor[HTML]{000000}{\fontsize{11}{11}\selectfont{\global\setmainfont{Arial}{0.556}}}} & \multicolumn{1}{>{\raggedright}m{\dimexpr 0.72in+0\tabcolsep}}{\textcolor[HTML]{000000}{\fontsize{11}{11}\selectfont{\global\setmainfont{Arial}{0.800}}}} & \multicolumn{1}{>{\raggedright}m{\dimexpr 0.76in+0\tabcolsep}}{\textcolor[HTML]{000000}{\fontsize{11}{11}\selectfont{\global\setmainfont{Arial}{0.500}}}} & \multicolumn{1}{>{\raggedright}m{\dimexpr 0.75in+0\tabcolsep}}{\textcolor[HTML]{000000}{\fontsize{11}{11}\selectfont{\global\setmainfont{Arial}{0.728}}}} & \multicolumn{1}{>{\raggedright}m{\dimexpr 0.63in+0\tabcolsep}}{\textcolor[HTML]{000000}{\fontsize{11}{11}\selectfont{\global\setmainfont{Arial}{}}}} \\

\ascline{1.5pt}{666666}{1-6}



\end{longtable}



\arrayrulecolor[HTML]{000000}

\global\setlength{\arrayrulewidth}{\Oldarrayrulewidth}

\global\setlength{\tabcolsep}{\Oldtabcolsep}

\renewcommand*{\arraystretch}{1}

\begin{Shaded}
\begin{Highlighting}[]
\NormalTok{as\_eval}\SpecialCharTok{$}\NormalTok{classify\_newdata}
\end{Highlighting}
\end{Shaded}

\global\setlength{\Oldarrayrulewidth}{\arrayrulewidth}

\global\setlength{\Oldtabcolsep}{\tabcolsep}

\setlength{\tabcolsep}{2pt}

\renewcommand*{\arraystretch}{1.5}



\providecommand{\ascline}[3]{\noalign{\global\arrayrulewidth #1}\arrayrulecolor[HTML]{#2}\cline{#3}}

\begin{longtable}[c]{|p{1.57in}|p{0.82in}|p{0.72in}|p{0.76in}|p{0.75in}|p{0.62in}}



\ascline{1.5pt}{666666}{1-6}

\multicolumn{6}{>{\raggedright}m{\dimexpr 5.24in+10\tabcolsep}}{\textcolor[HTML]{000000}{\fontsize{11}{11}\selectfont{\global\setmainfont{Arial}{Classification\ Matrix\ -\ New\ data}}}} \\

\ascline{1.5pt}{666666}{1-6}



\multicolumn{6}{>{\raggedright}m{\dimexpr 5.24in+10\tabcolsep}}{\textcolor[HTML]{000000}{\fontsize{11}{11}\selectfont{\global\setmainfont{Arial}{Accuracy\ =\ 0.607}}}} \\

\ascline{1.5pt}{666666}{1-6}



\multicolumn{6}{>{\raggedright}m{\dimexpr 5.24in+10\tabcolsep}}{\textcolor[HTML]{000000}{\fontsize{11}{11}\selectfont{\global\setmainfont{Arial}{PCC\ =\ 0.331}}}} \\

\ascline{1.5pt}{666666}{1-6}



\multicolumn{1}{>{\raggedright}m{\dimexpr 1.57in+0\tabcolsep}}{\textcolor[HTML]{000000}{\fontsize{11}{11}\selectfont{\global\setmainfont{Arial}{}}}} & \multicolumn{4}{>{\raggedright}m{\dimexpr 3.05in+6\tabcolsep}}{\textcolor[HTML]{000000}{\fontsize{11}{11}\selectfont{\global\setmainfont{Arial}{Reference}}}} & \multicolumn{1}{>{\raggedright}m{\dimexpr 0.62in+0\tabcolsep}}{\textcolor[HTML]{000000}{\fontsize{11}{11}\selectfont{\global\setmainfont{Arial}{}}}} \\

\ascline{1.5pt}{666666}{1-6}



\multicolumn{1}{>{\raggedright}m{\dimexpr 1.57in+0\tabcolsep}}{\textcolor[HTML]{000000}{\fontsize{11}{11}\selectfont{\global\setmainfont{Arial}{Predicted}}}} & \multicolumn{1}{>{\raggedright}m{\dimexpr 0.82in+0\tabcolsep}}{\textcolor[HTML]{000000}{\fontsize{11}{11}\selectfont{\global\setmainfont{Arial}{Dannon}}}} & \multicolumn{1}{>{\raggedright}m{\dimexpr 0.72in+0\tabcolsep}}{\textcolor[HTML]{000000}{\fontsize{11}{11}\selectfont{\global\setmainfont{Arial}{Hiland}}}} & \multicolumn{1}{>{\raggedright}m{\dimexpr 0.76in+0\tabcolsep}}{\textcolor[HTML]{000000}{\fontsize{11}{11}\selectfont{\global\setmainfont{Arial}{Weight}}}} & \multicolumn{1}{>{\raggedright}m{\dimexpr 0.75in+0\tabcolsep}}{\textcolor[HTML]{000000}{\fontsize{11}{11}\selectfont{\global\setmainfont{Arial}{Yoplait}}}} & \multicolumn{1}{>{\raggedright}m{\dimexpr 0.62in+0\tabcolsep}}{\textcolor[HTML]{000000}{\fontsize{11}{11}\selectfont{\global\setmainfont{Arial}{Total}}}} \\

\ascline{1.5pt}{666666}{1-6}\endfirsthead 

\ascline{1.5pt}{666666}{1-6}

\multicolumn{6}{>{\raggedright}m{\dimexpr 5.24in+10\tabcolsep}}{\textcolor[HTML]{000000}{\fontsize{11}{11}\selectfont{\global\setmainfont{Arial}{Classification\ Matrix\ -\ New\ data}}}} \\

\ascline{1.5pt}{666666}{1-6}



\multicolumn{6}{>{\raggedright}m{\dimexpr 5.24in+10\tabcolsep}}{\textcolor[HTML]{000000}{\fontsize{11}{11}\selectfont{\global\setmainfont{Arial}{Accuracy\ =\ 0.607}}}} \\

\ascline{1.5pt}{666666}{1-6}



\multicolumn{6}{>{\raggedright}m{\dimexpr 5.24in+10\tabcolsep}}{\textcolor[HTML]{000000}{\fontsize{11}{11}\selectfont{\global\setmainfont{Arial}{PCC\ =\ 0.331}}}} \\

\ascline{1.5pt}{666666}{1-6}



\multicolumn{1}{>{\raggedright}m{\dimexpr 1.57in+0\tabcolsep}}{\textcolor[HTML]{000000}{\fontsize{11}{11}\selectfont{\global\setmainfont{Arial}{}}}} & \multicolumn{4}{>{\raggedright}m{\dimexpr 3.05in+6\tabcolsep}}{\textcolor[HTML]{000000}{\fontsize{11}{11}\selectfont{\global\setmainfont{Arial}{Reference}}}} & \multicolumn{1}{>{\raggedright}m{\dimexpr 0.62in+0\tabcolsep}}{\textcolor[HTML]{000000}{\fontsize{11}{11}\selectfont{\global\setmainfont{Arial}{}}}} \\

\ascline{1.5pt}{666666}{1-6}



\multicolumn{1}{>{\raggedright}m{\dimexpr 1.57in+0\tabcolsep}}{\textcolor[HTML]{000000}{\fontsize{11}{11}\selectfont{\global\setmainfont{Arial}{Predicted}}}} & \multicolumn{1}{>{\raggedright}m{\dimexpr 0.82in+0\tabcolsep}}{\textcolor[HTML]{000000}{\fontsize{11}{11}\selectfont{\global\setmainfont{Arial}{Dannon}}}} & \multicolumn{1}{>{\raggedright}m{\dimexpr 0.72in+0\tabcolsep}}{\textcolor[HTML]{000000}{\fontsize{11}{11}\selectfont{\global\setmainfont{Arial}{Hiland}}}} & \multicolumn{1}{>{\raggedright}m{\dimexpr 0.76in+0\tabcolsep}}{\textcolor[HTML]{000000}{\fontsize{11}{11}\selectfont{\global\setmainfont{Arial}{Weight}}}} & \multicolumn{1}{>{\raggedright}m{\dimexpr 0.75in+0\tabcolsep}}{\textcolor[HTML]{000000}{\fontsize{11}{11}\selectfont{\global\setmainfont{Arial}{Yoplait}}}} & \multicolumn{1}{>{\raggedright}m{\dimexpr 0.62in+0\tabcolsep}}{\textcolor[HTML]{000000}{\fontsize{11}{11}\selectfont{\global\setmainfont{Arial}{Total}}}} \\

\ascline{1.5pt}{666666}{1-6}\endhead



\multicolumn{1}{>{\raggedright}m{\dimexpr 1.57in+0\tabcolsep}}{\textcolor[HTML]{000000}{\fontsize{11}{11}\selectfont{\global\setmainfont{Arial}{Dannon}}}} & \multicolumn{1}{>{\raggedright}m{\dimexpr 0.82in+0\tabcolsep}}{\textcolor[HTML]{000000}{\fontsize{11}{11}\selectfont{\global\setmainfont{Arial}{199}}}} & \multicolumn{1}{>{\raggedright}m{\dimexpr 0.72in+0\tabcolsep}}{\textcolor[HTML]{000000}{\fontsize{11}{11}\selectfont{\global\setmainfont{Arial}{14}}}} & \multicolumn{1}{>{\raggedright}m{\dimexpr 0.76in+0\tabcolsep}}{\textcolor[HTML]{000000}{\fontsize{11}{11}\selectfont{\global\setmainfont{Arial}{104}}}} & \multicolumn{1}{>{\raggedright}m{\dimexpr 0.75in+0\tabcolsep}}{\textcolor[HTML]{000000}{\fontsize{11}{11}\selectfont{\global\setmainfont{Arial}{38}}}} & \multicolumn{1}{>{\raggedright}m{\dimexpr 0.62in+0\tabcolsep}}{\textcolor[HTML]{000000}{\fontsize{11}{11}\selectfont{\global\setmainfont{Arial}{355}}}} \\





\multicolumn{1}{>{\raggedright}m{\dimexpr 1.57in+0\tabcolsep}}{\textcolor[HTML]{000000}{\fontsize{11}{11}\selectfont{\global\setmainfont{Arial}{Hiland}}}} & \multicolumn{1}{>{\raggedright}m{\dimexpr 0.82in+0\tabcolsep}}{\textcolor[HTML]{000000}{\fontsize{11}{11}\selectfont{\global\setmainfont{Arial}{2}}}} & \multicolumn{1}{>{\raggedright}m{\dimexpr 0.72in+0\tabcolsep}}{\textcolor[HTML]{000000}{\fontsize{11}{11}\selectfont{\global\setmainfont{Arial}{2}}}} & \multicolumn{1}{>{\raggedright}m{\dimexpr 0.76in+0\tabcolsep}}{\textcolor[HTML]{000000}{\fontsize{11}{11}\selectfont{\global\setmainfont{Arial}{1}}}} & \multicolumn{1}{>{\raggedright}m{\dimexpr 0.75in+0\tabcolsep}}{\textcolor[HTML]{000000}{\fontsize{11}{11}\selectfont{\global\setmainfont{Arial}{1}}}} & \multicolumn{1}{>{\raggedright}m{\dimexpr 0.62in+0\tabcolsep}}{\textcolor[HTML]{000000}{\fontsize{11}{11}\selectfont{\global\setmainfont{Arial}{6}}}} \\





\multicolumn{1}{>{\raggedright}m{\dimexpr 1.57in+0\tabcolsep}}{\textcolor[HTML]{000000}{\fontsize{11}{11}\selectfont{\global\setmainfont{Arial}{Weight}}}} & \multicolumn{1}{>{\raggedright}m{\dimexpr 0.82in+0\tabcolsep}}{\textcolor[HTML]{000000}{\fontsize{11}{11}\selectfont{\global\setmainfont{Arial}{8}}}} & \multicolumn{1}{>{\raggedright}m{\dimexpr 0.72in+0\tabcolsep}}{\textcolor[HTML]{000000}{\fontsize{11}{11}\selectfont{\global\setmainfont{Arial}{1}}}} & \multicolumn{1}{>{\raggedright}m{\dimexpr 0.76in+0\tabcolsep}}{\textcolor[HTML]{000000}{\fontsize{11}{11}\selectfont{\global\setmainfont{Arial}{12}}}} & \multicolumn{1}{>{\raggedright}m{\dimexpr 0.75in+0\tabcolsep}}{\textcolor[HTML]{000000}{\fontsize{11}{11}\selectfont{\global\setmainfont{Arial}{13}}}} & \multicolumn{1}{>{\raggedright}m{\dimexpr 0.62in+0\tabcolsep}}{\textcolor[HTML]{000000}{\fontsize{11}{11}\selectfont{\global\setmainfont{Arial}{34}}}} \\





\multicolumn{1}{>{\raggedright}m{\dimexpr 1.57in+0\tabcolsep}}{\textcolor[HTML]{000000}{\fontsize{11}{11}\selectfont{\global\setmainfont{Arial}{Yoplait}}}} & \multicolumn{1}{>{\raggedright}m{\dimexpr 0.82in+0\tabcolsep}}{\textcolor[HTML]{000000}{\fontsize{11}{11}\selectfont{\global\setmainfont{Arial}{33}}}} & \multicolumn{1}{>{\raggedright}m{\dimexpr 0.72in+0\tabcolsep}}{\textcolor[HTML]{000000}{\fontsize{11}{11}\selectfont{\global\setmainfont{Arial}{0}}}} & \multicolumn{1}{>{\raggedright}m{\dimexpr 0.76in+0\tabcolsep}}{\textcolor[HTML]{000000}{\fontsize{11}{11}\selectfont{\global\setmainfont{Arial}{21}}}} & \multicolumn{1}{>{\raggedright}m{\dimexpr 0.75in+0\tabcolsep}}{\textcolor[HTML]{000000}{\fontsize{11}{11}\selectfont{\global\setmainfont{Arial}{152}}}} & \multicolumn{1}{>{\raggedright}m{\dimexpr 0.62in+0\tabcolsep}}{\textcolor[HTML]{000000}{\fontsize{11}{11}\selectfont{\global\setmainfont{Arial}{206}}}} \\





\multicolumn{1}{>{\raggedright}m{\dimexpr 1.57in+0\tabcolsep}}{\textcolor[HTML]{000000}{\fontsize{11}{11}\selectfont{\global\setmainfont{Arial}{Total}}}} & \multicolumn{1}{>{\raggedright}m{\dimexpr 0.82in+0\tabcolsep}}{\textcolor[HTML]{000000}{\fontsize{11}{11}\selectfont{\global\setmainfont{Arial}{242}}}} & \multicolumn{1}{>{\raggedright}m{\dimexpr 0.72in+0\tabcolsep}}{\textcolor[HTML]{000000}{\fontsize{11}{11}\selectfont{\global\setmainfont{Arial}{17}}}} & \multicolumn{1}{>{\raggedright}m{\dimexpr 0.76in+0\tabcolsep}}{\textcolor[HTML]{000000}{\fontsize{11}{11}\selectfont{\global\setmainfont{Arial}{138}}}} & \multicolumn{1}{>{\raggedright}m{\dimexpr 0.75in+0\tabcolsep}}{\textcolor[HTML]{000000}{\fontsize{11}{11}\selectfont{\global\setmainfont{Arial}{204}}}} & \multicolumn{1}{>{\raggedright}m{\dimexpr 0.62in+0\tabcolsep}}{\textcolor[HTML]{000000}{\fontsize{11}{11}\selectfont{\global\setmainfont{Arial}{601}}}} \\





\multicolumn{1}{>{\raggedright}m{\dimexpr 1.57in+0\tabcolsep}}{\textcolor[HTML]{000000}{\fontsize{11}{11}\selectfont{\global\setmainfont{Arial}{Statistics\ by\ Class:}}}} & \multicolumn{1}{>{\raggedright}m{\dimexpr 0.82in+0\tabcolsep}}{\textcolor[HTML]{000000}{\fontsize{11}{11}\selectfont{\global\setmainfont{Arial}{}}}} & \multicolumn{1}{>{\raggedright}m{\dimexpr 0.72in+0\tabcolsep}}{\textcolor[HTML]{000000}{\fontsize{11}{11}\selectfont{\global\setmainfont{Arial}{}}}} & \multicolumn{1}{>{\raggedright}m{\dimexpr 0.76in+0\tabcolsep}}{\textcolor[HTML]{000000}{\fontsize{11}{11}\selectfont{\global\setmainfont{Arial}{}}}} & \multicolumn{1}{>{\raggedright}m{\dimexpr 0.75in+0\tabcolsep}}{\textcolor[HTML]{000000}{\fontsize{11}{11}\selectfont{\global\setmainfont{Arial}{}}}} & \multicolumn{1}{>{\raggedright}m{\dimexpr 0.62in+0\tabcolsep}}{\textcolor[HTML]{000000}{\fontsize{11}{11}\selectfont{\global\setmainfont{Arial}{}}}} \\





\multicolumn{1}{>{\raggedright}m{\dimexpr 1.57in+0\tabcolsep}}{\textcolor[HTML]{000000}{\fontsize{11}{11}\selectfont{\global\setmainfont{Arial}{Sensitivity}}}} & \multicolumn{1}{>{\raggedright}m{\dimexpr 0.82in+0\tabcolsep}}{\textcolor[HTML]{000000}{\fontsize{11}{11}\selectfont{\global\setmainfont{Arial}{0.822}}}} & \multicolumn{1}{>{\raggedright}m{\dimexpr 0.72in+0\tabcolsep}}{\textcolor[HTML]{000000}{\fontsize{11}{11}\selectfont{\global\setmainfont{Arial}{0.118}}}} & \multicolumn{1}{>{\raggedright}m{\dimexpr 0.76in+0\tabcolsep}}{\textcolor[HTML]{000000}{\fontsize{11}{11}\selectfont{\global\setmainfont{Arial}{0.087}}}} & \multicolumn{1}{>{\raggedright}m{\dimexpr 0.75in+0\tabcolsep}}{\textcolor[HTML]{000000}{\fontsize{11}{11}\selectfont{\global\setmainfont{Arial}{0.745}}}} & \multicolumn{1}{>{\raggedright}m{\dimexpr 0.62in+0\tabcolsep}}{\textcolor[HTML]{000000}{\fontsize{11}{11}\selectfont{\global\setmainfont{Arial}{}}}} \\





\multicolumn{1}{>{\raggedright}m{\dimexpr 1.57in+0\tabcolsep}}{\textcolor[HTML]{000000}{\fontsize{11}{11}\selectfont{\global\setmainfont{Arial}{Specificity}}}} & \multicolumn{1}{>{\raggedright}m{\dimexpr 0.82in+0\tabcolsep}}{\textcolor[HTML]{000000}{\fontsize{11}{11}\selectfont{\global\setmainfont{Arial}{0.565}}}} & \multicolumn{1}{>{\raggedright}m{\dimexpr 0.72in+0\tabcolsep}}{\textcolor[HTML]{000000}{\fontsize{11}{11}\selectfont{\global\setmainfont{Arial}{0.993}}}} & \multicolumn{1}{>{\raggedright}m{\dimexpr 0.76in+0\tabcolsep}}{\textcolor[HTML]{000000}{\fontsize{11}{11}\selectfont{\global\setmainfont{Arial}{0.952}}}} & \multicolumn{1}{>{\raggedright}m{\dimexpr 0.75in+0\tabcolsep}}{\textcolor[HTML]{000000}{\fontsize{11}{11}\selectfont{\global\setmainfont{Arial}{0.864}}}} & \multicolumn{1}{>{\raggedright}m{\dimexpr 0.62in+0\tabcolsep}}{\textcolor[HTML]{000000}{\fontsize{11}{11}\selectfont{\global\setmainfont{Arial}{}}}} \\





\multicolumn{1}{>{\raggedright}m{\dimexpr 1.57in+0\tabcolsep}}{\textcolor[HTML]{000000}{\fontsize{11}{11}\selectfont{\global\setmainfont{Arial}{Precision}}}} & \multicolumn{1}{>{\raggedright}m{\dimexpr 0.82in+0\tabcolsep}}{\textcolor[HTML]{000000}{\fontsize{11}{11}\selectfont{\global\setmainfont{Arial}{0.561}}}} & \multicolumn{1}{>{\raggedright}m{\dimexpr 0.72in+0\tabcolsep}}{\textcolor[HTML]{000000}{\fontsize{11}{11}\selectfont{\global\setmainfont{Arial}{0.333}}}} & \multicolumn{1}{>{\raggedright}m{\dimexpr 0.76in+0\tabcolsep}}{\textcolor[HTML]{000000}{\fontsize{11}{11}\selectfont{\global\setmainfont{Arial}{0.353}}}} & \multicolumn{1}{>{\raggedright}m{\dimexpr 0.75in+0\tabcolsep}}{\textcolor[HTML]{000000}{\fontsize{11}{11}\selectfont{\global\setmainfont{Arial}{0.738}}}} & \multicolumn{1}{>{\raggedright}m{\dimexpr 0.62in+0\tabcolsep}}{\textcolor[HTML]{000000}{\fontsize{11}{11}\selectfont{\global\setmainfont{Arial}{}}}} \\

\ascline{1.5pt}{666666}{1-6}



\end{longtable}



\arrayrulecolor[HTML]{000000}

\global\setlength{\arrayrulewidth}{\Oldarrayrulewidth}

\global\setlength{\tabcolsep}{\Oldtabcolsep}

\renewcommand*{\arraystretch}{1}

\subsection{Classification Performance}\label{classification-performance-1}

Classification is evaluated at the \textbf{choice-set level}:

\begin{itemize}
\tightlist
\item
  The predicted brand is the one with the highest predicted probability
\item
  Accuracy reflects correct brand predictions
\item
  PCC provides a baseline comparison
\end{itemize}

This approach mirrors how managers think about predicting actual consumer choices.

\begin{center}\rule{0.5\linewidth}{0.5pt}\end{center}

\section{Predicted Probabilities and Marginal Effects}\label{predicted-probabilities-and-marginal-effects}

\subsection{Why Predicted Probabilities Matter}\label{why-predicted-probabilities-matter}

Coefficients are not always intuitive. Predicted probabilities translate the model into outcomes managers care about:

\begin{itemize}
\tightlist
\item
  Market shares
\item
  Brand switching
\item
  Competitive responses
\end{itemize}

\subsection{Why Marginal Effects Are Useful}\label{why-marginal-effects-are-useful}

Marginal effects quantify how much choice probabilities change in response to a small change in an attribute, holding everything else constant. Marginal effects can be computed in two common ways:

\begin{itemize}
\tightlist
\item
  \textbf{At observed values (Average Marginal Effects, AME)}\\
  Marginal effects are calculated for each observation using its actual attribute values and then averaged.
\item
  \textbf{At means (Marginal Effects at the Mean, MEM)}\\
  Marginal effects are calculated at a single ``average'' profile, where each attribute is set to its sample mean.
\end{itemize}

Both approaches summarize how sensitive choice probabilities are to changes in attributes, but they differ in interpretation.

Marginal effects \textbf{at observed values}:

\begin{itemize}
\tightlist
\item
  Reflect the full distribution of the data
\item
  Avoid relying on a potentially unrealistic ``average consumer''
\item
  Are often preferred for descriptive and policy interpretation
\end{itemize}

Marginal effects \textbf{at means}:

\begin{itemize}
\tightlist
\item
  Are easier to reproduce by hand or with software defaults
\item
  Provide a clear, single reference point
\item
  Can be useful for illustrating model mechanics and comparing effects across variables
\end{itemize}

The marginal effects tables can therefore answer questions such as:

\begin{itemize}
\tightlist
\item
  ``On average, how does a \$1 increase in price affect brand choice?''
\item
  ``How would choice probabilities change for a typical consumer if an attribute increased slightly?''
\item
  ``Which brands are most sensitive to changes in a specific attribute?''
\end{itemize}

In practice, the choice between observed values and means depends on the goal of the analysis. For interpretation and real-world impact, average marginal effects at observed values are often preferred. For teaching, demonstration, or simplified comparisons, marginal effects at means can be equally informative.

\subsection{\texorpdfstring{The \texttt{pp\_as\_mnl()} Function}{The pp\_as\_mnl() Function}}\label{the-pp_as_mnl-function}

For both case-specific and alternative-specific predictors, we use the \texttt{pp\_as\_mnl()} function from the \texttt{MKT4320BGSU} package to get both predicted probabilities and marginal effects.

Usage:

\begin{itemize}
\tightlist
\item
  \texttt{pp\_as\_mnl(model,focal\_var,\ focal\_type\ =\ c("auto",\ "alt",\ "case"),}\strut \\
  \texttt{grid\_n\ =\ 25,\ digits\ =\ 4,\ ft\ =\ FALSE,\ marginal\ =\ TRUE,}\strut \\
  \texttt{me\_method\ =\ c("observed",\ "means"),\ me\_step\ =\ 1)}
\item
  where:

  \begin{itemize}
  \tightlist
  \item
    \texttt{model} is a fitted mlogit model.
  \item
    \texttt{focal\_var} is a character string name of the focal variable.
  \item
    \texttt{focal\_type} is a character string; one of ``case'', ``alt'', or ``auto'' (default = ``auto'').
  \item
    \texttt{grid\_n} is an integer; number of points used to construct the grid of focal values for predicted probability plots when the focal variable is continuous (default = 25).
  \item
    \texttt{digits} is an integer; rounding for numeric output (default = 4).
  \item
    \texttt{ft} is logical; if TRUE, return tables as flextable objects (default = FALSE).
  \item
    \texttt{marginal} is logical; if TRUE, compute marginal effects (default = TRUE).
  \item
    \texttt{me\_method} is a character string; one of ``observed'' AME or ``means'' (default = ``observed'').
  \item
    \texttt{me\_step} is numeric; finite-difference step size for AME (default = 1).
  \end{itemize}
\end{itemize}

\subsection{Case-Specific Predictors}\label{case-specific-predictors}

We first examine how a consumer-level variable affects brand choice probabilities.

\begin{Shaded}
\begin{Highlighting}[]
\NormalTok{pp\_income }\OtherTok{\textless{}{-}} \FunctionTok{pp\_as\_mnl}\NormalTok{(as\_mnl\_fit, }\AttributeTok{focal\_var =} \StringTok{"income"}\NormalTok{, }\AttributeTok{ft =} \ConstantTok{TRUE}\NormalTok{, }\AttributeTok{me\_method=}\StringTok{"means"}\NormalTok{)}
\NormalTok{pp\_income}\SpecialCharTok{$}\NormalTok{me\_table}
\end{Highlighting}
\end{Shaded}

\global\setlength{\Oldarrayrulewidth}{\arrayrulewidth}

\global\setlength{\Oldtabcolsep}{\tabcolsep}

\setlength{\tabcolsep}{2pt}

\renewcommand*{\arraystretch}{1.5}



\providecommand{\ascline}[3]{\noalign{\global\arrayrulewidth #1}\arrayrulecolor[HTML]{#2}\cline{#3}}

\begin{longtable}[c]{|p{0.82in}|p{0.80in}|p{0.80in}|p{0.75in}}



\ascline{1.5pt}{666666}{1-4}

\multicolumn{4}{>{\raggedleft}m{\dimexpr 3.18in+6\tabcolsep}}{\textcolor[HTML]{000000}{\fontsize{11}{11}\selectfont{\global\setmainfont{Arial}{Marginal\ effects\ for\ }}}\textcolor[HTML]{000000}{\fontsize{10}{10}\selectfont{\global\setmainfont{Arial}{\textbf{\textit{income}}}}}\textcolor[HTML]{000000}{\fontsize{11}{11}\selectfont{\global\setmainfont{Arial}{\ (at\ means)}}}} \\

\ascline{1.5pt}{666666}{1-4}



\multicolumn{1}{>{\raggedleft}m{\dimexpr 0.82in+0\tabcolsep}}{\textcolor[HTML]{000000}{\fontsize{11}{11}\selectfont{\global\setmainfont{Arial}{Dannon}}}} & \multicolumn{1}{>{\raggedleft}m{\dimexpr 0.8in+0\tabcolsep}}{\textcolor[HTML]{000000}{\fontsize{11}{11}\selectfont{\global\setmainfont{Arial}{Hiland}}}} & \multicolumn{1}{>{\raggedleft}m{\dimexpr 0.8in+0\tabcolsep}}{\textcolor[HTML]{000000}{\fontsize{11}{11}\selectfont{\global\setmainfont{Arial}{Weight}}}} & \multicolumn{1}{>{\raggedleft}m{\dimexpr 0.75in+0\tabcolsep}}{\textcolor[HTML]{000000}{\fontsize{11}{11}\selectfont{\global\setmainfont{Arial}{Yoplait}}}} \\

\ascline{1.5pt}{666666}{1-4}\endfirsthead 

\ascline{1.5pt}{666666}{1-4}

\multicolumn{4}{>{\raggedleft}m{\dimexpr 3.18in+6\tabcolsep}}{\textcolor[HTML]{000000}{\fontsize{11}{11}\selectfont{\global\setmainfont{Arial}{Marginal\ effects\ for\ }}}\textcolor[HTML]{000000}{\fontsize{10}{10}\selectfont{\global\setmainfont{Arial}{\textbf{\textit{income}}}}}\textcolor[HTML]{000000}{\fontsize{11}{11}\selectfont{\global\setmainfont{Arial}{\ (at\ means)}}}} \\

\ascline{1.5pt}{666666}{1-4}



\multicolumn{1}{>{\raggedleft}m{\dimexpr 0.82in+0\tabcolsep}}{\textcolor[HTML]{000000}{\fontsize{11}{11}\selectfont{\global\setmainfont{Arial}{Dannon}}}} & \multicolumn{1}{>{\raggedleft}m{\dimexpr 0.8in+0\tabcolsep}}{\textcolor[HTML]{000000}{\fontsize{11}{11}\selectfont{\global\setmainfont{Arial}{Hiland}}}} & \multicolumn{1}{>{\raggedleft}m{\dimexpr 0.8in+0\tabcolsep}}{\textcolor[HTML]{000000}{\fontsize{11}{11}\selectfont{\global\setmainfont{Arial}{Weight}}}} & \multicolumn{1}{>{\raggedleft}m{\dimexpr 0.75in+0\tabcolsep}}{\textcolor[HTML]{000000}{\fontsize{11}{11}\selectfont{\global\setmainfont{Arial}{Yoplait}}}} \\

\ascline{1.5pt}{666666}{1-4}\endhead



\multicolumn{1}{>{\raggedleft}m{\dimexpr 0.82in+0\tabcolsep}}{\textcolor[HTML]{000000}{\fontsize{11}{11}\selectfont{\global\setmainfont{Arial}{-0.0073}}}} & \multicolumn{1}{>{\raggedleft}m{\dimexpr 0.8in+0\tabcolsep}}{\textcolor[HTML]{000000}{\fontsize{11}{11}\selectfont{\global\setmainfont{Arial}{-0.0006}}}} & \multicolumn{1}{>{\raggedleft}m{\dimexpr 0.8in+0\tabcolsep}}{\textcolor[HTML]{000000}{\fontsize{11}{11}\selectfont{\global\setmainfont{Arial}{-0.0068}}}} & \multicolumn{1}{>{\raggedleft}m{\dimexpr 0.75in+0\tabcolsep}}{\textcolor[HTML]{000000}{\fontsize{11}{11}\selectfont{\global\setmainfont{Arial}{0.0147}}}} \\

\ascline{1.5pt}{666666}{1-4}



\end{longtable}



\arrayrulecolor[HTML]{000000}

\global\setlength{\arrayrulewidth}{\Oldarrayrulewidth}

\global\setlength{\tabcolsep}{\Oldtabcolsep}

\renewcommand*{\arraystretch}{1}

\begin{Shaded}
\begin{Highlighting}[]
\NormalTok{pp\_income}\SpecialCharTok{$}\NormalTok{pp\_table}
\end{Highlighting}
\end{Shaded}

\global\setlength{\Oldarrayrulewidth}{\arrayrulewidth}

\global\setlength{\Oldtabcolsep}{\tabcolsep}

\setlength{\tabcolsep}{2pt}

\renewcommand*{\arraystretch}{1.5}



\providecommand{\ascline}[3]{\noalign{\global\arrayrulewidth #1}\arrayrulecolor[HTML]{#2}\cline{#3}}

\begin{longtable}[c]{|p{1.06in}|p{0.82in}|p{0.75in}|p{0.76in}|p{0.75in}}



\ascline{1.5pt}{666666}{1-5}

\multicolumn{5}{>{\raggedleft}m{\dimexpr 4.14in+8\tabcolsep}}{\textcolor[HTML]{000000}{\fontsize{11}{11}\selectfont{\global\setmainfont{Arial}{Predicted\ Probability\ Table\ (income)\ -\ Model\ data}}}} \\

\ascline{1.5pt}{666666}{1-5}



\multicolumn{1}{>{\raggedleft}m{\dimexpr 1.06in+0\tabcolsep}}{\textcolor[HTML]{000000}{\fontsize{11}{11}\selectfont{\global\setmainfont{Arial}{focal\_value}}}} & \multicolumn{1}{>{\raggedleft}m{\dimexpr 0.82in+0\tabcolsep}}{\textcolor[HTML]{000000}{\fontsize{11}{11}\selectfont{\global\setmainfont{Arial}{Dannon}}}} & \multicolumn{1}{>{\raggedleft}m{\dimexpr 0.75in+0\tabcolsep}}{\textcolor[HTML]{000000}{\fontsize{11}{11}\selectfont{\global\setmainfont{Arial}{Hiland}}}} & \multicolumn{1}{>{\raggedleft}m{\dimexpr 0.76in+0\tabcolsep}}{\textcolor[HTML]{000000}{\fontsize{11}{11}\selectfont{\global\setmainfont{Arial}{Weight}}}} & \multicolumn{1}{>{\raggedleft}m{\dimexpr 0.75in+0\tabcolsep}}{\textcolor[HTML]{000000}{\fontsize{11}{11}\selectfont{\global\setmainfont{Arial}{Yoplait}}}} \\

\ascline{1.5pt}{666666}{1-5}\endfirsthead 

\ascline{1.5pt}{666666}{1-5}

\multicolumn{5}{>{\raggedleft}m{\dimexpr 4.14in+8\tabcolsep}}{\textcolor[HTML]{000000}{\fontsize{11}{11}\selectfont{\global\setmainfont{Arial}{Predicted\ Probability\ Table\ (income)\ -\ Model\ data}}}} \\

\ascline{1.5pt}{666666}{1-5}



\multicolumn{1}{>{\raggedleft}m{\dimexpr 1.06in+0\tabcolsep}}{\textcolor[HTML]{000000}{\fontsize{11}{11}\selectfont{\global\setmainfont{Arial}{focal\_value}}}} & \multicolumn{1}{>{\raggedleft}m{\dimexpr 0.82in+0\tabcolsep}}{\textcolor[HTML]{000000}{\fontsize{11}{11}\selectfont{\global\setmainfont{Arial}{Dannon}}}} & \multicolumn{1}{>{\raggedleft}m{\dimexpr 0.75in+0\tabcolsep}}{\textcolor[HTML]{000000}{\fontsize{11}{11}\selectfont{\global\setmainfont{Arial}{Hiland}}}} & \multicolumn{1}{>{\raggedleft}m{\dimexpr 0.76in+0\tabcolsep}}{\textcolor[HTML]{000000}{\fontsize{11}{11}\selectfont{\global\setmainfont{Arial}{Weight}}}} & \multicolumn{1}{>{\raggedleft}m{\dimexpr 0.75in+0\tabcolsep}}{\textcolor[HTML]{000000}{\fontsize{11}{11}\selectfont{\global\setmainfont{Arial}{Yoplait}}}} \\

\ascline{1.5pt}{666666}{1-5}\endhead



\multicolumn{5}{>{\raggedright}m{\dimexpr 4.14in+8\tabcolsep}}{\textcolor[HTML]{000000}{\fontsize{11}{11}\selectfont{\global\setmainfont{Arial}{Because\ }}}\textcolor[HTML]{000000}{\fontsize{10}{10}\selectfont{\global\setmainfont{Arial}{\textbf{\textit{income}}}}}\textcolor[HTML]{000000}{\fontsize{11}{11}\selectfont{\global\setmainfont{Arial}{\ is\ continuous,\ the\ values\ shown\ include\ the\ mean\ and\ +/-\ 1\ unit.}}}} \\

\endlastfoot



\multicolumn{1}{>{\raggedleft}m{\dimexpr 1.06in+0\tabcolsep}}{\textcolor[HTML]{000000}{\fontsize{11}{11}\selectfont{\global\setmainfont{Arial}{60.1438}}}} & \multicolumn{1}{>{\raggedleft}m{\dimexpr 0.82in+0\tabcolsep}}{\textcolor[HTML]{000000}{\fontsize{11}{11}\selectfont{\global\setmainfont{Arial}{0.4788}}}} & \multicolumn{1}{>{\raggedleft}m{\dimexpr 0.75in+0\tabcolsep}}{\textcolor[HTML]{000000}{\fontsize{11}{11}\selectfont{\global\setmainfont{Arial}{0.0060}}}} & \multicolumn{1}{>{\raggedleft}m{\dimexpr 0.76in+0\tabcolsep}}{\textcolor[HTML]{000000}{\fontsize{11}{11}\selectfont{\global\setmainfont{Arial}{0.2608}}}} & \multicolumn{1}{>{\raggedleft}m{\dimexpr 0.75in+0\tabcolsep}}{\textcolor[HTML]{000000}{\fontsize{11}{11}\selectfont{\global\setmainfont{Arial}{0.2544}}}} \\





\multicolumn{1}{>{\raggedleft}m{\dimexpr 1.06in+0\tabcolsep}}{\textcolor[HTML]{000000}{\fontsize{11}{11}\selectfont{\global\setmainfont{Arial}{61.1438}}}} & \multicolumn{1}{>{\raggedleft}m{\dimexpr 0.82in+0\tabcolsep}}{\textcolor[HTML]{000000}{\fontsize{11}{11}\selectfont{\global\setmainfont{Arial}{0.4729}}}} & \multicolumn{1}{>{\raggedleft}m{\dimexpr 0.75in+0\tabcolsep}}{\textcolor[HTML]{000000}{\fontsize{11}{11}\selectfont{\global\setmainfont{Arial}{0.0053}}}} & \multicolumn{1}{>{\raggedleft}m{\dimexpr 0.76in+0\tabcolsep}}{\textcolor[HTML]{000000}{\fontsize{11}{11}\selectfont{\global\setmainfont{Arial}{0.2545}}}} & \multicolumn{1}{>{\raggedleft}m{\dimexpr 0.75in+0\tabcolsep}}{\textcolor[HTML]{000000}{\fontsize{11}{11}\selectfont{\global\setmainfont{Arial}{0.2672}}}} \\





\multicolumn{1}{>{\raggedleft}m{\dimexpr 1.06in+0\tabcolsep}}{\textcolor[HTML]{000000}{\fontsize{11}{11}\selectfont{\global\setmainfont{Arial}{62.1438}}}} & \multicolumn{1}{>{\raggedleft}m{\dimexpr 0.82in+0\tabcolsep}}{\textcolor[HTML]{000000}{\fontsize{11}{11}\selectfont{\global\setmainfont{Arial}{0.4667}}}} & \multicolumn{1}{>{\raggedleft}m{\dimexpr 0.75in+0\tabcolsep}}{\textcolor[HTML]{000000}{\fontsize{11}{11}\selectfont{\global\setmainfont{Arial}{0.0047}}}} & \multicolumn{1}{>{\raggedleft}m{\dimexpr 0.76in+0\tabcolsep}}{\textcolor[HTML]{000000}{\fontsize{11}{11}\selectfont{\global\setmainfont{Arial}{0.2482}}}} & \multicolumn{1}{>{\raggedleft}m{\dimexpr 0.75in+0\tabcolsep}}{\textcolor[HTML]{000000}{\fontsize{11}{11}\selectfont{\global\setmainfont{Arial}{0.2804}}}} \\

\ascline{1.5pt}{666666}{1-5}



\end{longtable}



\arrayrulecolor[HTML]{000000}

\global\setlength{\arrayrulewidth}{\Oldarrayrulewidth}

\global\setlength{\tabcolsep}{\Oldtabcolsep}

\renewcommand*{\arraystretch}{1}

\begin{Shaded}
\begin{Highlighting}[]
\NormalTok{pp\_income}\SpecialCharTok{$}\NormalTok{pp\_plot}
\end{Highlighting}
\end{Shaded}

\pandocbounded{\includegraphics[keepaspectratio]{MKT4320_R_Tutorial_files/figure-latex/pp-case-1.pdf}}

\subsection{Alternative-Specific Predictors}\label{alternative-specific-predictors}

Now we examine a brand-specific variable such as price (a continuous variable) and feature (a categorical variable).

\begin{Shaded}
\begin{Highlighting}[]
\NormalTok{pp\_price }\OtherTok{\textless{}{-}} \FunctionTok{pp\_as\_mnl}\NormalTok{(as\_mnl\_fit, }\AttributeTok{focal\_var =} \StringTok{"price"}\NormalTok{, }\AttributeTok{ft=}\ConstantTok{TRUE}\NormalTok{, }\AttributeTok{me\_method=}\StringTok{"means"}\NormalTok{)}
\NormalTok{pp\_price}\SpecialCharTok{$}\NormalTok{me\_table}
\end{Highlighting}
\end{Shaded}

\global\setlength{\Oldarrayrulewidth}{\arrayrulewidth}

\global\setlength{\Oldtabcolsep}{\tabcolsep}

\setlength{\tabcolsep}{2pt}

\renewcommand*{\arraystretch}{1.5}



\providecommand{\ascline}[3]{\noalign{\global\arrayrulewidth #1}\arrayrulecolor[HTML]{#2}\cline{#3}}

\begin{longtable}[c]{|p{1.01in}|p{0.82in}|p{0.80in}|p{0.80in}|p{0.80in}}



\ascline{1.5pt}{666666}{1-5}

\multicolumn{5}{>{\raggedright}m{\dimexpr 4.24in+8\tabcolsep}}{\textcolor[HTML]{000000}{\fontsize{11}{11}\selectfont{\global\setmainfont{Arial}{Marginal\ effects\ for\ }}}\textcolor[HTML]{000000}{\fontsize{10}{10}\selectfont{\global\setmainfont{Arial}{\textbf{\textit{price}}}}}\textcolor[HTML]{000000}{\fontsize{11}{11}\selectfont{\global\setmainfont{Arial}{\ (at\ means)}}}} \\

\ascline{1.5pt}{666666}{1-5}



\multicolumn{1}{>{\raggedright}m{\dimexpr 1.01in+0\tabcolsep}}{\textcolor[HTML]{000000}{\fontsize{11}{11}\selectfont{\global\setmainfont{Arial}{Alternative}}}} & \multicolumn{1}{>{\raggedleft}m{\dimexpr 0.82in+0\tabcolsep}}{\textcolor[HTML]{000000}{\fontsize{11}{11}\selectfont{\global\setmainfont{Arial}{Dannon}}}} & \multicolumn{1}{>{\raggedleft}m{\dimexpr 0.8in+0\tabcolsep}}{\textcolor[HTML]{000000}{\fontsize{11}{11}\selectfont{\global\setmainfont{Arial}{Hiland}}}} & \multicolumn{1}{>{\raggedleft}m{\dimexpr 0.8in+0\tabcolsep}}{\textcolor[HTML]{000000}{\fontsize{11}{11}\selectfont{\global\setmainfont{Arial}{Weight}}}} & \multicolumn{1}{>{\raggedleft}m{\dimexpr 0.8in+0\tabcolsep}}{\textcolor[HTML]{000000}{\fontsize{11}{11}\selectfont{\global\setmainfont{Arial}{Yoplait}}}} \\

\ascline{1.5pt}{666666}{1-5}\endfirsthead 

\ascline{1.5pt}{666666}{1-5}

\multicolumn{5}{>{\raggedright}m{\dimexpr 4.24in+8\tabcolsep}}{\textcolor[HTML]{000000}{\fontsize{11}{11}\selectfont{\global\setmainfont{Arial}{Marginal\ effects\ for\ }}}\textcolor[HTML]{000000}{\fontsize{10}{10}\selectfont{\global\setmainfont{Arial}{\textbf{\textit{price}}}}}\textcolor[HTML]{000000}{\fontsize{11}{11}\selectfont{\global\setmainfont{Arial}{\ (at\ means)}}}} \\

\ascline{1.5pt}{666666}{1-5}



\multicolumn{1}{>{\raggedright}m{\dimexpr 1.01in+0\tabcolsep}}{\textcolor[HTML]{000000}{\fontsize{11}{11}\selectfont{\global\setmainfont{Arial}{Alternative}}}} & \multicolumn{1}{>{\raggedleft}m{\dimexpr 0.82in+0\tabcolsep}}{\textcolor[HTML]{000000}{\fontsize{11}{11}\selectfont{\global\setmainfont{Arial}{Dannon}}}} & \multicolumn{1}{>{\raggedleft}m{\dimexpr 0.8in+0\tabcolsep}}{\textcolor[HTML]{000000}{\fontsize{11}{11}\selectfont{\global\setmainfont{Arial}{Hiland}}}} & \multicolumn{1}{>{\raggedleft}m{\dimexpr 0.8in+0\tabcolsep}}{\textcolor[HTML]{000000}{\fontsize{11}{11}\selectfont{\global\setmainfont{Arial}{Weight}}}} & \multicolumn{1}{>{\raggedleft}m{\dimexpr 0.8in+0\tabcolsep}}{\textcolor[HTML]{000000}{\fontsize{11}{11}\selectfont{\global\setmainfont{Arial}{Yoplait}}}} \\

\ascline{1.5pt}{666666}{1-5}\endhead



\multicolumn{1}{>{\raggedright}m{\dimexpr 1.01in+0\tabcolsep}}{\textcolor[HTML]{000000}{\fontsize{11}{11}\selectfont{\global\setmainfont{Arial}{Dannon}}}} & \multicolumn{1}{>{\raggedleft}m{\dimexpr 0.82in+0\tabcolsep}}{\textcolor[HTML]{000000}{\fontsize{11}{11}\selectfont{\global\setmainfont{Arial}{-0.1105}}}} & \multicolumn{1}{>{\raggedleft}m{\dimexpr 0.8in+0\tabcolsep}}{\textcolor[HTML]{000000}{\fontsize{11}{11}\selectfont{\global\setmainfont{Arial}{0.0010}}}} & \multicolumn{1}{>{\raggedleft}m{\dimexpr 0.8in+0\tabcolsep}}{\textcolor[HTML]{000000}{\fontsize{11}{11}\selectfont{\global\setmainfont{Arial}{0.0550}}}} & \multicolumn{1}{>{\raggedleft}m{\dimexpr 0.8in+0\tabcolsep}}{\textcolor[HTML]{000000}{\fontsize{11}{11}\selectfont{\global\setmainfont{Arial}{0.0545}}}} \\





\multicolumn{1}{>{\raggedright}m{\dimexpr 1.01in+0\tabcolsep}}{\textcolor[HTML]{000000}{\fontsize{11}{11}\selectfont{\global\setmainfont{Arial}{Hiland}}}} & \multicolumn{1}{>{\raggedleft}m{\dimexpr 0.82in+0\tabcolsep}}{\textcolor[HTML]{000000}{\fontsize{11}{11}\selectfont{\global\setmainfont{Arial}{0.0010}}}} & \multicolumn{1}{>{\raggedleft}m{\dimexpr 0.8in+0\tabcolsep}}{\textcolor[HTML]{000000}{\fontsize{11}{11}\selectfont{\global\setmainfont{Arial}{-0.0021}}}} & \multicolumn{1}{>{\raggedleft}m{\dimexpr 0.8in+0\tabcolsep}}{\textcolor[HTML]{000000}{\fontsize{11}{11}\selectfont{\global\setmainfont{Arial}{0.0005}}}} & \multicolumn{1}{>{\raggedleft}m{\dimexpr 0.8in+0\tabcolsep}}{\textcolor[HTML]{000000}{\fontsize{11}{11}\selectfont{\global\setmainfont{Arial}{0.0005}}}} \\





\multicolumn{1}{>{\raggedright}m{\dimexpr 1.01in+0\tabcolsep}}{\textcolor[HTML]{000000}{\fontsize{11}{11}\selectfont{\global\setmainfont{Arial}{Weight}}}} & \multicolumn{1}{>{\raggedleft}m{\dimexpr 0.82in+0\tabcolsep}}{\textcolor[HTML]{000000}{\fontsize{11}{11}\selectfont{\global\setmainfont{Arial}{0.0550}}}} & \multicolumn{1}{>{\raggedleft}m{\dimexpr 0.8in+0\tabcolsep}}{\textcolor[HTML]{000000}{\fontsize{11}{11}\selectfont{\global\setmainfont{Arial}{0.0005}}}} & \multicolumn{1}{>{\raggedleft}m{\dimexpr 0.8in+0\tabcolsep}}{\textcolor[HTML]{000000}{\fontsize{11}{11}\selectfont{\global\setmainfont{Arial}{-0.0845}}}} & \multicolumn{1}{>{\raggedleft}m{\dimexpr 0.8in+0\tabcolsep}}{\textcolor[HTML]{000000}{\fontsize{11}{11}\selectfont{\global\setmainfont{Arial}{0.0290}}}} \\





\multicolumn{1}{>{\raggedright}m{\dimexpr 1.01in+0\tabcolsep}}{\textcolor[HTML]{000000}{\fontsize{11}{11}\selectfont{\global\setmainfont{Arial}{Yoplait}}}} & \multicolumn{1}{>{\raggedleft}m{\dimexpr 0.82in+0\tabcolsep}}{\textcolor[HTML]{000000}{\fontsize{11}{11}\selectfont{\global\setmainfont{Arial}{0.0545}}}} & \multicolumn{1}{>{\raggedleft}m{\dimexpr 0.8in+0\tabcolsep}}{\textcolor[HTML]{000000}{\fontsize{11}{11}\selectfont{\global\setmainfont{Arial}{0.0005}}}} & \multicolumn{1}{>{\raggedleft}m{\dimexpr 0.8in+0\tabcolsep}}{\textcolor[HTML]{000000}{\fontsize{11}{11}\selectfont{\global\setmainfont{Arial}{0.0290}}}} & \multicolumn{1}{>{\raggedleft}m{\dimexpr 0.8in+0\tabcolsep}}{\textcolor[HTML]{000000}{\fontsize{11}{11}\selectfont{\global\setmainfont{Arial}{-0.0840}}}} \\

\ascline{1.5pt}{666666}{1-5}



\end{longtable}



\arrayrulecolor[HTML]{000000}

\global\setlength{\arrayrulewidth}{\Oldarrayrulewidth}

\global\setlength{\tabcolsep}{\Oldtabcolsep}

\renewcommand*{\arraystretch}{1}

\begin{Shaded}
\begin{Highlighting}[]
\NormalTok{pp\_price}\SpecialCharTok{$}\NormalTok{pp\_table}
\end{Highlighting}
\end{Shaded}

\global\setlength{\Oldarrayrulewidth}{\arrayrulewidth}

\global\setlength{\Oldtabcolsep}{\tabcolsep}

\setlength{\tabcolsep}{2pt}

\renewcommand*{\arraystretch}{1.5}



\providecommand{\ascline}[3]{\noalign{\global\arrayrulewidth #1}\arrayrulecolor[HTML]{#2}\cline{#3}}

\begin{longtable}[c]{|p{0.95in}|p{1.06in}|p{0.82in}|p{0.75in}|p{0.76in}|p{0.75in}}



\ascline{1.5pt}{666666}{1-6}

\multicolumn{6}{>{\raggedright}m{\dimexpr 5.09in+10\tabcolsep}}{\textcolor[HTML]{000000}{\fontsize{11}{11}\selectfont{\global\setmainfont{Arial}{Predicted\ Probability\ Table\ (price)\ -\ Model\ data}}}} \\

\ascline{1.5pt}{666666}{1-6}



\multicolumn{1}{>{\raggedright}m{\dimexpr 0.95in+0\tabcolsep}}{\textcolor[HTML]{000000}{\fontsize{11}{11}\selectfont{\global\setmainfont{Arial}{varied\_alt}}}} & \multicolumn{1}{>{\raggedleft}m{\dimexpr 1.06in+0\tabcolsep}}{\textcolor[HTML]{000000}{\fontsize{11}{11}\selectfont{\global\setmainfont{Arial}{focal\_value}}}} & \multicolumn{1}{>{\raggedleft}m{\dimexpr 0.82in+0\tabcolsep}}{\textcolor[HTML]{000000}{\fontsize{11}{11}\selectfont{\global\setmainfont{Arial}{Dannon}}}} & \multicolumn{1}{>{\raggedleft}m{\dimexpr 0.75in+0\tabcolsep}}{\textcolor[HTML]{000000}{\fontsize{11}{11}\selectfont{\global\setmainfont{Arial}{Hiland}}}} & \multicolumn{1}{>{\raggedleft}m{\dimexpr 0.76in+0\tabcolsep}}{\textcolor[HTML]{000000}{\fontsize{11}{11}\selectfont{\global\setmainfont{Arial}{Weight}}}} & \multicolumn{1}{>{\raggedleft}m{\dimexpr 0.75in+0\tabcolsep}}{\textcolor[HTML]{000000}{\fontsize{11}{11}\selectfont{\global\setmainfont{Arial}{Yoplait}}}} \\

\ascline{1.5pt}{666666}{1-6}\endfirsthead 

\ascline{1.5pt}{666666}{1-6}

\multicolumn{6}{>{\raggedright}m{\dimexpr 5.09in+10\tabcolsep}}{\textcolor[HTML]{000000}{\fontsize{11}{11}\selectfont{\global\setmainfont{Arial}{Predicted\ Probability\ Table\ (price)\ -\ Model\ data}}}} \\

\ascline{1.5pt}{666666}{1-6}



\multicolumn{1}{>{\raggedright}m{\dimexpr 0.95in+0\tabcolsep}}{\textcolor[HTML]{000000}{\fontsize{11}{11}\selectfont{\global\setmainfont{Arial}{varied\_alt}}}} & \multicolumn{1}{>{\raggedleft}m{\dimexpr 1.06in+0\tabcolsep}}{\textcolor[HTML]{000000}{\fontsize{11}{11}\selectfont{\global\setmainfont{Arial}{focal\_value}}}} & \multicolumn{1}{>{\raggedleft}m{\dimexpr 0.82in+0\tabcolsep}}{\textcolor[HTML]{000000}{\fontsize{11}{11}\selectfont{\global\setmainfont{Arial}{Dannon}}}} & \multicolumn{1}{>{\raggedleft}m{\dimexpr 0.75in+0\tabcolsep}}{\textcolor[HTML]{000000}{\fontsize{11}{11}\selectfont{\global\setmainfont{Arial}{Hiland}}}} & \multicolumn{1}{>{\raggedleft}m{\dimexpr 0.76in+0\tabcolsep}}{\textcolor[HTML]{000000}{\fontsize{11}{11}\selectfont{\global\setmainfont{Arial}{Weight}}}} & \multicolumn{1}{>{\raggedleft}m{\dimexpr 0.75in+0\tabcolsep}}{\textcolor[HTML]{000000}{\fontsize{11}{11}\selectfont{\global\setmainfont{Arial}{Yoplait}}}} \\

\ascline{1.5pt}{666666}{1-6}\endhead



\multicolumn{6}{>{\raggedright}m{\dimexpr 5.09in+10\tabcolsep}}{\textcolor[HTML]{000000}{\fontsize{11}{11}\selectfont{\global\setmainfont{Arial}{Because\ }}}\textcolor[HTML]{000000}{\fontsize{10}{10}\selectfont{\global\setmainfont{Arial}{\textbf{\textit{price}}}}}\textcolor[HTML]{000000}{\fontsize{11}{11}\selectfont{\global\setmainfont{Arial}{\ is\ continuous,\ the\ values\ shown\ include\ the\ mean\ and\ +/-\ 1\ unit.}}}} \\

\endlastfoot



\multicolumn{1}{>{\raggedright}m{\dimexpr 0.95in+0\tabcolsep}}{\textcolor[HTML]{000000}{\fontsize{11}{11}\selectfont{\global\setmainfont{Arial}{Dannon}}}} & \multicolumn{1}{>{\raggedleft}m{\dimexpr 1.06in+0\tabcolsep}}{\textcolor[HTML]{000000}{\fontsize{11}{11}\selectfont{\global\setmainfont{Arial}{7.1628}}}} & \multicolumn{1}{>{\raggedleft}m{\dimexpr 0.82in+0\tabcolsep}}{\textcolor[HTML]{000000}{\fontsize{11}{11}\selectfont{\global\setmainfont{Arial}{0.4918}}}} & \multicolumn{1}{>{\raggedleft}m{\dimexpr 0.75in+0\tabcolsep}}{\textcolor[HTML]{000000}{\fontsize{11}{11}\selectfont{\global\setmainfont{Arial}{0.0250}}}} & \multicolumn{1}{>{\raggedleft}m{\dimexpr 0.76in+0\tabcolsep}}{\textcolor[HTML]{000000}{\fontsize{11}{11}\selectfont{\global\setmainfont{Arial}{0.1883}}}} & \multicolumn{1}{>{\raggedleft}m{\dimexpr 0.75in+0\tabcolsep}}{\textcolor[HTML]{000000}{\fontsize{11}{11}\selectfont{\global\setmainfont{Arial}{0.2949}}}} \\





\multicolumn{1}{>{\raggedright}m{\dimexpr 0.95in+0\tabcolsep}}{\textcolor[HTML]{000000}{\fontsize{11}{11}\selectfont{\global\setmainfont{Arial}{Dannon}}}} & \multicolumn{1}{>{\raggedleft}m{\dimexpr 1.06in+0\tabcolsep}}{\textcolor[HTML]{000000}{\fontsize{11}{11}\selectfont{\global\setmainfont{Arial}{8.1628}}}} & \multicolumn{1}{>{\raggedleft}m{\dimexpr 0.82in+0\tabcolsep}}{\textcolor[HTML]{000000}{\fontsize{11}{11}\selectfont{\global\setmainfont{Arial}{0.3980}}}} & \multicolumn{1}{>{\raggedleft}m{\dimexpr 0.75in+0\tabcolsep}}{\textcolor[HTML]{000000}{\fontsize{11}{11}\selectfont{\global\setmainfont{Arial}{0.0313}}}} & \multicolumn{1}{>{\raggedleft}m{\dimexpr 0.76in+0\tabcolsep}}{\textcolor[HTML]{000000}{\fontsize{11}{11}\selectfont{\global\setmainfont{Arial}{0.2353}}}} & \multicolumn{1}{>{\raggedleft}m{\dimexpr 0.75in+0\tabcolsep}}{\textcolor[HTML]{000000}{\fontsize{11}{11}\selectfont{\global\setmainfont{Arial}{0.3355}}}} \\





\multicolumn{1}{>{\raggedright}m{\dimexpr 0.95in+0\tabcolsep}}{\textcolor[HTML]{000000}{\fontsize{11}{11}\selectfont{\global\setmainfont{Arial}{Dannon}}}} & \multicolumn{1}{>{\raggedleft}m{\dimexpr 1.06in+0\tabcolsep}}{\textcolor[HTML]{000000}{\fontsize{11}{11}\selectfont{\global\setmainfont{Arial}{9.1628}}}} & \multicolumn{1}{>{\raggedleft}m{\dimexpr 0.82in+0\tabcolsep}}{\textcolor[HTML]{000000}{\fontsize{11}{11}\selectfont{\global\setmainfont{Arial}{0.3088}}}} & \multicolumn{1}{>{\raggedleft}m{\dimexpr 0.75in+0\tabcolsep}}{\textcolor[HTML]{000000}{\fontsize{11}{11}\selectfont{\global\setmainfont{Arial}{0.0375}}}} & \multicolumn{1}{>{\raggedleft}m{\dimexpr 0.76in+0\tabcolsep}}{\textcolor[HTML]{000000}{\fontsize{11}{11}\selectfont{\global\setmainfont{Arial}{0.2821}}}} & \multicolumn{1}{>{\raggedleft}m{\dimexpr 0.75in+0\tabcolsep}}{\textcolor[HTML]{000000}{\fontsize{11}{11}\selectfont{\global\setmainfont{Arial}{0.3716}}}} \\





\multicolumn{1}{>{\raggedright}m{\dimexpr 0.95in+0\tabcolsep}}{\textcolor[HTML]{000000}{\fontsize{11}{11}\selectfont{\global\setmainfont{Arial}{Hiland}}}} & \multicolumn{1}{>{\raggedleft}m{\dimexpr 1.06in+0\tabcolsep}}{\textcolor[HTML]{000000}{\fontsize{11}{11}\selectfont{\global\setmainfont{Arial}{4.3663}}}} & \multicolumn{1}{>{\raggedleft}m{\dimexpr 0.82in+0\tabcolsep}}{\textcolor[HTML]{000000}{\fontsize{11}{11}\selectfont{\global\setmainfont{Arial}{0.3966}}}} & \multicolumn{1}{>{\raggedleft}m{\dimexpr 0.75in+0\tabcolsep}}{\textcolor[HTML]{000000}{\fontsize{11}{11}\selectfont{\global\setmainfont{Arial}{0.0394}}}} & \multicolumn{1}{>{\raggedleft}m{\dimexpr 0.76in+0\tabcolsep}}{\textcolor[HTML]{000000}{\fontsize{11}{11}\selectfont{\global\setmainfont{Arial}{0.2258}}}} & \multicolumn{1}{>{\raggedleft}m{\dimexpr 0.75in+0\tabcolsep}}{\textcolor[HTML]{000000}{\fontsize{11}{11}\selectfont{\global\setmainfont{Arial}{0.3382}}}} \\





\multicolumn{1}{>{\raggedright}m{\dimexpr 0.95in+0\tabcolsep}}{\textcolor[HTML]{000000}{\fontsize{11}{11}\selectfont{\global\setmainfont{Arial}{Hiland}}}} & \multicolumn{1}{>{\raggedleft}m{\dimexpr 1.06in+0\tabcolsep}}{\textcolor[HTML]{000000}{\fontsize{11}{11}\selectfont{\global\setmainfont{Arial}{5.3663}}}} & \multicolumn{1}{>{\raggedleft}m{\dimexpr 0.82in+0\tabcolsep}}{\textcolor[HTML]{000000}{\fontsize{11}{11}\selectfont{\global\setmainfont{Arial}{0.4035}}}} & \multicolumn{1}{>{\raggedleft}m{\dimexpr 0.75in+0\tabcolsep}}{\textcolor[HTML]{000000}{\fontsize{11}{11}\selectfont{\global\setmainfont{Arial}{0.0268}}}} & \multicolumn{1}{>{\raggedleft}m{\dimexpr 0.76in+0\tabcolsep}}{\textcolor[HTML]{000000}{\fontsize{11}{11}\selectfont{\global\setmainfont{Arial}{0.2305}}}} & \multicolumn{1}{>{\raggedleft}m{\dimexpr 0.75in+0\tabcolsep}}{\textcolor[HTML]{000000}{\fontsize{11}{11}\selectfont{\global\setmainfont{Arial}{0.3392}}}} \\





\multicolumn{1}{>{\raggedright}m{\dimexpr 0.95in+0\tabcolsep}}{\textcolor[HTML]{000000}{\fontsize{11}{11}\selectfont{\global\setmainfont{Arial}{Hiland}}}} & \multicolumn{1}{>{\raggedleft}m{\dimexpr 1.06in+0\tabcolsep}}{\textcolor[HTML]{000000}{\fontsize{11}{11}\selectfont{\global\setmainfont{Arial}{6.3663}}}} & \multicolumn{1}{>{\raggedleft}m{\dimexpr 0.82in+0\tabcolsep}}{\textcolor[HTML]{000000}{\fontsize{11}{11}\selectfont{\global\setmainfont{Arial}{0.4083}}}} & \multicolumn{1}{>{\raggedleft}m{\dimexpr 0.75in+0\tabcolsep}}{\textcolor[HTML]{000000}{\fontsize{11}{11}\selectfont{\global\setmainfont{Arial}{0.0179}}}} & \multicolumn{1}{>{\raggedleft}m{\dimexpr 0.76in+0\tabcolsep}}{\textcolor[HTML]{000000}{\fontsize{11}{11}\selectfont{\global\setmainfont{Arial}{0.2338}}}} & \multicolumn{1}{>{\raggedleft}m{\dimexpr 0.75in+0\tabcolsep}}{\textcolor[HTML]{000000}{\fontsize{11}{11}\selectfont{\global\setmainfont{Arial}{0.3399}}}} \\





\multicolumn{1}{>{\raggedright}m{\dimexpr 0.95in+0\tabcolsep}}{\textcolor[HTML]{000000}{\fontsize{11}{11}\selectfont{\global\setmainfont{Arial}{Weight}}}} & \multicolumn{1}{>{\raggedleft}m{\dimexpr 1.06in+0\tabcolsep}}{\textcolor[HTML]{000000}{\fontsize{11}{11}\selectfont{\global\setmainfont{Arial}{6.9421}}}} & \multicolumn{1}{>{\raggedleft}m{\dimexpr 0.82in+0\tabcolsep}}{\textcolor[HTML]{000000}{\fontsize{11}{11}\selectfont{\global\setmainfont{Arial}{0.3516}}}} & \multicolumn{1}{>{\raggedleft}m{\dimexpr 0.75in+0\tabcolsep}}{\textcolor[HTML]{000000}{\fontsize{11}{11}\selectfont{\global\setmainfont{Arial}{0.0256}}}} & \multicolumn{1}{>{\raggedleft}m{\dimexpr 0.76in+0\tabcolsep}}{\textcolor[HTML]{000000}{\fontsize{11}{11}\selectfont{\global\setmainfont{Arial}{0.3060}}}} & \multicolumn{1}{>{\raggedleft}m{\dimexpr 0.75in+0\tabcolsep}}{\textcolor[HTML]{000000}{\fontsize{11}{11}\selectfont{\global\setmainfont{Arial}{0.3169}}}} \\





\multicolumn{1}{>{\raggedright}m{\dimexpr 0.95in+0\tabcolsep}}{\textcolor[HTML]{000000}{\fontsize{11}{11}\selectfont{\global\setmainfont{Arial}{Weight}}}} & \multicolumn{1}{>{\raggedleft}m{\dimexpr 1.06in+0\tabcolsep}}{\textcolor[HTML]{000000}{\fontsize{11}{11}\selectfont{\global\setmainfont{Arial}{7.9421}}}} & \multicolumn{1}{>{\raggedleft}m{\dimexpr 0.82in+0\tabcolsep}}{\textcolor[HTML]{000000}{\fontsize{11}{11}\selectfont{\global\setmainfont{Arial}{0.4019}}}} & \multicolumn{1}{>{\raggedleft}m{\dimexpr 0.75in+0\tabcolsep}}{\textcolor[HTML]{000000}{\fontsize{11}{11}\selectfont{\global\setmainfont{Arial}{0.0301}}}} & \multicolumn{1}{>{\raggedleft}m{\dimexpr 0.76in+0\tabcolsep}}{\textcolor[HTML]{000000}{\fontsize{11}{11}\selectfont{\global\setmainfont{Arial}{0.2287}}}} & \multicolumn{1}{>{\raggedleft}m{\dimexpr 0.75in+0\tabcolsep}}{\textcolor[HTML]{000000}{\fontsize{11}{11}\selectfont{\global\setmainfont{Arial}{0.3392}}}} \\





\multicolumn{1}{>{\raggedright}m{\dimexpr 0.95in+0\tabcolsep}}{\textcolor[HTML]{000000}{\fontsize{11}{11}\selectfont{\global\setmainfont{Arial}{Weight}}}} & \multicolumn{1}{>{\raggedleft}m{\dimexpr 1.06in+0\tabcolsep}}{\textcolor[HTML]{000000}{\fontsize{11}{11}\selectfont{\global\setmainfont{Arial}{8.9421}}}} & \multicolumn{1}{>{\raggedleft}m{\dimexpr 0.82in+0\tabcolsep}}{\textcolor[HTML]{000000}{\fontsize{11}{11}\selectfont{\global\setmainfont{Arial}{0.4446}}}} & \multicolumn{1}{>{\raggedleft}m{\dimexpr 0.75in+0\tabcolsep}}{\textcolor[HTML]{000000}{\fontsize{11}{11}\selectfont{\global\setmainfont{Arial}{0.0340}}}} & \multicolumn{1}{>{\raggedleft}m{\dimexpr 0.76in+0\tabcolsep}}{\textcolor[HTML]{000000}{\fontsize{11}{11}\selectfont{\global\setmainfont{Arial}{0.1648}}}} & \multicolumn{1}{>{\raggedleft}m{\dimexpr 0.75in+0\tabcolsep}}{\textcolor[HTML]{000000}{\fontsize{11}{11}\selectfont{\global\setmainfont{Arial}{0.3566}}}} \\





\multicolumn{1}{>{\raggedright}m{\dimexpr 0.95in+0\tabcolsep}}{\textcolor[HTML]{000000}{\fontsize{11}{11}\selectfont{\global\setmainfont{Arial}{Yoplait}}}} & \multicolumn{1}{>{\raggedleft}m{\dimexpr 1.06in+0\tabcolsep}}{\textcolor[HTML]{000000}{\fontsize{11}{11}\selectfont{\global\setmainfont{Arial}{9.6874}}}} & \multicolumn{1}{>{\raggedleft}m{\dimexpr 0.82in+0\tabcolsep}}{\textcolor[HTML]{000000}{\fontsize{11}{11}\selectfont{\global\setmainfont{Arial}{0.3673}}}} & \multicolumn{1}{>{\raggedleft}m{\dimexpr 0.75in+0\tabcolsep}}{\textcolor[HTML]{000000}{\fontsize{11}{11}\selectfont{\global\setmainfont{Arial}{0.0302}}}} & \multicolumn{1}{>{\raggedleft}m{\dimexpr 0.76in+0\tabcolsep}}{\textcolor[HTML]{000000}{\fontsize{11}{11}\selectfont{\global\setmainfont{Arial}{0.2138}}}} & \multicolumn{1}{>{\raggedleft}m{\dimexpr 0.75in+0\tabcolsep}}{\textcolor[HTML]{000000}{\fontsize{11}{11}\selectfont{\global\setmainfont{Arial}{0.3886}}}} \\





\multicolumn{1}{>{\raggedright}m{\dimexpr 0.95in+0\tabcolsep}}{\textcolor[HTML]{000000}{\fontsize{11}{11}\selectfont{\global\setmainfont{Arial}{Yoplait}}}} & \multicolumn{1}{>{\raggedleft}m{\dimexpr 1.06in+0\tabcolsep}}{\textcolor[HTML]{000000}{\fontsize{11}{11}\selectfont{\global\setmainfont{Arial}{10.6874}}}} & \multicolumn{1}{>{\raggedleft}m{\dimexpr 0.82in+0\tabcolsep}}{\textcolor[HTML]{000000}{\fontsize{11}{11}\selectfont{\global\setmainfont{Arial}{0.4069}}}} & \multicolumn{1}{>{\raggedleft}m{\dimexpr 0.75in+0\tabcolsep}}{\textcolor[HTML]{000000}{\fontsize{11}{11}\selectfont{\global\setmainfont{Arial}{0.0311}}}} & \multicolumn{1}{>{\raggedleft}m{\dimexpr 0.76in+0\tabcolsep}}{\textcolor[HTML]{000000}{\fontsize{11}{11}\selectfont{\global\setmainfont{Arial}{0.2350}}}} & \multicolumn{1}{>{\raggedleft}m{\dimexpr 0.75in+0\tabcolsep}}{\textcolor[HTML]{000000}{\fontsize{11}{11}\selectfont{\global\setmainfont{Arial}{0.3269}}}} \\





\multicolumn{1}{>{\raggedright}m{\dimexpr 0.95in+0\tabcolsep}}{\textcolor[HTML]{000000}{\fontsize{11}{11}\selectfont{\global\setmainfont{Arial}{Yoplait}}}} & \multicolumn{1}{>{\raggedleft}m{\dimexpr 1.06in+0\tabcolsep}}{\textcolor[HTML]{000000}{\fontsize{11}{11}\selectfont{\global\setmainfont{Arial}{11.6874}}}} & \multicolumn{1}{>{\raggedleft}m{\dimexpr 0.82in+0\tabcolsep}}{\textcolor[HTML]{000000}{\fontsize{11}{11}\selectfont{\global\setmainfont{Arial}{0.4438}}}} & \multicolumn{1}{>{\raggedleft}m{\dimexpr 0.75in+0\tabcolsep}}{\textcolor[HTML]{000000}{\fontsize{11}{11}\selectfont{\global\setmainfont{Arial}{0.0318}}}} & \multicolumn{1}{>{\raggedleft}m{\dimexpr 0.76in+0\tabcolsep}}{\textcolor[HTML]{000000}{\fontsize{11}{11}\selectfont{\global\setmainfont{Arial}{0.2545}}}} & \multicolumn{1}{>{\raggedleft}m{\dimexpr 0.75in+0\tabcolsep}}{\textcolor[HTML]{000000}{\fontsize{11}{11}\selectfont{\global\setmainfont{Arial}{0.2699}}}} \\

\ascline{1.5pt}{666666}{1-6}



\end{longtable}



\arrayrulecolor[HTML]{000000}

\global\setlength{\arrayrulewidth}{\Oldarrayrulewidth}

\global\setlength{\tabcolsep}{\Oldtabcolsep}

\renewcommand*{\arraystretch}{1}

\begin{Shaded}
\begin{Highlighting}[]
\NormalTok{pp\_price}\SpecialCharTok{$}\NormalTok{pp\_plot}
\end{Highlighting}
\end{Shaded}

\pandocbounded{\includegraphics[keepaspectratio]{MKT4320_R_Tutorial_files/figure-latex/pp-alt-1.pdf}}

\begin{Shaded}
\begin{Highlighting}[]
\NormalTok{pp\_feat }\OtherTok{\textless{}{-}} \FunctionTok{pp\_as\_mnl}\NormalTok{(as\_mnl\_fit, }\AttributeTok{focal\_var =} \StringTok{"feat"}\NormalTok{, }\AttributeTok{ft=}\ConstantTok{TRUE}\NormalTok{, }\AttributeTok{me\_method=}\StringTok{"means"}\NormalTok{)}
\NormalTok{pp\_feat}\SpecialCharTok{$}\NormalTok{me\_table}
\end{Highlighting}
\end{Shaded}

\global\setlength{\Oldarrayrulewidth}{\arrayrulewidth}

\global\setlength{\Oldtabcolsep}{\tabcolsep}

\setlength{\tabcolsep}{2pt}

\renewcommand*{\arraystretch}{1.5}



\providecommand{\ascline}[3]{\noalign{\global\arrayrulewidth #1}\arrayrulecolor[HTML]{#2}\cline{#3}}

\begin{longtable}[c]{|p{1.01in}|p{0.82in}|p{0.80in}|p{0.80in}|p{0.80in}}



\ascline{1.5pt}{666666}{1-5}

\multicolumn{5}{>{\raggedright}m{\dimexpr 4.24in+8\tabcolsep}}{\textcolor[HTML]{000000}{\fontsize{11}{11}\selectfont{\global\setmainfont{Arial}{Marginal\ effects\ for\ }}}\textcolor[HTML]{000000}{\fontsize{10}{10}\selectfont{\global\setmainfont{Arial}{\textbf{\textit{feat}}}}}\textcolor[HTML]{000000}{\fontsize{11}{11}\selectfont{\global\setmainfont{Arial}{\ (at\ means)}}}} \\

\ascline{1.5pt}{666666}{1-5}



\multicolumn{1}{>{\raggedright}m{\dimexpr 1.01in+0\tabcolsep}}{\textcolor[HTML]{000000}{\fontsize{11}{11}\selectfont{\global\setmainfont{Arial}{Alternative}}}} & \multicolumn{1}{>{\raggedleft}m{\dimexpr 0.82in+0\tabcolsep}}{\textcolor[HTML]{000000}{\fontsize{11}{11}\selectfont{\global\setmainfont{Arial}{Dannon}}}} & \multicolumn{1}{>{\raggedleft}m{\dimexpr 0.8in+0\tabcolsep}}{\textcolor[HTML]{000000}{\fontsize{11}{11}\selectfont{\global\setmainfont{Arial}{Hiland}}}} & \multicolumn{1}{>{\raggedleft}m{\dimexpr 0.8in+0\tabcolsep}}{\textcolor[HTML]{000000}{\fontsize{11}{11}\selectfont{\global\setmainfont{Arial}{Weight}}}} & \multicolumn{1}{>{\raggedleft}m{\dimexpr 0.8in+0\tabcolsep}}{\textcolor[HTML]{000000}{\fontsize{11}{11}\selectfont{\global\setmainfont{Arial}{Yoplait}}}} \\

\ascline{1.5pt}{666666}{1-5}\endfirsthead 

\ascline{1.5pt}{666666}{1-5}

\multicolumn{5}{>{\raggedright}m{\dimexpr 4.24in+8\tabcolsep}}{\textcolor[HTML]{000000}{\fontsize{11}{11}\selectfont{\global\setmainfont{Arial}{Marginal\ effects\ for\ }}}\textcolor[HTML]{000000}{\fontsize{10}{10}\selectfont{\global\setmainfont{Arial}{\textbf{\textit{feat}}}}}\textcolor[HTML]{000000}{\fontsize{11}{11}\selectfont{\global\setmainfont{Arial}{\ (at\ means)}}}} \\

\ascline{1.5pt}{666666}{1-5}



\multicolumn{1}{>{\raggedright}m{\dimexpr 1.01in+0\tabcolsep}}{\textcolor[HTML]{000000}{\fontsize{11}{11}\selectfont{\global\setmainfont{Arial}{Alternative}}}} & \multicolumn{1}{>{\raggedleft}m{\dimexpr 0.82in+0\tabcolsep}}{\textcolor[HTML]{000000}{\fontsize{11}{11}\selectfont{\global\setmainfont{Arial}{Dannon}}}} & \multicolumn{1}{>{\raggedleft}m{\dimexpr 0.8in+0\tabcolsep}}{\textcolor[HTML]{000000}{\fontsize{11}{11}\selectfont{\global\setmainfont{Arial}{Hiland}}}} & \multicolumn{1}{>{\raggedleft}m{\dimexpr 0.8in+0\tabcolsep}}{\textcolor[HTML]{000000}{\fontsize{11}{11}\selectfont{\global\setmainfont{Arial}{Weight}}}} & \multicolumn{1}{>{\raggedleft}m{\dimexpr 0.8in+0\tabcolsep}}{\textcolor[HTML]{000000}{\fontsize{11}{11}\selectfont{\global\setmainfont{Arial}{Yoplait}}}} \\

\ascline{1.5pt}{666666}{1-5}\endhead



\multicolumn{1}{>{\raggedright}m{\dimexpr 1.01in+0\tabcolsep}}{\textcolor[HTML]{000000}{\fontsize{11}{11}\selectfont{\global\setmainfont{Arial}{Dannon}}}} & \multicolumn{1}{>{\raggedleft}m{\dimexpr 0.82in+0\tabcolsep}}{\textcolor[HTML]{000000}{\fontsize{11}{11}\selectfont{\global\setmainfont{Arial}{0.1057}}}} & \multicolumn{1}{>{\raggedleft}m{\dimexpr 0.8in+0\tabcolsep}}{\textcolor[HTML]{000000}{\fontsize{11}{11}\selectfont{\global\setmainfont{Arial}{-0.0010}}}} & \multicolumn{1}{>{\raggedleft}m{\dimexpr 0.8in+0\tabcolsep}}{\textcolor[HTML]{000000}{\fontsize{11}{11}\selectfont{\global\setmainfont{Arial}{-0.0526}}}} & \multicolumn{1}{>{\raggedleft}m{\dimexpr 0.8in+0\tabcolsep}}{\textcolor[HTML]{000000}{\fontsize{11}{11}\selectfont{\global\setmainfont{Arial}{-0.0521}}}} \\





\multicolumn{1}{>{\raggedright}m{\dimexpr 1.01in+0\tabcolsep}}{\textcolor[HTML]{000000}{\fontsize{11}{11}\selectfont{\global\setmainfont{Arial}{Hiland}}}} & \multicolumn{1}{>{\raggedleft}m{\dimexpr 0.82in+0\tabcolsep}}{\textcolor[HTML]{000000}{\fontsize{11}{11}\selectfont{\global\setmainfont{Arial}{-0.0010}}}} & \multicolumn{1}{>{\raggedleft}m{\dimexpr 0.8in+0\tabcolsep}}{\textcolor[HTML]{000000}{\fontsize{11}{11}\selectfont{\global\setmainfont{Arial}{0.0020}}}} & \multicolumn{1}{>{\raggedleft}m{\dimexpr 0.8in+0\tabcolsep}}{\textcolor[HTML]{000000}{\fontsize{11}{11}\selectfont{\global\setmainfont{Arial}{-0.0005}}}} & \multicolumn{1}{>{\raggedleft}m{\dimexpr 0.8in+0\tabcolsep}}{\textcolor[HTML]{000000}{\fontsize{11}{11}\selectfont{\global\setmainfont{Arial}{-0.0005}}}} \\





\multicolumn{1}{>{\raggedright}m{\dimexpr 1.01in+0\tabcolsep}}{\textcolor[HTML]{000000}{\fontsize{11}{11}\selectfont{\global\setmainfont{Arial}{Weight}}}} & \multicolumn{1}{>{\raggedleft}m{\dimexpr 0.82in+0\tabcolsep}}{\textcolor[HTML]{000000}{\fontsize{11}{11}\selectfont{\global\setmainfont{Arial}{-0.0526}}}} & \multicolumn{1}{>{\raggedleft}m{\dimexpr 0.8in+0\tabcolsep}}{\textcolor[HTML]{000000}{\fontsize{11}{11}\selectfont{\global\setmainfont{Arial}{-0.0005}}}} & \multicolumn{1}{>{\raggedleft}m{\dimexpr 0.8in+0\tabcolsep}}{\textcolor[HTML]{000000}{\fontsize{11}{11}\selectfont{\global\setmainfont{Arial}{0.0808}}}} & \multicolumn{1}{>{\raggedleft}m{\dimexpr 0.8in+0\tabcolsep}}{\textcolor[HTML]{000000}{\fontsize{11}{11}\selectfont{\global\setmainfont{Arial}{-0.0277}}}} \\





\multicolumn{1}{>{\raggedright}m{\dimexpr 1.01in+0\tabcolsep}}{\textcolor[HTML]{000000}{\fontsize{11}{11}\selectfont{\global\setmainfont{Arial}{Yoplait}}}} & \multicolumn{1}{>{\raggedleft}m{\dimexpr 0.82in+0\tabcolsep}}{\textcolor[HTML]{000000}{\fontsize{11}{11}\selectfont{\global\setmainfont{Arial}{-0.0521}}}} & \multicolumn{1}{>{\raggedleft}m{\dimexpr 0.8in+0\tabcolsep}}{\textcolor[HTML]{000000}{\fontsize{11}{11}\selectfont{\global\setmainfont{Arial}{-0.0005}}}} & \multicolumn{1}{>{\raggedleft}m{\dimexpr 0.8in+0\tabcolsep}}{\textcolor[HTML]{000000}{\fontsize{11}{11}\selectfont{\global\setmainfont{Arial}{-0.0277}}}} & \multicolumn{1}{>{\raggedleft}m{\dimexpr 0.8in+0\tabcolsep}}{\textcolor[HTML]{000000}{\fontsize{11}{11}\selectfont{\global\setmainfont{Arial}{0.0804}}}} \\

\ascline{1.5pt}{666666}{1-5}



\end{longtable}



\arrayrulecolor[HTML]{000000}

\global\setlength{\arrayrulewidth}{\Oldarrayrulewidth}

\global\setlength{\tabcolsep}{\Oldtabcolsep}

\renewcommand*{\arraystretch}{1}

\begin{Shaded}
\begin{Highlighting}[]
\NormalTok{pp\_feat}\SpecialCharTok{$}\NormalTok{pp\_table}
\end{Highlighting}
\end{Shaded}

\global\setlength{\Oldarrayrulewidth}{\arrayrulewidth}

\global\setlength{\Oldtabcolsep}{\tabcolsep}

\setlength{\tabcolsep}{2pt}

\renewcommand*{\arraystretch}{1.5}



\providecommand{\ascline}[3]{\noalign{\global\arrayrulewidth #1}\arrayrulecolor[HTML]{#2}\cline{#3}}

\begin{longtable}[c]{|p{0.95in}|p{1.06in}|p{0.82in}|p{0.75in}|p{0.76in}|p{0.75in}}



\ascline{1.5pt}{666666}{1-6}

\multicolumn{6}{>{\raggedright}m{\dimexpr 5.09in+10\tabcolsep}}{\textcolor[HTML]{000000}{\fontsize{11}{11}\selectfont{\global\setmainfont{Arial}{Predicted\ Probability\ Table\ (feat)\ -\ Model\ data}}}} \\

\ascline{1.5pt}{666666}{1-6}



\multicolumn{1}{>{\raggedright}m{\dimexpr 0.95in+0\tabcolsep}}{\textcolor[HTML]{000000}{\fontsize{11}{11}\selectfont{\global\setmainfont{Arial}{varied\_alt}}}} & \multicolumn{1}{>{\raggedleft}m{\dimexpr 1.06in+0\tabcolsep}}{\textcolor[HTML]{000000}{\fontsize{11}{11}\selectfont{\global\setmainfont{Arial}{focal\_value}}}} & \multicolumn{1}{>{\raggedleft}m{\dimexpr 0.82in+0\tabcolsep}}{\textcolor[HTML]{000000}{\fontsize{11}{11}\selectfont{\global\setmainfont{Arial}{Dannon}}}} & \multicolumn{1}{>{\raggedleft}m{\dimexpr 0.75in+0\tabcolsep}}{\textcolor[HTML]{000000}{\fontsize{11}{11}\selectfont{\global\setmainfont{Arial}{Hiland}}}} & \multicolumn{1}{>{\raggedleft}m{\dimexpr 0.76in+0\tabcolsep}}{\textcolor[HTML]{000000}{\fontsize{11}{11}\selectfont{\global\setmainfont{Arial}{Weight}}}} & \multicolumn{1}{>{\raggedleft}m{\dimexpr 0.75in+0\tabcolsep}}{\textcolor[HTML]{000000}{\fontsize{11}{11}\selectfont{\global\setmainfont{Arial}{Yoplait}}}} \\

\ascline{1.5pt}{666666}{1-6}\endfirsthead 

\ascline{1.5pt}{666666}{1-6}

\multicolumn{6}{>{\raggedright}m{\dimexpr 5.09in+10\tabcolsep}}{\textcolor[HTML]{000000}{\fontsize{11}{11}\selectfont{\global\setmainfont{Arial}{Predicted\ Probability\ Table\ (feat)\ -\ Model\ data}}}} \\

\ascline{1.5pt}{666666}{1-6}



\multicolumn{1}{>{\raggedright}m{\dimexpr 0.95in+0\tabcolsep}}{\textcolor[HTML]{000000}{\fontsize{11}{11}\selectfont{\global\setmainfont{Arial}{varied\_alt}}}} & \multicolumn{1}{>{\raggedleft}m{\dimexpr 1.06in+0\tabcolsep}}{\textcolor[HTML]{000000}{\fontsize{11}{11}\selectfont{\global\setmainfont{Arial}{focal\_value}}}} & \multicolumn{1}{>{\raggedleft}m{\dimexpr 0.82in+0\tabcolsep}}{\textcolor[HTML]{000000}{\fontsize{11}{11}\selectfont{\global\setmainfont{Arial}{Dannon}}}} & \multicolumn{1}{>{\raggedleft}m{\dimexpr 0.75in+0\tabcolsep}}{\textcolor[HTML]{000000}{\fontsize{11}{11}\selectfont{\global\setmainfont{Arial}{Hiland}}}} & \multicolumn{1}{>{\raggedleft}m{\dimexpr 0.76in+0\tabcolsep}}{\textcolor[HTML]{000000}{\fontsize{11}{11}\selectfont{\global\setmainfont{Arial}{Weight}}}} & \multicolumn{1}{>{\raggedleft}m{\dimexpr 0.75in+0\tabcolsep}}{\textcolor[HTML]{000000}{\fontsize{11}{11}\selectfont{\global\setmainfont{Arial}{Yoplait}}}} \\

\ascline{1.5pt}{666666}{1-6}\endhead



\multicolumn{6}{>{\raggedright}m{\dimexpr 5.09in+10\tabcolsep}}{\textcolor[HTML]{000000}{\fontsize{11}{11}\selectfont{\global\setmainfont{Arial}{Because\ }}}\textcolor[HTML]{000000}{\fontsize{10}{10}\selectfont{\global\setmainfont{Arial}{\textbf{\textit{feat}}}}}\textcolor[HTML]{000000}{\fontsize{11}{11}\selectfont{\global\setmainfont{Arial}{\ is\ binary,\ only\ the\ two\ observed\ values\ are\ shown.}}}} \\

\endlastfoot



\multicolumn{1}{>{\raggedright}m{\dimexpr 0.95in+0\tabcolsep}}{\textcolor[HTML]{000000}{\fontsize{11}{11}\selectfont{\global\setmainfont{Arial}{Dannon}}}} & \multicolumn{1}{>{\raggedleft}m{\dimexpr 1.06in+0\tabcolsep}}{\textcolor[HTML]{000000}{\fontsize{11}{11}\selectfont{\global\setmainfont{Arial}{0}}}} & \multicolumn{1}{>{\raggedleft}m{\dimexpr 0.82in+0\tabcolsep}}{\textcolor[HTML]{000000}{\fontsize{11}{11}\selectfont{\global\setmainfont{Arial}{0.3988}}}} & \multicolumn{1}{>{\raggedleft}m{\dimexpr 0.75in+0\tabcolsep}}{\textcolor[HTML]{000000}{\fontsize{11}{11}\selectfont{\global\setmainfont{Arial}{0.0301}}}} & \multicolumn{1}{>{\raggedleft}m{\dimexpr 0.76in+0\tabcolsep}}{\textcolor[HTML]{000000}{\fontsize{11}{11}\selectfont{\global\setmainfont{Arial}{0.2308}}}} & \multicolumn{1}{>{\raggedleft}m{\dimexpr 0.75in+0\tabcolsep}}{\textcolor[HTML]{000000}{\fontsize{11}{11}\selectfont{\global\setmainfont{Arial}{0.3403}}}} \\





\multicolumn{1}{>{\raggedright}m{\dimexpr 0.95in+0\tabcolsep}}{\textcolor[HTML]{000000}{\fontsize{11}{11}\selectfont{\global\setmainfont{Arial}{Dannon}}}} & \multicolumn{1}{>{\raggedleft}m{\dimexpr 1.06in+0\tabcolsep}}{\textcolor[HTML]{000000}{\fontsize{11}{11}\selectfont{\global\setmainfont{Arial}{1}}}} & \multicolumn{1}{>{\raggedleft}m{\dimexpr 0.82in+0\tabcolsep}}{\textcolor[HTML]{000000}{\fontsize{11}{11}\selectfont{\global\setmainfont{Arial}{0.4853}}}} & \multicolumn{1}{>{\raggedleft}m{\dimexpr 0.75in+0\tabcolsep}}{\textcolor[HTML]{000000}{\fontsize{11}{11}\selectfont{\global\setmainfont{Arial}{0.0244}}}} & \multicolumn{1}{>{\raggedleft}m{\dimexpr 0.76in+0\tabcolsep}}{\textcolor[HTML]{000000}{\fontsize{11}{11}\selectfont{\global\setmainfont{Arial}{0.1870}}}} & \multicolumn{1}{>{\raggedleft}m{\dimexpr 0.75in+0\tabcolsep}}{\textcolor[HTML]{000000}{\fontsize{11}{11}\selectfont{\global\setmainfont{Arial}{0.3033}}}} \\





\multicolumn{1}{>{\raggedright}m{\dimexpr 0.95in+0\tabcolsep}}{\textcolor[HTML]{000000}{\fontsize{11}{11}\selectfont{\global\setmainfont{Arial}{Hiland}}}} & \multicolumn{1}{>{\raggedleft}m{\dimexpr 1.06in+0\tabcolsep}}{\textcolor[HTML]{000000}{\fontsize{11}{11}\selectfont{\global\setmainfont{Arial}{0}}}} & \multicolumn{1}{>{\raggedleft}m{\dimexpr 0.82in+0\tabcolsep}}{\textcolor[HTML]{000000}{\fontsize{11}{11}\selectfont{\global\setmainfont{Arial}{0.4027}}}} & \multicolumn{1}{>{\raggedleft}m{\dimexpr 0.75in+0\tabcolsep}}{\textcolor[HTML]{000000}{\fontsize{11}{11}\selectfont{\global\setmainfont{Arial}{0.0285}}}} & \multicolumn{1}{>{\raggedleft}m{\dimexpr 0.76in+0\tabcolsep}}{\textcolor[HTML]{000000}{\fontsize{11}{11}\selectfont{\global\setmainfont{Arial}{0.2297}}}} & \multicolumn{1}{>{\raggedleft}m{\dimexpr 0.75in+0\tabcolsep}}{\textcolor[HTML]{000000}{\fontsize{11}{11}\selectfont{\global\setmainfont{Arial}{0.3391}}}} \\





\multicolumn{1}{>{\raggedright}m{\dimexpr 0.95in+0\tabcolsep}}{\textcolor[HTML]{000000}{\fontsize{11}{11}\selectfont{\global\setmainfont{Arial}{Hiland}}}} & \multicolumn{1}{>{\raggedleft}m{\dimexpr 1.06in+0\tabcolsep}}{\textcolor[HTML]{000000}{\fontsize{11}{11}\selectfont{\global\setmainfont{Arial}{1}}}} & \multicolumn{1}{>{\raggedleft}m{\dimexpr 0.82in+0\tabcolsep}}{\textcolor[HTML]{000000}{\fontsize{11}{11}\selectfont{\global\setmainfont{Arial}{0.3960}}}} & \multicolumn{1}{>{\raggedleft}m{\dimexpr 0.75in+0\tabcolsep}}{\textcolor[HTML]{000000}{\fontsize{11}{11}\selectfont{\global\setmainfont{Arial}{0.0407}}}} & \multicolumn{1}{>{\raggedleft}m{\dimexpr 0.76in+0\tabcolsep}}{\textcolor[HTML]{000000}{\fontsize{11}{11}\selectfont{\global\setmainfont{Arial}{0.2251}}}} & \multicolumn{1}{>{\raggedleft}m{\dimexpr 0.75in+0\tabcolsep}}{\textcolor[HTML]{000000}{\fontsize{11}{11}\selectfont{\global\setmainfont{Arial}{0.3381}}}} \\





\multicolumn{1}{>{\raggedright}m{\dimexpr 0.95in+0\tabcolsep}}{\textcolor[HTML]{000000}{\fontsize{11}{11}\selectfont{\global\setmainfont{Arial}{Weight}}}} & \multicolumn{1}{>{\raggedleft}m{\dimexpr 1.06in+0\tabcolsep}}{\textcolor[HTML]{000000}{\fontsize{11}{11}\selectfont{\global\setmainfont{Arial}{0}}}} & \multicolumn{1}{>{\raggedleft}m{\dimexpr 0.82in+0\tabcolsep}}{\textcolor[HTML]{000000}{\fontsize{11}{11}\selectfont{\global\setmainfont{Arial}{0.4038}}}} & \multicolumn{1}{>{\raggedleft}m{\dimexpr 0.75in+0\tabcolsep}}{\textcolor[HTML]{000000}{\fontsize{11}{11}\selectfont{\global\setmainfont{Arial}{0.0300}}}} & \multicolumn{1}{>{\raggedleft}m{\dimexpr 0.76in+0\tabcolsep}}{\textcolor[HTML]{000000}{\fontsize{11}{11}\selectfont{\global\setmainfont{Arial}{0.2264}}}} & \multicolumn{1}{>{\raggedleft}m{\dimexpr 0.75in+0\tabcolsep}}{\textcolor[HTML]{000000}{\fontsize{11}{11}\selectfont{\global\setmainfont{Arial}{0.3398}}}} \\





\multicolumn{1}{>{\raggedright}m{\dimexpr 0.95in+0\tabcolsep}}{\textcolor[HTML]{000000}{\fontsize{11}{11}\selectfont{\global\setmainfont{Arial}{Weight}}}} & \multicolumn{1}{>{\raggedleft}m{\dimexpr 1.06in+0\tabcolsep}}{\textcolor[HTML]{000000}{\fontsize{11}{11}\selectfont{\global\setmainfont{Arial}{1}}}} & \multicolumn{1}{>{\raggedleft}m{\dimexpr 0.82in+0\tabcolsep}}{\textcolor[HTML]{000000}{\fontsize{11}{11}\selectfont{\global\setmainfont{Arial}{0.3565}}}} & \multicolumn{1}{>{\raggedleft}m{\dimexpr 0.75in+0\tabcolsep}}{\textcolor[HTML]{000000}{\fontsize{11}{11}\selectfont{\global\setmainfont{Arial}{0.0257}}}} & \multicolumn{1}{>{\raggedleft}m{\dimexpr 0.76in+0\tabcolsep}}{\textcolor[HTML]{000000}{\fontsize{11}{11}\selectfont{\global\setmainfont{Arial}{0.2990}}}} & \multicolumn{1}{>{\raggedleft}m{\dimexpr 0.75in+0\tabcolsep}}{\textcolor[HTML]{000000}{\fontsize{11}{11}\selectfont{\global\setmainfont{Arial}{0.3188}}}} \\





\multicolumn{1}{>{\raggedright}m{\dimexpr 0.95in+0\tabcolsep}}{\textcolor[HTML]{000000}{\fontsize{11}{11}\selectfont{\global\setmainfont{Arial}{Yoplait}}}} & \multicolumn{1}{>{\raggedleft}m{\dimexpr 1.06in+0\tabcolsep}}{\textcolor[HTML]{000000}{\fontsize{11}{11}\selectfont{\global\setmainfont{Arial}{0}}}} & \multicolumn{1}{>{\raggedleft}m{\dimexpr 0.82in+0\tabcolsep}}{\textcolor[HTML]{000000}{\fontsize{11}{11}\selectfont{\global\setmainfont{Arial}{0.4043}}}} & \multicolumn{1}{>{\raggedleft}m{\dimexpr 0.75in+0\tabcolsep}}{\textcolor[HTML]{000000}{\fontsize{11}{11}\selectfont{\global\setmainfont{Arial}{0.0299}}}} & \multicolumn{1}{>{\raggedleft}m{\dimexpr 0.76in+0\tabcolsep}}{\textcolor[HTML]{000000}{\fontsize{11}{11}\selectfont{\global\setmainfont{Arial}{0.2306}}}} & \multicolumn{1}{>{\raggedleft}m{\dimexpr 0.75in+0\tabcolsep}}{\textcolor[HTML]{000000}{\fontsize{11}{11}\selectfont{\global\setmainfont{Arial}{0.3352}}}} \\





\multicolumn{1}{>{\raggedright}m{\dimexpr 0.95in+0\tabcolsep}}{\textcolor[HTML]{000000}{\fontsize{11}{11}\selectfont{\global\setmainfont{Arial}{Yoplait}}}} & \multicolumn{1}{>{\raggedleft}m{\dimexpr 1.06in+0\tabcolsep}}{\textcolor[HTML]{000000}{\fontsize{11}{11}\selectfont{\global\setmainfont{Arial}{1}}}} & \multicolumn{1}{>{\raggedleft}m{\dimexpr 0.82in+0\tabcolsep}}{\textcolor[HTML]{000000}{\fontsize{11}{11}\selectfont{\global\setmainfont{Arial}{0.3681}}}} & \multicolumn{1}{>{\raggedleft}m{\dimexpr 0.75in+0\tabcolsep}}{\textcolor[HTML]{000000}{\fontsize{11}{11}\selectfont{\global\setmainfont{Arial}{0.0289}}}} & \multicolumn{1}{>{\raggedleft}m{\dimexpr 0.76in+0\tabcolsep}}{\textcolor[HTML]{000000}{\fontsize{11}{11}\selectfont{\global\setmainfont{Arial}{0.2112}}}} & \multicolumn{1}{>{\raggedleft}m{\dimexpr 0.75in+0\tabcolsep}}{\textcolor[HTML]{000000}{\fontsize{11}{11}\selectfont{\global\setmainfont{Arial}{0.3918}}}} \\

\ascline{1.5pt}{666666}{1-6}



\end{longtable}



\arrayrulecolor[HTML]{000000}

\global\setlength{\arrayrulewidth}{\Oldarrayrulewidth}

\global\setlength{\tabcolsep}{\Oldtabcolsep}

\renewcommand*{\arraystretch}{1}

\begin{Shaded}
\begin{Highlighting}[]
\NormalTok{pp\_feat}\SpecialCharTok{$}\NormalTok{pp\_plot}
\end{Highlighting}
\end{Shaded}

\pandocbounded{\includegraphics[keepaspectratio]{MKT4320_R_Tutorial_files/figure-latex/pp-alt-2.pdf}}

\begin{center}\rule{0.5\linewidth}{0.5pt}\end{center}

\section{Managerial Insights}\label{managerial-insights}

Alternative-specific MNL models allow managers to:

\begin{itemize}
\tightlist
\item
  Evaluate pricing and promotion strategies
\item
  Understand competitive substitution patterns
\item
  Predict market share changes under different scenarios
\end{itemize}

Compared to standard MNL models, AS-MNL models provide more realistic insights when brand attributes vary within choice sets.

\bibliography{book.bib,packages.bib}

\end{document}
